\subsection{Multikey-Quicksort}
\label{section:mkqs}
Der \currentauthor{Oliver Magiera und Marvin Böcker} Multikey-Quicksort von Bentley und Sedgewick basiert auf
ternärem Quicksort und sortiert Strings eines Arrays \textbf{A} zeichenweise \cite{multikey_quicksort}.
Das Funktionsargument \textit{d} beschreibt, welches Zeichen momentan verglichen werden soll.
Initial ist dieser Parameter 0, um das erste Zeichen von den Strings in \textbf{A} zu vergleichen.
Wenn dieses Zeichen kleiner oder größer ist, kann der String in die \glqq <\grqq- bzw \glqq >\grqq-Partition sortiert werden.

\begin{figure}
	\begin{minted}[mathescape=true,escapeinside=||]{python}
def mkqs(A: Array<SuffixStartPos>, d: Integer):
  if |A| <= 1:
    return end

  p := choose a pivot element

  # Rearrange A so, that T[A[..i]] has a character at position d
  # that is smaller than p, T[A[i..j]] has p at position d and 
  # T[A[j..]] has a character at position d that is larger than p.
  i, j := partition(A, d, p)

  mkqs(A[0, i), d)
  # Sort Equal-Partition by next character.
  mkqs(A[i, j), d+1)
  mkqs(A[j, |A|), d)
	\end{minted}
	\caption{Bentley-Sedgewick Multikey-Quicksort~\cite{multikey_quicksort}}
	\label{alg:mkqs}
\end{figure}

% An Referenzen in Zwischenbericht anpassen.
Bei Gleichheit hingegen muss das nächste Zeichen betrachtet werden,
um die Strings eindeutig zu sortieren, während für die \glqq <\grqq- bzw. \glqq >\grqq-Partitionen nach dem aktuellen Zeichen genauer sortiert wird.
Abbildung \ref{fg:ternary_partitions} stellt die Partitionierung eines Sortierschritts bei Pivotelement $p$ an.

Der Pseudocode \ref{alg:mkqs} skizziert die Vorgehensweise des MK-QS Algorithmus.
Es wird erst ein Pivotelement $p$ bestimmt, welches zur Partitionierung benötigt wird.
Dabei werden alle Elemente kleiner als das Pivot in $[0, i)$ einsortiert, alle gleich dem Pivotelement in $[i, j)$ und schließlich
alle größeren Elemente in $[j, |A|)$.
Zunächst wird die \glqq <\grqq-Partition nach dem $d$-ten Zeichen rekursiv
weiter sortiert, bevor für die \glqq =\grqq-Partition zum Sortieren das $d+1$-te Zeichen verwendet wird.
Zuletzt wird die Rekursion für die \glqq >\grqq-Partition durchgeführt.

