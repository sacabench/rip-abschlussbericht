\subsection{sacabench list}
\label{framework:cli:sacabench-list}

{
\begin{wrapfigure}[12]{R}[5mm]{.5\textwidth}
    \vspace{-1.5\baselineskip}
    \includegraphics[page=1, viewport=0cm 32.8cm 20.5cm 42.5cm, clip, width=.5\textwidth]{{kapitel/3_framework/cli/sacabench-list/sacabench-list}.pdf}\\
    \includegraphics[page=1, viewport=0cm 25cm 20.5cm 26.3cm, clip, width=.5\textwidth]{{kapitel/3_framework/cli/sacabench-list/sacabench-list}.pdf}\\
    \includegraphics[page=1, viewport=0cm 0cm 20.5cm 1.5cm, clip, width=.5\textwidth]{{kapitel/3_framework/cli/sacabench-list/sacabench-list}.pdf}
    \caption{Gekürzte Ausgabe von \texttt{man sacabench list}.}
    \label{manpage:sacabench-list}
\end{wrapfigure}
Mit dem Befehl \texttt{sacabench list} können alle verfügbaren Algorithmen aufgelistet werden.
Nach dem Aufruf erscheint eine Liste aller im Rahmen der Projektgruppe implementierten Algorithmen, sowie deren Referenzimplementierungen. 
Letztere sind in der angegebenen Abkürzung durch \termfont{\_ref} markiert.\par
Über die Option \termfont{-{}-no-description} kann außerdem die Ausgabe der Kurz\-be\-schrei\-bungen unterdrückt werden, sodass nur noch eine Liste aller Kürzel erscheint. 
Die dort aufgelisteten Kürzel entsprechen den Bezeichnungen der Algorithmen, über die sie vom Framework referenziert werden.
Zusätzlich ermöglicht es die Option \termfont{-j} oder \termfont{-{}-json}, die Namen der Algorithmen als ein JSON Array auszugeben.\par
}
