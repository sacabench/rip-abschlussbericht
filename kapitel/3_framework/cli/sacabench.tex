\subsection{sacabench}
\label{framework:cli:sacabench}

Das Hauptprogramm wird mit dem Befehl \texttt{sacabench} ausgeführt.
Für die einzelnen Funktionen stehen ent\-sprech\-ende Subcommands zur Verfügung.
Alle Komponenten beinhalten eine Hilfefunktion, die sich durch die Option \termfont{-h} bzw. \termfont{-{}-help} aufrufen lässt.
Die gleiche Hilfe wird außerdem ausgegeben, wenn das Programm mit ungültigen Parametern aufgerufen wird.
Für präzisere Erläuterungen der Aufrufe und Optionen stehen außerdem Man-Pages für alle Subcommands zur Verfügung.
Diese beinhalten eine Beschreibung der Befehlssyntax sowie Erläuterungen zur Verwendung der Optionen und Verweise auf ähnliche Befehle.\par
Ohne vorangegangene Installation sind die Man-Pages nur lokal aufrufbar: Aus dem \sacabench Wurzelverzeichnis lässt sich beispielsweise die Man-Page für \termfont{sacabench batch} über den Befehl \termfont{man ./man/man1/sacabench-batch.1.gz} aufrufen. Für eine systemweite Installation steht das Shell-Skript \termfont{install\-manpages.sh} zur Verfügung, welches mit entsprechenden Rechten ausgeführt werden muss, um die benötigten Dateien ins Verzeichnis \termfont{/usr/local/man/man1/} zu kopieren.
Eine gekürzte Version der Man-Page für den Befehl \termfont{sacabench} ist in \cref{manpage:sacabench} zu sehen.
