\subsection{sacabench}
\label{framework:cli:sacabench}

Das Hauptprogramm wird mit dem Befehl \texttt{sacabench} ausgeführt.
Für die einzelnen Funktionen stehen ent\-sprech\-ende Subcommands zur Verfügung.
Alle Komponenten beinhalten eine Hilfefunktion, die sich durch die Option \termfont{-h} bzw. \termfont{-{}-help} aufrufen lässt.
Die gleiche Hilfe wird außerdem ausgegeben, wenn das Programm mit ungültigen Parametern aufgerufen wird.
Für präzisere Erläuterungen der Aufrufe und Optionen stehen außerdem Man-Pages für alle Subcommands zur Verfügung.
Diese sind wie üblich über \termfont{man sacabench} bzw. \termfont{man sacabench [SUBCOMMAND]} aufzurufen \todo{man pages mit cmake installieren?} und beinhalten eine Beschreibung der Befehlssyntax sowie Erläuterungen zur Verwendung der Optionen und Verweise auf ähnliche Befehle.
Eine gekürzte Version der Man-Page für den Befehl \termfont{sacabench} ist in \cref{manpage:sacabench} zu sehen.
