\subsection{Automatische PDF-Generierung mit SqlPlotTools}
\label{framework:bechmark:sqlplottools}

\currentauthor{David Piper und Florian Grieskamp}
Wird \texttt{sacabench construct} zur Analyse eines einzelnen Algorithmus und dessen Phasen oder \texttt{sacabench batch} zum Vergleich mehrerer Algorithmen untereinander mit zusätzlicher Option \texttt{-{}-latexplot} ausgeführt, werden die erhaltenen Messdaten direkt in einer PDF-Datei aufgearbeitet.
Hierzu wird zunächst vom \texttt{sacabench} Haupttool eine Konfigurations-Datei mit Metainformationen über die durchgeführte Messung erstellt.
Diese beinhaltet Informationen zum System, auf dem die Analyse ausgeführt wird und zu dem Text, welcher als Input für die Algorithmen verwendet wurde.
Zum System werden die Anzahl der CPUs, die Anzahl an Threads pro Socket, das Modell und das verwendete Betriebssystem erfasst.
Bezüglich des Textes enthält die Konfigurationsdatei den Dateinamen des Textes, Präfixgröße und Anzahl der Wiederholungen der Messung.
Diese Datei wird im JSON-Format im Verzeichnis \texttt{zbmessung/sqlplot} gespeichert, welches ebenfalls ein Skript names automation.sh beinhaltet.\par
Im Anschluss an das Benchmarktool wird dieses Skript ausgeführt.
Hierdurch wird zunächst ein temporäres Verzeichnis erzeugt, welches zur Generierung der fertigen PDF-Dateien dient.
Neben der zuvor erstellten Konfigurations-Datei beinhaltet dieses Verzeichnis auch zwei LaTeX-Dateien, welche als Vorlagen für die zu erstellende PDF-Datei dienen, das für die Konvertierung des Datenformats benötigte Python-Skript \texttt{json\_to\_result\_converter.py} und ein Makefile, welches spätere Abläufe koordiniert.
Zusätzlich wird die JSON-Datei, welche bei der Messung durch das Benchmarktool erstellt wurde, in das temporäre Verzeichnis kopiert.
Damit sind alle benötigten Vorbereitungen getroffen und das Makefile im Unterverzeichnis \texttt{sqlplot} wird durch das Skript ausgeführt.\par\smallskip
Das Makefile führt das Python-Skript \texttt{json\_to\_result\_converter.py} aus, welches als erstes die Messergebnisse unter Berücksichtigung der Metadaten in ein \texttt{RESULT}-Format konvertiert.
Dieses beinhaltet die Daten in einem für \emph{SqlPlotTools} \cite{sqlplottools} lesbaren Format.
Da abhängig davon, ob \texttt{construct} oder \texttt{batch} aufgerufen wurde, unterschiedliche Daten im PDF benötigt werden, werden zwei unterschiedliche Result-Dateien erstellt:
Eine Datei enthält Daten für die genauere Analyse der unterschiedlichen Phasen aller Algorithmen und die andere Datei beinhaltet die Messwerte für gesamte Algorithmen.
Die Daten der einzelnen Phasen werden für die von \texttt{sacabench construct} erzeugten Diagramme benötigt, während die Messwerte der gesamten Algorithmen in den von \texttt{sacabench batch} erzeugten Diagrammen visualisiert werden.
Zusätzlich generiert das Python-Skript die LaTeX-Dateien, aus denen im Anschluss die fertigen PDF-Dateien erzeugt werden. 
Hierzu werden die beiden LaTeX-Vorlagen im Unterordner \texttt{templates} genutzt.
Nachdem all diese Dateien generiert wurden, klont das Makefile (sofern noch nicht im Verzeichnis \texttt{zbmessung/sqlplot} vorhanden) das SqlPlotTools-Repository von GitHub \cite{sqlplottools} in den temporären Ordner und baut dort das Projekt.
Jede durch das Python-Skript erstellte LaTeX-Datei wird durch einen Aufruf des SqlPlotTools mit den für die Plots benötigten Daten befüllt.
Dieser Aufruf wird ebenfalls von dem Makefile ausgelöst. 
Daraufhin können die LaTeX-Dateien gesetzt werden, wodurch die fertige PDF-Datei mit den neuen Messdaten entsteht.\par\smallskip
Im Anschluss kopiert das Skript \texttt{automation.sh} die generierte PDF-Datei in ihr Zielverzeichnis.
Zuletzt wird das temporäre Verzeichnis wieder vom System entfernt.
\newcommand{\comment}[1]{}
\comment{
In sacabench.cpp:
1. Erstellung einer Config-Datei mit Metainformationen über die durchgeführte Messung.
2. Speichern der Config-Datei in dem Ordner zbmessung/sqlplot.
3. Starte das Script zbmessung/automation.sh.

In zbmessung/automation.sh:
1. Erstelle temporären Ordner
2. Kopiere Ordner zbmessung/sqlplot in temporären Ordner
3. Kopiere durch Benchmark erstellte JSON-Datei in temporären Ordner
4. Rufe make im temporären Ordner auf.
5. Kopiere generierte PDF-Dateien an Zielort.
6. Lösche temporären Ordner.

Im temporären Ordner durch das Makefile:
1. Das Python-Skript json\_to\_result\_converter.py erstellt Result-Dateien, die von SqlPlotTools verarbeitet werden können. 
Diese enthalten die Daten aus der durch sacabench.cpp erstellten Config-Datei und der durch das Benchmark erstellten und durch automation.sh kopierten JSON-Datei mit den Messergebnissen.
Dazu werden zwei verschiedene Result-Dateien erstellt, eine mit Daten für den Vergleich mehrerer Algoirhtmen untereinander nach batch, und eine mit Daten für die genauere Analyse der einzelnen Phasen eines einzelnen Algorithmus nach construct.
Dieses Python-Skript generiert neben den Result-Dateien auch die LaTeX-Dateien, welche durch SqlPlotTools mit den Daten aus der Result-Datei befüllt werden.
Im Unterordner templates befinden sich zwei LaTeX-Dateien, eine für das Ergebnis eines construct-Aufrufs und eine für das Ergebnis eines batch-Aufrufs.
Abhänig vom Befülle die im Ordner templates befindlichen LaTeX-Dateien (je nach Ausführung) mit den Ergebnissen der Abfragen aus SqlPlotTools.
2. Klone SqlPlotTools von GitHub und baue das Projekt.
3. Jede durch das Python-Skript erstellte LaTeX-Datei wird durch einen Aufruf des SqlPlotTools mit Daten befüllt.
Dieser Aufruf wird ebenfalls von dem Makefile ausgelöst. 
4. Nachdem nun die LaTeX-Dateien generiert und mit den Daten befüllt sind, werden sie gesetzt.
}
