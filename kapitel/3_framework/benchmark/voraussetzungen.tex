\subsection{Voraussetzungen}

\begin{description}
	\item[\termfont{R - Version}] Die Ergebnisse des Benchmark Systems werden mit Hilfe der Programmiersprache \texttt{R} ausgewertet. \texttt{R} ist eine \emph{Open-Source-Software} für statistische Berechnungen und Grafiken. Daher muss auf dem auszuführenden Computer ein \texttt{R}-Compiler installiert sein\cite{r-project}. Das \texttt{R}-Skript verwendet unter anderem das Paket \termfont{rjson}, welches erst ab der Version \texttt{R} $\geq 3.1.0$ unterstützt wird\cite{rjson}. Aus diesem Grund wird die Version \texttt{R} $\geq 3.1.0$ vorausgesetzt.
	\item[\termfont{Administrationsrechte}] Das \texttt{R}-Skript benötigt neben dem Paket \termfont{rjson} auch das Paket \termfont{RColorBrewer} \cite{rcolorbrewer}. Beide Pakete werden automatisch installiert, wenn sie nicht vorher schon manuell installiert worden sind. Für diese Installation werden jedoch gegebenenfalls Administrationsrechte benötigt. Außerdem sind zum Beispiel für Computer mit Betriebssystem \texttt{macOS} Schreib- und Leserechte bestimmter Verzeichnisse notwendig. Daher wird empfohlen, die Befehle als Administrator auszuführen, indem zum Beispiel das Schlüsselwort \termfont{sudo} verwendet wird.
\end{description}
