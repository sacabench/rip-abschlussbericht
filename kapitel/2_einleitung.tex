\chapter{Einleitung}

Im \currentauthor{Christopher Poeplau und Marvin Böcker} Zuge der Projektgruppe 2018/2019 beschäftigt sich
\emph{\sacabench} mit Suffix-Array-Konstruktionsalgorithmen. \emph{SACA} ist das Akronym für
\textbf{S}uffix \textbf{A}rray \textbf{C}on\-struction \textbf{A}lgorithm und \emph{Bench} die Abkürzung für Benchmark,
also dem Messen von Laufzeit und Speicherplatz der Algorithmen.

Die Forschung an effizienten Konstruktionsalgorithmen für Suffix-Arrays hat in den letzten Jahren durch
immer komplexer werdende Anwendungen enorm an Bedeutung gewonnen.
Gerade in der Bioinformatik,
in der man es im Bereich der Genomforschung mit Größenordnungen von Milliarden von Zeichen zu tun hat,
sind effiziente Algorithmen in Bezug auf Zeit und Speicherplatz notwendig~\cite[Kap.~1]{saca:6}.

Ein Suffix-Array repräsentiert die Suffixe eines Strings \inputtext{} in lexikographischer Reihenfolge.
Sei $\inputtext{} = \text{suffix}$. Für den String \inputtext{} ergeben sich folgende Suffixe:
%

\begin{table}[htb]
\centering
  \begin{tabular}{ | l | c | r }
    \hline
        $i$ & $\inputtext{}[i, n)$ \\ \hline
        0 & suffix \\ \hline
        1 & uffix \\ \hline
        2 & ffix \\ \hline
        3 & fix \\ \hline
        4 & ix \\ \hline
        5 & x \\ \hline
        6 & \$ \\
    \hline

  \end{tabular}
  \caption{Suffixe des Strings \inputtext}
\end{table}

%
Es ist sofort ersichtlich, dass $\inputtext{}[0, n) = \inputtext{}$ und $\inputtext{}[n-2, n)=\$$ gilt.
\newpage \noindent Sortiert man die Suffixe nun lexikografisch, ergibt sich das Suffix-Array:
%
\begin{table}[H]
\centering
  \begin{tabular}{ | l | c | r }
    \hline
        $i$ & $\inputtext{}[i, n)$ \\ \hline
        6 & \$ \\ \hline
        3 & fix \\ \hline
        2 & ffix \\ \hline
        4 & ix \\ \hline
        0 & suffix \\ \hline
        1 & uffix \\ \hline
        5 & x \\
    \hline
  \end{tabular}
  \caption{Sortiertes Suffix-Array des Strings \inputtext}
\end{table}

%
Somit ergibt sich das Suffix-Array $\sa = \{ 6,3,2,4,0,1,5 \}$.
Ziel ist die effiziente Konstruktion dieses Arrays.
Dabei gibt es Algorithmen, die den Fokus auf die Laufzeit setzen,
andere wiederum auf die Speicheroptimierung und wieder andere versuchen, den besten Kompromiss aus beiden Welten zu finden.
Die Aufgabe der Projektgruppe ist das Schaffen einer umfangreichen Library der bekanntesten SACAs.
Diese sind eingebettet in ein Framework, das es ermöglicht, auf intuitive Art und Weise Algorithmen auf beliebigen
Texten zu testen und die Performance miteinander zu vergleichen.
Grundziel des Frameworks ist die Vereinheitlichung:
Viele der Algorithmen existieren in einzelnen Repositorys und die Algorithmen werden in den meisten Fällen 
nicht auf vergleichbarer Basis getestet und analysiert.
Diese Algorithmen gilt es zunächst zu verstehen
und dann zu implementieren, sodass sie den Schnittstellen des Frameworks genügen.
Es soll also ein erweiterbares Gesamtkonstrukt geschaffen werden, das bestehende SACAs
sammelt und repräsentatives Vergleichen der Algorithmen ermöglicht.

\section{Notation und Definitionen}
\todo{Macht die Definitionen lieber wirklich als Definitionen und nicht als subsections}
Einige\currentauthor{Rosa Pink und\\Marvin Löbel} grundlegende Notationen und Definitionen werden hier kurz vorangestellt.

\begin{definition}[Intervall]
Wir nutzen $[i, j] = \{i, \dots, j\}$ und $[i, j) = [i, j - 1]$ als Kurzschreibweise für Integer-Intervalle.
\end{definition}

\begin{definition}[Alphabet und lexikographische Sortierung]
Das konstante Alphabet ist definiert als $\Sigma = \{\sigma_1, \sigma_2, ..., \sigma_{|\Sigma|-1}\} \cup \{\$\}$.
Es besteht aus Symbolen bzw. Zeichen $\sigma_i$, die im Eingabe-String vorkommen dürfen, und dem Sentinel-Symbol \$, welches das Ende des Strings markiert. 
Die Zeichen sind wie folgt lexikographisch (nach \textit{lexikographischer Sortierung})
geordnet: $\$ < \sigma_1 < \sigma_2 < ... < \sigma_{|\Sigma|-1}$. 
Sie lassen sich durch Integer-Werte repräsentieren, indem wir \$ den Wert 0 zuweisen und $\sigma_i$ den Wert $i$.
$\Sigma^+$ bezeichnet weiter die Menge aller Strings über diesem Alphabet mit echt positiver Länge.
\end{definition}

\begin{definition}[Input-String und Suffix]
Der Input-String wird \inputtext genannt und hat die Länge $n$. Das $i$-te Zeichen im
Input-String ist \inputtext[i], das $i$-te Suffix  (kurz: Suffix $i$) ist 
\begin{align*}
\suffix{i} := \inputtext[i,n) = \inputtext[i]\inputtext[i+1]...\inputtext[n-1]
\end{align*} \inputtext[n]
ist das Terminalsymbol \$ (sowie alle \inputtext[m] mit $m>n$) und ist formal nicht Teil des Eingabe-Strings.
Der Eingabe-String beginnt bei \inputtext[0].
\end{definition}

\begin{definition}[Suffix-Array]
Das Suffix-Array, kurz \sa, bezeichnet ein Array, in dem in lexikographischer Reihenfolge
die Suffix-Indizes (Positionen des Anfangsbuchstabens) gespeichert sind.
Das bedeutet, $\sa[j] = i$ genau dann, wenn $\mathsf{T}[i,n)$ das $j$-te Suffix von \inputtext in lexikografischer Ordnung ist.
\end{definition}

\begin{definition}[Bucket]
    \label{def:bucket}
    Alle Suffixe, die mit demselben Zeichen $c_0 \in \Sigma$ beginnen, formen ein zusammenhängendes Intervall im Suffix-Array.
    Dieses Intervall wird $c_0$-Bucket genannt und mit $\bucket{c_0}$ bezeichnet.
    Der $(c_0,c_1)$-Bucket $\bucket{c_0, c_1}$ bezeichnet das Intervall, dessen Suffixe mit denselben zwei Zeichen $c_0, c_1 \in \Sigma$ beginnen.\par
    Ähnlich dazu bezeichnet der Bucket $\bucket{\omega}$ alle Suffixe im \sa, die alle mit dem String $\omega \in \Sigma^m$ mit \(m > 0\) anfangen. \(\bucket{\omega}\) heißt dann auch Level-\(m\)-Bucket.
\end{definition}

\begin{definition}[Textwiederholung]
\label{def:repetition}
Eine Wiederholung in $\mathsf{T}$ ist ein Teilstring $\mathsf{T}[i, i + rp]$ mit $ r \geq 2, p \geq 0$ und $i, i + rp \in [0, n)$, sodass $\mathsf{T}[i, i+p) = \mathsf{T}[i + p, i + 2p) = \dots = \mathsf{T}[i + (r-1)p, i + rp)$.
\end{definition}

\subsection{Praxisrelevante Grenzen}

Wir gehen im Folgenden davon aus, dass für unsere Eingabe $|\Sigma| = 256$ gilt, da sich so jedes Zeichen durch ein Byte repräsentieren lässt. Dies erlaubt auch das direkte Verarbeiten von Texten in Standardkodierungen wie ASCII und UTF-8~\cite{grundlagen:utf8}, die durch Byte-Arrays ohne 0-Bytes repräsentiert werden.

Auch gehen wir davon aus, das die Länge $n$ von Datenstrukturen, insbesondere der Eingabe und des Suffix-Arrays, durch die maximale Größe nativer Integer-Datentypen in Consumer-Computersystemen begrenzt ist. Es gilt somit in der Regel $n \leq 2^{32}$ oder $n \leq 2^{64}$, womit sich Indizes durch 4- bzw. 8-Byte Integer repräsentieren lassen. Wir betrachten außerdem $n = 2^{40}$ als 5-Byte Kompromiss zwischen den Beiden.

\subsection{Codebeispiele}

Wir geben alle algorithmischen Codebeispiele in an Python angelehnten Pseudocode an.

%\section{Notationen und Definitionen}
%\label{definitions}
%Sei \currentauthor{Christopher Poeplau} $T$ ein Text mit Zeichenanzahl $\vert T \vert = n$
%\begin{itemize}
%    \item $\Sigma (T)$ := Alphabet von $T$
%    \item Ein Substring von $T[i, j) = T[i]...T[j-1]$
%    \item $T$ wird durch das lexikografisch kleinste Zeichen, dem $\$$ terminiert (engl. sentinel)
%\end{itemize} 
%\bigskip
%Suffix $S_i=T[i,n)$ als das in $i$ beginnende Suffix in $T$. 
%\begin{itemize}
%    \item Das Suffix ist ein S(hort)-Type-Suffix $\iff S_i<S_{i+1}$
%    \item Das Suffix ist ein L(arge)-Type-Suffix $\iff S_i>S_{i+1}$
%    \item $Type('\$') := S$
%    \item Ein Zeichen $T[i]$ ist S-Type $\iff Type(S_i) = S$
%    \item Ein Zeichen $T[i]$ ist L-Type $\iff Type(S_i) = L$
%\end{itemize}
%
%\subsection{Leftmost S-Type}
%Ein Zeichen $T[i]$ des Strings $T$ ist genau dann LMS, wenn $Type(T[i-1])=L$, also das Vorgängerzeichen vom Typ $L$ ist. Gleichzeitig ist ein Suffix $T[i, n)$ ein LMS-Suffix, wenn $T[i]$ LMS ist.\\
%Ein Substring $T[i, j]$, $i\neq j$, ist LMS, wenn $Type(T[i])=S$ und $Type(T[j])=S$. Zudem darf in diesem Substring kein weiteres LMS-Zeichen vorhanden sein. Insbesondere ist $T[j]$ dadurch exklusiv und gehört nicht zu dem Substring. \\
%Für die Gleichheit zweier LMS-Substrings gilt das folgende: \\
%Seien $T_1$ und $T_2$ LMS-Substrings.
%\begin{center}
%    $S_1=S_2 \iff \vert S_1 \vert = \vert S_2 \vert $ $\wedge$ $S_1$ besitzt dieselben Zeichen wie $S_2$ in derselben Reihenfolge
%\end{center}

