\section{Zusammenfassung}

Wir \currentauthor{Janina Michaelis und Marvin Böcker} haben das Tool SACABench vorgestellt, welches viele verschiedene SACAs enthält.
Durch dieses Commandline Tool können alle Algorithmen inklusiv bestehender Referenzimplementationen einfach ausgeführt werden
und in Laufzeit und Speicherverbrauch auf Texten mit verschiedenen Eigenschaften gegeneinander getestet werden.

Es wurden gemeinsame Komponenten wie beispielsweise verschiedene Sortierverfahren ausgelagert und optimiert,
um doppelten Code zu verhindern und bessere Implementierungen dieser Algorithmen zu erreichen.

Zusammenfassend sieht es so aus, dass unsere Implementierung teils einen besseren Speicherverbrauch vorweist,
dafür aber meistens eine schlechtere Laufzeit als die Referenz hat.
Insgesamt wurde das Ziel, die angegebenen Algorithmen zu implementieren und
durch ein Benchmark-Framework miteinander zu vergleichen, erreicht.
Es zeigt sich, dass DivSufSort immernoch auf den meisten Texten der schnellste SACA ist.

Im zweiten Semester haben wir eine Auswahl von Algorithmen auf GPU und CPU parallelisiert:
zuerst wurde dabei eine naive Parallelisierung versucht, indem die Sortieralgorithmen in den SACAs durch
parallele Varianten ersetzt wurden, beispielsweise den parallelen \ipsviero (\cref{section:ips4o}) oder
den stabilen Sortierer aus der GNU Standardbibliothek.
Dabei dominieren die Sortierer aus der Standardbibliothek klar.

Zuletzt wurden komplexere Parallelisierungen der SACAs verfolgt, wie beispielsweise für pSAIS
oder Deep-Shallow:
Diese zeigen jedoch noch Optimierungspotential.
