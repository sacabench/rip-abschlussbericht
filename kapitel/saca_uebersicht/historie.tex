\section{Historie}
Alle \currentauthor{David Piper} in dieser Ausarbeitung vorgestellte Algorithmen zur Konstruktion eines Suffix Arrays lassen sich in drei Gruppen einteilen, welche die grundlegenden Sortierverfahren beschreiben. \par
Die erste Gruppe bilden die sogenannten \textit{prefix doubler}, zu welcher die Algorithmen Prefix Doubling und qSufSort geh{\"o}ren.
Wie in Abschnitt \ref{sec:ansatz:doubler} beschrieben wird, sortieren diese die Pr{\"a}fixe der Suffixe und k{\"o}nnen durch geschickte Ideen die Sortierung beschleunigen, indem die Anzahl der betrachteten Pr{\"a}fixe in jedem Schritt verdoppelt wird. 
Dabei setzte der SACA Prefix Doubling die Grundlagen f{\"u}r diese Kategorie, auf der dann qSufSort aufbaute, um eine effizientere Konstruktion des Suffix Arrays zu erm{\"o}glichen. 
Neben diesen beiden Algorithmen geh{\"o}rt auch der Algorithmus BPR teilweise in diese Gruppe, nutzt jedoch auch Ideen eines zweiten Verfahrens, dem \textit{Induzieren}. \par

Zur Gruppe der Induzierer geh{\"o}ren ebenfalls die Algorithmen Deep-Shallow, DivSufSort, mSufSort, SAIS und GSACA.
Sie sortieren Suffixe anhand bereits zuvor sortierter Suffixe.
Dieses Konzept wird n{\"a}her im Absatz \ref{section:induzierer} beschrieben.
Deep-Shallow war von den hier behandelten Algorithmen der erste, welcher dieses Verfahren nutzte.
Von ihm wurde der Algorithmus DivSufSort inspiriert, dessen Implementierungsdetails teilweise in die Entwicklung von SAIS eingingen.
SAIS f{\"u}hrte Ideen ein, welche auch im Algorithmus GSACA aufgegriffen wurden.
Hingegen bildet mSufSort eine eigene Vererbungslinie und baut auf keinem der zuvor genannten SACAs auf.
Auch nzSufSort und SACA-K geh{\"o}rten zu dieser Gruppe, kombinieren das induzierte Sortieren jedoch mit dem Konzept der letzten Gruppe. \par

Diese nutzt Rekursion um kleinere Teilmengen von Zeichen zu Sortieren, welche dann zum finalen Suffix Array zusammengesetzt werden.
Zus{\"a}tzlich zu nzSufSort und SACA-K wird dieses Sortierverfahren auch von den SACAs DC3, SADS und GOTO verwenden. 
Der erste Algorithmus, welcher rekursiv arbeitete um ein Suffix Array zu erstellen, war DC3. 
Zusammen mit einem anderen Algorithmus, welcher die Konzepte des induzierten Sortierens und der Rekursion kombinierte, inspirierte DC3 den Algorithmus nzSufSort. 
Dieser andere Algorithmus war ebenfalls die Grundlage f{\"u}r den zuvor bereits genannten SAIS sowie zusätzlich für den änlichen Algorithmus SADS.
SACA-K, welcher sowohl induziert als auch rekursiv arbeitet, ist eine Weiterentwicklung der Ideen von SAIS und SADS und stellt dabei sowohl eine Verbesserung der Laufzeit als auch des Speicherverbrauchs dar.
Zuletzt leitet sich GOTO direkt von SACA-K ab. \par

\begin{figure}[H]
	\centering
	\includegraphics[width=\linewidth]{kapitel/saca_uebersicht/history/history2}
	\caption[Geschichtliche Entwicklung von SACAs.]{Geschichtliche Entwicklung von SACAs.}
	\label{fig_banane_1_2}
\end{figure}
