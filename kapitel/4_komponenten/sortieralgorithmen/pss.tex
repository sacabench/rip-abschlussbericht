\subsection{Parallel Stable Sort}
\label{section:pss}

Für einige Algorithmen war es wichtig, eine stabiles Sortierverfahren zu implementieren.
Wir verwenden dafür eine Implementierung von Intel\textregistered{} eines parallelen \emph{Merge Sorts}~\cite{pss_intel}.
Merge Sort ist ein rekursives, stabiles Sortierverfahren, welches eine theoretische Worst-Case-Komplexität von $\mathcal O (n \log n)$ hat.
Diese Struktur macht es möglich, diesen Algorithmus effizient zu parallelisieren.

Sei $\mathsf A$ das zu sortierende Array, so werden zuerst die beiden
Hälften $\mathsf A[0, \frac{|A|}{2})$ und $\mathsf A[\frac{|A|}{2}, |\mathsf A|)$ rekursiv sortiert.
Der Rekursionsabbruch ist dabei eine zu sortierende Menge der Größe 1, welche automatisch sortiert ist.
Danach kann die Sortierung der Gesamtmenge aus den sortierten Teilmengen zusammengesetzt werden,
indem beide Hälften in der sortierten Reihenfolge gleichzeitig durchlaufen werden.
Dabei wird immer das kleinere, aktuelle Element der beiden Partitionen verwendet,
um die Gesamtsortierung zu konstruieren.

Um diesen Algorithmus zu parallelisieren,
können beide Partitionen von unabhängigen Kernen sortiert werden.
Der \emph{Merge-Schritt} muss dennoch sequentiell ausgeführt werden.