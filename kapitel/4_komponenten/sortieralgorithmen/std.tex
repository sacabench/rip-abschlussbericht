\subsection{Standardbibliothek-Sortierer}
\label{section:stdsort}

In \sacabench verwenden wir außerdem die Sortierer aus der GNU Standardbibliothek von C++.
Wir verwenden insbesondere die parallelen Varianten von \texttt{std::sort}
und \texttt{std::stable\_sort}.
Diese erwiesen sich als zuverlässiger als \ipsviero auf vielen Kernen
und werden daher für die meisten naiven Parallelisierungen der SACAs verwendet.

Im nicht-stabilen Sortierer kann entweder \emph{Parallel Multiway Mergesort} oder \emph{Parallel Load-Balanced Quicksort} verwendet werden.
Die Mergesort-Variante teilt dabei das Problem nicht in zwei Teile sondern $k$ Teile, wovon jeder von einem anderen Kern bearbeitet wird.
Um Verlangsamungen durch kleine Partitionen zu vermeiden, teilt die Quicksort-Variante Threads,
die mit ihrem Anteil des Arrays schon fertig sind, neue Teile zu, um eine höhere Effizienz zu erreichen.
Da der erste Algorithmus stabil sortiert, wird er auch als stabiler Sortierer verwendet.
Die Details der verwendeten Algorithmen werden im Paper von Singler und Kosnik von 2008~\cite{parallelstdsort} erläutert.