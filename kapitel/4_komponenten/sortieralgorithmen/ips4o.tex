\subsection{Inplace Parallel Super Scalar Samplesort (\ipsviero)}
\label{section:ips4o}

In \sacabench verwenden wir auch andere Sortieralgorithmen, die wir nicht selbst implementiert haben.
Der erste dieser Algorithmen ist \ipsviero~\cite{axtmann2017}.
Dieser existiert in einer sequentiellen und einer parallelen Variante,
die mittels OpenMP parallel funktioniert.

\ipsviero baut auf Samplesort auf, welcher wiederrum auf Quicksort aufbaut.
Statt nur ein Pivot wie bei Quicksort werden dabei aber mehrere Pivot verwendet.
Dadurch erhält man einen Algorithmus \glqq der cache-effizient ist, datenparallel arbeitet und \emph{branch mispredictions} verhindert\grqq~\cite{axtmann2017}.
\ipsviero verbessert diesen Algorithmus und implementiert ihn in-place, also mit $\mathcal O(1)$ Extraspeicher.
Da dies ein komplexes, vergleichsbasiertes Sortierverfahren ist,
sei an dieser Stelle für weitere Details auf das Paper von Axtmann et al.~\cite{axtmann2017}
beziehungsweise das öffentliche Repository von \ipsviero~\cite{ips4o:repo} verwiesen.