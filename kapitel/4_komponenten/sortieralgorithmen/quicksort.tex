\subsection{Quicksort}
\label{section:quicksort}

Der \currentauthor{Nico Bertram} Quicksort ist ein Sortieralgorithmus,
der nach dem Teile \& Herrsche Prinzip funktioniert und
in der ersten Version von Hoare vorgestellt wurde~\cite{quicksort}.
Der Algorithmus erreicht zwar mit $\mathcal O(n^2)$ im Gegensatz zu anderen
vergleichsbasierten Sortieralgorithmen eine schlechtere Worst-Case Laufzeit,
aber in der Praxis ist Quicksort besser als andere Algorithmen.

Das Vorgehen ist dabei wie folgt: Sei $\mathsf{A}$ das zu sortierende Array.
Es wird ein geeignetes Pivotelement $p$ bestimmt und das Array durch Vertauschungen in
zwei Bereiche $\mathsf{A}_{\text{left}}$ und $\mathsf{A}_{\text{right}}$ aufgeteilt.
Dabei soll für alle $i$ gelten, dass $\mathsf{A}_{\text{left}}[i] \le p$ und $\mathsf{A}_{\text{right}}[i] > p$.
Die beiden Teilarrays werden anschließend rekursiv sortiert.
Der Rekursionsabbruch erfolgt, wenn das Eingabearray eine Größe von $0$ oder $1$ besitzt.
In diesen Fällen ist die Eingabe bereits sortiert.

Im Folgenden stellen wir mit Ternary Quicksort eine weitere Variante des Quicksort vor.
Anschließend beschreiben wir wie möglichst effizient das Pivotelement gewonnen werden kann.

\begin{listing}[!h]
    \begin{minted}{python}
    def bentley_mcilroy_tq(A: Array<SuffixStartPos>):
      if |A| > 1:
        p := choose a pivot element
    
        # Rearrange A so, that every element in A[0, left) is
        # smaller than p, every element in A[left, right) is equal
        # to p and every element in A[right, |A|) is greater than p.
        left, right := partition(A, p)
    
        bentley_mcilroy_tq(A[0, left))
        bentley_mcilroy_tq(A[right, |A|))
    \end{minted}
    \caption{Bentley-McIlroy ternary Quicksort~\cite{ternary_quicksort}.}
\end{listing}    

\subsubsection{Ternary Quicksort}
\label{section:ternary_quicksort}

\noindent
Ternary Quicksort ist eine Variante des Quicksort, die besonders geeignet ist,
wenn in der Eingabe mehrere gleiche Elemente vorkommen~\cite{ternary_quicksort}.
Wir wählen hier wieder ein geeignetes Pivotelement $p$.
Statt die Eingabe in zwei Bereiche aufzuteilen, teilen wir die Eingabe in drei Bereiche $\mathsf{A}_{\text{left}}$, $\mathsf{A}_{\text{middle}}$ und $\mathsf{A}_{\text{right}}$ auf (siehe \cref{fg:ternary_partitions}).
In $\mathsf{A}_{\text{left}}$ sind alle Elemente enthalten, die echt kleiner sind als $p$.
In $\mathsf{A}_{\text{middle}}$ sind alle Elemente enthalten, die gleich $p$ sind, und in $\mathsf{A}_{\text{right}}$ sind alle Elemente enthalten, die größer als $p$ sind.
Die Größe dieser Bereiche lässt sich durch Zählen der Elemente bestimmen.
Anschließend können die Elemente durch Vertauschen in den richtigen Bereich geschrieben werden.

Nachdem wir alle Elemente in die drei Bereiche aufgeteilt haben, müssen $\mathsf{A}_{\text{left}}$ und $\mathsf{A}_{\text{right}}$ rekursiv sortiert werden.

\begin{figure}[!h]
	\centering
	\begin{tikzpicture}
	\draw (0,0) -- (10,0) -- (10,1) -- (0,1) -- cycle;
	\draw (3,0) -- (3,1);
	\draw (8,0) -- (8,1);
	
	\node at (1.5, 0.5) {<};
	\node at (5.5, 0.5) {=};
	\node at (9, 0.5) {>};
	
	\draw[red] (6,0.5) -- (6,2) node[above] {\texttt{p}};
	
	\begin{scope}[yshift=-10pt]
	\draw[decorate, decoration={mirror, brace,amplitude=10pt}] (0.1,0) -- (2.9,0) node[midway, below = 10pt] {$A_\text{left}$};
	\draw[decorate, decoration={mirror, brace,amplitude=10pt}] (3.1,0) -- (7.9,0) node[midway, below = 10pt] {$A_\text{middle}$};
	\draw[decorate, decoration={mirror, brace,amplitude=10pt}] (8.1,0) -- (9.9,0) node[midway, below = 10pt] {$A_\text{right}$};
	\end{scope}
	\end{tikzpicture}
	\caption[Partitionen bei ternary Quicksort]{Partitionen bei ternary Quicksort. (Abbildung ähnlich in \cite{ternary_quicksort}.)}
	\label{fg:ternary_partitions}
\end{figure}

\subsubsection{Auswahl des Pivotelements}

Es gibt unterschiedliche Ansätze, um in den hier beschriebenen Quicksort-Varianten das Pivotelement zu wählen.
Das Problem ist der Trade-Off zwischen aufwändiger Pivot-Wahl (Pivotwahl hat hohe Laufzeit) und
degeneriertem Quicksort durch eine schlechte Wahl des Pivotelements (Quicksort hat hohe Laufzeit).

Einfache Ansätze sind es das erste, mittlere oder das letzte Element der Eingabe zu wählen.
Diese Ansätze sind problematisch, wenn das gewählte Pivotelement ein sehr kleines oder ein sehr großes Element ist.
Dadurch wird die Eingabe nicht in zwei annähernd gleich große Bereiche aufgeteilt und die rekursiven Aufrufe werden ineffizient (degeneriertes Quicksort).

Daher verwenden wir im Quicksort und im Ternary Quicksort eine verbesserte Variante~\cite{ternary_quicksort}.
Falls die Größe des Eingabearrays kleiner als ein bestimmter Threshold ist,
bestimmen wir den \emph{Median of 3}. Das bedeutet, dass wir uns das erste,
das mittlere und das letzte Element der Eingabe anschauen und den Median von diesen drei Elementen wählen.
In der Implementierung haben wir als Threshold $40$ gewählt.
Falls die Größe des Eingabearrays größer als als der gewählte Threshold ist,
bestimmen wir den \emph{Median of 9}. Dazu teilen wir das Array in drei gleich große Teilarrays auf
und bestimmen für jedes dieser Teilarrays den \emph{Median of 3}.
Unter diesen drei Elementen wird wiederum der Median gebildet, um das Pivotelement zu erhalten.