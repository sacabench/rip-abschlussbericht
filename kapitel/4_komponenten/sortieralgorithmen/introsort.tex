\subsection{Introsort}
\label{section:introsort}

Nicht \currentauthor{Oliver Magiera} immer kann eine Laufzeit von $\mathcal O(n\log{n})$ beim Quicksort garantiert werden.
Für diesen Zweck wurde der Introsort Algorithmus (auf Basis des binären Quicksort) entworfen \cite{Musser97}:
nach einer Rekursionstiefe $h$ durch den Quicksort Algorithmus wird Heapsort verwendet,
um ein Teilarray \textbf{A'} von \textbf{A} endgültig zu sortieren.
Als Wert für die maximale Tiefe $h$ hat sich $2 \lfloor \log_2{|A|}\rfloor$ als empirisch geeignet herausgestellt.
Eine weitere Optimierung liegt in der Verwendung von Insertionsort,
wenn die Anzahl der Elemente einen Schwellwert $\theta$ unterschreitet.

\begin{figure}
\begin{minted}[escapeinside=||, linenos=true, autogobble]{python}
def introsort(A: Array<Integer>, f: Integer, b: Integer):
  introsort_loop(A, f, b, |$2 \lfloor \log_2{(b-f)}\rfloor$|)
  insertionsort(A, f, b)
\end{minted}

\begin{minted}[escapeinside=||, mathescape=true, linenos=true, autogobble]{python}
def introsort_loop(A: Array<Integer>, f: Integer,
                   b: Integer, depth_limit: Integer)
  while b - f > |$\theta$|
    if depth_limit == 0
      heapsort(A, f, b)
      return end
    depth_limit := depth_limit - 1
    p := Choose pivot element

    # Rearrange A such that all elements $\leq$ p are in $A[f\dots i]$
    # and all elements > p in A$[i\dots b]$
    i := partition(A, f, b, p)
    # Recursively sort >-Partition
    introsort_loop(A, i, b, depth_limit)
    # Sort left partition ($\leq$) in next iteration
    b := i

\end{minted}
\caption{Introsort mit \glqq $\leq$\grqq- und \glqq >\grqq-Partition.}
\label{alg:introsort}
\end{figure}

Der Pseudocode in \ref{alg:introsort} beschreibt den Ablauf des Algorithmus~\cite{Musser97}.
Dabei befindet sich in \texttt{introsort\_loop} die eigentliche Funktion: In Zeile 3 wird geprüft,
ob die Größe des (Teil-)Arrays den Schwellwert unterschreitet. Ist dies der Fall, bricht die Rekursion ab.
Andernfalls muss geprüft werden, ob die maximale Tiefe bereits erreicht wurde.
Falls ja, wird das (Teil-)Array vollständig mit Heapsort sortiert.
Tritt dies ebenfalls nicht ein, wird die verbleibende Rekursionstiefe um eins reduziert,
das Pivotelement bestimmt und das Array anhand dessen partitioniert,
d.h. alle größeren Elemente landen in der rechten Partition, alle Anderen in der Linken.
Zuletzt wird rekursiv \texttt{introsort\_loop} auf die rechte Partition ausgeführt
und danach die rechte Grenze $b$ auf die linke Partition angepasst, welche iterativ weiter sortiert wird.

Die Methode \texttt{introsort} ruft \texttt{introsort\_loop} mit dem übergebenen Anfang $f$ und Ende $b$ sowie
mit der empirisch geeigneten Maximaltiefe aufgerufen. Daraufhin wird Insertionsort auf das übergebene Array ausgeführt.
Durch die vorige Vorverarbeitung über die innere Introsort-Schleife müssen nur noch Intervalle
der Länge $\leq \theta$ final sortiert werden.
Dies funktioniert deshalb effizient, da z.~B. Elemente im Intervall $[0, \theta)$ noch nicht
endgültig sortiert sind, sie befinden sich jedoch bereits im richtigen Intervall.
Somit wird kein Element aus $[\theta, 2 \theta)$ oder einem hinteren Intervall in \textbf{A} kleiner
sein als ein beliebiges Element aus dem Intervall $[0,\theta)$.
