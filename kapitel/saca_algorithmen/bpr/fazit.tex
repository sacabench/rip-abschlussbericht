\subsection{Fazit}
\label{bpr:fazit}

Wie wir gesehen haben, handelt es ich bei \bpr um einen Konstruktionsalgorithmus für Suffixarrays, der sich aufgrund seiner einfachen Struktur unkompliziert programmieren lässt. Darüber hinaus erzielt der Algorithmus besonders im Anwendungsgebiet der Bioinformatik gute Laufzeiten und eignet sich daher auch für den praktischen Einsatz.\par
Bei der Implementierung hat sich gezeigt, dass sich die Laufzeit des Algorithmus insbesondere durch die Verwendung von IPS\(^4\)o noch weiter beschleunigen lässt. Für die Umstrukturierung der Sortierphasen ist noch zu untersuchen, ob sich Fälle konstruieren lassen, in denen die geänderte Reihenfolge Nachteile gegenüber der originalen Version hat. Bisher konnten weder für kleine Eingaben, noch für große Dateien nennenswerte Unterschiede gemessen werden.\par\smallskip
Auf theoretischer Seite bleibt weiterhin von Interesse, ob die von Schürmann und Stoye beschriebene asymptotische Schranke von \(\O(n^2)\) \cite[Kapitel~5]{schuermann2005} auch für \(d < \log n\) eingehalten bzw. sogar noch verfeinert werden kann und welche Auswirkungen dies insbesondere auf den Speicherbedarf in Phase 1 des Algorithmus hat. Sollte die Schranke gültig sein, so ist zu analysieren, und ob für allgemeine Eingaben eventuell sogar eine kleinere obere Schranke existiert. Im Zuge dessen kann es hilfreich sein, komplexere Beispieleingaben zu konstruieren, die die Laufzeit des Verfahrens sowohl in der Rekursionstiefe als auch im Sortieraufwand maximieren.
