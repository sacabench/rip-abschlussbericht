\section{DC3}
\label{algorithm:dc3}

Im \currentauthor{Johannes Bahne} Jahre 2003 veröffentlichten Juha Kärkkäinen und Peter Sanders einen der ersten Algorithmen, der ein Suffix-Array in linearer Laufzeit $\mathcal{O}(n)$ erstellen kann \cite{saca:9}. Dieser Algorithmus heißt \emph{DC3}, wird aber auch \emph{skew} genannt. Der \emph{DC3}-Algorithmus wird in den meisten wissenschaftlichen Veröffentlichungen gerne als Vergleichsalgorithmus angesehen, sodass dieser auch im Rahmen der Projektgruppe \sacabench nicht fehlen darf. Außerdem dient der \emph{DC3} als Grundlage für den Algorithmus \emph{nzSufSort}, der im Kapitel \ref{algorithm:nzSufSort} näher erläutert wird.

\subsection{Einführung}
\label{dc3:einfuehrung}

Der \emph{DC3}-Algorithmus wird in der wissenschaftlichen Ausarbeitung von Juha Kärkkäinen und Peter Sanders wiederholt als simpler linearer Algorithmus zur Konstruktion eines Suffix-Arrays dargestellt. Der Wortlaut \emph{simpel} bezieht sich jedoch auf die Implementierung des Algorithmus. Diese benötigt nämlich le\-dig\-lich 50 Zeilen Code in der Programmiersprache \emph{C++}. Die Theorie, die sich dabei im Hintergrund abspielt, ist jedoch nicht ganz trivial. Daher werden wir uns mit der Theorie näher auseinandersetzen und diese anhand von Beispielen besser verdeutlichen. Dabei gehen wir am Ende auch auf Erweiterungen und Ver\-bes\-se\-rungs\-vor\-schlä\-ge ein und bewerten diese anhand von Laufzeit- und Speicherplatz-Messungen.

\subsection{Vorüberlegungen}
\label{dc3:vorueberlegungen}

\newtheorem{example}{Beispiel}
Der \emph{DC3}-Algorithmus baut auf dem \emph{Divide-and-Conquer} Prinzip auf. Diese Art von Algorithmen teilen das Hauptproblem rekursiv in kleinere Teilprobleme auf, bis diese leichter zu lösen sind. Im Anschluss werden die Teilprobleme zu einer Gesamtlösung zusammengefügt.

Das zweite Prinzip, das sich hinter dem \emph{DC3}-Algorithmus verbirgt, ist das sogenannte \emph{Difference Cover}. Diesem Prinzip verdankt der Algorithmus auch seinen Namen.

\begin{definition}[\emph{Difference Cover}]
	\label{def:differenceCover}
	Eine Menge \(D\) $\subseteq$ \([0,v)\) ist ein Difference Cover modulo \(v\), wenn die Werte in \([0,v)\) als Differenz zweier Werte aus \(D\) $\subseteq$ \([0,v)\) ausgedrückt werden können. Mathematisch dargestellt:\\ \\
	$\makebox[\linewidth]{\{(i - j) \text{modulo} \(v\) $\mid$ $i,j$ $\in$ \(D\)\} $=$ \([0,v)\)}$\\
			
\end{definition}

\begin{example}
	Ein Beispiel für das Difference Cover \emph{modulo} \(7\): \(D\) $=$ \{1, 2, 4\} 
	\begin{table}[!htbp]
		\centering
		\begin{tabular}{lllll}
			1 - 1 = 0 &  & 1 - 4 = -3 $\equiv$ 4 &  & (\(\mod\) \(7\) )\\
			2 - 1 = 1 &  & 2 - 4 = -2 $\equiv$ 5 &  & (\(\mod\) \(7\) )\\
			4 - 2 = 2 &  & 1 - 2 = -1 $\equiv$ 6 &  & (\(\mod\) \(7\) )\\
			4 - 1 = 3 &  &                       &  & 
		\end{tabular}
	\end{table}
	
	Das bedeutet, dass wir mit der Menge \(D\) $=$ \{1, 2, 4\} die Zahlen aus \([0,7)\) mit der Hilfe von Differenzen zweier Werte aus \(D\) ausdrücken können.
	
	Ein Beispiel für das Difference Cover \emph{modulo} \(3\): \(D\) $=$ \{1, 2\} 
\end{example}

\subsection{Algorithmus}
\label{dc3:algorithmus}

In dem vorherigen Kapitel \ref{dc3:vorueberlegungen} ist die Definition des \emph{Difference-Cover}-Prinzips näher erläutert worden. Nun wird beschrieben, wie dieses Verfahren eingesetzt werden kann, um ein Suffix-Array zu erhalten - wobei wir uns in diesem Kapitel nur mit dem \emph{Difference-Cover} modulo $3$ beschäftigen, da dieser auch in der wissenschaftlichen Ausarbeitung von Juha Kärkkäinen und Peter Sanders behandelt ist. Dieser wird dabei in drei Phasen aufgeteilt. Allgemein ausgedrückt werden in der ersten Phase die Suffixe mit Startpositionen aus dem \emph{Difference-Cover} \(D\) $=$ \{1, 2\} sortiert. In der zweiten Phase werden die restlichen Suffixe unter Verwendung des Ergebnisses des vorherigen Schrittes sortiert. Und - wie im \emph{Divide-and-Conquer} Prinzip üblich - werden in der dritten Phase die beiden sortierten Lösungen aus der ersten und zweiten Phase zusammengeführt, sodass am Ende eine sortierte Menge aller Suffixe des Ausgangstextes entsteht - das Suffix-Array.


\subsubsection{Erste Phase - Sortierung des \emph{Difference Covers}}
\label{dc3:algorithmus:phase1}

In dem ersten Schritt des Algorithmus werden die Substrings $\inputtext[i, i+2]$ sortiert, wobei $i$ ein Element aus dem \emph{Difference Cover} $D = \{1, 2\}$ ist. Das bedeutet, es werden alle Substrings der Länge drei - auch Triplets genannt - startend in den Positionen $i \text{ modulo } 3 = 1$, also $i = 1, 4, 7, 10,...$, und den Positionen $i \text{ modulo } 3 = 2$, also $i = 2, 5, 8, 11,...$, des Ausgangstextes \inputtext aufsteigend sortiert.
Anschließend werden den Triplets der Reihenfolge nach lexikographische Namen $t_i \in [0,\lceil2n/3\rceil)$ mit der Eigenschaft, dass $t_i < t_j$ wenn $\inputtext[i, i+2] < \inputtext[j, j+2]$ und $t_i = t_j$ wenn $\inputtext[i, i+2] = \inputtext[j, j+2]$, zugewiesen. Das bedeutet, dass das kleinste Triplet den lexikographisch kleinsten Namen $0$ erhält, das zweitkleinste Triplet die $1$ und so weiter. Sind mehrere Triplets gleich, das heißt, sowohl die erste, zweite und dritte Stelle des Substrings sind jeweils gleich, dann erhalten diese Triplets die gleichen lexikographischen Namen.
Wenn jedem Triplet ein lexikographischer Name zugeordnet worden ist, wird überprüft, ob diese Namen eindeutig sind. Wenn in dem Ausgangstext keine gleichen Triplets mit den beschriebenen Eigenschaften vorkommen, sind wir mit dem ersten Schritt des \emph{DC3}-Algorithmus fertig und können mit der zweiten Phase fortfahren. Kommen in dem Ausgangstext \inputtext jedoch gleiche Triplets startend in den Positionen $i \text{ modulo } 3 \neq 0$ vor, so können wir aktuell noch keine eindeutige Aussage über die Reihenfolge der gleichen Suffixe treffen. Um diese eindeutige Aussage treffen zu können, schreiben wir die lexikographischen Namen so um, dass die Ordnung der Triplets beibehalten wird. Dabei werden die Namen $t_i$ in die Mengen $i \text{ modulo } 3 = 1$ und $i \text{ modulo } 3 = 2$ aufgeteilt und diese zu einem neuen String $\mathsf{T}_{12}$ konkateniert. Mathematisch dargestellt, bedeutet das, dass der String
\begin{center}
	$\mathsf{T}_{12} = [t_i \mid i \text{ modulo } 3 = 1] \circ [t_i \mid i \text{ modulo } 3 = 2]$ 
\end{center}
als neuer Ausgangstext angesehen und der \emph{DC3}-Algorithmus auf diesem String ausgeführt wird, sodass wir nach erneutem Ausführen des Algorithmus die endgültige Sortierung der Triplets erhalten und mit dem zweiten Schritt fortfahren können.

Es gibt jedoch einen Spezialfall, bei dem dieser Schritt zu einem falschem Suffix-Array führen kann. Der Spezialfall tritt auf, wenn folgende drei Punkte gleichzeitig zutreffen:

$\begin{array}{ll}
1. & \text{ die Textlänge ist } n \text{ modulo } 3 = 1\\ 
2. & \text{ das Triplet an der Position } n - 3 \text{ kommt in dem Text mehrmals vor}\\ 
3. & \text{ das Triplet an der Position } n - 2 \text{ ist nicht das kleinste Triplet}
\end{array}$

Denn dann ist der String mit den lexikographischen Namen falsch, weil der Algorithmus nicht weiß, dass mit dem Triplet an der Position $n - 3$ das Ende des Textes erreicht ist.
Der Spezialfall tritt zum Beispiel bei dem Text $\inputtext = aabcabc$ auf. Der String mit den lexikographischen Namen wäre $t_{12} = [1, 1] \circ [3, 2]$, also $\mathsf{T}_{12} = 1132$. Wird der String in dieser Form rekursiv aufgerufen, wird am Ende des Algorithmus das Suffix $\inputtext$$[1, n)$ kleiner sein als das Suffix $\inputtext$$[4, n)$, weil das Triplet $113$ in der Rekursion kleiner ist als $132$. Um dagegen vorzubeugen, wird vor Beginn des Algorithmus dem Ausgangstext ein sogenanntes Dummy-Triplet angehangen, das aus Sentinels besteht. Dadurch wird dafür gesorgt, dass das Ende des Textes mit in die Sortierung eingeht.
\subsubsection{Zweite Phase - Induzierung}
\label{dc3:algorithmus:phase2}

Wir haben nun in der ersten Phase die eindeutige Reihenfolge der Suffixe aus dem \emph{Difference Cover} bestimmt. Jetzt können wir das Prinzip des \emph{Difference Covers} anwenden und die Reihenfolge der restlichen Suffixe beginnend in Position $i \text{ modulo } 3 = 0$ induzieren. Dafür benötigen wir die Ränge der Suffixe aus dem \emph{Difference Cover}, die sich aus den lexikographischen Namen der ersten Phase ableiten lassen. Also berechnen wir das inverse Suffix-Array $\isa_{12}$, das die jeweiligen Ränge repräsentiert.
\begin{center}
	$\sa_{12}[i] = j$ genau dann, wenn $\isa_{12}[j] = i$
\end{center}
Jetzt lassen sich Paare aufstellen, die sich aus einem Zeichen $\inputtext[i]$ des Aus\-gangs\-text\-es und dem Rang des darauffolgenden Suffixes zusammensetzen, wobei $i \text{ modulo } 3 = 0$ ist. Nachdem alle Paare aufgestellt sind, können diese ebenfalls aufsteigend sortiert werden und anschließend die jeweiligen Positionen $i$ der Paare in ein Array $\sa_0$ abgespeichert werden. Somit haben wir auch die Suffixe aufsteigend sortiert, die nicht in dem \emph{Difference Cover} sind, und können zur dritten Phase übergehen.

Dieses Prinzip funktioniert, da die Suffixe startend in Position $i \text{ modulo } 3 = 0$ nur ein Zeichen am Anfang des Suffixes mehr aufweisen als die Suffixe beginnend in $i \text{ modulo } 3 = 1$ und der Rest gleich ist. Und die Suffixe in $i \text{ modulo } 3 = 1$ sind bereits in der ersten Phase eindeutig sortiert worden. Dieses hilft uns in der zweiten Phase weiter, denn dann reicht es aus, die jeweiligen Zeichen in $i \text{ modulo } 3 = 0$ zu vergleichen und bei Gleichheit die Ränge der Suffixe einer Position hinter den jeweiligen Zeichen anzuschauen. Dadurch können zwar Paare das gleiche Zeichen $\inputtext[i]$ aber niemals den gleichen Rang aufweisen und das führt zu einer eindeutigen Sortierung der Suffixe startend in Position $i \text{ modulo } 3 = 0$.
\subsubsection{Dritte Phase - Merge}
\label{dc3:algorithmus:phase3}

\crefname{enumi}{Fall}{Fällen}

Wir haben aus den ersten beiden Phasen die jeweils sortierten Suffixe des \emph{Difference Covers} $\sa_{12}$ und diejenigen, die nicht in dem \emph{Difference Covers} $\sa_{0}$ sind, vorliegen. Im letzten Schritt müssen diese beiden Mengen vereinigt werden, um an das Suffix-Array \sa zu gelangen. Dabei nutzen wir aus, dass die jeweiligen Mengen bereits sortiert sind. Somit vergleichen wir immer das kleinste Suffix aus der Menge $\sa_{12}$ mit dem kleinsten aus der Menge $\sa_{0}$. Dabei können bei dem Vergleich von $\sa_{12}[i]$ und $\sa_{0}[j]$ folgende vier Fälle auftreten.
\begin{enumerate}
	\item $\sa_{12}[i] \text{ modulo } 3 = 1$ \label{option1}
	\item $\sa_{12}[i] \text{ modulo } 3 = 2$ \label{option2}
	\item $i$ > length($\sa_{12}$) \label{option3}
	\item $j$ > length($\sa_{0}$) \label{option4}
\end{enumerate}

Tritt \cref{option1} ein, bedeutet das, dass ein Suffix aus $\sa_{0}$ mit einem Suffix aus $\sa_{12}$, dessen Startposition $\sa_{12}[i] \text{ modulo } 3 = 1$ ist, miteinander verglichen wird. Für diesen Vergleich benötigen wir die Paare $(\inputtext[\sa_{12}[i]], \isa_{12}[\sa_{12}[i]+1])$ und $(\inputtext[\sa_{0}[j]], \isa_{12}[\sa_{0}[j]+1])$. Wenn wir diese zwei Paare ermittelt haben, lassen sie sich miteinander vergleichen. Der kleinere Index von beiden wird dann dem Suffix-Array \sa hinzugefügt und der andere Index wird mit dem nächsten verglichen. Bei \cref{option1} wird ein Zeichen und der Rang des darauffolgenden Suffix miteinander verglichen. Dies funktioniert, da sowohl der Rang der Position nach $i \text{ modulo } 3 = 0$ als auch der Rang des Suffixes nach $i \text{ modulo } 3 = 1$ bekannt ist. \\
Tritt \cref{option2} ein, bedeutet das, dass ein Suffix aus $\sa_{0}$ mit einem Suffix aus $\sa_{12}$, dessen Startposition $\sa_{12}[i] \text{ modulo } 3 = 2$ ist, miteinander verglichen wird. Für diesen Vergleich stellen wir - anders als in \cref{option1} - die Triplets $(\inputtext[\sa_{12}[i]]$, $\inputtext[\sa_{12}[i]+1]$, $\isa_{12}[\sa_{12}[i]+2])$ und $(\inputtext[\sa_{0}[j]]$, $\inputtext[\sa_{0}[j]+1]$, $\isa_{12}[\sa_{0}[j]+2])$ auf. Wie zuvor werden diese beiden Triplets nun wieder verglichen und der Index des kleineren Triplets dem Suffix-Array angehangen. Bei \cref{option2} werden zwei Zeichen und der Rang des Suffix, das zwei Zeichen später beginnt, miteinander verglichen. Dies funktioniert, da sowohl der Rang des Suffixes zwei Positionen nach $j \text{ modulo } 3 = 0$ als auch der Rang zwei Positionen nach $i \text{ modulo } 3 = 2$ bekannt ist. Ein Zeichen und der darauffolgende Rang - wie in \cref{option1} - reicht für einen Vergleich nicht aus, da der Rang nach $i \text{ modulo } 3 = 2$ eine Position startend in $i+1 \text{ modulo } 3 = 0$ ist und somit nicht miteinander vergleichbar ist.\\
Bei den \cref{option3,option4} sind eines der beiden Mengen $\sa_{0}$ oder $\sa_{12}$ abgearbeitet und der Rest des Suffix-Arrays kann mit den jeweils restlichen Indizes der übrig gebliebenen Menge aufgefüllt werden.\\
Sind beide Mengen durchlaufen worden, haben wir am Ende ein vollständiges Suffix-Array und der Algorithmus ist terminiert.



\subsubsection{Laufzeit}
\label{dc3:algorithmus:laufzeit}

In diesem Kapitel untersuchen wir die \emph{Worst-Case}-Laufzeit des \emph{DC3}-Al\-go\-rith\-mus, indem wir zuerst die Laufzeiten der einzelnen Phasen betrachten und diese am Ende zu einer Gesamtlaufzeit zusammenführen.
In der ersten Phase werden Triplets aufgestellt und diese anschließend sortiert. Für die Sortierung könnte zum Beispiel das Sortierverfahren \emph{Radix-Sort} verwendet werden. Der \emph{LSD-Radix-Sort}, der im Kapitel \ref{sort:radix:lsd} näher erläutert worden ist, benötigt eine Laufzeit von $\mathcal{O}(kn)$, wobei $k$ für die Länge der zu ver\-gleich\-en\-den Werte steht. In unserem Fall ist $k = 3$. Für die Vergabe der lexikographischen Namen wird ebenfalls eine lineare Laufzeit benötigt. Zusammengefasst benötigt die erste Phase eine Laufzeit von $T(n) = \mathcal{O}(n)$.

Die zweite Phase stellt Paare auf, welche erneut mit Hilfe von \emph{Radix-Sort} sortiert werden. Dies benötigt eine Laufzeit von $T(n) = \mathcal{O}(n)$.

In der dritten Phase werden beide Mengen $SA_{0}$ und $SA_{12}$ jeweils einmal durchlaufen und einfache Vergleiche ausgeführt. Dies führt ebenfalls zu einer Laufzeit von $T(n) = \mathcal{O}(n)$.

Also weist jede Phase eine Laufzeit von $\mathcal{O}(n)$ auf. Im \emph{Worst-Case} sind die lexikographischen Namen aus der ersten Phase jedoch nicht eindeutig, sodass ein rekursiver Aufruf erfolgt. Dabei wird der Algorithmus mit einer Textlänge von $2n/3$ aufgerufen. Zusammengefasst erhalten wir somit für den gesamten Algorithmus eine Laufzeit von $T(n) = \mathcal{O}(n) + T(2n/3)$. Da der Faktor $2/3$ kleiner als $1$ ist, lässt sich diese Rekursionsgleichung zu $T(n) = \mathcal{O}(n)$ auflösen. Somit ist die Laufzeit des \emph{DC3}-Algorithmus linear.
\subsubsection{Beispiel}
\label{dc3:algorithmus:beispiel}

\crefname{enumi}{Fall}{Fälle}

In diesem Kapitel wollen wir die Theorie des \emph{DC3}-Al\-go\-rith\-mus anhand eines Beispieles verdeutlichen. Dazu wenden wir den Al\-go\-rith\-mus auf den String \inputtext = caabaccaabacaa\$ an, um das gesuchte Suffix-Array zu erhalten.

\begin{table}[H]
	\centering
	\begin{tabular}{c| c c c c c c c c c c c c c c c}
		$i$ & 0 & 1 & 2 & 3 & 4 & 5 & 6 & 7 & 8 & 9 & 10 & 11 & 12 & 13 & 14 \\
		$\inputtext[i]$ & c & a & a & b & a & c & c & a & a & b & a & c & a & a & \$
	\end{tabular}
\end{table}

Bei der ersten Phase werden alle Triplets beginnend in Positionen\\$i \text{ modulo } 3 = 1$ und $i \text{ modulo } 3 = 2$ aufgestellt, wie in \cref{tab:unsortierteTriplets} zu erkennen.

\begin{table}[H]
	\centering
	\begin{tabular}{c| c c c c c ||  c c c c c }
		$i$ & 1 & 4 & 7 & 10 & 13 & 2 & 5 & 8 & 11 & 14\\
		$\inputtext[i, i+2]$ & aab & acc & aab & aca & a\$\$ & aba & cca & aba & caa & \$\$\$
	\end{tabular}
	\caption{Triplets $i \text{ modulo } 3 \neq 0$}
	\label{tab:unsortierteTriplets}
\end{table}

Dabei fällt auf, dass die Triplets an der Stelle $i = 13$ und $i = 14$ mit Sentinels aufgefüllt werden, damit sie besser mit den anderen zu vergleichen sind. Nun können wir diese Triplets zum Beispiel mit Hilfe des Sortieralgorithmus \emph{Radix-Sort} aufsteigend sortieren. Das Ergebnis ist in der \cref{tab:sortierteTriplets} festgehalten.

\begin{table}[H]
	\centering
	\begin{tabular}{c| c c c c c c c c c c}
		$i$ & 14 & 13 & 1 & 7 & 2 & 8 & 10 & 4 & 11 & 5\\
		$T[i, i+2]$ & \$\$\$ & a\$\$ & aab & aab & aba & aba & aca & acc & caa & cca 
	\end{tabular}
	\caption{sortierte Triplets $i \text{ modulo } 3 \neq 0$}
	\label{tab:sortierteTriplets}
\end{table}

Nun können wir die lexikographischen Namen vergeben. Dafür werden die Positionen aufsteigend nummeriert. Falls mehrere Triplets gleich sind, erhalten sie den gleichen lexikographischen Namen. Anschließend werden diese lexikographischen Namen so umsortiert, dass wir die Namen für die Positionen $i \text{ modulo } 3 = 1$ und die Positionen $i \text{ modulo } 3 = 2$ konkatenieren können. Das Ergebnis dieser zwei Schritte ist in der \cref{tab:lexNamen} zu finden.

\begin{table}[H]
	\centering
	\begin{tabular}{c| c c c c c c c c c c}
		$i$ & 1 & 4 & 7 & 10 & 13 & 2 & 5 & 8 & 11 & 14\\
		$t_{12}$ & 2 & 5 & 2 & 4 & 1 & 3 & 7 & 3 & 6 & 0
	\end{tabular}
	\caption{Vergabe der lexikographischen Namen}
	\label{tab:lexNamen}
\end{table}

Hier ist zu erkennen, dass mehrere Triplets den gleichen lexikographischen Namen erhalten haben. Somit haben wir noch nicht die endgültige Reihenfolge ermitteln können. Dafür rufen wir den Algorithmus rekursiv mit diesen lexikographischen Namen $\mathsf{T}_{12} = 2524137360$ auf. Als Ergebnis der Rekursion erhalten wir $\sa_{12} = [9, 4, 2, 0, 7, 5, 3, 1, 8, 6]$. Das Ergebnis lässt sich so deuten, dass sich das kleinste Triplet an Stelle 9 von $\mathsf{T}_{12}$ befindet, das der Position $i = 14$ aus dem Ausgangstext \inputtext entspricht, also $\sa_{12} = [14, 13, 7, 1, 8, 2, 10, 4, 11, 5]$. 

Damit ist die erste Phase des Algorithmus abgeschlossen und wir gehen über zur zweiten Phase. Dafür werden zunächst die Ränge bestimmt mit Hilfe des inversen Suffix-Arrays, das sich wie folgt berechnen lässt: $\sa_{12}[i] = j$ genau dann, wenn $\isa_{12}[j] = i$.

\begin{table}[H]
	\centering
	\begin{tabular}{c| c c c c c c c c c c}
		$i$ & 1 & 4 & 7 & 10 & 13 & 2 & 5 & 8 & 11 & 14 \\
		$\isa_{12}$ & 4 & 8 & 3 & 7 & 2 & 6 & 10 & 5 & 9 & 1
	\end{tabular}
	\caption{Ränge der Suffixe $i \text{ modulo } 3 \neq 0$}
	\label{tab:ergebnis_rek}
\end{table}

Im Anschluss lassen sich die Suffixe $i \text{ modulo } 3 = 0$ mit Hilfe des Aus\-gangs\-text\-es \inputtext und der Ränge $\isa_{12}$ sortieren. An diesem Beispiel kann man die Theorie gut verdeutlichen. Denn anstatt sich - wie in der ersten Phase - erneut Triplets anzuschauen und möglicherweise vor dem Problem zu stehen, gleiche Triplets zu erhalten, wird die eindeutige Sortierung aus der ersten Phase genutzt. Schaut man sich das Zeichen 'c' an der ersten und siebten Position von \inputtext an, muss man sich nur die Ränge der Suffixe an der jeweils nächsten Position anschauen. Daraus lässt sich erschließen, dass das Suffix an Position $i = 6$ kleiner als das Suffix an Position $i = 0$ ist, weil der Rang des Suffixes an $i = 7$ kleiner ist als das an $i = 1$.

\begin{table}[H]
	\centering
	\begin{tabular}{c| c c c c c}
		$i$ & 12 & 9 & 3 & 6 & 0 \\
		$(\inputtext[i], \isa_{12}[i+1])$  & (a, 2) & (b, 7) & (b, 8) & (c, 3) &  (c, 4)
	\end{tabular}
	\caption{Sortierte Paare $(\inputtext[i], \isa_{12}[i+1]), i \text{ modulo } 3 = 0$}
	\label{tab:SA01}
\end{table}

Als Ergebnis erhalten wir $\sa_{0} = [12, 9, 3, 6, 0]$ - wie in der \cref{tab:SA01} nachzuvollziehen ist.

Nun können wir in der dritten Phase die bereits sortierten Mengen $\sa_{0}$ und $\sa_{12}$ mergen. Dafür stellen wir - wie im vorherigen Kapitel \ref{dc3:algorithmus:beispiel} beschrieben - die benötigten Paare und Triplets auf, je nachdem, welcher der \crefrange{option1}{option4} zutrifft. Als erstes vergleichen wir die beiden Suffixe beginnend in $\sa_{12}[i = 0] = 14$ und $\sa_{0}[j = 0] = 12$. Da $14$ modulo $3 = 2$ ist - also \cref{option2} zutrifft -, stellen wir die Triplets $(\$,\$,0)$ und $(a,a,1)$ auf. Der Vergleich ergibt, dass das Suffix startend in $\sa_{12}[i = 0] = 14$ kleiner ist. Somit wird dem Suffix-Array \sa der Index $14$ hinzugefügt und als nächstes $\sa_{12}[i = 1] = 13$ mit $\sa_{0}[j = 0] = 12$ verglichen. $13$ modulo $3$ ergibt $1$. Das heißt, der \cref{option1} ist eingetreten und wir vergleichen die Paare $(a,1)$ und $(a, 2)$. So werden beide Mengen einmal durchlaufen und wir erhalten am Ende das endgültige Suffix-Array $\sa = [14, 13, 12, 7, 1, 8, 2, 10, 4, 9, 3, 11, 6, 0, 5]$.
\subsection{Erweiterungen}
\label{dc3:erweiterung}


\subsubsection{DC7}
\label{dc7}

Wie in \cref{dc3:vorueberlegungen}, ist nicht nur ein Difference Cover modulo $3$ möglich, sondern auch zum Beispiel modulo $7, 13, 21$ und $31$. Im Rahmen der Projektgruppe \emph{SACABench} wurde zusätzlich zum \emph{DC3} auch der \emph{DC7} implementiert. Dafür werden jedoch ein paar Änderungen an dem Algorithmus vorgenommen, die der wissenschaftlichen Arbeit von Juha Kärkkäinen, Peter Sanders und Stefan Burkhardt entnommen worden ist \cite{dc3:new}. Für den \emph{DC7}-Algorithmus verwenden wir das Difference-Cover \(D\) $=$ \{1, 2, 4\}.

\subparagraph*{Erste Phase}

Die erste Phase ist analog zu der ersten Phase des \emph{DC3}-Algorithmus. Hier werden die Septets $\inputtext[i,i+6]$ an den Positionen $i \text{ modulo } 7 \in \{1, 2, 4\}$ aufsteigend sortiert. Anschließend werden lexikographische Namen vergeben und diese so umsortiert, dass die Ordnung der Septets beibehalten wird.
\begin{center}
	$\mathsf{T}_{124} = [t_i : i \text{ modulo } 7 = 1] \circ [t_i : i \text{ modulo } 7 = 2] \circ [t_i : i \text{ modulo } 7 = 4]$ 
\end{center}
Wenn die lexikographischen Namen eindeutig sind, kann mit der zweiten Phase fortgefahren werden, ansonsten wird der \emph{DC7}-Algorithmus erneut mit dem String $\mathsf{T}_{124}$ der Länge $\mathcal{O}(3n/7)$ ausgeführt.

\subparagraph*{Zweite Phase}

In der zweiten Phase werden die Suffixe an den Positionen $i \text{ modulo } 7 \notin \{1, 2, 4\}$ sortiert, indem die Tupel $(\inputtext[i, i+5], \mathsf{R}[i+6])$ aufgestellt und aufsteigend sortiert werden, wobei $\mathsf{R}[i+6]$ den Rang des Suffixes beginnend in Position $i + 6$ repräsentiert. Dafür müssen wir eine bestimmte Reihenfolge einhalten. Demnach werden zuerst die Suffixe an den Positionen $i \text{ modulo } 7 = 3$ und $i \text{ modulo } 7 = 5$ bestimmt, indem die Ränge der Positionen $(i+6) \text{ modulo } 7 = 2$ beziehungsweise $(i+6) \text{ modulo } 7 = 4$ zu Hilfe genommen werden. Anschließend werden die Ränge der Positionen $i \text{ modulo } 7 = 5$ bestimmt, da diese für die Sortierung der Suffixe an $i \text{ modulo } 7 = 6$ benötigt werden. Danach werden die Ränge von $i \text{ modulo } 7 = 6$ bestimmt, um damit die noch übrig gebliebenen Suffixe an $i \text{ modulo } 7 = 0$ zu sortieren.

\subparagraph*{Dritte Phase}

Für die dritte Phase gibt es zwei verschiedene Ansätze. Zuerst betrachten wir den naiven Ansatz. Hierbei wird - ähnlich wie bei dem \emph{DC3}-Algorithmus - der jeweils kleinste Werte der Mengen $\sa_{124}$, $\sa_0$, $\sa_3$, $\sa_5$ und  $\sa_6$, der noch nicht in dem endgültigen Suffix-Array einsortiert worden ist, miteinander verglichen. Dabei werden bei den Vergleichen zwischen $i$ und $j$ jeweils ein $l$ gesucht, sodass $(i + l) \text{ modulo } 7$ und $(j + l) \text{ modulo } 7$ aus dem Difference Cover $D = \{1, 2, 4\}$ sind. Zur Bestimmung der Länge $l$ kann die \cref{tab:merge} verwendet werden.

\begin{table}[H]
	\centering
	\begin{tabular}{c|lllllll}
		i/j & 0 & 1 & 2 & 3 & 4 & 5 & 6 \\\hline
		0   & 0 & 1 & 2 & 1 & 4 & 4 & 2 \\
		1   & 1 & 0 & 0 & 1 & 0 & 3 & 3 \\
		2   & 2 & 0 & 0 & 6 & 0 & 6 & 2 \\
		3   & 1 & 1 & 6 & 0 & 5 & 6 & 5 \\
		4   & 4 & 0 & 0 & 5 & 0 & 4 & 5 \\
		5   & 4 & 3 & 6 & 6 & 4 & 0 & 3 \\
		6   & 2 & 3 & 2 & 5 & 5 & 3 & 0
	\end{tabular}
	\caption{Merge-Tabelle}
	\label{tab:merge}
\end{table}

Nun werden Tupel der Länge $l+1$ aufgestellt, die aus $l$ Zeichen beginnend von $i$ beziehungsweise $j$ und dem Rang $\mathsf{R}$ der Position $i+l$ beziehungsweise $j+l$ bestehen, also $(\inputtext[i, i+l], \mathsf{R}[i+l+1])$ und $(\inputtext[j, j+l], \mathsf{R}[j+l+1])$. Wird dieser Ansatz verwendet, muss ein 7-Wege-Merge mit jeweils $\mathcal{O}(7)$-Vergleichen durchgeführt werden, was zu einer \emph{Worst-Case}-Laufzeit von $\mathcal{O}(7n \text{ log} (7)) = \mathcal{O}(n) $ führt.

Der zweite Ansatz verfolgt eine theoretisch bessere Idee im Bezug auf die Laufzeit. Dafür wird die Menge $\sa_{124}$ in die drei Mengen $\sa_1$, $\sa_2$ und $\sa_4$ aufgeteilt, wobei die jeweilige Sortierung beibehalten wird. Anschließend werden alle Mengen zu einer Menge $\sa_{0:6} = \sa_0 \circ \sa_1 \circ \sa_2 \circ \sa_3 \circ \sa_4 \circ \sa_5 \circ \sa_6$ konkateniert. Nun wird diese Menge stabil nach den ersten sieben Zeichen sortiert. Die Eigenschaft \emph{stabil} ist in diesem Fall sehr wichtig, da wir die eindeutige vorherige Sortierung aus den ersten zwei Phasen nicht durcheinander bringen dürfen. Als nächstes müssen die noch gleichen Tupel mit einem vergleichsbasiertem $7$-Wege-Merge sortiert werden. Dies passiert, indem wir die Ränge von $(i+l)$ und $(j+l)$ miteinander vergleichen. Dadurch erhalten wir das endgültige Suffix-Array. Der zweite Ansatz hat in der Theorie eine bessere \emph{Worst-Case}-Laufzeit von $\mathcal{O}(7n)$

%\subsubsection*{Vergleich $\emph{DC3}$ - $\emph{DC7}$}

%Interessant ist, ob der $\emph{DC7}$ gegenüber dem $\emph{DC3}$ Vorteile oder sogar auch Nach\-tei\-le aufweist. In der Theorie hat der \emph{Difference Cover} - Algorithmus eine Laufzeit von $\mathcal{O}(vn)$ und somit ist die Laufzeit des $\emph{DC3}$ - Algorithmus ($v = 3$) zwar besser also die des $\emph{DC7}$ - Algorithmus ($v = 7$), aber der String wird bei dem $\emph{DC3}$ - Algorithmus in jedem Rekursionsaufruf nur auf $T(2n/3)$ verkleinert. Bei dem $\emph{DC7}$ - Algorithmus wird der String in der Rekursion auf $T(3n/7)$ verkleinert. Wie dies in der Praxis aussieht, wird in dem Kapitel \ref{dc3:optim} näher untersucht.
\subsubsection{DC3-Lite}
\label{dc3:lite}


Es \currentauthor{Nico Bertram} ist möglich den Speicherverbrauch des \emph{DC3} so weit zu reduzieren, dass nur ein zusätzliches Hilfsarray $U$ der Länge $n$ verwendet wird. Diese Variante des \emph{DC3} wurde in ~\cite{saca:10} vorgestellt und wird als Komponente im \emph{nzSufSort} \cref{algorithm:nzSufSort} verwendet. Wir bezeichnen diese Variante in unserem Framework als \emph{DC3-Lite}. \\
Im Folgenden gehen wir auf die Unterschiede der einzelnen Phasen zwischen dem \emph{DC3} und dem \emph{DC3-Lite} ein. Da das Induzieren komplett analog zum \emph{DC3} ist, wird diese Phase hier nicht genauer betrachtet. 

\subparagraph*{Erste Phase}

Ähnlich wie im \emph{DC3} werden in der ersten Phase die Triplets $T[i,i+2]$ sortiert und lexikographische Namen vergeben. Im Unterschied zum \emph{DC3} werden aber alle Positionen $i$ von $T$ sortiert. Dies machen wir, um den Eingabetext mit den lexikographischen Rängen zu überschreiben, damit für $t_0$ und $t_{12}$ kein zusätzlicher Speicherbereich benötigt wird. Anders formuliert berechnet sich der überschriebene Text $T_{\text{new}}$ durch 
\begin{center}
	$T_{\text{new}} = [t_i : i \text{ modulo } 3 = 0] \circ [t_i : i \text{ modulo } 3 = 1] \circ [t_i : i \text{ modulo } 3 = 2]$ 
\end{center}
Dadurch bleiben die Informationen des ursprünglichen Textes erhalten, da in jedem lexikographischen Rang die Information des Zeichens an der betrachteten Position berücksichtigt wurde. In jedem Zeichen sind sogar mehr Informationen enthalten, da die beiden darauf folgenden Zeichen ebenfalls berücksichtigt werden. \\
Zunächst werden die Positionen von $T$ in das Array $U$ geschrieben und mithilfe der Inplace-Variante des Radixsort für große Alphabete \cref{sort:radix:big_alph} sortiert. Dabei wird der Speicherbereich für das $SA$ als Bucketarray verwendet. \\
Anschließend werden die lexikographischen Ränge vergeben und damit $T_{\text{new}}$ berechnet. Damit für den rekursiven Aufruf die lexikographischen Ränge durch die Textlänge beschränkt sind, berechnen wir parallel auch $t_{12}$ und schreiben dies in die Positionen von $T_{\text{new}}$, welche den Positionen $i$ mit $i \text{ modulo } 3 = 1$ und $i \text{ modulo } 3 = 2$ entsprechen. Die Positionen $i$ von $T_{\text{new}}$ mit $i \text{ modulo } 3 = 1$, werden vorher im ersten Drittel von $U$ und die Positionen mit $i \text{ modulo } 3 = 2$ im ersten Drittel von $SA$ zwischengespeichert. \\
Der rekursive Aufruf erfolgt dann mit $t_{12}$ und dem nicht verwendeten Speicher von $U$ und $SA$. Dadurch wird das Suffixarray $SA_{12}$ von $t_{12}$ berechnet und die zwischengespeicherten Positionen von $T_{\text{new}}$ werden wieder zurückgeschrieben.


\subparagraph*{Dritte Phase}

Durch das Induzieren in der zweiten Phase wurde das $SA_0$ berechnet. In dieser Phase werden nun $SA_0$ und $SA_{12}$ vereinigt. Dazu werden das $ISA_0$ und das $ISA_{12}$ hintereinander in $U$ berechnet. Im Unterschied zum \emph{DC3} wird das vereinigte Suffixarray nicht direkt in einen neuen Speicherbereich geschrieben, sondern die Einträge von $SA_0$ und $SA_{12}$ werden mit den Positionen im vereinigten Suffixarray überschrieben. \\
Beim Vereinigen werden jeweils die Suffixe an den Positionen $SA_0[i]$ und $SA_{12}[j]$ miteinander verglichen. Wenn $T[SA_0[i]]$ und $T[SA_{12}[j]]$ ungleich sind, kann die Position im vereinigten Suffixarray direkt bestimmt werden. Ansonsten kann die Ordnung der Suffixe durch Nachschauen in $ISA_0$ und $ISA_{12}$ bestimmt werden. Falls $SA_{12}[i] \text{ modulo } 3 = 1$ gilt, kann die Ordnung der Suffixe durch einen Vergleich von $ISA_{12}[SA_0[i]]$ und $ISA_{12}[p_2+SA_{12}[j]]$ bestimmt werden, wobei $p_2$ die erste Position in $ISA_{12}$ ist, die der Menge entspricht, die alle Positionen $i$ mit $i \text{ modulo } 3 = 2$ enthält. Falls $SA_{12}[i] \text{ modulo } 3 = 2$ gilt, wird die Ordnung der Suffixe durch einen Vergleich von $ISA_{12}[p_2+SA_0[i]]$ und $ISA_{12}[SA_{12}[j]-p_2+1]$ bestimmt. \\
Anschließend werden $ISA_0$ und $ISA_{12}$ mit den berechneten Positionen im vereinigten Suffixarray, die in $SA_0$ und $SA_{12}$ stehen, verknüpft. Genauer gesagt wird $ISA_0[i] = SA_0[ISA_0[i]]$ und $ISA_{12}[i] = SA_{12}[ISA_{12}[i]]$ gesetzt. Da $ISA_0$ und $ISA_{12}$ hintereinander in $U$ gespeichert wurden, steht in $U$ nun das inverse Suffixarray $ISA_{\text{new}}$ von $T_{\text{new}}$. Um $SA_{\text{new}}$ zu berechnen wird das Inverse von $U$ in $SA$ berechnet. \\
Bis jetzt haben wir nur das Suffixarray $SA_{\text{new}}$ des überschriebenen Eingabetextes berechnet. Da wir die Positionen des ursprünglichen Eingabetextes umsortiert haben, müssen wir diese wieder in die korrekte Reihenfolge bringen. Dies lässt sich durch einen Durchlauf mit der folgenden Funktion erreichen. \\
\begin{center}
	$SA[i] =
   \begin{cases}
     3SA_{\text{new}}[i] & \text{für } SA_{\text{new}}[i] \in[0, m_0) \\
     3(SA_{\text{new}}[i]-m_0)+1 & \text{für } SA_{\text{new}}[i] \in [m_0 ,m_0+\lceil \frac{m_{12}}{2} \rceil)) \\
     3(SA_{\text{new}}[i]-m_1)+2 & \text{für } SA_{\text{new}}[i] \in [m_0+\lceil \frac{m_{12}}{2} \rceil),m_0+m_{12})
   \end{cases}$ mit $m_0 = |t_0$|, $m_{12} = |t_{12}|$
\end{center}
\subsubsection{Parallel}
\label{dc3-parallel}

In \currentauthor{Johannes Bahne und Nico Bertram} diesem Kapitel wird der \emph{DC3} - Algorithmus auf Parallelität untersucht. Dabei gehen wir auf jede einzelne Phase des Algorithmus ein, wobei das Augenmerk auf die dritte Phase - dem Mergen - gelegt wird. Diese Entscheidung haben wir getroffen, da der dritte Teil der Hauptteil des Algorithmus ist und nach unserer Implementierung - um es kurz vorwegzunehmen - die längste Laufzeit aufweist.


\subparagraph*{Phase 1}
%Johannes

\subparagraph*{Phase 2}
%Johannes

\subparagraph*{Phase 3}

Die dritte Phase des \emph{DC3} - Algorithmus lässt sich hingegen nicht naiv parallelisieren, da die Reihenfolge der Ausführungen in der Variante, die im Kapitel \ref{dc3:algorithmus:phase3} vorgestellt worden ist, vorgegeben ist.
Daher werden wir uns in diesem Kapitel mit verschiedenen Merge-Verfahren auseinandersetzen, die parallelisiert werden können.

\input{kapitel/saca_algorithmen/dc3/erweiterung/parallel/naiv_merge}
\input{kapitel/saca_algorithmen/dc3/erweiterung/parallel/theorem2}
\input{kapitel/saca_algorithmen/dc3/erweiterung/parallel/suche}
\input{kapitel/saca_algorithmen/dc3/erweiterung/parallel/mergezweiseiten}
\input{kapitel/saca_algorithmen/dc3/erweiterung/parallel/theorem7}



\subsection{Optimierung und Evaluation}
\label{dc3:optim}

In \currentauthor{Johannes Bahne} diesem Kapitel wird auf verschiedene Optimierungen und deren Wir\-kungs\-wei\-sen eingegangen. Außerdem werden generelle Vergleiche der Mess\-er\-geb\-nis\-se zwischen unserer Implementierungen des \emph{DC3} - beziehungsweise \emph{DC7} - Algorithmus und der Referenzimplementierung des \emph{DC3} - Algorithmus, die der wissenschaftlichen Arbeit von Juha Kärkkäinen und Peter Sanders entnommen worden ist, gezogen \cite[p.~954,955]{saca:9}. Dabei nehmen wir an, dass die Referenzimplementierung die Basisversion ist.

\begin{description}
	\item[\texttt{Speicherverbrauch}]

	Die Referenzimplementierung benötigt folgende Arrays:

	$\begin{array}{lcl}
	1. \text{  } t_{12} & : & \text{ lexikographische Namen der Suffixe in } 	i \text{ modulo } 3 \neq 0\\
	2. \text{  } \sa_{12}& : & \text{ sortierte Positionen der Suffixe } 		i \text{ modulo } 3 \neq 0\\
	3. \text{  } t_0	& : & \text{ Positionen der Suffixe } 					i \text{ modulo } 3 = 0\\
	4. \text{  } \sa_0	& : & \text{ sortierte Positionen der Suffixe } 		i \text{ modulo } 3 = 0
	\end{array}$

	In unserer Implementierung benötigen wir jedoch nur drei Arrays. Und zwar können wir uns das Array $t_{0}$ sparen, da dieses nicht erst berechnet werden muss, sondern dem Sortieralgorithmus in Form einer \emph{Compare}-Funktion übermittelt werden kann, ohne ein neues Array anlegen zu müssen. Ein Array der Länge $1/3n$ wird somit weniger verbraucht. Daher hat unsere Implementierung einen etwas besseren Speicherverbrauch als die Referenzimplementierung.

	\item[\texttt{Sortieralgorithmus}]

	Die Referenzimplementierung von Juha Kärkkäinen und Peter Sanders verwendet als Sortieralgorithmus, die in den beiden ersten Phasen des \emph{DC3} - Algorithmus benötigt werden, den \emph{Radix-Sort}. Eine ähnliche Variante haben wir auch entwickelt, jedoch ist der Standard-Sortieralgorithmus der C++-Bibliothek schneller. Wir haben ebenfalls die sequentielle Variante des Algorithmus \emph{In-place Parallel Super Scalar Samplesort ($IPS^4o$)} ausprobiert, jedoch keinen Unterschied weder bezüglich der Laufzeit noch des Speicherverbrauches feststellen können, sodass wir uns für den Standard-Sortieralgorithmus entschieden haben, damit wir dafür keine externe Bibliothek verwenden müssen. Aufgrund des besseren Sortieralgorithmus weist unsere Implementierung ebenfalls eine etwas bessere Laufzeit auf als die der Referenzimplementierung.

	\item[\texttt{DC7}]

	Die Implementierung des \emph{DC7} - Algorithmus ist sehr ähnlich zu der des \emph{DC3} - Algorithmus. Es wird jedoch zusätzlich ein Array benötigt, um zuerst die Ränge der Suffixe beginnend in Position $i \text{ modulo } 7 = 5$ und anschließend die Ränge in $i \text{ modulo } 7 = 6$ zu speichern. Dieses Array kann jedoch anschließend gelöscht werden.
	Trotzdem ist der Speicherverbrauch niedriger als der unseres \emph{DC3} - Algorithmus, da die Rekursion mit einem kleineren String aufgerufen wird. Dies wurde bereits bei den theoretischen Überlegungen angenommen.

	Ein etwas größerer Unterschied der Implementierungen der beiden Algorithmen ist die dritte Phase - das Mergen. Hierbei haben wir beide Ansätze, die wir bereits im Kapitel \ref{dc7} besprochen haben, implementiert und miteinander verglichen. Als Ergebnis erhielten wir, dass der naive erste Ansatz schneller ist als der zweite Ansatz. Dies liegt vermutlich daran, dass zuerst alle sortierten Mengen $\sa_0$, $\sa_{124}$, $\sa_3$, $\sa_5$ und  $\sa_6$ zuerst einmal zu einer Menge konkateniert werden müssen und anschließend eine Sortierung nach den ersten sieben Zeichen erfolgen muss, um dann erst mergen zu können. Dieser Ansatz hat zwar in der Theorie eine bessere Laufzeit, jedoch sind dies zu viele Schritte in der Praxis. Aus diesem Grund liegt der zweite Ansatz zwar in unserem \sacabench - Framework vor, jedoch wird stattdessen nur der naive Ansatz bei den Auswertungen verwendet.
	Trotz der geringeren Rekursionstiefe weist der \emph{DC7} - Algorithmus mit dem naiven Ansatz eine langsamere Laufzeit gegenüber dem \emph{DC3} - Algorithmus auf. Diese lässt sich vermutlich auf das Mergen zurückführen, denn dies ist die Schwachstelle in dem Algorithmus.
	%Daher wäre es interessant, ob sich nicht sogar noch andere Ansätze anbieten würden, die diesen Schritt schneller ausführen können. Ein möglicher Ansatz zum Mergen wäre der sogenannte \emph{Looser-Tree}. Dieser ist in der Praxis einer der schnellsten Bäume, mit dem sich die Laufzeit möglicherweise noch verbessern lässt. Somit wäre es eine Überlegung Wert, diesen Ansatz im zweiten Teil der Projektgruppe \sacabench zu implementieren.

\end{description}

