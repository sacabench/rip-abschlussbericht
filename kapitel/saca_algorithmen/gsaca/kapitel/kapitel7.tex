\newpage
\subsection{Optimierungen}
\label{gsaca:chapter7}
% Einleitung / Ziel der Optimierung
Nachdem in den vorherigen Kapitel die Einzelheiten zu GSACA vorgestellt wurden, werden in diesem Kapitel Optimierungen und Änderungen am Algorithmus vorgestellt.
Dabei wurde das Augenmerk auf eine Reduktion des Speicherverbrauchs gelegt.
Ausgangspunkt war eine abgewandelte Implementierung des GSACA-Algorithmus, in dem die Liste \gsize als Vektor gespeichert wurde.
Wie zuvor beschrieben, speichert \gsize die Größen der Gruppen.
Dabei ist die eigentliche Größe lediglich am Startindex der Gruppe gespeichert, der Rest der Gruppeneinträge ist 0.
Hierdurch bot dieser Bestandteil des Algorithmus einen guten Einstiegspunkt, um den verbrauchten Speicherplatz zu verringern.
Die Idee zur Verbesserung, die verfolgt wurde, war, die Größe der einzelnen Elemente dieses Arrays zu reduzieren.
Ursprünglich wurden alle Elemente des Vektors als Integer gespeichert, so dass je nach Größe der Integer bis zu 64 Bit pro Eintrag benötigt wurden.
Durch geschickte Kodierung der Elemente gelang es, diese Größe auf zwei Bit pro Eintrag zu reduzieren. \par
% Beschreibung der Optimierung
Wie auch in der alten Version, wurden die Einträge in der neuen \gsize-Liste mit geringerem Speicherverbrauch als Vektor gespeichert.
Jedoch wurden sie nicht als Integer sondern als boolsche Werte gespeichert.
Je zwei aufeinanderfolgende Bool-Werte kodieren dabei einen Eintrag in \gsize, somit enthält der Vektor zwar doppelt so viele Einträge wie zuvor, aber jeder Eintrag belegt nur ein Bit.
Das erste Bit kennzeichnet den Start einer Gruppe.
Ist es auf true gesetzt, ist das entsprechende Element an dem Index das erste Element der Gruppe.
In der ursprünglichen Version enthielte dieser Index die Anzahl der Elemente der Gruppe.
Das zweite Bit markiert das Ende einer Gruppe.
Diese Information ist wichtig, da beim Umsortieren der Gruppenkontexte in Phase 1 von GSACA die hinteren Elemente der Gruppen abgespalten werden, ohne dass die Gruppengröße direkt aktualisiert wird.
Da die ursprüngliche Gruppengröße aber weiter verwendet wird, kann zu dem Zeitpunkt noch nicht das erste Element der Gruppe aktualisiert werden, es wird also gleichzeitig die alte Gruppengröße und die Größe der kleinen abgespaltenen Gruppen benötigt.
Wird nun die Größe einer Gruppe gespeichert, wird das erste Bit des ersten Elements der Gruppe auf 1 gesetzt.
Anschließend werden für alle weiteren Elemente der Gruppe sowohl das erste als auch das zweite Bit auf 0 gesetzt.
Schließlich wird das zweite Bit des letzten Elements der Gruppe auf 1 gesetzt.
Ein Beispiel für den Vergleich einer Gruppe mit der alten und der neuen Variante lässt sich in Tabelle \ref{compare_gsize_variants} sehen.
Das Lesen eines Wertes ist etwas komplexer als das Spei\-chern eines Wertes.
Wenn an der Index an einer beliebigen Stelle innerhalb einer Gruppe gelesen wird, ist das erste Bit 0, somit ist es nicht der Beginn der Gruppe.
Wie bei der ursprünglichen Variante wird als Gruppengröße 0 zurückgegeben.
Wird aber am Beginn einer Gruppe gelesen, ist also das erste Bit 1, muss die Größe der Gruppe erst bestimmt werden.
Hierzu wird der Vektor von dem abgefragten Index an durchlaufen, solange bis ein Element erreicht wird, dessen erstes Bit 1 ist, das also den Beginn der nächsten Gruppe darstellt.
Bei diesem Durchlauf wird die Anzahl der Elemente gezählt.
So kann aus der Struktur die Gruppengröße bestimmt werden. \par

\begin{table}[H]
\begin{tabular}{l|cc|cc|cc|cc|cc|cc}
Index              & \multicolumn{2}{c|}{0} & \multicolumn{2}{c|}{1} & \multicolumn{2}{c|}{2} & \multicolumn{2}{c|}{3} & \multicolumn{2}{c|}{4} & \multicolumn{2}{c}{5} \\
\gsize mit Integern & \multicolumn{2}{c|}{3} & \multicolumn{2}{c|}{0} & \multicolumn{2}{c|}{0} & \multicolumn{2}{c|}{2} & \multicolumn{2}{c|}{0} & \multicolumn{2}{c}{1} \\
\gsize mit Boolean  & 1          & 0         & 0          & 0         & 0          & 1         & 1          & 0         & 0          & 1         & 1         & 1        
\end{tabular}
\caption{Vergleich der Speicherung verschiedener Gruppengrößen zwischen der ursprünglichen Speicherung mit Integern und der neuen Speicherung mit boolschen Werten.}
\label{compare_gsize_variants}
\end{table}

% Beschreibung des Ansatzes
Um diesen neuen Ansatz zu entwickeln, war es nötig, die Gleichheit der alten und der neuen Version nach jeder Operation sicherzustellen.
Hierzu wurde zunächst eine Klasse GSIZE\_LIST erstellt, welche die Verwaltung der alten Version kapselt.
Diese Klasse bot Methoden zum lesenden und schreibenden Zugriff auf die einzelnen Element und wurde anschließend in dem Algorithmus verwendet.
Dann wurde eine neue Klasse GSIZE\_BOOL hinzugefügt, welche die glei\-chen Methoden wie GSIZE\_LIST bereitstellte.
Hierdurch konnte innerhalb des Algorithmus an nur einer Stelle die verwendete Klasse ausgewechselt werden, indem statt einer Instanz von GSIZE\_LIST eine Instanz von GSIZE\_BOOL initialisiert wurde.
Jetzt konnte zwar die Implementierung einfach ausgetauscht werden, jedoch konnten die beiden Versionen noch nicht direkt miteinander verglichen werden.
Um dies zu tun wurde als drittes die Klasse GSIZE\_COMPARE erstellt.
Auch diese Klasse enthielt die gleichen Methoden wie GSIZE\_LIST und GSIZE\_BOOL und konnte so ohne große Änderungen vom Algorithmus verwendet werden.
GSIZE\_COMPARE verwaltet je eine Instanz der beiden anderen Klassen und leitet alle Operationen an beide weiter.
Verwendet GSACA nun eine Instanz dieser dritten Klasse und nutzt diese beispielsweise um einen Wert von \gsize auszulesen, wird der Wert an dem abgefragten Index sowohl von GSIZE\_LIST als auch von GSIZE\_BOOL ausgelesen.
Auch beim Setzen eines neuen Wertes wendet GSIZE\_COMPARE diese Änderungen auf beide Implementierungen an.
Auf diese Art können in jedem Schritt die Werte von der alten und der neuen Variante verglichen werden.
Außerdem lässt sich bei Abweichungen direkt feststellen, in welchem Schritt es passiert ist, warum es gescheitert ist und, aufgrund der Richtigkeit von GSIZE\_LIST, welcher Wert eigentlich in GSIZE\_BOOL enthalten sein müsste.
Durch dieses Vorgehen konnte das Verhalten von GSIZE\_LIST auch in Grenzfällen passen analysiert und nachgebildet werden. \par
% Evaluation der Optimierung
Leider benötigt die neue Version GSIZE\_BOOL aber erheblich mehr Re\-chen\-auf\-wand, da sowohl beim Speichern eines neuen Wertes als auch beim Lesen eines Wertes die ursprüngliche Zahl berechnet werden muss.
Durch diesen Ansatz konnte der Speicherverbrauch im Vergleich zu der vorherigen Implementierung als Vektor reduziert werden.
Aufgrund der benötigten Umrechnung der Werte erhöhte sich jedoch die Laufzeit gegenüber dem vorherigen Ansatz deutlich.
Die Ergebnisse von Laufzeit- und Speicherverbrauchsmessungen auf Datensätzen verschiedener Größe lässt sich in \ref{compare_gsize_results} sehen.
Beide Varianten wurden drei mal ausgeführt, um Schwankungen auszugleichen.
Die gemessenen Werte beziehen sich dabei auf die Ausführungsdauer des Algorithmus und beziehen nicht die Initialisierungsphase oder das Einlesen und Vorbereiten der Texte mit ein.
Als Grundlage diente der Text \textit{english} des \textit{Pizza\&Chili Corpus} \cite{testdaten:pizzachilli2007}.
Wegen dieser starken Erhöhung der Laufzeit bei gleich\-zeitig nur relativ moderater Reduktion des Speicherverbrauchs ist der untersuchte Ansatz als keine sinnvolle Optimierung von GSACA anzusehen.
Jedoch hat sich der beschriebene Ansatz zum Vergleichen mehrerer alternativen Implementierungen als praktisch erwiesen.
Durch die parallele Ausführung mehrerer Varianten konnten schnell Abweichungen und Fehler gefunden werden.
Zusätz\-lich hatten die beiden Implementierungen von \gsize nach jeder Operation den gleichen Zustand, so dass sich Unterschiede schnell untersuchen lassen konnten und der für die Ungleichheit verantwortlich Schritt reproduzierbar verglichen werden konnte. \par

\begin{table}[H]
\begin{tabular}{rrrrr}
\multicolumn{1}{l}{} & \multicolumn{4}{c}{GSACA mit GSIZE\_LIST}                                                                                                                    \\
\multicolumn{1}{l}{} & \multicolumn{3}{c}{Zeit in Millisekunden}                                                        & \multicolumn{1}{c}{\multirow{2}{*}{Speicherpeak in Byte}} \\
\multicolumn{1}{l}{} & \multicolumn{1}{c}{1. Messung} & \multicolumn{1}{c}{2. Messung} & \multicolumn{1}{c}{3. Messung} & \multicolumn{1}{c}{}                                      \\
1K                   & 0,07                           & 0,05                           & 0,05                           & 17.296                                                    \\
10K                  & 0,65                           & 0,51                           & 0,64                           & 165.168                                                   \\
100K                 & 7,84                           & 6,10                           & 5,51                           & 1.639.888                                                 \\
1M                   & 176,87                         & 163,13                         & 174,05                         & 16.778.944                                                \\
\multicolumn{1}{l}{} & \multicolumn{1}{l}{}           & \multicolumn{1}{l}{}           & \multicolumn{1}{l}{}           & \multicolumn{1}{l}{}                                      \\
\multicolumn{1}{l}{} & \multicolumn{4}{c}{GSACA mit GSIZE\_BOOL}                                                                                                                    \\
\multicolumn{1}{l}{} & \multicolumn{3}{c}{Zeit in Millisekunden}                                                        & \multicolumn{1}{c}{\multirow{2}{*}{Speicherpeak in Byte}} \\
\multicolumn{1}{l}{} & \multicolumn{1}{c}{1. Messung} & \multicolumn{1}{c}{2. Messung} & \multicolumn{1}{c}{3. Messung} & \multicolumn{1}{c}{}                                      \\
1K                   & 0,16                           & 0,14                           & 0,15                           & 15.246                                                    \\
10K                  & 11,26                          & 8,34                           & 8,56                           & 144.686                                                   \\
100K                 & 645,56                         & 645,18                         & 644,93                         & 1.435.086                                                 \\
1M                   & 70.417,64                      & 70.432,93                      & 70.391,62                      & 14.681.790                                               
\end{tabular}
\caption{Messergebnisse des Vergleichs der Laufzeit und des Speicherbedarfs zwischen GSACA mit GSIZE-Liste aus Integer (oben) und GSIZE-Liste aus boolsche Werte (unten) auf Texten der Gr\�\�e 1K, 10K, 100K und 1M.}
\label{compare_gsize_results}
\end{table}
