\newpage
\subsection{Der Algorithmus}
\label{gsaca:chapter4}
%
Nachdem zuvor das allgemeine Vorgehen beschrieben und an einem Beispiel verdeutlicht wurde, wird in diesem Kapitel der konkrete Algorithmus als Pseudocode vorgestellt. 

In Abbildung \ref{saca:3:code_1} ist der Pseudocode zu Phase 1 von GSACA zu sehen.
Zunächst wird die initiale Einteilung in die lexikografisch sortierten Gruppen vorgenommen. 
Anschließend werden in die Gruppen in einer for-Schleife in absteigender Reihenfolge durchlaufen.
In der ersten inneren for-Schleife speichert für jedes Zeichen der aktuellen Gruppe einen Zeiger zum direkt vorhergehenden Zeichen im Wort.
Gibt es kein vorheriges Zeichen, wird die 0 als \prevpointer gespeichert. 
Dies ermöglicht die Überprüfung, ob ein Zeichen ein vohergehendes Zeichen hat, indem der \prevpointer mit 0 verglichen wird. 
Dies geschieht in einer späteren Codezeile in Phase 2. 
Anschließend werden alle lexikografisch kleineren Gruppenkontexte, auf die ein \prevpointer eines Elements der gerade betrachteten Gruppe zeigt, in Untermengen geteilt.
Diese Untermengen enthalten Elemente der lexikografisch kleineren Gruppen, auf die dieselbe Anzahl an \prevpointern zeigt.
In einer weiteren for-Schleife wird über diese Untermengen iteriert.
Dabei wird jede Untermenge weiter unterteilt, so dass jede Teilmenge einer Untermenge Suffixe der gleichen Gruppe beinhaltet.
Abschließend wird in einer letzten for-Schleife jede dieser neuen Teilmengen bearbeitet.
Dazu wird jedes Suffix dieser Teilmengen aus seiner ursprünglichen Gruppe entfernt und zu einer neuen Gruppe als direkter Nachfolger der alten Gruppe hinzugefügt.

\begin{figure}
\begin{minted}[escapeinside=@@, mathescape=true, linenos=true, autogobble, obeytabs=true, tabsize=2]{python}
def phase_1():
	# Order all suffixes into groups according to their first character
	groups = setupGroups()
	groups.sortByContext()
	for group in groups.reversed():
		for char in group:
			prev(char) = max(@\{@ @$j$@ @$\in$@ [1 ... char] | group(@$j$@) < group(char) @\}@ @$\cup$@ @\{@0@\}@)
		# prevoisSuffixes is the set of previous suffixes from $group$
		prevoisSuffixes = @\{ $j$ $\in$ [1 ... $n$] | prev(char) = $j$ for any char $\in$ group \}@

		# each subset $P_{l}$ contains suffixes whose number of prev pointers
		# from group pointing to them is equal to l
		listOfSubsets = prevoisSuffixes.splitIntoSubsets()
		for subset = listOfSubsets.count down to 1:
			# split each subset into new subsets, 
			# such that suffixes of same group are gathered in the same subset
			listOfSubgroups = subset.sortByGroup()
			for subgrop = 1 up to listOfSubgroups.count do
				remove suffixes of @$P_{lq}$@ from their group and put them into a new group placed as immediate successor of their old group.
\end{minted}
\caption[Algorithmus zur Konstruktion eines Suffix Arrays für einen gegebenen nullterminierten String $S$ der Länge $n$]{Algorithmus zur Konstruktion eines Suffix Arrays für einen gegebenen nullterminierten String $S$ der Länge $n$}
\label{saca:3:code}
\end{figure}

Abbildung \ref{saca:3:code_2} zeigt den Pseudocode zu Phase 2 von GSACA, in der das Suffix-Array aus den zuvor gebildeten Gruppen konstruiert wird.
Hierzu wird zunächst das letzte Zeichen des Wortes, also das Terminalsymbol \$, als erstes Element der Liste \sa gespeichert. 
Anschließend werden die Elemente dieser Liste in einer for-Schleife durchlaufen.
Auch wenn zu Beginn nur ein einziges Element in der Liste enthalten ist, wird jeder Durchlauf weitere Elemente hinzufügen.
In jeder Iteration ein das nächste Zeichen in \sa bearbeitet und zu diesem Zeichen wird der Vorgänger im Wort betrachtet. 
Die Liste der \prevpointer wird in einer nachfolgenden while-Schleife durchlaufen.
In dieser while-Schleife wird zunächst die Position des gerade betrachteten Zeichens bestimmt. 
Die Position ergibt sich aus der Kardinalität der Menge aller Suffixe in kleineren Gruppen, inkrementiert um 1. 
Anschließend wird überprüft, ob \sa bereits ein Zeichen an diesem Index enthält. 
Falls dies der Fall ist, wurden dieses Zeichen und die Elemente der \prevpointer schon betrachtet und die aktuelle Iteration der for-Schleife wird abgebrochen. 
Gibt es jedoch noch kein Zeichen an dieser Stelle in der Liste \sa, wird das aktuell betrachtete Zeichen dort gespeichert. 
Dann wird dieses Zeichen aus seiner aktuellen Gruppe entfernt und in eine neue, direkt vor der alten Gruppe liegende Gruppe hinzugefügt.
Am Schluss der while-Schleife wird das Zeichen, auf den der \prevpointer des aktuellen Zeichens zeigt, zum betrachteten Element für die nächste Iteration der while-Schleife.\\

\newpage
Im folgenden wird die erste Iteration der zweiten Phase exemplarisch am Beispiel Banane durchlaufen:\\
\sa[1] = 7\\
for-Schleife wird mit $i$ = 1 begonnen\\
$j$ = \sa[$i$] - 1 = \sa[1] - 1 = 6\\
$j$ $\neq$ 0, in while-Schleife wird eingetreten\\
$sr$ = |\{\$, anane, ane, b\}| = 4\\
\sa[$sr$ + 1] = nil, daher wird die 1. Iteration der for-Schleife nicht abgebrochen\\
\sa[$sr$ + 1] = $j$ = 6\\
$j$ wird aus der aktuellen Gruppe entfernt und in eine neue Gruppe eingefügt, die ein direkter Vorgänger der aktuellen Gruppe ist. Da $j$ das einzige Element dieser Gruppe ist, findet keine Änderung statt.\\
$j$ = prev($j$) = 4. Die nächste Iteration der while-Schleife wird mit $j$ = 4 durchlaufen.

\begin{figure}
\begin{minted}[escapeinside=@@,numbers=left, tabsize=2, breaklines]{python}
def phase_2():
	SA[1] = @$n$@ 
	for @$i$@ = 1 up to @$n$@:
		@$j$@ = SA[@$i$@] - 1 
		while @$j$@ @$\neq$@ 0:
			# number of suffixes placed in lower groups
			@$sr$@ =  |@\{@ @$s$@ @$\in$@ [1 ... @$n$@] | group(@$s$@) < group(@$j$@) @\}@|
			if SA[@$sr$@ + 1] @$\neq$@ nil: 
				break
			SA[@$sr$@ + 1] = @$j$@
			# remove @$j$@ from its group and put it into a new group
			oldGroup = group(@$j$@)
			group(@$j$@).remove(j)
			newGroup = Group()
			newGroup.add(j)
			# place newGroup as immediate predecessor of oldGroup
			insertPosition = groups.indexOf(oldGroup) - 1
			groups.insert(newGroup, insertPosition)
			@$j$@ = prev(@$j$@)
\end{minted}
\caption[Phase 2 von GSACA]{Phase 2 von GSACA}
\label{saca:3:code_2}
\end{figure}
\clearpage % add new page to move the next section down
