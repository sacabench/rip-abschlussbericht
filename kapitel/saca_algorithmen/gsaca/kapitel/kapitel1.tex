\subsection{Einleitung}
\label{gsaca:chapter1}
%
In \currentauthor{David Piper} diesem Abschnitt geht es um den Algorithmus GSACA. 
Grundlage hierf\�r bietet das Paper \textit{Linear-time Suffix Sorting - A New Approach for Suffix Array Construction} von Uwe Baier \cite{saca:3}. 
GSACA verfolgt bei der Konstruktion des Suffix Arrays einen neuen Ansatz und erstellt dieses in linearer Zeit ohne rekursiv zu arbeiten.
Eine vollst\�ndige Implementierung des Algorithmus l\�sst sich auf der GitHub-Seite \cite{saca:3:github} von Uwe Baier finden. \par
Zun\�chst wird in Kapitel \ref{gsaca:chapter2} das allgemeine Vorgehen des Algorithmus informell beschrieben. 
Kapitel \ref{gsaca:chapter3} wendet dies beispielhaft an dem Wort Banane an. 
Danach wird in Kapitel \ref{gsaca:chapter4} der Algorithmus vorgestellt und Zeile f\�r Zeile erkl\�rt. 
Anschlie\�end werden in Kapitel \ref{gsaca:chapter5} weitere Details der Implementierung besprochen.
Darauf folgt das Kapitel \ref{gsaca:chapter6} in dem GSACA auf die Eingabe \textit{caabaccaabacaa} angewendet wird, wie es zuvor auch schon bei anderen Algorithmen der Fall war.
Zum Schluss werden in Kapitel \ref{gsaca:chapter7} m\�gliche \�nderungen und Optimierungen besprochen. 
