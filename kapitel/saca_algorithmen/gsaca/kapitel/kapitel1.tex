\subsection{Einleitung}
\label{gsaca:chapter1}
%
Der \currentauthor{David Piper} Algorithmus GSACA verfolgt bei der Konstruktion des Suffix Arrays einen neuen Ansatz verfolgt, um dieses in linearer Zeit zu erstellen ohne rekursiv zu arbeiten.
Vorgestellt wurde dieser Algorithmus das erste mal im Paper \textit{Linear-time Suffix Sorting - A New Approach for Suffix Array Construction} von Uwe Baier \cite{saca:3}. 
Eine vollständige Implementierung des Algorithmus lässt sich auf seiner GitHub-Seite \cite{saca:3:github} finden. \par
Zunächst wird in Kapitel \ref{gsaca:chapter2} das allgemeine Vorgehen des Algorithmus informell beschrieben. 
Kapitel \ref{gsaca:chapter3} wendet ihn dann beispielhaft an dem Wort Banane an. 
Danach wird in Kapitel \ref{gsaca:chapter4} der Algorithmus detailliert erklärt. 
In Kapitel \ref{gsaca:chapter5} werden anschließend weitere Details der Implementierung besprochen.
Darauf folgt das Kapitel \ref{gsaca:chapter6}, in dem GSACA auf die Eingabe \textit{caabaccaabacaa} angewendet wird, wie es zuvor auch schon bei anderen Algorithmen der Fall war.
Zum Schluss werden in Kapitel \ref{gsaca:chapter7} mögliche Änderungen und Optimierungen besprochen. 
