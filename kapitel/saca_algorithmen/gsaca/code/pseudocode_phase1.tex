\begin{figure}
\begin{minted}[escapeinside=@@, mathescape=true, linenos=true, autogobble, obeytabs=true, tabsize=2]{python}
def phase_1():
	# Order all suffixes into groups according to their first character
	groups = setupGroups()
	groups.sortByContext()
	for group in groups.reversed():
		for char in group:
			prev(char) = max(@\{@ @$j$@ @$\in$@ [1 ... char] | group(@$j$@) < group(char) @\}@ @$\cup$@ @\{@0@\}@)
		# prevoisSuffixes is the set of previous suffixes from $group$
		prevoisSuffixes = @\{ $j$ $\in$ [1 ... $n$] | prev(char) = $j$ for any char $\in$ group \}@

		# each subset $P_{l}$ contains suffixes whose number of prev pointers
		# from group pointing to them is equal to l
		listOfSubsets = prevoisSuffixes.splitIntoSubsets()
		for subset = listOfSubsets.count down to 1:
			# split each subset into new subsets, 
			# such that suffixes of same group are gathered in the same subset
			listOfSubgroups = subset.sortByGroup()
			for subgrop = 1 up to listOfSubgroups.count do
				remove suffixes of @$P_{lq}$@ from their group and put them into a new group placed as immediate successor of their old group.
\end{minted}
\caption[Algorithmus zur Konstruktion eines Suffix Arrays für einen gegebenen nullterminierten String $S$ der Länge $n$]{Algorithmus zur Konstruktion eines Suffix Arrays für einen gegebenen nullterminierten String $S$ der Länge $n$}
\label{saca:3:code}
\end{figure}