\begin{enumerate}
\item 
Erzeuge anfängliche Zeichenpaar-Textposition Tupel. Sie entsprechen Präfixen der Länge 2.
\begin{center}
\small\begin{tabular}{lrrrrrrrrrrrrrr}
    \toprule 
    $S$ & \textcolor{gray}{0} & \textcolor{gray}{1} & \textcolor{gray}{2} & \textcolor{gray}{3} & \textcolor{gray}{4} & \textcolor{gray}{5} & \textcolor{gray}{6} & \textcolor{gray}{7} & \textcolor{gray}{8} & \textcolor{gray}{9} & \textcolor{gray}{10} & \textcolor{gray}{11} & \textcolor{gray}{12} & \textcolor{gray}{13}\\
    \midrule 
    $c_0$ & $c$ & $a$ & $a$ & $b$ & $a$ & $c$ & $c$ & $a$ & $a$ & $b$ & $a$ & $c$ & $a$ & $a$ \\
    $c_1$ & $a$ & $a$ & $b$ & $a$ & $c$ & $c$ & $a$ & $a$ & $b$ & $a$ & $c$ & $a$ & $a$ & $\$$ \\
    $i$ & $0$ & $1$ & $2$ & $3$ & $4$ & $5$ & $6$ & $7$ & $8$ & $9$ & $10$ & $11$ & $12$ & $13$ \\
    \bottomrule 
\end{tabular}
\end{center}
\item 
Sortiere $S$ lexikografisch.
\begin{center}
\small\begin{tabular}{lrrrrrrrrrrrrrr}
    \toprule 
    $S$ & \textcolor{gray}{0} & \textcolor{gray}{1} & \textcolor{gray}{2} & \textcolor{gray}{3} & \textcolor{gray}{4} & \textcolor{gray}{5} & \textcolor{gray}{6} & \textcolor{gray}{7} & \textcolor{gray}{8} & \textcolor{gray}{9} & \textcolor{gray}{10} & \textcolor{gray}{11} & \textcolor{gray}{12} & \textcolor{gray}{13}\\
    \midrule 
    $c_0$ & $a$ & $a$ & $a$ & $a$ & $a$ & $a$ & $a$ & $a$ & $b$ & $b$ & $c$ & $c$ & $c$ & $c$ \\
    $c_1$ & $\$$ & $a$ & $a$ & $a$ & $b$ & $b$ & $c$ & $c$ & $a$ & $a$ & $a$ & $a$ & $a$ & $c$ \\
    $i$ & $13$ & $1$ & $7$ & $12$ & $2$ & $8$ & $4$ & $10$ & $3$ & $9$ & $0$ & $6$ & $11$ & $5$ \\
    \bottomrule 
\end{tabular}
\end{center}

\item 
Benenne die Paare in $S$ lexikografische um ($U$ = name($S$)).
\begin{center}
\small\begin{tabular}{lrrrrrrrrrrrrrr}
    \toprule 
    $U$ & \textcolor{gray}{0} & \textcolor{gray}{1} & \textcolor{gray}{2} & \textcolor{gray}{3} & \textcolor{gray}{4} & \textcolor{gray}{5} & \textcolor{gray}{6} & \textcolor{gray}{7} & \textcolor{gray}{8} & \textcolor{gray}{9} & \textcolor{gray}{10} & \textcolor{gray}{11} & \textcolor{gray}{12} & \textcolor{gray}{13}\\
    \midrule 
    $c$ & $1$ & $2$ & $2$ & $2$ & $5$ & $5$ & $7$ & $7$ & $9$ & $9$ & $11$ & $11$ & $11$ & $14$ \\
    $i$ & $13$ & $1$ & $7$ & $12$ & $2$ & $8$ & $4$ & $10$ & $3$ & $9$ & $0$ & $6$ & $11$ & $5$ \\
    \bottomrule 
\end{tabular}
\end{center}

\item 
Die anfängliche Zuordnung der Namen zu Präfixen ist somit:
\begin{center}
\small\begin{tabular}{rl}
\toprule 
Name & Präfix\\
\midrule 
  $1$ & $a\$$\\
  $2$ & $aa$\\
  $5$ & $ab$\\
  $7$ & $ac$\\
\bottomrule 
\end{tabular}
\small\begin{tabular}{rl}
\toprule 
Name & Präfix\\
\midrule 
  $9$ & $ba$\\
  $11$ & $ca$\\
  $14$ & $cc$\\
  &\\
\bottomrule 
\end{tabular}
\end{center}

\item 
Initialisiere $P$ und $F$ als leere Sequenzen.
\item 
Iteriere bis zu $\ceil{\log_2 n} = 4$ mal.
\item 
Beginne Iteration $k = 1$.

\item 
Markiere alle einzigartigen Namen in U (Grün in der Tabelle).
\begin{center}
\small\begin{tabular}{lrrrrrrrrrrrrrr}
    \toprule 
    $U$ & \textcolor{gray}{0} & \textcolor{gray}{1} & \textcolor{gray}{2} & \textcolor{gray}{3} & \textcolor{gray}{4} & \textcolor{gray}{5} & \textcolor{gray}{6} & \textcolor{gray}{7} & \textcolor{gray}{8} & \textcolor{gray}{9} & \textcolor{gray}{10} & \textcolor{gray}{11} & \textcolor{gray}{12} & \textcolor{gray}{13}\\
    \midrule 
    $c$ & $1$ & $2$ & $2$ & $2$ & $5$ & $5$ & $7$ & $7$ & $9$ & $9$ & $11$ & $11$ & $11$ & $14$ \\
    $i$ & $13$ & $1$ & $7$ & $12$ & $2$ & $8$ & $4$ & $10$ & $3$ & $9$ & $0$ & $6$ & $11$ & $5$ \\
    $Uniq.$ & $$\cmarkc$$ & $$\xmarkc$$ & $$\xmarkc$$ & $$\xmarkc$$ & $$\xmarkc$$ & $$\xmarkc$$ & $$\xmarkc$$ & $$\xmarkc$$ & $$\xmarkc$$ & $$\xmarkc$$ & $$\xmarkc$$ & $$\xmarkc$$ & $$\xmarkc$$ & $$\cmarkc$$ \\
    \bottomrule 
\end{tabular}
\end{center}

\item 
Merge P in die Sequenz U. Keine Änderung, da P leer.

\item 
Sortiere $U$ anhand $(i \mod 2^k, i \div 2^k)$.
\begin{center}
\small\begin{tabular}{lrrrrrrrrrrrrrr}
    \toprule 
    $U$ & \textcolor{gray}{0} & \textcolor{gray}{1} & \textcolor{gray}{2} & \textcolor{gray}{3} & \textcolor{gray}{4} & \textcolor{gray}{5} & \textcolor{gray}{6} & \textcolor{gray}{7} & \textcolor{gray}{8} & \textcolor{gray}{9} & \textcolor{gray}{10} & \textcolor{gray}{11} & \textcolor{gray}{12} & \textcolor{gray}{13}\\
    \midrule 
    $c$ & $11$ & $5$ & $7$ & $11$ & $5$ & $7$ & $2$ & $2$ & $9$ & $14$ & $2$ & $9$ & $11$ & $1$ \\
    $i$ & $0$ & $2$ & $4$ & $6$ & $8$ & $10$ & $12$ & $1$ & $3$ & $5$ & $7$ & $9$ & $11$ & $13$ \\
    $Uniq.$ & $$\xmarkc$$ & $$\xmarkc$$ & $$\xmarkc$$ & $$\xmarkc$$ & $$\xmarkc$$ & $$\xmarkc$$ & $$\xmarkc$$ & $$\xmarkc$$ & $$\xmarkc$$ & $$\cmarkc$$ & $$\xmarkc$$ & $$\xmarkc$$ & $$\xmarkc$$ & $$\cmarkc$$ \\
    \bottomrule 
\end{tabular}
\end{center}

\item 
Iteriere durch U.
\begin{itemize}
\item Falls der Name einzigartig ist, und einer der beiden Vorgängernamen einzigartig ist, füge ihn zu F hinzu.
\item Falls der Name einzigartig ist, und keiner der beiden Vorgängernamen einzigartig ist, füge ihn zu P hinzu.
\item Sonst bilde aus dem Namen und seinen Nachfolger ein Paar, und füge es zu S hinzu.
\end{itemize}

\begin{center}
\small\begin{tabular}{rrcccc}
\toprule 
 $U_c$ & $U_i$ & $U_{uniq}$ &     $S$      &   $P$   &   $F$   \\
\midrule 
$11$ & 0 & $$\xmarkc$$ & $((11, 5), 0)$ &       &       \\
$ 5$ & 2 & $$\xmarkc$$ & $(( 5, 7), 2)$ &       &       \\
$ 7$ & 4 & $$\xmarkc$$ & $(( 7,11), 4)$ &       &       \\
$11$ & 6 & $$\xmarkc$$ & $((11, 5), 6)$ &       &       \\
$ 5$ & 8 & $$\xmarkc$$ & $(( 5, 7), 8)$ &       &       \\
$ 7$ & 10 & $$\xmarkc$$ & $(( 7, 2),10)$ &       &       \\
$ 2$ & 12 & $$\xmarkc$$ & $(( 2, 0),12)$ &       &       \\
$ 2$ & 1 & $$\xmarkc$$ & $(( 2, 9), 1)$ &       &       \\
$ 9$ & 3 & $$\xmarkc$$ & $(( 9,14), 3)$ &       &       \\
$14$ & 5 & $$\cmarkc$$  &            & $(14, 5)$ &       \\
$ 2$ & 7 & $$\xmarkc$$ & $(( 2, 9), 7)$ &       &       \\
$ 9$ & 9 & $$\xmarkc$$ & $(( 9,11), 9)$ &       &       \\
$11$ & 11 & $$\xmarkc$$ & $((11, 1),11)$ &       &       \\
$ 1$ & 13 & $$\cmarkc$$  &            & $( 1,13)$ &       \\
\bottomrule 
\end{tabular}
\end{center}

\item 
Überprüfe ob $S$ leer ist.
$\Rightarrow$ Nein, mache weiter.

\item 
Sortiere $S$ lexikografisch.
\begin{center}
\small\begin{tabular}{lrrrrrrrrrrrr}
    \toprule 
    $S$ & \textcolor{gray}{0} & \textcolor{gray}{1} & \textcolor{gray}{2} & \textcolor{gray}{3} & \textcolor{gray}{4} & \textcolor{gray}{5} & \textcolor{gray}{6} & \textcolor{gray}{7} & \textcolor{gray}{8} & \textcolor{gray}{9} & \textcolor{gray}{10} & \textcolor{gray}{11}\\
    \midrule 
    $c_0$ & $2$ & $2$ & $2$ & $5$ & $5$ & $7$ & $7$ & $9$ & $9$ & $11$ & $11$ & $11$ \\
    $c_1$ & $0$ & $9$ & $9$ & $7$ & $7$ & $2$ & $11$ & $11$ & $14$ & $1$ & $5$ & $5$ \\
    $i$ & $12$ & $1$ & $7$ & $2$ & $8$ & $10$ & $4$ & $9$ & $3$ & $11$ & $0$ & $6$ \\
    \bottomrule 
\end{tabular}
\end{center}

\item 
Benenne die Paare in $S$ lexikografische um.
\begin{center}
\small\begin{tabular}{lrrrrrrrrrrrr}
    \toprule 
    $U$ & \textcolor{gray}{0} & \textcolor{gray}{1} & \textcolor{gray}{2} & \textcolor{gray}{3} & \textcolor{gray}{4} & \textcolor{gray}{5} & \textcolor{gray}{6} & \textcolor{gray}{7} & \textcolor{gray}{8} & \textcolor{gray}{9} & \textcolor{gray}{10} & \textcolor{gray}{11}\\
    \midrule 
    $c$ & $2$ & $3$ & $3$ & $5$ & $5$ & $7$ & $8$ & $9$ & $10$ & $11$ & $12$ & $12$ \\
    $i$ & $12$ & $1$ & $7$ & $2$ & $8$ & $10$ & $4$ & $9$ & $3$ & $11$ & $0$ & $6$ \\
    \bottomrule 
\end{tabular}
\end{center}
\item 
Beginne Iteration $k = 2$.

\item 
Markiere alle einzigartigen Namen in U.
\begin{center}
\small\begin{tabular}{lrrrrrrrrrrrr}
    \toprule 
    $U$ & \textcolor{gray}{0} & \textcolor{gray}{1} & \textcolor{gray}{2} & \textcolor{gray}{3} & \textcolor{gray}{4} & \textcolor{gray}{5} & \textcolor{gray}{6} & \textcolor{gray}{7} & \textcolor{gray}{8} & \textcolor{gray}{9} & \textcolor{gray}{10} & \textcolor{gray}{11}\\
    \midrule 
    $c$ & $2$ & $3$ & $3$ & $5$ & $5$ & $7$ & $8$ & $9$ & $10$ & $11$ & $12$ & $12$ \\
    $i$ & $12$ & $1$ & $7$ & $2$ & $8$ & $10$ & $4$ & $9$ & $3$ & $11$ & $0$ & $6$ \\
    $Uniq.$ & $$\cmarkc$$ & $$\xmarkc$$ & $$\xmarkc$$ & $$\xmarkc$$ & $$\xmarkc$$ & $$\cmarkc$$ & $$\cmarkc$$ & $$\cmarkc$$ & $$\cmarkc$$ & $$\cmarkc$$ & $$\xmarkc$$ & $$\xmarkc$$ \\
    \bottomrule 
\end{tabular}
\end{center}

\item 
Merge P in die Sequenz U und setze P zurück. Die Tupel sind dabei immer einzigartig.
\begin{center}
\small\begin{tabular}{lrr}
    \toprule 
    $P$ & \textcolor{gray}{0} & \textcolor{gray}{1}\\
    \midrule 
    $c$ & $14$ & $1$ \\
    $i$ & $5$ & $13$ \\
    \bottomrule 
\end{tabular}
\end{center}
\begin{center}
$\Rightarrow$
\end{center}
\begin{center}
\small\begin{tabular}{lrrrrrrrrrrrrrr}
    \toprule 
    $U$ & \textcolor{gray}{0} & \textcolor{gray}{1} & \textcolor{gray}{2} & \textcolor{gray}{3} & \textcolor{gray}{4} & \textcolor{gray}{5} & \textcolor{gray}{6} & \textcolor{gray}{7} & \textcolor{gray}{8} & \textcolor{gray}{9} & \textcolor{gray}{10} & \textcolor{gray}{11} & \textcolor{gray}{12} & \textcolor{gray}{13}\\
    \midrule 
    $c$ & $1$ & $2$ & $3$ & $3$ & $5$ & $5$ & $7$ & $8$ & $9$ & $10$ & $11$ & $12$ & $12$ & $14$ \\
    $i$ & $13$ & $12$ & $1$ & $7$ & $2$ & $8$ & $10$ & $4$ & $9$ & $3$ & $11$ & $0$ & $6$ & $5$ \\
    $Uniq.$ & $$\cmarkc$$ & $$\cmarkc$$ & $$\xmarkc$$ & $$\xmarkc$$ & $$\xmarkc$$ & $$\xmarkc$$ & $$\cmarkc$$ & $$\cmarkc$$ & $$\cmarkc$$ & $$\cmarkc$$ & $$\cmarkc$$ & $$\xmarkc$$ & $$\xmarkc$$ & $$\cmarkc$$ \\
    \bottomrule 
\end{tabular}
\end{center}

\item 
Die Zuordnung der Namen zu Präfixen für diesen Schritt ist somit:
\begin{center}
\small\begin{tabular}{rl}
\toprule 
Name & Präfix \\
\midrule 
  $1$ & $a\$\$\$$\\
  $2$ & $aa\$\$$\\
  $3$ & $aaba$\\
  $5$ & $abac$\\
  $7$ & $acaa$\\
  $8$ & $acca$\\
\bottomrule 
\end{tabular}
\small\begin{tabular}{rl}
\toprule 
Name & Präfix \\
\midrule 
  $9$ & $baca$\\
  $10$ & $bacc$\\
  $11$ & $caa\$$\\
  $12$ & $caab$\\
  $14$ & $ccaa$\\
  &\\
\bottomrule 
\end{tabular}
\end{center}

\item 
Sortiere $U$ anhand $(i \mod 2^k, i \div 2^k)$.
\begin{center}
\small\begin{tabular}{lrrrrrrrrrrrrrr}
    \toprule 
    $U$ & \textcolor{gray}{0} & \textcolor{gray}{1} & \textcolor{gray}{2} & \textcolor{gray}{3} & \textcolor{gray}{4} & \textcolor{gray}{5} & \textcolor{gray}{6} & \textcolor{gray}{7} & \textcolor{gray}{8} & \textcolor{gray}{9} & \textcolor{gray}{10} & \textcolor{gray}{11} & \textcolor{gray}{12} & \textcolor{gray}{13}\\
    \midrule 
    $c$ & $12$ & $8$ & $5$ & $2$ & $3$ & $14$ & $9$ & $1$ & $5$ & $12$ & $7$ & $10$ & $3$ & $11$ \\
    $i$ & $0$ & $4$ & $8$ & $12$ & $1$ & $5$ & $9$ & $13$ & $2$ & $6$ & $10$ & $3$ & $7$ & $11$ \\
    $Uniq.$ & $$\xmarkc$$ & $$\cmarkc$$ & $$\xmarkc$$ & $$\cmarkc$$ & $$\xmarkc$$ & $$\cmarkc$$ & $$\cmarkc$$ & $$\cmarkc$$ & $$\xmarkc$$ & $$\xmarkc$$ & $$\cmarkc$$ & $$\cmarkc$$ & $$\xmarkc$$ & $$\cmarkc$$ \\
    \bottomrule 
\end{tabular}
\end{center}

\item 
Iteriere durch U.

\begin{center}
\small\begin{tabular}{rrcccc}
\toprule 
 $U_c$ & $U_i$ & $U_{uniq}$ &     $S$      &   $P$   &   $F$   \\
\midrule 
$12$ & 0 & $$\xmarkc$$ & $((12, 8), 0)$ &       &       \\
$ 8$ & 4 & $$\cmarkc$$  &            &       & $( 8, 4)$\\
$ 5$ & 8 & $$\xmarkc$$ & $(( 5, 2), 8)$ &       &       \\
$ 2$ & 12 & $$\cmarkc$$  &            &       & $( 2,12)$\\
$ 3$ & 1 & $$\xmarkc$$ & $(( 3,14), 1)$ &       &       \\
$14$ & 5 & $$\cmarkc$$  &            &       & $(14, 5)$\\
$ 9$ & 9 & $$\cmarkc$$  &            &       & $( 9, 9)$\\
$ 1$ & 13 & $$\cmarkc$$  &            &       & $( 1,13)$\\
$ 5$ & 2 & $$\xmarkc$$ & $(( 5,12), 2)$ &       &       \\
$12$ & 6 & $$\xmarkc$$ & $((12, 7), 6)$ &       &       \\
$ 7$ & 10 & $$\cmarkc$$  &            & $( 7,10)$ &       \\
$10$ & 3 & $$\cmarkc$$  &            &       & $(10, 3)$\\
$ 3$ & 7 & $$\xmarkc$$ & $(( 3,11), 7)$ &       &       \\
$11$ & 11 & $$\cmarkc$$  &            &       & $(11,11)$\\
\bottomrule 
\end{tabular}
\end{center}
\item 
$F$ nach diesem Schritt:
\begin{center}
\small\begin{tabular}{lrrrrrrr}
    \toprule 
    $F$ & \textcolor{gray}{0} & \textcolor{gray}{1} & \textcolor{gray}{2} & \textcolor{gray}{3} & \textcolor{gray}{4} & \textcolor{gray}{5} & \textcolor{gray}{6}\\
    \midrule 
    $c$ & $8$ & $2$ & $14$ & $9$ & $1$ & $10$ & $11$ \\
    $i$ & $4$ & $12$ & $5$ & $9$ & $13$ & $3$ & $11$ \\
    \bottomrule 
\end{tabular}
\end{center}

\item 
Überprüfe ob $S$ leer ist.
$\Rightarrow$ Nein, mache weiter.

\item 
Sortiere $S$ lexikografisch.
\begin{center}
\small\begin{tabular}{lrrrrrr}
    \toprule 
    $S$ & \textcolor{gray}{0} & \textcolor{gray}{1} & \textcolor{gray}{2} & \textcolor{gray}{3} & \textcolor{gray}{4} & \textcolor{gray}{5}\\
    \midrule 
    $c_0$ & $3$ & $3$ & $5$ & $5$ & $12$ & $12$ \\
    $c_1$ & $11$ & $14$ & $2$ & $12$ & $7$ & $8$ \\
    $i$ & $7$ & $1$ & $8$ & $2$ & $6$ & $0$ \\
    \bottomrule 
\end{tabular}
\end{center}

\item 
Benenne die Paare in $S$ lexikografische um.
\begin{center}
\small\begin{tabular}{lrrrrrr}
    \toprule 
    $U$ & \textcolor{gray}{0} & \textcolor{gray}{1} & \textcolor{gray}{2} & \textcolor{gray}{3} & \textcolor{gray}{4} & \textcolor{gray}{5}\\
    \midrule 
    $c$ & $3$ & $4$ & $5$ & $6$ & $12$ & $13$ \\
    $i$ & $7$ & $1$ & $8$ & $2$ & $6$ & $0$ \\
    \bottomrule 
\end{tabular}
\end{center}
\item 
Beginne Iteration $k = 3$.

\item 
Markiere alle einzigartigen Namen in U.
\begin{center}
\small\begin{tabular}{lrrrrrr}
    \toprule 
    $U$ & \textcolor{gray}{0} & \textcolor{gray}{1} & \textcolor{gray}{2} & \textcolor{gray}{3} & \textcolor{gray}{4} & \textcolor{gray}{5}\\
    \midrule 
    $c$ & $3$ & $4$ & $5$ & $6$ & $12$ & $13$ \\
    $i$ & $7$ & $1$ & $8$ & $2$ & $6$ & $0$ \\
    $Uniq.$ & $$\cmarkc$$ & $$\cmarkc$$ & $$\cmarkc$$ & $$\cmarkc$$ & $$\cmarkc$$ & $$\cmarkc$$ \\
    \bottomrule 
\end{tabular}
\end{center}

\item 
Merge P in die Sequenz U und setze P zurück.
\begin{center}
\small\begin{tabular}{lr}
    \toprule 
    $P$ & \textcolor{gray}{0}\\
    \midrule 
    $c$ & $7$ \\
    $i$ & $10$ \\
    \bottomrule 
\end{tabular}
~$\Rightarrow$~
\small\begin{tabular}{lrrrrrrr}
    \toprule 
    $U$ & \textcolor{gray}{0} & \textcolor{gray}{1} & \textcolor{gray}{2} & \textcolor{gray}{3} & \textcolor{gray}{4} & \textcolor{gray}{5} & \textcolor{gray}{6}\\
    \midrule 
    $c$ & $3$ & $4$ & $5$ & $6$ & $7$ & $12$ & $13$ \\
    $i$ & $7$ & $1$ & $8$ & $2$ & $10$ & $6$ & $0$ \\
    $Uniq.$ & $$\cmarkc$$ & $$\cmarkc$$ & $$\cmarkc$$ & $$\cmarkc$$ & $$\cmarkc$$ & $$\cmarkc$$ & $$\cmarkc$$ \\
    \bottomrule 
\end{tabular}
\end{center}

\item 
Die Zuordnung der Namen zu Präfixen für diesen Schritt ist somit:
\begin{center}
\small\begin{tabular}{rl}
\toprule 
Name & Präfix \\
\midrule 
  $3$ & $aabacaa\$$\\
  $4$ & $aabaccaa$\\
  $5$ & $abacaa\$\$$\\
  $6$ & $abaccaab$\\
\bottomrule 
\end{tabular}
\small\begin{tabular}{rl}
\toprule 
Name & Präfix \\
\midrule 
  $7$ & $acaa\$\$\$\$$\\
  $12$ & $caabacaa$\\
  $13$ & $caabacca$\\
  &\\
\bottomrule 
\end{tabular}
\end{center}

\item 
Sortiere $U$ anhand $(i \mod 2^k, i \div 2^k)$.
\begin{center}
\small\begin{tabular}{lrrrrrrr}
    \toprule 
    $U$ & \textcolor{gray}{0} & \textcolor{gray}{1} & \textcolor{gray}{2} & \textcolor{gray}{3} & \textcolor{gray}{4} & \textcolor{gray}{5} & \textcolor{gray}{6}\\
    \midrule 
    $c$ & $13$ & $5$ & $4$ & $6$ & $7$ & $12$ & $3$ \\
    $i$ & $0$ & $8$ & $1$ & $2$ & $10$ & $6$ & $7$ \\
    $Uniq.$ & $$\cmarkc$$ & $$\cmarkc$$ & $$\cmarkc$$ & $$\cmarkc$$ & $$\cmarkc$$ & $$\cmarkc$$ & $$\cmarkc$$ \\
    \bottomrule 
\end{tabular}
\end{center}

\item 
Iteriere durch U.

\begin{center}
\small\begin{tabular}{rrcccc}
\toprule 
 $U_c$ & $U_i$ & $U_{uniq}$ &     $S$      &   $P$   &   $F$   \\
\midrule 
$13$ & 0 & $$\cmarkc$$  &            &       & $(13, 0)$\\
$ 5$ & 8 & $$\cmarkc$$  &            &       & $( 5, 8)$\\
$ 4$ & 1 & $$\cmarkc$$  &            &       & $( 4, 1)$\\
$ 6$ & 2 & $$\cmarkc$$  &            &       & $( 6, 2)$\\
$ 7$ & 10 & $$\cmarkc$$  &            &       & $( 7,10)$\\
$12$ & 6 & $$\cmarkc$$  &            &       & $(12, 6)$\\
$ 3$ & 7 & $$\cmarkc$$  &            &       & $( 3, 7)$\\
\bottomrule 
\end{tabular}
\end{center}
\item 
$F$ nach diesem Schritt:
\begin{center}
\small\begin{tabular}{lrrrrrrrrrrrrrr}
    \toprule 
    $F$ & \textcolor{gray}{0} & \textcolor{gray}{1} & \textcolor{gray}{2} & \textcolor{gray}{3} & \textcolor{gray}{4} & \textcolor{gray}{5} & \textcolor{gray}{6} & \textcolor{gray}{7} & \textcolor{gray}{8} & \textcolor{gray}{9} & \textcolor{gray}{10} & \textcolor{gray}{11} & \textcolor{gray}{12} & \textcolor{gray}{13}\\
    \midrule 
    $c$ & $8$ & $2$ & $14$ & $9$ & $1$ & $10$ & $11$ & $13$ & $5$ & $4$ & $6$ & $7$ & $12$ & $3$ \\
    $i$ & $4$ & $12$ & $5$ & $9$ & $13$ & $3$ & $11$ & $0$ & $8$ & $1$ & $2$ & $10$ & $6$ & $7$ \\
    \bottomrule 
\end{tabular}
\end{center}

\item 
Überprüfe ob $S$ leer ist.
$\Rightarrow$ Ja, beende Algorithmus.
\item 
Sortiere $F$ lexikografisch.
\begin{center}
\small\begin{tabular}{lrrrrrrrrrrrrrr}
    \toprule 
    $F$ & \textcolor{gray}{0} & \textcolor{gray}{1} & \textcolor{gray}{2} & \textcolor{gray}{3} & \textcolor{gray}{4} & \textcolor{gray}{5} & \textcolor{gray}{6} & \textcolor{gray}{7} & \textcolor{gray}{8} & \textcolor{gray}{9} & \textcolor{gray}{10} & \textcolor{gray}{11} & \textcolor{gray}{12} & \textcolor{gray}{13}\\
    \midrule 
    $c$ & $1$ & $2$ & $3$ & $4$ & $5$ & $6$ & $7$ & $8$ & $9$ & $10$ & $11$ & $12$ & $13$ & $14$ \\
    $i$ & $13$ & $12$ & $7$ & $1$ & $8$ & $2$ & $10$ & $4$ & $9$ & $3$ & $11$ & $6$ & $0$ & $5$ \\
    \bottomrule 
\end{tabular}
\end{center}

\item 
Gib finales Suffix Array aus:
\begin{center}
\small\begin{tabular}{rl}
\toprule 
 Index & Suffix \\
\midrule 
  $13$ & $a$ \\
  $12$ & $aa$ \\
  $ 7$ & $aabacaa$ \\
  $ 1$ & $aabaccaabacaa$ \\
  $ 8$ & $abacaa$ \\
  $ 2$ & $abaccaabacaa$ \\
  $10$ & $acaa$ \\
  $ 4$ & $accaabacaa$ \\
  $ 9$ & $bacaa$ \\
  $ 3$ & $baccaabacaa$ \\
  $11$ & $caa$ \\
  $ 6$ & $caabacaa$ \\
  $ 0$ & $caabaccaabacaa$ \\
  $ 5$ & $ccaabacaa$ \\
\bottomrule 
\end{tabular}\\
\end{center}
\end{enumerate}
