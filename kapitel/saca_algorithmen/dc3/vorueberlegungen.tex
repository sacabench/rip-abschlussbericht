\subsection{Vorüberlegungen}
\label{dc3:vorueberlegungen}

\newtheorem{example}{Beispiel}
Der \emph{DC3}-Algorithmus baut auf dem \emph{Divide-and-Conquer} Prinzip auf. Diese Art von Algorithmen teilen das Hauptproblem rekursiv in kleinere Teilprobleme auf, bis diese leichter zu lösen sind. Im Anschluss werden die Teilprobleme zu einer Gesamtlösung zusammengefügt.

Das zweite Prinzip, das sich hinter dem \emph{DC3}-Algorithmus verbirgt, ist das sogenannte \emph{Difference Cover}. Diesem Prinzip verdankt der Algorithmus auch seinen Namen.

\begin{definition}[Difference Cover]
	\label{def:differenceCover}
	Eine Menge \(D\) $\subseteq$ \([0,v)\) ist ein Difference Cover \(modulo\) \(v\), wenn die Werte in \([0,v)\) als Differenz zweier Werte aus \(D\) $\subseteq$ \([0,v)\) ausgedrückt werden können. Anders dargestellt:\\ \\
	$\makebox[\linewidth]{\{(i - j) \(modulo\) \(v\) $\mid$ $i,j$ $\in$ \(D\)\} $=$ \([0,v)\)}$
			
\end{definition}

\begin{example}
	Ein Beispiel für das Difference Cover \(modulo\) \(7\):\\
	\(D\) $=$ \{1, 2, 4\} 
	\begin{table}[!htbp]
		\centering
		\begin{tabular}{lllll}
			1 - 1 = 0 &  & 1 - 4 = -3 $\equiv$ 4 &  & (\(\mod\) \(7\))\\
			2 - 1 = 1 &  & 2 - 4 = -2 $\equiv$ 5 &  & (\(\mod\) \(7\))\\
			4 - 2 = 2 &  & 1 - 2 = -1 $\equiv$ 6 &  & (\(\mod\) \(7\))\\
			4 - 1 = 3 &  &                       &  & 
		\end{tabular}
	\end{table}
	
	Das bedeutet, dass wir mit der Menge \(D\) $=$ \{1, 2, 4\}, die Zahlen aus \([0,7)\) mit der Hilfe von Differenzen zweier Werte aus \(D\) ausdrücken können.
	
	Ein Beispiel für das Difference Cover \(modulo\) \(3\): \(D\) $=$ \{1, 2\} 
\end{example}
