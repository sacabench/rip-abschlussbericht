\subsubsection{Dritte Phase - Merge}
\label{dc3:algorithmus:phase3}

\crefname{enumi}{Fall}{Fällen}

Wir haben aus den ersten beiden Phasen die jeweils sortierten Suffixe des \emph{Difference Covers} $\sa_{12}$ und diejenigen, die nicht in dem \emph{Difference Covers} $\sa_{0}$ sind, vorliegen. Im letzten Schritt müssen diese beiden Mengen vereinigt werden, um an das Suffix-Array \sa zu gelangen. Dabei nutzen wir aus, dass die jeweiligen Mengen bereits sortiert sind. Somit vergleichen wir immer das kleinste Suffix aus der Menge $\sa_{12}$ mit dem kleinsten aus der Menge $\sa_{0}$. Dabei können bei dem Vergleich von $\sa_{12}[i]$ und $\sa_{0}[j]$ folgende vier Fälle auftreten.
\begin{enumerate}
	\item $\sa_{12}[i] \text{ modulo } 3 = 1$ \label{option1}
	\item $\sa_{12}[i] \text{ modulo } 3 = 2$ \label{option2}
	\item $i$ > length($\sa_{12}$) \label{option3}
	\item $j$ > length($\sa_{0}$) \label{option4}
\end{enumerate}

Tritt \cref{option1} ein, bedeutet das, dass ein Suffix aus $\sa_{0}$ mit einem Suffix aus $\sa_{12}$, dessen Startposition $\sa_{12}[i] \text{ modulo } 3 = 1$ ist, miteinander verglichen wird. Für diesen Vergleich benötigen wir die Paare $(\inputtext[\sa_{12}[i]], \isa_{12}[\sa_{12}[i]+1])$ und $(\inputtext[\sa_{0}[j]], \isa_{12}[\sa_{0}[j]+1])$. Wenn wir diese zwei Paare ermittelt haben, lassen sie sich miteinander vergleichen. Der kleinere Index von beiden wird dann dem Suffix-Array \sa hinzugefügt und der andere Index wird mit dem nächsten verglichen. Bei \cref{option1} wird ein Zeichen und der Rang des darauffolgenden Suffix miteinander verglichen. Dies funktioniert, da sowohl der Rang der Position nach $i \text{ modulo } 3 = 0$ als auch der Rang des Suffixes nach $i \text{ modulo } 3 = 1$ bekannt ist. \\
Tritt \cref{option2} ein, bedeutet das, dass ein Suffix aus $\sa_{0}$ mit einem Suffix aus $\sa_{12}$, dessen Startposition $\sa_{12}[i] \text{ modulo } 3 = 2$ ist, miteinander verglichen wird. Für diesen Vergleich stellen wir - anders als in \cref{option1} - die Triplets $(\inputtext[\sa_{12}[i]]$, $\inputtext[\sa_{12}[i]+1]$, $\isa_{12}[\sa_{12}[i]+2])$ und $(\inputtext[\sa_{0}[j]]$, $\inputtext[\sa_{0}[j]+1]$, $\isa_{12}[\sa_{0}[j]+2])$ auf. Wie zuvor werden diese beiden Triplets nun wieder verglichen und der Index des kleineren Triplets dem Suffix-Array angehangen. Bei \cref{option2} werden zwei Zeichen und der Rang des Suffix, das zwei Zeichen später beginnt, miteinander verglichen. Dies funktioniert, da sowohl der Rang des Suffixes zwei Positionen nach $j \text{ modulo } 3 = 0$ als auch der Rang zwei Positionen nach $i \text{ modulo } 3 = 2$ bekannt ist. Ein Zeichen und der darauffolgende Rang - wie in \cref{option1} - reicht für einen Vergleich nicht aus, da der Rang nach $i \text{ modulo } 3 = 2$ eine Position startend in $i+1 \text{ modulo } 3 = 0$ ist und somit nicht miteinander vergleichbar ist.\\
Bei den \cref{option3,option4} sind eines der beiden Mengen $\sa_{0}$ oder $\sa_{12}$ abgearbeitet und der Rest des Suffix-Arrays kann mit den jeweils restlichen Indizes der übrig gebliebenen Menge aufgefüllt werden.\\
Sind beide Mengen durchlaufen worden, haben wir am Ende ein vollständiges Suffix-Array und der Algorithmus ist terminiert.

