\subsubsection{Zweite Phase - Induzierung}
\label{dc3:algorithmus:phase2}

Wir haben nun in der ersten Phase die eindeutige Reihenfolge der Suffixe aus dem \emph{Difference Cover} bestimmt. Jetzt können wir das Prinzip des \emph{Difference Covers} anwenden und die Reihenfolge der restlichen Suffixe beginnend in Position $i \text{ modulo } 3 = 0$ induzieren. Dafür benötigen wir die Ränge der Suffixe aus dem \emph{Difference Cover}, die sich aus den lexikographischen Namen der ersten Phase ableiten lassen. Also berechnen wir das inverse Suffix-Array $\isa_{12}$, das die jeweiligen Ränge repräsentiert.
\begin{center}
	$\sa_{12}[i] = j$ genau dann, wenn $\isa_{12}[j] = i$
\end{center}
Jetzt lassen sich Paare aufstellen, die sich aus einem Zeichen $\inputtext[i]$ des Aus\-gangs\-text\-es und dem Rang des darauffolgenden Suffixes zusammensetzen, wobei $i \text{ modulo } 3 = 0$ ist. Nachdem alle Paare aufgestellt sind, können diese ebenfalls aufsteigend sortiert werden und anschließend die jeweiligen Positionen $i$ der Paare in ein Array $\sa_0$ abgespeichert werden. Somit haben wir auch die Suffixe aufsteigend sortiert, die nicht in dem \emph{Difference Cover} sind, und können zur dritten Phase übergehen.

Dieses Prinzip funktioniert, da die Suffixe startend in Position $i \text{ modulo } 3 = 0$ nur ein Zeichen am Anfang des Suffixes mehr aufweisen als die Suffixe beginnend in $i \text{ modulo } 3 = 1$ und der Rest gleich ist. Und die Suffixe in $i \text{ modulo } 3 = 1$ sind bereits in der ersten Phase eindeutig sortiert worden. Dieses hilft uns in der zweiten Phase weiter, denn dann reicht es aus, die jeweiligen Zeichen in $i \text{ modulo } 3 = 0$ zu vergleichen und bei Gleichheit die Ränge der Suffixe einer Position hinter den jeweiligen Zeichen anzuschauen. Dadurch können zwar Paare das gleiche Zeichen $\inputtext[i]$ aber niemals den gleichen Rang aufweisen und das führt zu einer eindeutigen Sortierung der Suffixe startend in Position $i \text{ modulo } 3 = 0$.