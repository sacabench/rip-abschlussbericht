\subsection{Einführung}
\label{dc3:einfuehrung}

Der \emph{DC3}-Algorithmus wird in der wissenschaftlichen Ausarbeitung von Juha Kärkkäinen und Peter Sanders wiederholt als simpler linearer Algorithmus zur Konstruktion eines Suffix-Arrays dargestellt. Der Wortlaut \emph{simpel} bezieht sich jedoch auf die Implementierung des Algorithmus. Diese benötigt nämlich le\-dig\-lich 50 Zeilen Code in der Programmiersprache \emph{C++}. Die Theorie, die sich dabei im Hintergrund abspielt, ist jedoch nicht ganz trivial. Daher werden wir uns mit der Theorie näher auseinandersetzen und diese anhand von Beispielen besser verdeutlichen. Dabei gehen wir am Ende auch auf Erweiterungen und Ver\-bes\-se\-rungs\-vor\-schlä\-ge ein und bewerten diese anhand von Laufzeit- und Speicherplatz-Messungen.
