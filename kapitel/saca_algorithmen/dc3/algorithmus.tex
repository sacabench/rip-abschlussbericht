\subsection{Algorithmus}
\label{dc3:algorithmus}

In dem vorherigen Kapitel \ref{dc3:vorueberlegungen} ist die Definition des \emph{Difference-Cover}-Prinzips näher erläutert worden. Nun wird beschrieben, wie dieses Verfahren eingesetzt werden kann, um ein Suffix-Array zu erhalten - wobei wir uns in diesem Kapitel nur mit dem \emph{Difference-Cover} modulo $3$ beschäftigen, da dieser auch in der wissenschaftlichen Ausarbeitung von Juha Kärkkäinen und Peter Sanders behandelt ist. Dieser wird dabei in drei Phasen aufgeteilt. Allgemein ausgedrückt werden in der ersten Phase die Suffixe mit Startpositionen aus dem \emph{Difference-Cover} \(D\) $=$ \{1, 2\} sortiert. In der zweiten Phase werden die restlichen Suffixe unter Verwendung des Ergebnisses des vorherigen Schrittes sortiert. Und - wie im \emph{Divide-and-Conquer} Prinzip üblich - werden in der dritten Phase die beiden sortierten Lösungen aus der ersten und zweiten Phase zusammengeführt, sodass am Ende eine sortierte Menge aller Suffixe des Ausgangstextes entsteht - das Suffix-Array.


\subsubsection{Erste Phase - Sortierung des \emph{Difference Covers}}
\label{dc3:algorithmus:phase1}

In dem ersten Schritt des Algorithmus werden die Substrings $\inputtext[i, i+2]$ sortiert, wobei $i$ ein Element aus dem \emph{Difference Cover} $D = \{1, 2\}$ ist. Das bedeutet, es werden alle Substrings der Länge drei - auch Triplets genannt - startend in den Positionen $i \text{ modulo } 3 = 1$, also $i = 1, 4, 7, 10,...$, und den Positionen $i \text{ modulo } 3 = 2$, also $i = 2, 5, 8, 11,...$, des Ausgangstextes \inputtext aufsteigend sortiert.
Anschließend werden den Triplets der Reihenfolge nach lexikographische Namen $t_i \in [0,\lceil2n/3\rceil)$ mit der Eigenschaft, dass $t_i < t_j$ wenn $\inputtext[i, i+2] < \inputtext[j, j+2]$ und $t_i = t_j$ wenn $\inputtext[i, i+2] = \inputtext[j, j+2]$, zugewiesen. Das bedeutet, dass das kleinste Triplet den lexikographisch kleinsten Namen $0$ erhält, das zweitkleinste Triplet die $1$ und so weiter. Sind mehrere Triplets gleich, das heißt, sowohl die erste, zweite und dritte Stelle des Substrings sind jeweils gleich, dann erhalten diese Triplets die gleichen lexikographischen Namen.
Wenn jedem Triplet ein lexikographischer Name zugeordnet worden ist, wird überprüft, ob diese Namen eindeutig sind. Wenn in dem Ausgangstext keine gleichen Triplets mit den beschriebenen Eigenschaften vorkommen, sind wir mit dem ersten Schritt des \emph{DC3}-Algorithmus fertig und können mit der zweiten Phase fortfahren. Kommen in dem Ausgangstext \inputtext jedoch gleiche Triplets startend in den Positionen $i \text{ modulo } 3 \neq 0$ vor, so können wir aktuell noch keine eindeutige Aussage über die Reihenfolge der gleichen Suffixe treffen. Um diese eindeutige Aussage treffen zu können, schreiben wir die lexikographischen Namen so um, dass die Ordnung der Triplets beibehalten wird. Dabei werden die Namen $t_i$ in die Mengen $i \text{ modulo } 3 = 1$ und $i \text{ modulo } 3 = 2$ aufgeteilt und diese zu einem neuen String $\mathsf{T}_{12}$ konkateniert. Mathematisch dargestellt, bedeutet das, dass der String
\begin{center}
	$\mathsf{T}_{12} = [t_i \mid i \text{ modulo } 3 = 1] \circ [t_i \mid i \text{ modulo } 3 = 2]$ 
\end{center}
als neuer Ausgangstext angesehen und der \emph{DC3}-Algorithmus auf diesem String ausgeführt wird, sodass wir nach erneutem Ausführen des Algorithmus die endgültige Sortierung der Triplets erhalten und mit dem zweiten Schritt fortfahren können.

Es gibt jedoch einen Spezialfall, bei dem dieser Schritt zu einem falschem Suffix-Array führen kann. Der Spezialfall tritt auf, wenn folgende drei Punkte gleichzeitig zutreffen:

$\begin{array}{ll}
1. & \text{ die Textlänge ist } n \text{ modulo } 3 = 1\\ 
2. & \text{ das Triplet an der Position } n - 3 \text{ kommt in dem Text mehrmals vor}\\ 
3. & \text{ das Triplet an der Position } n - 2 \text{ ist nicht das kleinste Triplet}
\end{array}$

Denn dann ist der String mit den lexikographischen Namen falsch, weil der Algorithmus nicht weiß, dass mit dem Triplet an der Position $n - 3$ das Ende des Textes erreicht ist.
Der Spezialfall tritt zum Beispiel bei dem Text $\inputtext = aabcabc$ auf. Der String mit den lexikographischen Namen wäre $t_{12} = [1, 1] \circ [3, 2]$, also $\mathsf{T}_{12} = 1132$. Wird der String in dieser Form rekursiv aufgerufen, wird am Ende des Algorithmus das Suffix $\inputtext$$[1, n)$ kleiner sein als das Suffix $\inputtext$$[4, n)$, weil das Triplet $113$ in der Rekursion kleiner ist als $132$. Um dagegen vorzubeugen, wird vor Beginn des Algorithmus dem Ausgangstext ein sogenanntes Dummy-Triplet angehangen, das aus Sentinels besteht. Dadurch wird dafür gesorgt, dass das Ende des Textes mit in die Sortierung eingeht.
\subsubsection{Zweite Phase - Induzierung}
\label{dc3:algorithmus:phase2}

Wir haben nun in der ersten Phase die eindeutige Reihenfolge der Suffixe aus dem \emph{Difference Cover} bestimmt. Jetzt können wir das Prinzip des \emph{Difference Covers} anwenden und die Reihenfolge der restlichen Suffixe beginnend in Position $i \text{ modulo } 3 = 0$ induzieren. Dafür benötigen wir die Ränge der Suffixe aus dem \emph{Difference Cover}, die sich aus den lexikographischen Namen der ersten Phase ableiten lassen. Also berechnen wir das inverse Suffix-Array $\isa_{12}$, das die jeweiligen Ränge repräsentiert.
\begin{center}
	$\sa_{12}[i] = j$ genau dann, wenn $\isa_{12}[j] = i$
\end{center}
Jetzt lassen sich Paare aufstellen, die sich aus einem Zeichen $\inputtext[i]$ des Aus\-gangs\-text\-es und dem Rang des darauffolgenden Suffixes zusammensetzen, wobei $i \text{ modulo } 3 = 0$ ist. Nachdem alle Paare aufgestellt sind, können diese ebenfalls aufsteigend sortiert werden und anschließend die jeweiligen Positionen $i$ der Paare in ein Array $\sa_0$ abgespeichert werden. Somit haben wir auch die Suffixe aufsteigend sortiert, die nicht in dem \emph{Difference Cover} sind, und können zur dritten Phase übergehen.

Dieses Prinzip funktioniert, da die Suffixe startend in Position $i \text{ modulo } 3 = 0$ nur ein Zeichen am Anfang des Suffixes mehr aufweisen als die Suffixe beginnend in $i \text{ modulo } 3 = 1$ und der Rest gleich ist. Und die Suffixe in $i \text{ modulo } 3 = 1$ sind bereits in der ersten Phase eindeutig sortiert worden. Dieses hilft uns in der zweiten Phase weiter, denn dann reicht es aus, die jeweiligen Zeichen in $i \text{ modulo } 3 = 0$ zu vergleichen und bei Gleichheit die Ränge der Suffixe einer Position hinter den jeweiligen Zeichen anzuschauen. Dadurch können zwar Paare das gleiche Zeichen $\inputtext[i]$ aber niemals den gleichen Rang aufweisen und das führt zu einer eindeutigen Sortierung der Suffixe startend in Position $i \text{ modulo } 3 = 0$.
\subsubsection{Dritte Phase - Merge}
\label{dc3:algorithmus:phase3}

\crefname{enumi}{Fall}{Fällen}

Wir haben aus den ersten beiden Phasen die jeweils sortierten Suffixe des \emph{Difference Covers} $\sa_{12}$ und diejenigen, die nicht in dem \emph{Difference Covers} $\sa_{0}$ sind, vorliegen. Im letzten Schritt müssen diese beiden Mengen vereinigt werden, um an das Suffix-Array \sa zu gelangen. Dabei nutzen wir aus, dass die jeweiligen Mengen bereits sortiert sind. Somit vergleichen wir immer das kleinste Suffix aus der Menge $\sa_{12}$ mit dem kleinsten aus der Menge $\sa_{0}$. Dabei können bei dem Vergleich von $\sa_{12}[i]$ und $\sa_{0}[j]$ folgende vier Fälle auftreten.
\begin{enumerate}
	\item $\sa_{12}[i] \text{ modulo } 3 = 1$ \label{option1}
	\item $\sa_{12}[i] \text{ modulo } 3 = 2$ \label{option2}
	\item $i$ > length($\sa_{12}$) \label{option3}
	\item $j$ > length($\sa_{0}$) \label{option4}
\end{enumerate}

Tritt \cref{option1} ein, bedeutet das, dass ein Suffix aus $\sa_{0}$ mit einem Suffix aus $\sa_{12}$, dessen Startposition $\sa_{12}[i] \text{ modulo } 3 = 1$ ist, miteinander verglichen wird. Für diesen Vergleich benötigen wir die Paare $(\inputtext[\sa_{12}[i]], \isa_{12}[\sa_{12}[i]+1])$ und $(\inputtext[\sa_{0}[j]], \isa_{12}[\sa_{0}[j]+1])$. Wenn wir diese zwei Paare ermittelt haben, lassen sie sich miteinander vergleichen. Der kleinere Index von beiden wird dann dem Suffix-Array \sa hinzugefügt und der andere Index wird mit dem nächsten verglichen. Bei \cref{option1} wird ein Zeichen und der Rang des darauffolgenden Suffix miteinander verglichen. Dies funktioniert, da sowohl der Rang der Position nach $i \text{ modulo } 3 = 0$ als auch der Rang des Suffixes nach $i \text{ modulo } 3 = 1$ bekannt ist. \\
Tritt \cref{option2} ein, bedeutet das, dass ein Suffix aus $\sa_{0}$ mit einem Suffix aus $\sa_{12}$, dessen Startposition $\sa_{12}[i] \text{ modulo } 3 = 2$ ist, miteinander verglichen wird. Für diesen Vergleich stellen wir - anders als in \cref{option1} - die Triplets $(\inputtext[\sa_{12}[i]]$, $\inputtext[\sa_{12}[i]+1]$, $\isa_{12}[\sa_{12}[i]+2])$ und $(\inputtext[\sa_{0}[j]]$, $\inputtext[\sa_{0}[j]+1]$, $\isa_{12}[\sa_{0}[j]+2])$ auf. Wie zuvor werden diese beiden Triplets nun wieder verglichen und der Index des kleineren Triplets dem Suffix-Array angehangen. Bei \cref{option2} werden zwei Zeichen und der Rang des Suffix, das zwei Zeichen später beginnt, miteinander verglichen. Dies funktioniert, da sowohl der Rang des Suffixes zwei Positionen nach $j \text{ modulo } 3 = 0$ als auch der Rang zwei Positionen nach $i \text{ modulo } 3 = 2$ bekannt ist. Ein Zeichen und der darauffolgende Rang - wie in \cref{option1} - reicht für einen Vergleich nicht aus, da der Rang nach $i \text{ modulo } 3 = 2$ eine Position startend in $i+1 \text{ modulo } 3 = 0$ ist und somit nicht miteinander vergleichbar ist.\\
Bei den \cref{option3,option4} sind eines der beiden Mengen $\sa_{0}$ oder $\sa_{12}$ abgearbeitet und der Rest des Suffix-Arrays kann mit den jeweils restlichen Indizes der übrig gebliebenen Menge aufgefüllt werden.\\
Sind beide Mengen durchlaufen worden, haben wir am Ende ein vollständiges Suffix-Array und der Algorithmus ist terminiert.


