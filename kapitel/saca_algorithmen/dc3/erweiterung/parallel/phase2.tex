\subparagraph*{Phase 2}
%Johannes
In der zweiten Phase des Algorithmus wird die Reihenfolge der Suffixe \suffix{i} bestimmt, wobei $i$ kein Element aus dem Difference Cover \(D\) $=$ \{1, 2\} ist. Wie bereits im Kapitel \ref{dc3:algorithmus:phase2} beschrieben worden ist, kann dieser Schritt mithilfe der Induzierung durchgeführt werden. Dafür werden Tupel bestehend aus einem Zeichen \inputtext[i] und dem Rang des nachfolgenden Suffixes \suffix{i+1} aufgestellt. Die jeweiligen Ränge sind dem zuvor berechneten inversen Suffix-Array zu entnehmen. Diese Tupel lassen sich anschließend sortieren.
Dementsprechend lässt sich die zweite Phase ebenfalls naiv parallelisieren. Für die Berechnung des inversen Suffix-Arrays lässt sich die Schleife mit \emph{OpenMP} parallelisieren. Die Schleife für die anschließende Aufstellung der Tupel parallelisieren wir ebenfalls mithilfe von \emph{OpenMP}. Für die Sortierung dieser Tupel verwenden wir - wie bereits in der ersten Phase - den parallelen Standard-Sortieralgorithmus der C++-Bibliothek.