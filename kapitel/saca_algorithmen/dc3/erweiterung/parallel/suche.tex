\subparagraph*{Parallele Suche}

In diesem Absatz stellen wir die \textit{parallele Suche} auf sortierten Arrays vor. \cite{merge:kruskal} Dabei handelt es sich um eine Verallgemeinerung der binären Suche auf eine beliebige Anzahl $p$ an Threads.

Sei $A$ das sortierte Eingabearray und $k$ der Schlüssel, nach dem wir suchen. Ähnlich wie bei der binären Suche wählen wir bestimmte Elemente in $A$ und vergleichen diese mit $k$, um zu entscheiden, ob wir links oder rechts von $x$ weitersuchen. Um dies effizient zu parallelisieren, wählen wir $p$ Elemente $x_i$ an den Positionen $\frac{n(i+1)}{p+1}$ aus mit $i = 0,...,p-1$. Dadurch werden $p+1$ Segmente in $A$ induziert. Wir prüfen für jedes Element $x_i$, ob $k < x_i$ gilt und speichern die Ergebnisse dieser Vergleichsoperationen in ein Array $C$ der Größe $p$.
Nun wollen wir entscheiden, in welchem Segment von $A$ sich der Schlüssel $k$ befindet. Für dieses Segment muss gelten, dass das Element an der rechten Grenze größer als $k$ ist und das Element an der linken Grenze kleiner als $k$ ist. Wir müssen in $C$ also nach der Position $i$ suchen für die gilt $C[i-1] = 0$ und $C[i] = 1$, falls $1 \le i \le p-1$. Dann ist das Segment durch das Intervall $[\frac{ni}{p+1}, \frac{n(i+1)}{p+1}]$ gegeben. Zusätzlich müssen noch die Sonderfälle für das erste und letzte Segment betrachtet werden: Ist $C[0] = 1$ ist das Segment, in dem sich $k$ befindet, durch $[0, \frac{n}{p+1}]$ gegeben und ist $C[p-1] = 0$, ist das Segment durch $[\frac{n(p-1)}{p+1}, |A|-1]$ gegeben. Diese Bedingungen lassen sich parallelisiert testen. 
Auf dem Segment lässt sich nun rekursiv die parallele Suche fortsetzen bis das Segment maximal $p$ Elemente enthält. Dann lässt sich parallelisiert jedes Element mit $k$ vergleichen und die Position von $k$ in $A$ wird zurückgegeben.

Da wir in jedem Rekursionsschritt das Eingabearray in $p+1$ Segmente aufteilen, bricht die Suche nach $log_{p+1}(n) = \frac{log(n)}{log(p+1)}$ Rekursionsschritten ab. Da in jedem Rekursionsschritt jeder Thread eine konstante Anzahl an Schritten ausführt, ist die Laufzeit in $O(\frac{log(n)}{log(p+1)})$.