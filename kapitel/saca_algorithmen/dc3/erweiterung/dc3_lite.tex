\subsubsection{DC3-Lite}
\label{dc3:lite}


Es \currentauthor{Nico Bertram} ist möglich den Speicherverbrauch des \emph{DC3} so weit zu reduzieren, dass nur ein zusätzliches Hilfs-Array $\mathsf{U}$ der Länge $n$ verwendet wird. Diese Variante des \emph{DC3} wurde in ~\cite{saca:10} vorgestellt und wird als Komponente im \emph{nzSufSort} \cref{algorithm:nzSufSort} verwendet. Wir bezeichnen diese Variante in unserem Framework als \emph{DC3-Lite}. \par
Im Folgenden gehen wir auf die Unterschiede der einzelnen Phasen zwischen dem \emph{DC3} und dem \emph{DC3-Lite} ein. Da das Induzieren analog zum \emph{DC3} ist, wird diese Phase hier nicht genauer betrachtet. 

\subparagraph*{Erste Phase}

Ähnlich wie im \emph{DC3} werden in der ersten Phase die Triplets $\mathsf{T}[i,i+2]$ sortiert und lexikographische Namen vergeben. Im Unterschied zum \emph{DC3} werden aber alle Positionen $i$ von $\mathsf{T}$ sortiert. Dies machen wir, um den Eingabetext mit den lexikographischen Rängen zu überschreiben, damit für $\mathsf{T}_0$ und $\mathsf{T}_{12}$ kein zusätzlicher Speicherbereich benötigt wird. Formell berechnet sich der überschriebene Text $\mathsf{T}_{\text{new}}$ durch 
\begin{center}
	$\mathsf{T}_{\text{new}} = [t_i | i \text{ modulo } 3 = 0] \circ [t_i | i \text{ modulo } 3 = 1] \circ [t_i | i \text{ modulo } 3 = 2]$ 
\end{center}
Dadurch bleiben die Informationen des ursprünglichen Textes erhalten, da in jedem lexikographischen Rang die Information des Zeichens an der betrachteten Position berücksichtigt wurde. In jedem Zeichen sind sogar mehr Informationen enthalten, da die beiden darauf folgenden Zeichen ebenfalls berücksichtigt werden. \par
Zunächst werden die Positionen von $\mathsf{T}$ in das Array $\mathsf{U}$ geschrieben und mithilfe der Inplace-Variante des Radixsort für große Alphabete aus \cref{sort:radix:big_alph} sortiert. Dabei wird der Speicherbereich für das $\sa$ als Bucket-Array verwendet. \par
Anschließend werden die lexikographischen Ränge vergeben und damit $\mathsf{T}_{\text{new}}$ berechnet. Damit für den rekursiven Aufruf die lexikographischen Ränge durch die Textlänge beschränkt sind, berechnen wir parallel auch $\mathsf{T}_{12}$ und schreiben dies in die Positionen von $\mathsf{T}_{\text{new}}$, welche den Positionen $i$ mit $i \text{ modulo } 3 = 1$ und $i \text{ modulo } 3 = 2$ entsprechen. Die Positionen $i$ von $\mathsf{T}_{\text{new}}$ mit $i \text{ modulo } 3 = 1$, werden vorher im ersten Drittel von $\mathsf{U}$ und die Positionen mit $i \text{ modulo } 3 = 2$ im ersten Drittel von $\sa$ zwischengespeichert. \par
Der rekursive Aufruf erfolgt dann mit $\mathsf{T}_{12}$ und dem nicht verwendeten Speicher von $\mathsf{U}$ und $\sa$. Dadurch wird das Suffix-Array $\sa_{12}$ von $\mathsf{T}_{12}$ berechnet und die zwischengespeicherten Positionen von $\mathsf{T}_{\text{new}}$ werden wieder zurückgeschrieben.


\subparagraph*{Dritte Phase}

Durch das Induzieren in der zweiten Phase wurde das $\sa_0$ berechnet. In dieser Phase werden nun $\sa_0$ und $\sa_{12}$ vereinigt. Dazu werden das $\isa_0$ und das $\isa_{12}$ hintereinander in $\mathsf{U}$ berechnet. Im Unterschied zum \emph{DC3} wird das vereinigte Suffix-Array nicht direkt in einen neuen Speicherbereich geschrieben, sondern die Einträge von $\sa_0$ und $\sa_{12}$ werden mit den Positionen im vereinigten Suffix-Array überschrieben. \par
Beim Vereinigen werden jeweils die Suffixe an den Positionen $\sa_0[i]$ und $\sa_{12}[j]$ miteinander verglichen. Wenn $\mathsf{T}[\sa_0[i]]$ und $\mathsf{T}[\sa_{12}[j]]$ ungleich sind, kann die Position im vereinigten Suffix-Array direkt bestimmt werden. Ansonsten kann die Ordnung der Suffixe durch Nachschauen in $\isa_0$ und $\isa_{12}$ bestimmt werden. Falls $\sa_{12}[i] \text{ modulo } 3 = 1$ gilt, kann die Ordnung der Suffixe durch einen Vergleich von $\isa_{12}[\sa_0[i]]$ und $\isa_{12}[p_2+\sa_{12}[j]]$ bestimmt werden, wobei $p_2$ die Startposition der Menge aller Positionen $i$ mit $i \text{ modulo } 3 = 2$ in $\isa_{12}$ ist. Falls $\sa_{12}[i] \text{ modulo } 3 = 2$ gilt, wird die Ordnung der Suffixe durch einen Vergleich von $\isa_{12}[p_2+\sa_0[i]]$ und $\isa_{12}[\sa_{12}[j]-p_2+1]$ bestimmt. \par
Anschließend werden $\isa_0$ und $\isa_{12}$ mit den berechneten Positionen im vereinigten Suffix-Array, die in $\sa_0$ und $\sa_{12}$ stehen, verknüpft. Genauer gesagt wird $\isa_0[i] = \sa_0[\isa_0[i]]$ und $\isa_{12}[i] = \sa_{12}[\isa_{12}[i]]$ gesetzt. Da $\isa_0$ und $\isa_{12}$ hintereinander in $\mathsf{U}$ gespeichert wurden, steht in $\mathsf{U}$ nun das inverse Suffix-Array $\isa_{\text{new}}$ von $\mathsf{T}_{\text{new}}$. Um $\sa_{\text{new}}$ zu berechnen wird das Inverse von $\mathsf{U}$ in $\sa$ berechnet. \par
Bis jetzt haben wir nur das Suffix-Array $\sa_{\text{new}}$ des überschriebenen Eingabetextes berechnet. Da wir die Positionen des ursprünglichen Eingabetextes umsortiert haben, müssen wir diese wieder in die korrekte Reihenfolge bringen. Dies lässt sich durch einen Durchlauf mit der folgenden Funktion erreichen, wobei $m_0 = |\mathsf{T}_0|$ und $m_{12} = |\mathsf{T}_{12}|$. \par
\begin{center}
	$\sa[i] =
   \begin{cases}
     3\sa_{\text{new}}[i] & \text{für } \sa_{\text{new}}[i] \in[0, m_0) \\
     3(\sa_{\text{new}}[i]-m_0)+1 & \text{für } \sa_{\text{new}}[i] \in [m_0 ,m_0+\lceil \frac{m_{12}}{2} \rceil)) \\
     3(\sa_{\text{new}}[i]-m_1)+2 & \text{für } \sa_{\text{new}}[i] \in [m_0+\lceil \frac{m_{12}}{2} \rceil),m_0+m_{12})
   \end{cases}$
\end{center}