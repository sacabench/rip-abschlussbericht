\subsubsection{Laufzeit}
\label{dc3:algorithmus:laufzeit}

In diesem Kapitel untersuchen wir die \emph{Worst-Case}-Laufzeit des \emph{DC3}-Al\-go\-rith\-mus, indem wir zuerst die Laufzeiten der einzelnen Phasen betrachten und diese am Ende zu einer Gesamtlaufzeit zusammenführen.
In der ersten Phase werden Triplets aufgestellt und diese anschließend sortiert. Für die Sortierung könnte zum Beispiel das Sortierverfahren \emph{Radix-Sort} verwendet werden. Der \emph{LSD-Radix-Sort}, der im Kapitel \ref{sort:radix:lsd} näher erläutert worden ist, benötigt eine Laufzeit von $\mathcal{O}(kn)$, wobei $k$ für die Länge der zu ver\-gleich\-en\-den Werte steht. In unserem Fall ist $k = 3$. Für die Vergabe der lexikographischen Namen wird ebenfalls eine lineare Laufzeit benötigt. Zusammengefasst benötigt die erste Phase eine Laufzeit von $T(n) = \mathcal{O}(n)$.

Die zweite Phase stellt Paare auf, welche erneut mit Hilfe von \emph{Radix-Sort} sortiert werden. Dies benötigt eine Laufzeit von $T(n) = \mathcal{O}(n)$.

In der dritten Phase werden beide Mengen $SA_{0}$ und $SA_{12}$ jeweils einmal durchlaufen und einfache Vergleiche ausgeführt. Dies führt ebenfalls zu einer Laufzeit von $T(n) = \mathcal{O}(n)$.

Also weist jede Phase eine Laufzeit von $\mathcal{O}(n)$ auf. Im \emph{Worst-Case} sind die lexikographischen Namen aus der ersten Phase jedoch nicht eindeutig, sodass ein rekursiver Aufruf erfolgt. Dabei wird der Algorithmus mit einer Textlänge von $2n/3$ aufgerufen. Zusammengefasst erhalten wir somit für den gesamten Algorithmus eine Laufzeit von $T(n) = \mathcal{O}(n) + T(2n/3)$. Da der Faktor $2/3$ kleiner als $1$ ist, lässt sich diese Rekursionsgleichung zu $T(n) = \mathcal{O}(n)$ auflösen. Somit ist die Laufzeit des \emph{DC3}-Algorithmus linear.