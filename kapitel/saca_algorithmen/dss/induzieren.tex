
\subsubsection{Induzieren von L- und S-Suffixen}



% Please add the following required packages to your document preamble:
% \usepackage[table,xcdraw]{xcolor}
% If you use beamer only pass "xcolor=table" option, i.e. \documentclass[xcolor=table]{beamer}
% Please add the following required packages to your document preamble:
% \usepackage{graphicx}
% Please add the following required packages to your document preamble:
% \usepackage{graphicx}
% \usepackage[table,xcdraw]{xcolor}
% If you use beamer only pass "xcolor=table" option, i.e. \documentclass[xcolor=table]{beamer}
\begin{table}
	\centering
	\resizebox{\textwidth}{!}{%
		\begin{tabular}{l|lllllllllllllll|l}
			$i$  & 0                          & 1  & 2                & 3                                       & 4                                       & 5                                       & 6                                       & 7                          & 8                         & 9 & 10 & 11 & 12 & 13 & 14 & $\mathsf{Bucket\_S}${[}a, a{]}       \\ \hline
			$\mathsf{SA}$ & 14                         & 2  & $\underline{10}$ & $\underline{4}$                         & 1                                       & 8                                       & 2                                       & $\underline{10}$           & $\underline{\textbf{4}}$  & 0 & 0  & 2  & 4  & 8  & 10 & 4                         \\ \hline
			$\mathsf{SA}$ & 14                         & 13 & $\underline{10}$ & $\underline{4}$                         & 1                                       & 8                                       & 2                                       & $\underline{\textbf{10}}$  &\cellcolor[HTML]{34CDF9}4 & 0 & 0  & 2  & 4  & 8  & 10 & 4                         \\ \hline
			$\mathsf{SA}$ & 14                         & 13 & $\underline{10}$ & $\underline{4}$                         & 1                                       & 8                                       & \textbf{2}                              & \cellcolor[HTML]{34CDF9}10 & 4                         & 0 & 0  & 2  & 4  & 8  & 10 & 4                         \\ \hline
			$\mathsf{SA}$ & 14                         & 13 & $\underline{10}$ & $\underline{4}$                         & \cellcolor[HTML]{32CB00}$\underline{1}$ & \textbf{8}                              & \cellcolor[HTML]{34CDF9}$\underline{2}$ & 10                         & 4                         & 0 & 0  & 2  & 4  & 8  & 10 & \cellcolor[HTML]{32CB00}3 \\ \hline
			$\mathsf{SA}$ & 14                         & 13 & $\underline{10}$ & \cellcolor[HTML]{32CB00}$\underline{7}$ & $\underline{\textbf{1}}$                & \cellcolor[HTML]{34CDF9}$\underline{8}$ & $\underline{2}$                         & 10                         & 4                         & 0 & 0  & 2  & 4  & 8  & 10 & \cellcolor[HTML]{32CB00}2 \\ \hline
			$\mathsf{SA}$ & 14                         & 13 & $\underline{10}$ & $\underline{\textbf{7}}$                & \cellcolor[HTML]{34CDF9}1               & $\underline{8}$                         & $\underline{2}$                         & 10                         & 4                         & 0 & 0  & 2  & 4  & 8  & 10 & 2                         \\ \hline
			$\mathsf{SA}$ & 14                         & 13 & $\underline{10}$ & \cellcolor[HTML]{34CDF9}7               & 1                                       & $\underline{8}$                         & $\underline{2}$                         & 10                         & 4                         & 0 & 0  & 2  & 4  & 8  & 10 & 2                         \\ \hline
			$\mathsf{SA}$ & \cellcolor[HTML]{32CB00}14 & 2  & $\underline{10}$ & 7                                       & 1                                       & $\underline{8}$                         & $\underline{2}$                         & 10                         & 4                         & 0 & 0  & 2  & 4  & 8  & 10 & 2 \\ \hline                       
		\end{tabular}%
	}
	\caption{Induzieren der S-Suffixe. Neu eingefügte Indizes sind grün hinterlegt, wohingegen blau hinterlegte Werte negiert wurden. Es wird immer der fett markierte Index zum Induzieren betrachtet.}
	\label{dss:table:induce-s}
\end{table}

% Please add the following required packages to your document preamble:
% \usepackage[table,xcdraw]{xcolor}
% If you use beamer only pass "xcolor=table" option, i.e. \documentclass[xcolor=table]{beamer}
\begin{table}
	\centering
	\resizebox{\textwidth}{!}{%
		\begin{tabular}{l|l|lllllllllllllll}
			Schritt & $i$  & 0           & 1                                   & 2                                   & 3          & 4          & 5                         & 6                         & 7           & 8          & 9                                       & 10                                      & 11                                       & 12                        & 13                        & 14                        \\ \hline
			0       & $\mathsf{SA}$ & \textbf{14} & 2                                   & $\underline{10}$                    & 7          & 1          & $\underline{8}$           & $\underline{2}$           & 10          & 4          & 0                                       & 0                                       & 2                                        & 4                         & 8                         & 10                        \\ \hline
			1       & $\mathsf{SA}$ & 14          & \cellcolor[HTML]{32CB00}\textbf{13} & $\underline{10}$                    & 7          & 1          & $\underline{8}$           & $\underline{2}$           & 10          & 4          & 0                                       & 0                                       & 2                                        & 4                         & 8                         & 10                        \\ \hline
			2       & $\mathsf{SA}$ & 14          & 13                                  & \cellcolor[HTML]{32CB00}\textbf{12} & 7          & 1          & $\underline{8}$           & $\underline{2}$           & 10          & 4          & 0                                       & 0                                       & 2                                        & 4                         & 8                         & 10                        \\ \hline
			3       & $\mathsf{SA}$ & 14          & 13                                  & 12                                  & \textbf{7} & 1          & $\underline{8}$           & $\underline{2}$           & 10          & 4          & 0                                       & 0                                       & \cellcolor[HTML]{32CB00}$\underline{11}$ & 4                         & 8                         & 10                        \\ \hline
			4       & $\mathsf{SA}$ & 14          & 13                                  & 12                                  & 7          & \textbf{1} & $\underline{8}$           & $\underline{2}$           & 10          & 4          & 0                                       & 0                                       & $\underline{11}$                         & \cellcolor[HTML]{32CB00}6 & 8                         & 10                        \\ \hline
			5       & $\mathsf{SA}$ & 14          & 13                                  & 12                                  & 7          & 1          & $\underline{\textbf{8}}$  & $\underline{2}$           & 10          & 4          & 0                                       & 0                                       & $\underline{11}$                         & 6                         & \cellcolor[HTML]{32CB00}0 & 10                        \\ \hline
			6       & $\mathsf{SA}$ & 14          & 13                                  & 12                                  & 7          & 1          & \cellcolor[HTML]{34CDF9}8 & $\underline{\textbf{2}}$  & 10          & 4          & 0                                       & 0                                       & $\underline{11}$                         & 6                         & 0                         & 10                        \\ \hline
			7       & $\mathsf{SA}$ & 14          & 13                                  & 12                                  & 7          & 1          & 8                         & \cellcolor[HTML]{34CDF9}2 & \textbf{10} & 4          & 0                                       & 0                                       & $\underline{11}$                         & 6                         & 0                         & 10                        \\ \hline
			8       & $\mathsf{SA}$ & 14          & 13                                  & 12                                  & 7          & 1          & 8                         & 2                         & 10          & \textbf{4} & \cellcolor[HTML]{32CB00}$\underline{9}$ & 0                                       & $\underline{11}$                         & 6                         & 0                         & 10                        \\ \hline
			9       & $\mathsf{SA}$ & 14          & 13                                  & 12                                  & 7          & 1          & 8                         & 2                         & 10          & 4          & $\underline{\textbf{9}}$                & \cellcolor[HTML]{32CB00}$\underline{3}$ & $\underline{11}$                         & 6                         & 0                         & 10                        \\ \hline
			10      & $\mathsf{SA}$ & 14          & 13                                  & 12                                  & 7          & 1          & 8                         & 2                         & 10          & 4          & \cellcolor[HTML]{34CDF9}9               & $\underline{\textbf{3}}$                & $\underline{11}$                         & 6                         & 0                         & 10                        \\ \hline
			11      & $\mathsf{SA}$ & 14          & 13                                  & 12                                  & 7          & 1          & 8                         & 2                         & 10          & 4          & 9                                       & \cellcolor[HTML]{34CDF9}3               & $\underline{\textbf{11}}$                & 6                         & 0                         & 10                        \\ \hline
			12      & $\mathsf{SA}$ & 14          & 13                                  & 12                                  & 7          & 1          & 8                         & 2                         & 10          & 4          & 9                                       & 3                                       & \cellcolor[HTML]{34CDF9}11               & \textbf{6}                & 0                         & 10                        \\ \hline
			13      & $\mathsf{SA}$ & 14          & 13                                  & 12                                  & 7          & 1          & 8                         & 2                         & 10          & 4          & 9                                       & 3                                       & 11                                       & 6                         & 0                         & \cellcolor[HTML]{32CB00}5 \\ \hline
		\end{tabular}%
	}
% Please add the following required packages to your document preamble:
% \usepackage{graphicx}
	\centering
	\begin{tabular}{l|l|l|l|l}
		Schritt & $\mathsf{Bucket\_L}${[}\${]} & $\mathsf{Bucket\_L}${[}a{]}          & $\mathsf{Bucket\_L}${[}b{]}           & $\mathsf{Bucket\_L}${[}c{]}           \\ \hline
		0       & 1                 & 1                         & 9                          & 11                         \\ \hline
		1       & 1                 & \cellcolor[HTML]{32CB00}2 & 9                          & 11                         \\ \hline
		2       & 1                 & \cellcolor[HTML]{32CB00}3 & 9                          & 11                         \\ \hline
		3       & 1                 & 3                         & 9                          & \cellcolor[HTML]{32CB00}12 \\ \hline
		4       & 1                 & 3                         & 9                          & \cellcolor[HTML]{32CB00}13 \\ \hline
		5       & 1                 & 3                         & 9                          & \cellcolor[HTML]{32CB00}14 \\ \hline
		6       & 1                 & 3                         & 9                          & 14                         \\ \hline
		7       & 1                 & 3                         & 9                          & 14                         \\ \hline
		8       & 1                 & 3                         & \cellcolor[HTML]{32CB00}10 & 14                         \\ \hline
		9       & 1                 & 3                         & \cellcolor[HTML]{32CB00}11 & 14                         \\ \hline
		10      & 1                 & 3                         & 11                         & 14                         \\ \hline
		11      & 1                 & 3                         & 11                         & 14                         \\ \hline
		12      & 1                 & 3                         & 11                         & 14                         \\ \hline
		13      & 1                 & 3                         & 11                         & \cellcolor[HTML]{32CB00}15 \\ \hline
	\end{tabular}%
	\caption{Induzieren der L-Suffixe. Grün markierte Indizes wurden neu eingefügt, fett hervorgehobene Indizes werden zum Induzieren verwendet und blau hinterlegte Werte wurden negiert.}
	\label{dss:table:induce-l}
\end{table}


Durch die Typen der Suffixe wissen wir, dass in beliebigen Buckets $\mathsf{b}_{c0,c1}$ L-Suffixe lexikografisch kleiner als S-Suffixe sowie RMS-Suffixe kleiner als S-Suffixe sind. Wir wissen ebenfalls, dass bei lexikografischer Ordnung alle direkt aneinandergereihten S-Suffixe links von mindestens einem RMS-Suffix sind (welches zu einem L-Suffix wechselt), da $\mathsf{S}_{n-1}$ ein L-Suffix ist. Ebenso sind (in lexikografischer Reihenfolge) alle L-Suffixe rechts von mindestens einem S-Suffix. Das Suffix-Array wird nun zwei Mal durchlaufen. Beim ersten Mal laufen wir von rechts nach links und induzieren alle S-Suffixe, beim zweiten Mal laufen wir von links nach rechts und induzieren die L-Suffixe.

Beim ersten Scan des Suffix-Arrays $\mathsf{SA}$ speichern wir jedes Mal den Eintrag $i-1$ an die rechteste freie Position im Bucket $\mathsf{b}_{c0,c1}$, wenn wir einen Eintrag $i > 0$ lesen. Falls $\mathsf{T}[i-2] > \mathsf{T}[i-1]$, so ist $\mathsf{S}_{i-2}$ ein L-Suffix und wir negieren den Wert von $i-1$ bitweise, da $\mathsf{S}_{i-2}$ in diesem Schritt nicht induziert wird. Bei diesem Durchlauf wird jeder Wert durch seinen bitweise negierten Wert überschrieben. Falls eine Stelle bereits bitweise negiert war, so wird sie im nächsten Scan betrachtet, da es induziert wurde und das dazugehörige Suffix ein L-Suffix ist (daher in diesem Durchlauf nicht relevant). Alle Suffixe, welche zur Induzierung verwendet wurden, haben ihre Position bitweise negiert, da sie ein S-Suffix induziert haben. Alle Anderen werden hingegen durch ihre Position repräsentiert und sind für den nächsten Durchlauf relevant. Alle induzierten Suffixe sind dabei lexikografisch kleiner als jene, von denen induziert wurde, da in diesem Durchlauf S-Suffixe betrachtet wurden und damit $c0 \leq c1$ für alle Buckets $\mathsf{b}_{c0,c1}$ gelten muss. Ebenso können wir nur in $\mathsf{b}_{c0,c1}$ mit $c1 \leq c0$ induzieren, da nur S-Suffixe betrachtet wurden. Für unser Beispiel können wir in Tabelle \ref{dss:table:induce-s} sehen, wie Schrittweise die Indizes für $\mathsf{b}_{a}$ überprüft und induziert werden. Der letzte zu prüfende Index 3 ist in $\mathsf{BUCK\-ET\_RMS}[a,b]$ gespeichert (s. Tabelle \ref{dss:table:last-buckets}).

Vor dem zweiten Durchlauf von links nach rechts wird $n-1$ an den Anfang des $\mathsf{T}[n-1]$-Buckets gesetzt. Falls $\mathsf{S}_{n-2}$ ein L-Suffix ist, so speichern wir $n-1$, da wir $\mathsf{S}_{n-2}$ induzieren wollen. Andernfalls soll der bitweise negierte Wert von $n-1$ abgespeichert werden. Nun kann der Durchlauf beginnen.
Liegt ein Eintrag $i < 0$ vor, so wurde dieser bereits an seine richtige Position gesetzt und bitweise negiert, damit die korrekte Position an dieser Stelle steht. Ist $i > 0$, so muss das Suffix $\mathsf{S}_{i-1}$ an die linkeste freie Position im Bucket $\mathsf{b}_{\mathsf{T}[i-1]}$ induziert werden. Alle übrigen Suffixe werden in diesem Durchlauf induziert, sodass die linke Grenze für $\mathsf{b}_{c0}$, welche in $\mathsf{BUCK\-ET\_L}[c0]$ gespeichert wird, ausreichend ist. Wenn das induzierte Suffix $\mathsf{S}_{i-1}$ ein S-Suffix induzieren würde, so wird stattdessen der bitweise negierte Wert induziert, da das Suffix beim scannen des Indizes übersprungen (nur negiert) wird. In Tabelle \ref{dss:table:induce-l} sind im oberen Teil die Schritte für die Indizes beim Induzieren dargestellt. Im unteren Teil werden die Änderungen der jeweiligen $\mathsf{BUCK\-ET\_L}$ nach Einfügen eines Suffixes aufgelistet.
