\subsubsection{RMS-Suffixe sortieren}
Das Sortieren der RMS-Suffixe kann in drei Schritte unterteilt werden. Im ersten Schritt werden die RMS-Teilstrings für jeden Bucket $\mathsf{b}_{c0,c1}$ sortiert. Schritt zwei besteht aus der Erzeugung eines partiellen inversen Suffix-Arrays $\mathsf{ISAb}$, in welchem die Ränge der nach den RMS-Teilstrings partiell sortierten RMS-Suffixe enthalten sind. Diese Ränge werden verwendet, um mit einem Verfahren, ähnlich zum Prefix-Doubling, die lexikografische Ordnung aller RMS-Suffixe zu bestimmen. Dies wird mit dem Ansatz der \textit{repetition detection} erweitert.

\begin{table}
	\begin{tabular}{l|l|l}
		Index & Referenzindex & RMS-Teilstring \\ \hline
		2     & 0             & abac           \\ \hline
		8     & 2             & abac           \\ \hline
		10    & 3             & acaa\$         \\ \hline
		4     & 1             & accaab         \\ \hline
	\end{tabular}
	\caption{Lexikografisch sortierte RMS-Teilstrings. Die Suffixe an Position 2 und 8 haben denselben Teilstring und sind daher nicht eindeutig sortiert.}
	\label{dss:table:substrings}
\end{table}

\subsubsection{Sortieren der RMS-Teilstrings}
Der aktuell nicht verwendete Bereich im Suffix-Array $\mathsf{SA}[m\dots n-m)$ dient als Buffer für das Sortieren der RMS-Teilstrings. Den Buffer referenzieren wir folgend als $\mathsf{buf}[i] = \mathsf{SA}[m+i]$ mit $0 \leq i < n - 2m$. In jedem $\mathsf{BUCK\-ET\_RMS}$ werden die RMS-Substrings sortiert. Dies bedeutet, dass mehrere $\mathsf{BUCK\-ET\_RMS}$ parallel sortiert werden können. Bei $p$ Prozessen erhält jeder Prozess eine Buffergröße von $\frac{|\mathsf{buf}|}{p}$. Die Anzahl der gleichzeitig zu sortierenden Elemente ist beschränkt durch einen Parameter, dessen Default-Wert 1024 beträgt. Ist die Größe des verfügbaren Buffers kleiner als 1024 (d.h. als der Parameter) oder kleiner als die Größe des aktuell zu sortierenden Buckets, so wird der Bucket in kleinere Teilbuckets aufgeteilt, die nach dem Sortieren gemerged werden. Falls der aktuell sortierte Bucket den letzten RMS-Teilstring enthält, so wird dieser bereits an die richtige Position gesetzt, da dieser nicht verglichen werden kann. 


% Please add the following required packages to your document preamble:
% \usepackage[table,xcdraw]{xcolor}
% If you use beamer only pass "xcolor=table" option, i.e. \documentclass[xcolor=table]{beamer}
\begin{table}
	\begin{tabular}{l|lllllllllllllll}
		$i$  & 0 & 1                                       & 2 & 3 & 4 & 5 & 6 & 7 & 8 & 9 & 10 & 11 & 12 & 13 & 14 \\ \hline
		$\mathsf{SA}$ & 0 & \cellcolor[HTML]{32CB00}2               & 3 & 1 & 0 & 0 & 0 & 0 & 0 & 0 & 0  & 2  & 4  & 8  & 10 \\ \hline
		$\mathsf{SA}$ & 0 & \cellcolor[HTML]{32CB00}$\underline{2}$ & 3 & 1 & 0 & 0 & 0 & 0 & 0 & 0 & 0  & 2  & 4  & 8  & 10 \\ \hline
	\end{tabular}
	\caption{RMS-Substrings sortiert. Bei identischen Substrings sind alle bis auf den ersten negiert (gekennzeichnet durch $\underline{\ }$).}
	\label{dss:table:substring-sorted}
\end{table}


Das richtige Sortieren wird über Intro Sort (\textit{ISS}) durchgeführt. Dabei wird Multikey Quicksort $\lfloor \lg (\text{last} - \text{first}) \rfloor$ Male ausgeführt, bevor Heapsort verwendet wird. Es wird dabei nicht rekursiv aufgerufen, sondern über einen Stack implementiert, welcher die unsortierten Teilintervalle enthält. Damit werden die kleineren Teilintervalle immer zuvor verarbeitet. Dies garantiert eine maximale Stack-Größe von $\lg l$, wobei $l$ die Größe des initialen Intervalls bezeichne. Unterschreitet die Größe des aktuell zu sortierenden (Teil-)Buckets einen Schwellwert (Defaultwert ist 8), so wird stattdessen Insertionsort verwendet. Beim Vergleich von Insertionsort werden die RMS-Teilstrings zeichenweise verglichen, angefangen bei der aktuellen Tiefe des Sortierens.

Beim Sortieren kann es vorkommen, dass einige der Teilstrings nicht vollständig sortiert werden können, d.h. sie sind identisch. Alle bis auf den ersten dieser Teilstrings in einem Intervall gleicher Teilstrings werden durch ihre bitweise negierte Referenz abgespeichert. Die erste Referenz in diesem Intervall repräsentiert den Beginn jenes Intervalls. Diese Uneindeutigkeit wird später aufgelöst. Tabelle \ref{dss:table:substrings} listet alle Teilstrings für unser Beispiel an. RMS-Suffixe 2 und 8 haben dabei einen identischen Teilstring. In Tabelle \ref{dss:table:substring-sorted} sind die sortierten Indizes in $\mathsf{PAb}$ enthalten. Da Index 8 (Referenzindex 2) identisch mit einem anderen war, wird dieser Index negiert.



% Please add the following required packages to your document preamble:
% \usepackage[table,xcdraw]{xcolor}
% If you use beamer only pass "xcolor=table" option, i.e. \documentclass[xcolor=table]{beamer}
% Please add the following required packages to your document preamble:
% \usepackage[table,xcdraw]{xcolor}
% If you use beamer only pass "xcolor=table" option, i.e. \documentclass[xcolor=table]{beamer}
\begin{table}
	\begin{tabular}{l|llll|llll|lllllll}
		$i$  & 0 & 1                         & 2                          & 3 & 4                         & 5                         & 6                         & 7                         & 8 & 9 & 10 & 11 & 12 & 13 & 14 \\ \hline
		$\mathsf{SA}$ & 0 & $\underline{2}$           & 3                          & 1 & 0                         & 0                         & 0                         & 0                         & 0 & 0 & 0  & 2  & 4  & 8  & 10 \\ \hline
		$\mathsf{SA}$ & 0 & $\underline{2}$           & 3                          & 1 & 0                         & \cellcolor[HTML]{32CB00}3 & 0                         & 0                         & 0 & 0 & 0  & 2  & 4  & 8  & 10 \\ \hline
		$\mathsf{SA}$ & 0 & $\underline{2}$           & 3                          & 1 & 0                         & 3                         & 0                         & \cellcolor[HTML]{32CB00}2 & 0 & 0 & 0  & 2  & 4  & 8  & 10 \\ \hline
		$\mathsf{SA}$ & 0 & \cellcolor[HTML]{34CDF9}2 & \cellcolor[HTML]{32CB00}-2 & 1 & 0                         & 3                         & \cellcolor[HTML]{32CB00}1 & 2                         & 0 & 0 & 0  & 2  & 4  & 8  & 10 \\ \hline
		$\mathsf{SA}$ & 0 & 2                         & -2                         & 1 & \cellcolor[HTML]{32CB00}1 & 3                         & 1                         & 2                         & 0 & 0 & 0  & 2  & 4  & 8  & 10 \\ \hline
		& \multicolumn{4}{l|}{$\mathsf{PAb}$}                                       & \multicolumn{4}{l|}{$\mathsf{ISAb}$}                                                                                     &   &   &    &    &    &    &   
	\end{tabular}
	\caption{Berechnung des initialen, partiellen inversen Suffix-Arrays $\mathsf{ISAb}$. Unterstrichene Werte wurden im vorigen Schritt nicht eindeutig sortiert, während negative Werte bereits sortierte Intervalle darstellen.}
	\label{dss:table:isa}
\end{table}

\subsubsection{Partielles inverses Suffix-Array berechnen}
Als nächstes können wir ein partielles inverses Suffix-Array berechnen, welches die Ränge der RMS-Suffixe über die bereits partiell sortierten RMS-Teilstrings wiedergibt. Das inverse Suffix-Array repräsentieren wir als $\mathsf{ISAb}[i] = \mathsf{SA}[m + i]$ mit $0 \leq i < m$, d.h. es wird in $\mathsf{SA}[m\dots 2m)$ gespeichert. In $\mathsf{ISAb}[i]$ finden wir somit die Anzahl der kleineren RMS-Suffixe des $i$-ten RMS-Suffixes. Falls $m > \frac{n}{3}$ gilt, so überschneidet sich $\mathsf{ISAb}$ mit $\mathsf{PAb}$. Dies stellt kein Problem dar, da wir die Textpositionen nicht mehr benötigen.

Es wird $\mathsf{SA}[0\dots m)$ von rechts nach links gescannt. Wenn ein Wert kleiner 0 ist, so erreichen wir ein Intervall, in dem die RMS-Suffixe im vorherigen Schritt nicht eindeutig sortiert wurden. Wir weisen allen von ihnen den größtmöglichen Rang $m-i$ zu. $i$ entspricht dabei der Anzahl der größeren RMS-Suffixe. Zusätzlich müssen die Referenzen bitweise negiert werden (da sie nun \glqq vergleichbar\grqq{} sind). Falls der Wert jedoch $\geq 0$ ist und dieser nicht nach einem negierten Wert gescannt wurde (d.h. nicht der Anfang eines unsortierten Intervalls ist), so weisen wir den korrekten Rang $m-i$ zu. Wann immer ein komplett sortiertes Intervall erkannt wird, so wird die Anfangsposition dieses Intervalls in $\mathsf{SA}[0\dots m)$ mit $-k$ markiert, wobei $k$ der Größe dieses Intervalls entspricht. Somit können alle sortierten Intervalle erkannt werden, da diese mit einem negativen Wert beginnen, dessen Absolutwert der Länge dieses Intervalls gleicht. Tabelle \ref{dss:table:isa} zeigt die schrittweise Berechnung für das initiale partielle ISA. Die letzten beiden Indizes ergeben ein sortiertes Intervall, weshalb an $\mathsf{PAb}[2]$ der Wert -2 eingetragen wird. Die relativen Indizes 0 und 2 erhalten denselben Rang, da ihre Teilstrings identisch sind.


\subsubsection{RMS-Suffixe sortieren}
Zu guter Letzt können wir die korrekten Ränge aller RMS-Suffixe bestimmen und diese in $\mathsf{ISAb}$ abspeichern. Die Ränge sind dabei zum Sortieren ausreichend, d.h. $\mathsf{PAb}$ und der Zugriff auf den originalen Text wird nicht mehr benötigt. Der verfolgte Ansatz ähnelt dabei dem des Prefix-Doubling: in jeder Iteration $k$ betrachten wir nicht die doppelte Präfixlänge, sondern die doppelte Anzahl der zu betrachtenden RMS-Teilstrings, welche eine beliebige Länge haben können. Mit $\mathsf{ISAd}[i]$ referenzieren wir den Rang des $i + 2^k$-ten RMS-Suffixes. Die Ränge der RMS-Suffixe müssen dabei aktualisiert werden, wenn die Anzahl der betrachteten RMS-Teilstrings verdoppelt wird. Da die Ränge der RMS-Suffixe in $\mathsf{ISA}$ in Textreihenfolge gegeben sind, kann der Rang des nächstgelegenen RMS-Teilstrings jederzeit für beliebige Teilstrings abgerufen werden.

% Please add the following required packages to your document preamble:
% \usepackage[table,xcdraw]{xcolor}
% If you use beamer only pass "xcolor=table" option, i.e. \documentclass[xcolor=table]{beamer}
\begin{table}
	\begin{tabular}{l|lllllllllllllll}
		$i$  & 0                         & 1                         & 2                                        & 3                                       & 4 & 5 & 6 & 7 & 8 & 9 & 10 & 11 & 12 & 13 & 14 \\ \hline
		$\mathsf{T}$  & c                         & a                         & a                                        & b                                       & a & c & c & a & a & b & a  & c  & a  & a  & \$ \\ \hline
		$\mathsf{SA}$ & -4                        & 0                         & -2                                       & 1                                       & 1 & 3 & 0 & 2 & 0 & 0 & 0  & 2  & 4  & 8  & 10 \\ \hline
		$\mathsf{SA}$ & \cellcolor[HTML]{32CB00}8 & \cellcolor[HTML]{32CB00}2 & \cellcolor[HTML]{32CB00}$\underline{10}$ & \cellcolor[HTML]{32CB00}$\underline{4}$ & 1 & 3 & 0 & 2 & 0 & 0 & 0  & 2  & 4  & 8  & 10 \\ \hline
	\end{tabular}
	\caption{Bestimmung der absoluten RMS-Suffixindizes in der korrekten Reihenfolge (in $\mathsf{SA}[0\dots 4)$) anhand der Werte in $\mathsf{ISAb}$. Unterstrichene Werte kennzeichnen die Negierung vor dem ersten Induzieren.}
	\label{dss:table:correct-indices}
\end{table}

\subsubsection{Repetition-Detection}
Bevor wir uns mit dem Induzieren der L- und S-Suffixe beschäftigen, folgt zuvor noch eine Anpassung mittels \textit{repetition detection}. Diese Anpassung beschreibt den Sortierschritt im vorigen Absatz.

Definition \ref{def:repetition} besagt, wie eine Wiederholung in einem Text definiert ist. Dies kann problematisch werden, falls $\mathsf{S}_i$ ein RMS-Suffix ist, denn dann wäre $\mathsf{S}_{kp}$ ebenfalls ein RMS-Suffix für $k \leq r$. Um dies aufzulösen, können wir das erste Symbol betrachten, welches nicht Teil der Wiederholung ist, d.h. $\mathsf{T}[i + l] \neq \mathsf{T}[i + rp + l]$. Falls $\mathsf{T}[i + rp + l] < \mathsf{T}[i + l]$, dann gilt für $1 < i \leq r$:$\mathsf{T}[i + (r-1)p + 1, i+rp] < \mathsf{T}[(i-1) + (r-1)p + 1, (i-1) + rp]$. Mit anderen Worten, die RMS-Suffixe dieser Wiederholung sind in absteigender Reihenfolge sortiert. Der analoge Fall gilt für $\mathsf{T}[i + rp + l] > \mathsf{T}[i + l]$, d.h. $\mathsf{T}[i + (r-1)p + 1, i + rp] > \mathsf{T}[(i-1) + (r-1)p + 1, (i-1) + rp]$, die RMS-Suffixe liegen in aufsteigender Reihenfolge vor.

Verwenden wir Quicksort mit den zuvor sortierten Rängen als Schlüssel, so können wir diese \textit{repetition detection} verwenden. Wir verwenden den Rang des Medians der aktuell betrachteten RMS-Suffixe als Pivotelement. Ist der aktuelle Rang des ersten RMS-Suffixes im aktuell betrachteten Teilintervall gleich zu dem Pivotelement, so liegt eine Wiederholung vor ($\mathsf{ISAb}[i] = \mathsf{ISAd}[i]$). Wir verwenden wieder für $\lfloor \lg (\text{last} - \text{first})\rfloor $ Male Quicksort, bevor wir Heapsort zum Sortieren verwenden. Nachdem alle RMS-Suffixe durch das Sortieren in lexikografischer Reihenfolge liegen, durchlaufen wir den Text $\mathsf{T}$ erneut von rechts nach links. Beobachten wir RMS-Suffix $\mathsf{S}_i$ an Position $j$, so speichern wir den Index ab in $\mathsf{SA}[\mathsf{ISAb}[i]] = j$. Da wir im darauf folgenden Durchlauf zunächst nur S-Suffixe, aber keine L-Suffixe induzieren möchten, speichern wir die bitweise Negation von $j$, falls $S_{j-1}$ ein L-Suffix ist. Die korrekten RMS-Indizes für unser Beispiel sind in Tabelle \ref{dss:table:correct-indices} dargestellt.

% Please add the following required packages to your document preamble:
% \usepackage[table,xcdraw]{xcolor}
% If you use beamer only pass "xcolor=table" option, i.e. \documentclass[xcolor=table]{beamer}
\begin{table}
	\begin{tabular}{l|lllllllllllllll}
		& \multicolumn{1}{l|}{\$} & \multicolumn{8}{l|}{a}                                                                                                                                                                  & \multicolumn{2}{l|}{b} & \multicolumn{4}{l|}{c} \\ \hline
		$i$  & 0                       & 1 & 2                & 3               & 4 & 5                         & 6                         & 7                                        & 8                                       & 9         & 10        & 11   & 12  & 13  & 14  \\ \hline
		$\mathsf{T}$  & c                       & a & a                & b               & a & c                         & c                         & a                                        & a                                       & b         & a         & c    & a   & a   & \$  \\ \hline
		$\mathsf{SA}$ & 8                       & 2 & $\underline{10}$ & $\underline{4}$ & 1 & \cellcolor[HTML]{32CB00}8 & \cellcolor[HTML]{32CB00}2 & \cellcolor[HTML]{32CB00}$\underline{10}$ & \cellcolor[HTML]{32CB00}$\underline{4}$ & 0         & 0         & 2    & 4   & 8   & 10  \\ \hline
	\end{tabular}
	\caption{Setzen der Indizes der sortierten RMS-Suffixe an die richtige Position in $\mathsf{SA}$.}
	\label{dss:table:final-sort}
\end{table}

Nun sind die Positionen aller RMS-Suffixe im Text in $\mathsf{SA}[0\dots m)$ in lexikografischer Reihenfolge abgespeichert. Diese müssen als nächstes an ihre richtige Position gebracht werden. Das Ergebnis für unser Beispiel sieht man in Tabelle \ref{dss:table:final-sort}. Währenddessen können $\mathsf{BUCK\-ET\_S}$ sowie $\mathsf{BUCK\-ET\_RMS}$ derart angepasst werden, dass alle S-Buckets die rechte Grenze des jeweiligen Buckets beinhalten, um beim Induzieren das S-Suffix direkt an die richtige Position zu setzen. Für RMS-Suffixe wird dabei nur der Bucket $\mathsf{b}_{c0,c0+1}$ aktualisiert, da diese die linke Grenze eines Buckets $\mathsf{b}_{c0}$ beinhalten, bis zu welcher S-Suffixe eingefügt werden können. Tabelle \ref{dss:table:last-buckets} gibt die Berechnung der Buckets für unseren Beispielstring an.

% Please add the following required packages to your document preamble:
% \usepackage[table,xcdraw]{xcolor}
% If you use beamer only pass "xcolor=table" option, i.e. \documentclass[xcolor=table]{beamer}
\begin{table}
	\begin{tabular}{l|l|l|l|l|l|l|l|}
		\cline{2-8}
		& \$ & a & b & c  & (a,a)                     & (a,b)                     & (a,c)                     \\ \hline
		\multicolumn{1}{|l|}{$\mathsf{Bucket\_L}$}   & 0  & 1 & 9 & 11 &                           &                           &                           \\ \hline
		\multicolumn{1}{|l|}{$\mathsf{Bucket\_S}$}   &    &   &   &    & \cellcolor[HTML]{32CB00}4 & \cellcolor[HTML]{32CB00}6 & \cellcolor[HTML]{32CB00}8 \\ \hline
		\multicolumn{1}{|l|}{$\mathsf{Bucket\_RMS}$} &    &   &   &    &                           & \cellcolor[HTML]{34CDF9}3 & 2                         \\ \hline
	\end{tabular}
	\caption{Berechnung der S- und RMS-Buckets für das Induzieren. Für die S-Buckets wurden die rechten Grenzen der jeweiligen Buckets bestimmt (grün markiert). Bei RMS-Buckets wird nur der Bucket $\mathsf{b}_{c0, c0+1}$ berechnet (blau markiert), um die linke Grenze für S-Suffixe zu markieren (läuft beim Induzieren der S-Buckets nur bis zu dieser Grenze für $\mathsf{b}_{c0}$)}
	\label{dss:table:last-buckets}
\end{table}
%Beispiel für Suffix-Typen?
