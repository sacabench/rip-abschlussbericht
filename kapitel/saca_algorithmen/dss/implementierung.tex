\subsection{Implementierung}

In unserer Implementierung wurden einige Stellen vereinfacht umgesetzt als bei der Referenz und in den vorigen Abschnitten beschrieben. Die Typen wurden (bis auf den Linksdurchlauf beim Induzieren der L-Suffixe) anhand von externen Methoden bestimmt, welche den Typen des Nachfolgers übergeben bekommt, um den Typen bei gleichen Zeichen korrekt zu bestimmen. Beim Sortieren der Teilstrings wurde statt des Multikey-Quicksorts innerhalb des Introsorts eine Vergleichsfunktion gewählt, welche alle Teilstrings enthält und diese (bei zwei gegebenen Suffix-Indizes) Zeichenweise vergleicht. 
Der größte Unterschied liegt beim Sortieren der RMS-Suffixe, nachdem die Teilstrings vorsortiert und das initiale partielle ISA bestimmt wurde. In der Referenz wurde die Repetition-Detection innerhalb des Quick- bzw. Introsorts integriert. In unserer Implementierung hingegen ist noch eine einfachere, iterative Idee verbaut: Wir durchlaufen das $\mathsf{PAb}$, d.h. die Referenzindizes bzw. Indikatoren für sortierte Intervalle und suchen unsortierte Intervalle bei einem Durchlauf von links nach rechts. Bei sortierten Intervallen können über den eingetragenen negierten Wert das komplette Intervall übersprungen werden. Bei unsortierten Intervallen müssen wir zum Einen prüfen, ob es sich um ein einzelnes Intervall handelt. Dies können wir daran erkennen, dass alle Ränge identisch sind. Bei einem unterschiedlichen Rang beginnt ein weiteres, unsortiertes Intervall. Ist ein komplettes unsortiertes Intervall bestimmt, kann anhand des Teilstring-Doublings über den Rang des nächsten Teilstrings durch Verdoppelung sortiert werden. Danach müssen in diesem Intervall alle Ränge neu berechnet werden, um zwischen neuen sortierten und unterschiedlichen unsortierten Intervallen weiterhin unterscheiden zu können. Dies wird solange wiederholt, bis in einem vollständigen Durchlauf der Referenzen kein unsortiertes Intervall vorhanden ist. Dann erst sind alle Ränge eindeutig bestimmt und die Suffixe können an die richtige Position ins Suffix-Array gesetzt werden.