\section{SADS}
\subsection{D-Critical}
\label{dc}

\currentauthor{Christopher Poeplau}

Der  Kern des \textbf{Radix Sorting Fixed Length D-Critical Substring-Algo\-rithmus}\cite[Kap.~4]{saca:6} (kurz: SADS) sind die \textbf{d-critical} (kurz: dc) Zeichen des Strings \inputtext. Ein Zeichen ist dc, für ein beliebiges aber festes $d \geq 2$, wenn
\begin{itemize}
    \item \inputtext[i] ist LMS oder
    \item \inputtext[i-d] ist dc, \inputtext[i+1] ist kein LMS und kein weiteres Zeichen in \inputtext[i-d+1,i-1] ist dc. Um die Notation einfach zu halten, sei $d_1=d+1$ für den Rest des Kapitels.
\end{itemize}

\noindent Für den Rest des Kapitels gilt $\mathbf{d=2}$.

Es folgt ein Beispiel für eine Klassifizierung, die dc-Zeichen (hier mit ** markiert) enthält:

\begin{table}[H]
\centering
  \begin{tabular}{ | l | c | c| c| c| c| c| c| c| c| c| c| c| c| c| c| c| c | }
    \hline
        $i$ & 0 & 1 & 2 & 3 & 4 & 5 \\ \hline
        $T$ & b & a & c & a & a & \$ \\ \hline
        $ $ & L & S & L & L & L & S \\ \hline
      	$ $ & & * & & ** & & * \\
    \hline
  \end{tabular}
  \caption{Beispiel einer dc-Berechnung}
\end{table}
\bigskip

\noindent \inputtext[1] und \inputtext[5] sind Zeichen, die als LMS klassifiziert werden. Dadurch sind sie auch dc-Zeichen. Auch das Zeichen \inputtext[3] ist ein dc-Zeichen, denn $\inputtext[3 - 2] = \inputtext[1]$ ist ein dc-Zeichen, $\inputtext[3 + 1] = \inputtext[4]$ ist kein LMS-Zeichen und in dem Intervall $\inputtext[3-2+1,3] = \inputtext[2,3]$ ist kein weiteres dc-Zeichen. Damit ist \inputtext[3] auch ein dc-Zeichen. Im Gegensatz zu den Zeichen \inputtext[1] und \inputtext[5] ist \inputtext[3] also ein Zeichen, das nicht vom SAIS als LMS, sondern nur vom SADS als dc-Zeichen klassifiziert wird.

\bigskip
\noindent Es folgen Definitionen, die für den weiteren Verlauf des Algorithmus notwendig sind:
\begin{itemize}
    \item   Ein Suffix \inputtext[i,n-1] gilt als dc, genau dann wenn \inputtext[i] dc ist.
    \item   Seien \inputtext[i] und \inputtext[j] zwei dc-Zeichen. Sie gelten als \textbf{benachbart}, wenn zwischen ihnen kein weiteres dc-Zeichen vorkommt.
    \item   Der Substring \inputtext[i, i+d_1] ist der dc-Substring des Zeichens \inputtext[i]. Dabei entspricht die Länge des Substrings immer $d_1+1$, sodass er wenn nötig mit dem Sentinel ergänzt wird.
    \item   Sei \inputtext[i] dc, dann gilt, dass \inputtext[i-1] und \inputtext[i+1] nicht dc sind.
    \item   \inputtext[0] ist kein dc und \inputtext[n] ist ein dc mit $n=|\inputtext|-1$
\end{itemize}
\subsubsection{Definition $\omega$-Gewichtung}
\label{weighting}
Sei $\omega(\inputtext,i)$ die $\omega$-gewichtende Funktion, definiert als $\omega(\inputtext,i) = 2\inputtext[i]+t[i]$. Ein S-Typ entspricht dabei einer $1$ und ein L-Typ einer $0$.

\subsection{Framework}
Ähnlich des Frameworks des SAIS-Algorithmus in Kap.~\ref{saisAlg}, ist auch der SADS-Algorithmus in einem Framework\cite[Fig.~3]{saca:6} modelliert:
\label{sadsAlg}
\begin{figure}[h]
\begin{minted}[escapeinside=@@,numbers=left, breaklines]{python}
SADS(T, SA)
# T is the input string;
# SA is the output suffix array of T
# t: array[@$0,n-1$@] of boolean;
# P@$_1$@, S@$_1$@: array[@$0,n_1-1$@] of integer;
# B: array[@$0,|\Sigma (T)|-1$@] of integer;
Scan T once to classify all the characters as L- or S-type into t;
Scan t once to find all the d-critical-substrings in T into P@$_1$@;
Bucket sort all the d-critical substrings using P@$_1$@ and B;
Name each d-critical-substring in T by its bucket index to get a new shortened string T@$_1$@;
if @$|T_1|$@ = Number of Buckets
   Directly compute SA@$_1$@ from T@$_1$@;
else
   SADS(T@$_1$@, SA@$_1$@); # Fire a recursive call
Induce SA from SA@$_1$@
\end{minted}
\caption{SADS-Framework}
\end{figure}


\noindent Analog zum ersten Algorithmus wird das Problem in Probleme kleinerer Größe aufgeteilt, um abschließend die Lösung des Problems zu induzieren.
Sei der String $\inputtext = caabaccaabacaa\$$ wieder der String, zu dem das Suffix-Array konstruiert wird. Wie oben festgelegt gilt $d = 2$. Der erste Schritt besteht daraus die Typen aus \inputtext
zu bestimmen, sodass folgende, schon aus dem SAIS bekannte, Klassifizierung entsteht:

\begin{table}[H]
\center
  \begin{tabular}{ | l | c | c | c | c | c | c | c | c | c | c | c | c | c | c | c | c | }
    \hline
        $i$ & 0 & 1 & 2 & 3 & 4 & 5 & 6 & 7 & 8 & 9 & 10 & 11 & 12 & 13 & 14 \\ \hline
        $\inputtext$ & c & a & a & b & a & c & c & a & a & b & a & c & a & a & \$ \\ \hline
        $t$ & L & S & S & L & S & L & L & S & S & L & S & L & L & L & S \\
    \hline
  \end{tabular}
  \caption{Klassifizierung des Strings \inputtext}
\end{table}
\bigskip

\newpage
\noindent Mit Hilfe der Definitionen aus Kap.~\ref{dc} werden die d-critical Zeichen bestimmt:

\begin{table}[H]
\centering
  \begin{tabular}{ | l | c | c| c| c| c| c| c| c| c| c| c| c| c| c| c| c| c | }
    \hline
        $i$ & 0 & 1 & 2 & 3 & 4 & 5 & 6 & 7 & 8 & 9 & 10 & 11 & 12 & 13 & 14 \\ \hline
        $\inputtext$ & c & a & a & b & a & c & c & a & a & b & a & c & a & a & \$ \\ \hline
        $t$ & L & S & S & L & S & L & L & S & S & L & S & L & L & L & S \\ \hline
        $P_1$ & & * & & & * & & & * & & & * & & ** & & * \\
    \hline
  \end{tabular}
  \caption{dc-Berechnung}
\end{table}
\bigskip
Die einfachen dc-Zeichen sind hier $normale$ LMS-Zeichen, die genauso auch beim SAIS berechnet wurden. Das doppelte dc-Zeichen entsteht durch die Anpassungen des SADS, die dafür sorgt, dass maximal Substrings der Größe $d_1$ entstehen. Durch die Definitionen aus Kap.~\ref{dc} wird außerdem deutlich, dass es maximal $\lfloor\frac{n}{2}\rfloor$ dc-Zeichen geben kann.
\bigskip
\\Die gefundenen sechs Substrings müssen nun sortiert werden. Dazu nutzt der Algorithmus ein Bucket-Sort mit vier Passes, beginnend mit dem LSC(Least Significant Character). Zunächst werden die Wörter anhand ihres jeweiligen LSC ihren Buckets zugeordnet. Dort werden sie anhand der $\omega$-Gewichtung aus Kap.~\ref{weighting} in ihren Buckets sortiert. In den darauffolgenden Schritten wird das nächste Zeichen des jeweiligen Wortes analysiert und das Wort wird dem passenden Bucket zugeordnet. Angewandt auf die dc-Substrings ergibt sich so folgende Reihenfolge:

\begin{table}[H]
\centering
\begin{adjustbox}{width=1\textwidth}
  \begin{tabular}{ | l | c | c | c | c | c | c | c | c | c | c | c | c | c | c | c | c | }
    \hline
       $\$$ & 14 \$\$\$\textbf{\underline{\$}} & 14 \$\$\textbf{\underline{\$}}\$ & 14 \$\textbf{\underline{\$}}\$\$ & 14 \textbf{\underline{\$}}\$\$\$ & 0 \\
            & 12 aa\$\textbf{\underline{\$}} & 12 aa\textbf{\underline{\$}}\$ & & & \\ \hline
            & 10 aca\textbf{\underline{a}} & 10 ac\textbf{\underline{a}}a & 12 a\textbf{\underline{a}}\$\$ & 12 \textbf{\underline{a}}a\$\$ & 1 \\
            & 7 aab\textbf{\underline{a}} & & 7 a\textbf{\underline{a}}ba & 7 \textbf{\underline{a}}aba & 2 \\
        $a$ & 4 acc\textbf{\underline{a}} & & 1 a\textbf{\underline{a}}ba & 1 \textbf{\underline{a}}aba & 2  \\
            & 1 aab\textbf{\underline{a}} & & & 10 \textbf{\underline{a}}caa & 3 \\
            & & & & 4 \textbf{\underline{a}}cca & 4  \\ \hline
        $b$ & & 7 aa\textbf{\underline{b}}a & & & \\
            & & 1 aa\textbf{\underline{b}}a & & & \\ \hline
        $c$ & & 4 ac\textbf{\underline{c}}a & 10 a\textbf{\underline{c}}aa & & \\
            & & & 4 a\textbf{\underline{c}}ca & & \\ \hline
            & Gewichten & Zeichensortierung & Zeichensortierung & Zeichensortierung & Bucketing \\
    \hline
  \end{tabular}
  \end{adjustbox}
  \caption{Radix-Sorting}
\end{table}
\bigskip

\noindent Im ersten Schritt findet die $\omega$-Gewichtung statt. Dadurch findet eine initiale Vorsortierung in den jeweiligen LSC-Buckets statt. Es folgt die eigentliche Radix-Sortierung. Die Suffixe $\{14, 12, 10, 7, 1 , 4\}$ werden anhand des vorletzten Zeichens sortiert. Dabei bleiben $\{14, 12, 10\}$ an derselben Stelle und in denselben Buckets, da ihr Zeichen identisch zum Nachfolgerzeichen ist. Es findet also keine Umsortierung statt. Substrings $\{7, 1\}$ gehören aufgrund ihres Zeichens nun zum \textit{b-Bucket} und sind aufgrund ihrer Reihenfolge im Schritt davor implizit in ihren Buckets sortiert worden. Analog dazu befindet sich $\{4\}$ als einziger String im \textit{c-Bucket}.
Im nächsten Schritt werden $\{12, 7, 1\}$ zum \textit{a-Bucket} zusortiert. Dabei ist die $12$ an erster Stelle, da sie aus dem obersten Bucket kommt. Äquivalent dazu ist im letzten Schritt die Sortierung final und die Indizes können ihren Substrings zugeordnet werden. Dabei ist zu beachten, dass identische Strings denselben Index erhalten.

\noindent Aus dem Bucketsort ist folgende Bucket-Zuordnung der dc-Zeichen enstanden:

\begin{table}[H]
\centering
  \begin{tabular}{ | l | c | c | c | c | c | c | c | c | c | c | c | c | c | c | c | c | }
    \hline
        $T_1$ & 2 & 4 & 2 & 3 & 1 & 0 \\ \hline
              & 1 & 4 & 7 & 10 & 12 & 14  \\
    \hline
  \end{tabular}
  \caption{dc-Bucket-Zuordnung}
\end{table}
\bigskip
Ähnlich zum SAIS ist auch hier eine Rekursion der Tiefe $1$ nötig, da der Bucket $2$ doppelt vergeben ist, wodurch noch nicht auf die eindeutige Reihenfolge geschlossen werden kann. Das Teilproblem beschreibt sich wie folgt:

\begin{table}[H]
\centering
  \begin{tabular}{ | l | c | c | c | c | c | c | c | c | c | c | c | c | c | c | c | c | }
    \hline
        $i$ & 0 & 1 & 2 & 3 & 4 & 5 \\ \hline
      $T_1$ & 2 & 4 & 2 & 3 & 1 & 0  \\ \hline
      $t_1$ & S & L & S\textsuperscript{*} & L & L & S\textsuperscript{*} \\
    \hline
  \end{tabular}
  \caption{Beschreibung des Teilproblems mit String $T_1$}
\end{table}
\bigskip

\noindent Analog zum vorherigen Bucket-Sort werden auch hier die dc-Zeichen sortiert, sodass sich folgende Berechnung ergibt:

\begin{table}[H]
\centering
  \begin{tabular}{ | l | c | c | c | c | c | c | c | c | c | c | c | c | c | c | c | c | }
    \hline
        $SA_1$ & 1 & 0 \\ \hline
        $dc$ & 2 & 5 \\
    \hline
  \end{tabular}
  \caption{dc-Sortierung}
\end{table}
\bigskip

\noindent Der letzte Schritt des Algorithmus besteht darin, das Suffix-Array \sa rekursiv aus den Teillösungen zu induzieren. Dazu wird das bestehende $T_1$ aus dem Rekursionsschritt sortiert, sodass sich die Buckets $\{0,1,2,2,3,4\}$ ergeben. Folgender Algorithmus wird nun für den ersten Schritt des Induzierens angewandt\cite[Kap.~4.5]{saca:6}:\newpage
\begin{figure}[H]
\begin{minted}[mathescape=true, escapeinside=||, frame=single]{text}
    Initialize each item of SA as |$-1$|. Find the end of
    each bucket in |$SA$| for all the suffixes in |$S$|. Scan
    |$SA_1$| once from right to left, if |$suf(T, P_1[SA_1[i]])$|
    is a LMS suffix then put |$P_1[SA_1[i]]$| to the current
    end of the bucket for |$suf(T, P_1[SA_1[i]])$| in |$SA$| and
    forward the bucket's end one item to the left.
\end{minted}
\caption{Algorithmus zur Induktion}
\end{figure}
\noindent Aus Gründen der Übersichtlichkeit, entsprechen leere Felder einem unbeschriebenen Feld, anstelle von $-1$.

\begin{table}[H]
\centering
  \begin{tabular}{ | l | c | c | c | c | c | c | c | c | c | c | c | c | c | c | c | c | }
    \hline
     $SA_1$ & 1 & 0 & & & & \\ \hline
   $Bucket$ & 0 & 1 & \multicolumn{1}{c}{2} & & 3 & 4 \\ \hline
      $t$   & S & L & \multicolumn{1}{c}{S} & & L & L \\ \hline
            & 5 & & & & &  \\ \hline
            & & 4 & & & 3 & 1 \\ \hline
            & & & 2 & 0 & & \\
    \hline
  \end{tabular}
  \caption{Induzieren von $SA_1$}
\end{table}
\bigskip

\noindent Es ergibt sich das Suffix-Array $SA_1 = \{5, 4, 2, 0, 3, 1\}$, mit dem jetzt das finale Suffix-Array \sa induziert wird:

\begin{table}[H]
\centering
\begin{adjustbox}{width=1\textwidth}
  \begin{tabular}{ | l | c | c | c | c | c | c | c | c | c | c | c | c | c | c | c | c | }
    \hline
     $SA_1$ & 5 & 4 & 2 & 0 & 3 & 1 & & & & & & & & & \\ \hline
   $Bucket$ &\$ & \multicolumn{7}{c}{a} &   & \multicolumn{1}{c}{b} &   & \multicolumn{3}{c}{c} & \\ \hline
      $Typ$ & S & \multicolumn{1}{c}{L} &   & \multicolumn{5}{c}{S} &   & \multicolumn{1}{c}{L} &    & \multicolumn{3}{c}{L} &\\ \hline
            & 14 & & & & & (7) & (1) & (10) & (4) & & & & & & \\ \hline
            & & 13 & 12 & & & & & & & 9 & 3 & 11 & 6 & 0 & 5 \\ \hline
            & & & & 7 & 1 & 10 & 4 & 2 & 8 & & & & & & \\
    \hline
  \end{tabular}
  \end{adjustbox}
  \caption{Induzieren von \sa}
\end{table}
\bigskip
\noindent Von links nach rechts ergibt sich das endgültige Suffix-Array \\
$\sa = \{14, 13, 12, 7, 1, 8, 2, 10, 4, 9, 3, 11, 6, 0, 5\}$.

\subsection{Optimierungsmöglichkeiten}
Analog zum SAIS werden beim SADS die Typen in einem Bitvektor gespeichert. Für die dc-Zeichen wird kein eigenes Array benötigt, da hier das SA-Array genutzt werden kann. Der Algorithmus kann auf Grund des Designs unseres Frameworks nicht nur für 8-Bit, sondern auch für 16-, 32-, 48-Bittypen usw. ausgeführt werden. Im Gegensatz zum SAIS scheint eine Optimierung des SADS nicht lohnenswert, da auch die Implementierung von Yuta Mori schlechtere Ergebnisse liefert als der SAIS.
 \newpage





























