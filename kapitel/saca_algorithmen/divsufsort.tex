\section{DivSufSort}
\label{algorithm:divsufsort}

\subsection{Einleitung}
In \currentauthor{Oliver Magiera} vielen Bereichen der Algorithmik sind Laufzeiten der entscheidende Faktor beim Vergleich von Algorithmen. Dabei werden bei den Analysen zumeist eher theoretische worst-case Laufzeitschranken betrachtet.
In der Praxis hingegen können die Ergebnisse anders aussehen. Ein bedeutendes Beispiel ist dabei der \glqq DivSufSort\grqq{} von Y. Mori \cite{saca:5:repo}. Mit einer worst-case Laufzeit von $O(n\log{n})$ und einem Speicherbedarf von $5n + O(1)$ mit $n$ als Textlänge liefert der Algorithmus bereits gute Schranken, jedoch nicht die Besten. In der Praxis hingegen liegt mit diesem Algorithmus eine der schnellsten Berechnungen für Suffixarrays vor. Da Y. Mori den Algorithmus nur als (kaum kommentierten) Quellcode veröffentlicht hat, werden wir im Folgenden den Algorithmus anhand der Beschreibung von J. Fischer und F. Kurpicz \cite{saca:5} erläutern. Dafür werden wir uns zunächst die zugrunde liegenden Algorithmen im Abschnitt Grundlagen anschauen, bevor wir uns den Definitionen und einigen Vorüberlegungen zu Suffix Arrays widmen. Danach beschreiben wir den eigentlichen Algorithmus, bevor wir die Ausarbeitung abschließen.

\iffalse
\subsection{Grundlagen}
Da einige bereits vorhandene Algorithmen und Berechnungen im DivSufSort-Algorithmus verwendet werden, sollen diese zunächst näher erläutert werden. Dafür beginnen wir mit der Präfixsumme.


\subsubsection{Heapsort}
Heapsort basiert auf der Struktur von Heaps, einem speziellen Binärbaum mit Heap-Bedingung. Bei Max-Heaps ist jeder Knoten größer als seine Kindknoten (Heap-Bedingung). Ein Heap kann auch als Array betrachtet werden. Dabei sind die Kinder eines Knotens $i$ die Knoten $2i$ und $2i + 1$. Ein Elter kann entsprechend erreicht werden über $\lfloor i/2 \rfloor$. Eine wichtige Operation für Heaps ist \textit{Max-Heapify}. Für einen gegebenen Knoten $i$ sei die Heap-Bedingung nicht erfüllt. Dann tauschen wir $i$ so lange mit dem größeren seiner Kinder, bis die Heap-Bedingung erfüllt ist. \textit{Build-Max-Heap} funktioniert wie folgt: Führe für alle Knoten, welche keine Blätter sind, \textit{Max-Heapify} aus, d.h. von $\lfloor \text{length}[H]/2\rfloor$ bis 1 (H bezeichnet Array für den Heap).

Heapsort läuft nun wie folgt ab: Erzeuge einen Max-Heap über \textit{Build-Max-Heap}. Die Wurzel enthält das größte Element, daher tausche es mit dem $n$-ten aus und verkleinere den Heap um 1. Führe daraufhin \textit{Max-Heapify} auf und wiederhole die Schritte, bis nur noch zwei Knoten vorhanden sind \cite{Cormen2009}[Kapitel 6].

\subsection{Quicksort}
Ein sehr beliebtes Verfahren Dank seiner Schnelligkeit ist Quicksort. Die Idee ist dabei simpel: Partitioniere ein (Teil-)Array anhand eines Pivotelements. Links von dem Pivot sind alle Elemente kleiner als das Pivot, rechts davon alle, die größer als das Pivot sind. Führe rekursiv Quicksort für die Teilarrays links und rechts vom Pivot aus. Das Partitionieren kann dabei wie folgt durchgeführt werden: Durchlaufe das Array von links nach rechts. Suche nach einem Element, welches kleiner gleich dem Pivot (letztes Element im Array) ist . Ist dieses gefunden, tausche es mit dem ersten Element, welches größer ist als das Pivot (über verwalteten Pointer). Wiederhole bis das vorletzte Element geprüft wurde. Der verwaltete Pointer gibt den Index an, welcher auf die Trennung der beiden Partitionen zeigt \cite{Cormen2009}[Kapitel 7].

\subsection{Multikey Quicksort}
Multikey Quicksort ist eine Erweiterung des Quicksort, welches ermöglicht, Elemente mit mehreren Schlüsseln wie Strings zu sortieren. Dabei geht der Algorithmus wie folgt vor: Verwalte eine Referenz zur aktuellen Tiefe $k$. Diese Referenz repräsentiert, nach welchem Zeichen aktuell sortiert wird. Alle Elemente kleiner als das Pivot landen im linken Teilarray und alle Elemente größer als das Pivot landen im rechten Teilarray, d.h. das $k$-te Zeichen wird jeweils verglichen. In diesen Teilarrays wird erneut mittels Multikey Quicksort nach dem $k$-ten Zeichen sortiert. Alle Elemente, deren $k$-tes Zeichen dem des Pivots gleicht, landen im mittleren Teilarray. Dieses Teilarray wird mittels Multikey Quicksort nach Zeichen $k+1$-sortiert \cite{bese97}. 

\subsection{Introspective Sort}
Zuletzt betrachten wir einen Sortieralgorithmus, welcher Quicksort mit Heapsort und Insertionsort kombiniert. Dabei verwenden wir eine spezielle Form des Quicksort, den Median-of-3 Quicksort. Um das Pivotelement zu bestimmen, betrachtet man das erste, das mittlere und das letze Element im betrachteten Array. Wir wählen davon den Median als Pivot. Nun Partitionieren wir das Array mittels berechnetem Pivot und sortieren erst das rechte, dann das linke Teilarray mittels Introspective Sort. Es gibt jedoch zwei Ausnahmen, wo wir kein Quicksort verwenden. Zum einen sortieren wir ab einer bestimmten Tiefe mittels Heapsort. Die maximale Tiefe ist dabei beschränkt durch $\lfloor \lg (last - first) \rfloor$, d.h. der Logarithmus von der Länge des ursprünglichen Arrays, abgerundet. Dies soll dafür sorgen, dass bei sogenannten Median-of-3 \glqq Killer-Sequenzen\grqq{}, das sind Sequenzen, welche die am schlechtesten Mögliche Eingabe für den Quicksort enthalten, die Laufzeit des Quicksort nicht $\Theta (n^2)$ beträgt. Zum anderen existiert ein Schwellwert (z.B. 8), welcher dafür sorgen soll, dass bei besonders kleinen Intervallen kein zu großer Overhead durch Quicksort bzw Heapsort entsteht. Fällt die Größe des betrachteten (Teil-)Arrays nämlich unterhalb dieses Schwellwerts, so wird Insertionsort angewandt. Die Prüfung auf den Schwellwert ist dabei die äußerste Schleife des Sortierverfahrens \cite{Musser97}.

\fi

\subsection{Grundlagen}
Bevor wir zum Algorithmus kommen, müssen zunächst noch einige Definitionen und Grundideen geklärt werden. Dafür betrachten wir zunächst die Präfixsumme:

\begin{definition}
	Betrachten wir eine Menge $\{a_0, a_1, \dots , a_{n-1}\}$ und den + Operator, dann ist die Menge der Präfixsummen wie folgt definiert:
	$$
	[a_0, (a_0 + a_1), \dots , (a_0 + a_1 + \dots + a_{n-1})]
	$$
\end{definition}
Die Präfixsumme kann demnach als ein Vektor betrachtet werden, welcher an der $i$-ten Stelle die ersten $i$ Elemente aufsummiert hat \cite{Blelloch90}. Später werden wir die Präfixsumme in leicht abgewandelter Form sehen.

In einem Text können Wiederholungen derselben Sequenz von Zeichen auftreten. Diese seien wie folgt definiert:
\begin{definition}
	\label{repetition}
	Eine Wiederholung in $T$ ist ein Teilstring $T[i, i + rp]$ mit $ r \geq 2, p \geq 0$ und $i, i + rp \in [0, n)$, sodass $T[i, i+p) = T[i + p, i + 2p) = \dots = T[i + (r-1)p, i + rp)$.
\end{definition}

Wir wollen unsere Suffix-Arrays noch weiter charakterisieren können. Dafür unterteilen wir die Suffixe in sogenannte Buckets.

\begin{definition}
	Alle Suffixe, die mit demselben Zeichen $c0 \in \Sigma$ beginnen, formen ein zusammenhängendes Intervall, das $c0$-Bucket $b_{\textit{c0}}$. Das $(c0,c1)$-Bucket $b_{\textit{c0,c1}}$ bezeichnet das Intervall, dessen Suffixe mit denselben zwei Zeichen $c0,c1 \in \Sigma$ beginnen.
\end{definition}

Die Buckets werden später wichtig, um die Beziehungen zwischen den Suffix-Typen herzuleiten. Dabei kann $c0=c1$ gelten. Suffixe können in drei Typen eingeteilt werden: L-Suffixe, S-Suffixe sowie RMS-Suffixe.

\begin{definition}
	Suffix $S_i$ ist ein
	\begin{itemize}
		\item L-Suffix, falls $T[i] > T[i+1]$ oder $i=n-1$.
		\item S-Suffix, falls $T[i] < T[i+1]$.
		\item RMS-Suffix, falls $S_i$ ein S-Suffix und $S_{i+1}$ ein L-Suffix ist.
	\end{itemize}
	Falls $T[i] = T[i+1]$, so ist $S_i$ je nach Typ von $S_{i+1}$ entweder ein L- oder ein S-Suffix.
\end{definition}

Somit liegen drei verschiedene Typen von Suffixen vor, welche eine Abhängigkeit der Textfolge repräsentieren. Dabei können RMS-Suffixe nicht durch Gleichheit zweier aufeinander folgender Zeichen übertragen werden. Es können maximal $\frac{n}{2}$ RMS-Suffixe vorliegen, da diese durch L-Suffixe definiert sind. Insbesondere schränkt die Zuweisung von Typen zu den Suffixen die Verteilung dieser auf Buckets ein: Ein Bucket $b_{\textit{c0,c1}}$ kann nur L-Suffixe enthalten, wenn $|c0| \geq |c1|$ gilt, d.h. der Rang von $c0$ größer als der Rang von $c1$ ist. Ebenso können nur S-Suffixe enthalten sein, wenn $|c0| \leq |c1|$. RMS-Suffixe können nicht bei $|c0|=|c1|$, d.h. in b$_{\textit{c0,c0}}$ vorliegen.
Durch diese Einschränkungen wird eine partielle Ordnung unter den Suffixen erzwungen:
	Seien $S_i, S_j$ zwei Suffixe. Dann gilt:
	\begin{itemize}
		\item $S_i < S_j$, falls $S_i$ ein L-Suffix, $S_j$ ein S-Suffix ist und $T[i] = T[j]$
		\item $S_i < S_j$, falls $S_i$ ein RMS-Suffix, $S_j$ ein S-, aber kein RMS-Suffix ist und $T[i, i+1] = T[j, j+1]$
	\end{itemize}

	L- und S-Suffixe können nur gleichzeitig in $b_{\textit{c0,c0}}$ auftauchen. Wir nehmen an, $S_i$ und $S_j$ beginnen mit $c0c0$, gefolgt von beliebig vielen $c0$ und $S_i, S_j$ sind ein L- bzw. S-Suffix. Sei $u = T[i+lcp(i, j)]$ und $v = T[j+lcp(i, j)]$, d.h. $u \neq v$ ist das erste Zeichen, wo sich $S_i$ und $S_j$ unterscheiden. Da $S_i$ ein L-Suffix ist und sich zuvor vom S-Suffix $S_j$ nicht unterschieden hat, muss $u \leq c0$ sein. Ebenso gilt $v \geq c0$. Eine der Ungleichungen ist strikt erfüllt, da $u \neq v$, damit ist $S_i < S_j$.
	Analog gilt für den zweiten Fall: Die ersten beiden Zeichen von $S_i, S_j$ sind identisch. Da $S_i$ ein RMS-Suffix ist, gilt $T[i] \neq T[i+1]$. Ebenso gilt $T[i+1] > T[i+2]$, da nach der Definition von RMS-Suffixen $S_{i+2}$ ein L-Suffix sein muss. Jedoch ist $S_j$ kein RMS-Suffix, sodass $T[i+2] < T[i+1] =  T[j+1] \leq T[j+2]$ und damit $S_i < S_j$. 

Um RMS-Suffixe später sortieren zu können, müssen wir zunächst RMS-Teilstrings betrachten.

\begin{definition}
	Gegeben seien zwei aufeinander folgende RMS-Suffixe $S_i$ und $S_j$, d.h. es existiert kein RMS-Suffix $S_k$ mit $i < k < j$. Wir bezeichnen den Teilstring $T[i, j+2)$ Als RMS-Teilstring. Für das letzte RMS-Suffix $S_i$ ($S_k$ mit $ i < k < n$ ist kein RMS-Suffix) ist der Teilstring $T[i, n)$ ebenfalls ein RMS-Teilstring.
\end{definition}

Nun haben wir neben den Suffix-Typen und Buckets auch RMS-Teilstrings kennen gelernt. Damit können wir mit der Beschreibung des Algorithmus beginnen.


\subsection{Der DivSufSort-Algorithmus}

Der Algorithmus selbst kann in drei Phasen unterteilt werden. In der ersten Phase müssen zunächst den Suffixen ihre Typen zugewiesen werden. Dafür muss der Text ein Mal durchlaufen werden. Während dieses Durchlaufes werden ebenfalls die Grenzen für die Buckets $b_{c0}$ bzw. $b_{c0,c1}$ berechnet. In Phase zwei werden die RMS-Suffixe in lexikografischer Reihenfolge sortiert und bereits an die korrekten Positionen im Suffix-Array \textit{SA} gesetzt. Hierfür werden zunächst die RMS-Teilstrings sortiert und deren Ränge bestimmt. Mit diesen Rängen können letztlich die RMS-Suffixe sortiert werden. In der dritten und letzten Phase müssen noch die L- und S-Suffixe an ihre richtigen Positionen gebracht werden. Dafür werden in einem Durchlauf von rechts nach links erst alle S-Suffixe und in einem zweiten Durchlauf von links nach rechts alle L-Suffixe induziert.

Damit diese drei Schritte reibungslos ablaufen können, benötigen wir noch etwas zusätzlichen Speicher. Wir verwenden zwei zusätzliche Arrays \textit{BUCK\-ET\_L} für L-Suffixe und \textit{BUCK\-ET\_S} für S- und RMS-Suffixe, um Informationen über die Buckets abspeichern zu können. \textit{BUCK\-ET\_L} der Größe $\sigma = |\Sigma |$ wird über einen Character $c0$ abgerufen, \textit{BUCK\-ET\_S} der Größe $\sigma ^2$ hingegen über zwei Character $(c0,c1)$. Um zwischen Referenzen für S- und RMS-Suffixen zu unterscheiden, werden diese über \textit{BUCK\-ET\_S$[c0,c1] = $ BUCK\-ET\_S$[|c0| \cdot \sigma + |c1|]$} bzw. \textit{BUCK\-ET\_RMS$[c0,c1] = $ BUCK\-ET\_S$[|c1| \cdot \sigma + |c0|]$} abgerufen. Die Buckets für S- und RMS-Suffixe können in demselben Array abgespeichert werden, da in $b_{c0,c0}$ keine RMS-Suffixe und in $b_{c0,c1}$ mit $c0 > c1$ keine S-Suffixe enthalten sein können. Die Anzahl der RMS-Suffixe wird mit $m$ bezeichnet.

Kommen wir nun zur ersten Phase des Algorithmus, der Initialisierung.
\subsubsection{Initialisierung}


\begin{table}
	\centering
	\resizebox{\columnwidth}{!} {%
		\begin{tabular}{l|l|l|l|l|l|l|l|l|l|l|l|l|l|l|l}
			$i$      & 0 & 1 & 2   & 3 & 4   & 5 & 6 & 7 & 8   & 9 & 10  & 11 & 12 & 13 & 14 \\ \hline
			$\mathsf{T}$      & c & a & a   & b & a   & c & c & a & a   & b & a   & c  & a  & a  & \$ \\ \hline
			$\mathsf{SA}$     & 0 & 0 & 0   & 0 & 0   & 0 & 0 & 0 & 0   & 0 & 0   & \cellcolor[HTML]{32CB00}2  & \cellcolor[HTML]{32CB00}4  & \cellcolor[HTML]{32CB00}8  & \cellcolor[HTML]{32CB00}10 \\ \hline
			SA-Typ & L & S & RMS & L & RMS & L & L & S & RMS & L & RMS & L  & L  & L  & L 
		\end{tabular}%
	}
	\caption{Eingabetext mit Suffixtypen und initialisiertem Suffix-Array.}
	\label{table:sa-init}
\end{table}





Wir beginnen mit einem Durchlauf des Textes $\mathsf{T}$, welcher von rechts nach links durchgeführt wird. In diesem Durchlauf werden zum einen die Typen der Suffixe festgelegt, zum anderen die Größe der Buckets abgespeichert. Zusätzlich wird am Ende des Suffix-Arrays $\mathsf{SA}$ die Position jedes RMS-Suffixes im Text abgespeichert, d.h. in $\mathsf{SA}[n-m\dots n)$. Dies bezeichnen wir der Einfachheit halber mit $\mathsf{PAb}[i] = \mathsf{SA}[n-m+i]$. Der Durchlauf von rechts nach links ermöglicht es, auch bei $\mathsf{T}[i] = \mathsf{T}[i-1]$ sofort den Typen bestimmen zu können, was bei einem Durchlauf von links nach rechts nicht so leicht möglich wäre. Tabelle \ref{table:sa-init} zeigt das initiale Suffix-Array sowie die Suffix-Typen für den Beispielstring $ \mathsf{T} = \text{caabaccaabacaa\$}$.



% Please add the following required packages to your document preamble:
% \usepackage[table,xcdraw]{xcolor}
% If you use beamer only pass "xcolor=table" option, i.e. \documentclass[xcolor=table]{beamer}
\begin{table}[]
\begin{tabular}{|l|l|l|l|l|l|l|l|}
\hline
            & \$                        & a                         & b                         & c                          & (a,a) & (a,b)                     & (a,c)                     \\ \hline
$\mathsf{Bucket\_L}$   & 1                         & 2                         & 2                         & 4                          &       &                           &                           \\ \hline
$\mathsf{Bucket\_S}$   &                           &                           &                           &                            & 2     &                           &                           \\ \hline
$\mathsf{Bucket\_RMS}$ &                           &                           &                           &                            &       & 2                         & 2                         \\ \hline
$\mathsf{Bucket\_L}$   & \cellcolor[HTML]{34CDF9}0 & \cellcolor[HTML]{34CDF9}1 & \cellcolor[HTML]{34CDF9}9 & \cellcolor[HTML]{34CDF9}11 &       &                           &                           \\ \hline
$\mathsf{Bucket\_S}$   &                           &                           &                           &                            & 2     &                           &                           \\ \hline
$\mathsf{Bucket\_RMS}$ &                           &                           &                           &                            &       & \cellcolor[HTML]{32CB00}2 & \cellcolor[HTML]{32CB00}4 \\ \hline
\end{tabular}
	\caption{Berechnung der Bucketgrößen und der Präfixsummen für L- und RMS-Buckets. Die L-Buckets enthalten die Präfixsummen über alle Suffix-Typen (linke Grenze des Buckets), die RMS-Buckets beinhalten die rechten Grenzen des jeweiligen Buckets (bestehend aus zwei Zeichen).}
	\label{table:prefixsum}
\end{table}


Als nächstes werden die Präfixsummen für $\mathsf{BUCK\-ET\_L}$ und $\mathsf{BUCK\-ET\_RMS}$ berechnet. Dabei werden für $\mathsf{BUCKET\_L}[c0]$ die Mengen der Buckets aller vorherigen Symbole $c_i$ aufaddiert (inkl. $\mathsf{BUCKET\_S}$ und $\mathsf{BUCKET\_RMS}$). Für die Buckets $\mathsf{BUCKET\_RMS}[c0,c1]$ hingegen muss die Präfixsumme über alle vorigen $\mathsf{BUCK\-ET\_RMS}[c_i,c_j]$  mit $c_i \leq c0, c_j \leq c1$ berechnet werden. Dies führt dazu, dass $\mathsf{BUCKET\_L}[c0]$ die linkeste Position jedes $\mathsf{b}_{c0}$ enthält. Für $\mathsf{BUCKET\_RMS}[c0,c1]$ wird die rechteste Position der RMS-Suffixe in Relation zu anderen RMS-Suffixen abgespeichert, d.h. die Positionen sind aus dem Intervall $[0,m)$. Tabelle \ref{table:prefixsum} gibt die initiale Größe der Buckets sowie die berechneten Präfixsummen für den Beispielstring an.

Bevor wir mit dem vollständigen Sortieren der RMS-Suffixe beginnen, werden die Referenzen in $\mathsf{PAb}$ auf diese nach den ersten zwei Zeichen sortiert. Die Referenz auf das letzte RMS-Suffix wird an den Anfang des jeweiligen Buckets gesetzt, da kein nachfolgendes RMS-Suffix für dieses Suffix vorhanden ist, welcher jedoch für den Vergleich von RMS-Teilstrings benötigt wird. Nach dieser Sortierung zeigt $\mathsf{BUCKET\_RMS}[c0,c1]$ auf die linkeste Position aus $[0,m)$. In Tabelle \ref{table:bucket-order} sind die Referenzen auf die RMS-Suffixe aus unserem Beispiel derart sortiert.

% Please add the following required packages to your document preamble:
% \usepackage[table,xcdraw]{xcolor}
% If you use beamer only pass "xcolor=table" option, i.e. \documentclass[xcolor=table]{beamer}
\begin{table}
	\centering
	\begin{tabular}[t]{l|lllllllllllllll}
		$i$  & 0                         & 1                         & 2                         & 3                         & 4 & 5 & 6 & 7 & 8 & 9 & 10 & 11 & 12 & 13 & 14 \\ \hline
		$\mathsf{SA}$ & \cellcolor[HTML]{32CB00}0 & \cellcolor[HTML]{32CB00}2 & \cellcolor[HTML]{32CB00}3 & \cellcolor[HTML]{32CB00}1 & 0 & 0 & 0 & 0 & 0 & 0 & 0  & 2  & 4  & 8  & 10 \\ \hline
	\end{tabular}	\newline

	\caption{Finaler Schritt der Initialisierung: Sortierte Referenzen $\mathsf{PAb}$ auf Indizes der RMS-Suffixe für den Beispielstring caabaccaabacaa\$.}
	\label{table:bucket-order}
\end{table}

\begin{table}
	\begin{tabular}[t]{l|c|c|c|c|c|c|c|}
		\cline{2-8}
		& \multicolumn{1}{l|}{\$} & \multicolumn{1}{l|}{a} & \multicolumn{1}{l|}{b} & \multicolumn{1}{l|}{c} & \multicolumn{1}{l|}{(a,a)} & \multicolumn{1}{l|}{(a,b)} & \multicolumn{1}{l|}{(a,c)} \\ \hline
		\multicolumn{1}{|l|}{$\mathsf{Bucket\_L}$}   & 0                       & 1                      & 9                      & 11                     &                            &                            &                            \\ \hline
		\multicolumn{1}{|l|}{$\mathsf{Bucket\_S}$}   &                         &                        &                        &                        & 2                          &                            &                            \\ \hline
		\multicolumn{1}{|l|}{$\mathsf{Bucket\_RMS}$} &                         &                        &                        &                        &                            & \cellcolor[HTML]{32CB00}0  & \cellcolor[HTML]{32CB00}2  \\ \hline
	\end{tabular}
	\caption{Finaler Schritt der Initialisierung: Alle RMS-Buckets zeigen auf die linke Grenze des Buckets.}
\end{table}

Nachdem die Initialisierung abgeschlossen wurde, folgt nun der Kern dieses Algorithmus: Das Sortieren der RMS-Suffixe.


\subsubsection{RMS-Suffixe sortieren}
Das Sortieren der RMS-Suffixe kann in drei Schritte unterteilt werden. Im ersten Schritt werden die RMS-Teilstrings für jeden Bucket $\mathsf{b}_{c0,c1}$ sortiert. Schritt zwei besteht aus der Erzeugung eines partiellen inversen Suffix-Arrays $\mathsf{ISAb}$, in welchem die Ränge der nach den RMS-Teilstrings partiell sortierten RMS-Suffixe enthalten sind. Diese Ränge werden verwendet, um mit einem Verfahren ähnlich zum Prefix-Doubling die lexikografische Ordnung aller RMS-Suffixe zu bestimmen. Dies wird mit dem Ansatz der \textit{repetition detection} erweitert.

\begin{table}
	\begin{tabular}{l|l|l}
		Index & Referenzindex & RMS-Teilstring \\ \hline
		2     & 0             & abac           \\ \hline
		8     & 2             & abac           \\ \hline
		10    & 3             & acaa\$         \\ \hline
		4     & 1             & accaab         \\ \hline
	\end{tabular}
	\caption{Lexikografisch sortierte RMS-Teilstrings. Die Suffixe an Position 2 und 8 haben denselben Teilstring und sind daher nicht eindeutig sortiert.}
	\label{dss:table:substrings}
\end{table}

\subsubsection{Sortieren der RMS-Teilstrings}
Der aktuell nicht verwendete Bereich im Suffix-Array $\mathsf{SA}[m\dots n-m)$ dient als Puffer für das Sortieren der RMS-Teilstrings. Den Puffer referenzieren wir folgend als $\mathsf{buf}[i] = \mathsf{SA}[m+i]$ mit $0 \leq i < n - 2m$. In jedem $\mathsf{BUCK\-ET\_RMS}$ werden die RMS-Substrings sortiert. Dies bedeutet, dass mehrere $\mathsf{BUCK\-ET\_RMS}$ parallel sortiert werden können. Bei $p$ Prozessen erhält jeder Prozess eine Puffergröße von $\frac{|\mathsf{buf}|}{p}$. Die Anzahl der gleichzeitig zu sortierenden Elemente ist beschränkt durch einen Parameter, dessen Default-Wert 1024 beträgt. Ist die Größe des verfügbaren Puffers kleiner als 1024 (d.h. als der Parameter) oder kleiner als die Größe des aktuell zu sortierenden Buckets, so wird der Bucket in kleinere Teilbuckets aufgeteilt, die nach dem Sortieren gemerged werden. Falls der aktuell sortierte Bucket den letzten RMS-Teilstring enthält, so wird dieser bereits an die richtige Position gesetzt, da dieser nicht verglichen werden kann. 


% Please add the following required packages to your document preamble:
% \usepackage[table,xcdraw]{xcolor}
% If you use beamer only pass "xcolor=table" option, i.e. \documentclass[xcolor=table]{beamer}
\begin{table}
	\begin{tabular}{l|lllllllllllllll}
		$i$  & 0 & 1                                       & 2 & 3 & 4 & 5 & 6 & 7 & 8 & 9 & 10 & 11 & 12 & 13 & 14 \\ \hline
		$\mathsf{SA}$ & 0 & \cellcolor[HTML]{32CB00}2               & 3 & 1 & 0 & 0 & 0 & 0 & 0 & 0 & 0  & 2  & 4  & 8  & 10 \\ \hline
		$\mathsf{SA}$ & 0 & \cellcolor[HTML]{32CB00}$\underline{2}$ & 3 & 1 & 0 & 0 & 0 & 0 & 0 & 0 & 0  & 2  & 4  & 8  & 10 \\ \hline
	\end{tabular}
	\caption{RMS-Substrings sortiert. Bei identischen Substrings sind alle bis auf den ersten negiert (gekennzeichnet durch $\underline{\ }$).}
	\label{dss:table:substring-sorted}
\end{table}


Das richtige Sortieren wird über Intro Sort (\textit{ISS}) durchgeführt. Dabei wird Multikey Quicksort $\lfloor \lg (\text{last} - \text{first}) \rfloor$ Male ausgeführt, bevor Heapsort verwendet wird. Es wird dabei nicht rekursiv aufgerufen, sondern über einen Stack implementiert, welcher die unsortierten Teilintervalle enthält. Damit werden die kleineren Teilintervalle immer zuvor verarbeitet. Dies garantiert eine maximale Stack-Größe von $\lg l$, wobei $l$ die Größe des initialen Intervalls bezeichne. Unterschreitet die Größe des aktuell zu sortierenden (Teil-)Buckets einen Schwellwert (Defaultwert ist 8), so wird stattdessen Insertionsort verwendet. Beim Vergleich von Insertionsort werden die RMS-Teilstrings zeichenweise verglichen, angefangen bei der aktuellen Tiefe des Sortierens.

Beim Sortieren kann es vorkommen, dass einige der Teilstrings nicht vollständig sortiert werden können, d.h. sie sind identisch. Alle bis auf den ersten dieser Teilstrings in einem Intervall gleicher Teilstrings werden durch ihre bitweise negierte Referenz abgespeichert. Die erste Referenz in diesem Intervall repräsentiert den Beginn jenes Intervalls. Diese Uneindeutigkeit wird später aufgelöst. Tabelle \ref{dss:table:substrings} listet alle Teilstrings für unser Beispiel an. RMS-Suffixe 2 und 8 haben dabei einen identischen Teilstring. In Tabelle \ref{dss:table:substring-sorted} sind die sortierten Indizes in $\mathsf{PAb}$ enthalten. Da Index 8 (Referenzindex 2) identisch mit einem anderen war, wird dieser Index negiert.



% Please add the following required packages to your document preamble:
% \usepackage[table,xcdraw]{xcolor}
% If you use beamer only pass "xcolor=table" option, i.e. \documentclass[xcolor=table]{beamer}
% Please add the following required packages to your document preamble:
% \usepackage[table,xcdraw]{xcolor}
% If you use beamer only pass "xcolor=table" option, i.e. \documentclass[xcolor=table]{beamer}
\begin{table}
	\begin{tabular}{l|llll|llll|lllllll}
		$i$  & 0 & 1                         & 2                          & 3 & 4                         & 5                         & 6                         & 7                         & 8 & 9 & 10 & 11 & 12 & 13 & 14 \\ \hline
		$\mathsf{SA}$ & 0 & $\underline{2}$           & 3                          & 1 & 0                         & 0                         & 0                         & 0                         & 0 & 0 & 0  & 2  & 4  & 8  & 10 \\ \hline
		$\mathsf{SA}$ & 0 & $\underline{2}$           & 3                          & 1 & 0                         & \cellcolor[HTML]{32CB00}3 & 0                         & 0                         & 0 & 0 & 0  & 2  & 4  & 8  & 10 \\ \hline
		$\mathsf{SA}$ & 0 & $\underline{2}$           & 3                          & 1 & 0                         & 3                         & 0                         & \cellcolor[HTML]{32CB00}2 & 0 & 0 & 0  & 2  & 4  & 8  & 10 \\ \hline
		$\mathsf{SA}$ & 0 & \cellcolor[HTML]{34CDF9}2 & \cellcolor[HTML]{32CB00}-2 & 1 & 0                         & 3                         & \cellcolor[HTML]{32CB00}1 & 2                         & 0 & 0 & 0  & 2  & 4  & 8  & 10 \\ \hline
		$\mathsf{SA}$ & 0 & 2                         & -2                         & 1 & \cellcolor[HTML]{32CB00}1 & 3                         & 1                         & 2                         & 0 & 0 & 0  & 2  & 4  & 8  & 10 \\ \hline
		& \multicolumn{4}{l|}{$\mathsf{PAb}$}                                       & \multicolumn{4}{l|}{$\mathsf{ISAb}$}                                                                                     &   &   &    &    &    &    &   
	\end{tabular}
	\caption{Berechnung des initialen partiellen ISA. Unterstrichene Werte wurden im vorigen Schritt nicht eindeutig sortiert, während negative Werte bereits sortierte Intervalle darstellen.}
	\label{dss:table:isa}
\end{table}

\subsubsection{Partielles inverses Suffix-Array berechnen}
Als nächstes können wir ein partielles inverses Suffix-Array berechnen, welches die Ränge der RMS-Suffixe über die bereits partiell sortierten RMS-Teilstrings wiedergibt. Das inverse Suffix-Array repräsentieren wir als $\mathsf{ISAb}[i] = \mathsf{SA}[m + i]$ mit $0 \leq i < m$, d.h. es wird in $\mathsf{SA}[m\dots 2m)$ gespeichert. In $\mathsf{ISAb}[i]$ finden wir somit die Anzahl der kleineren RMS-Suffixe des $i$-ten RMS-Suffixes. Falls $m > \frac{n}{3}$ gilt, so überschneidet sich $\mathsf{ISAb}$ mit $\mathsf{PAb}$. Dies ist nicht weiter schlimm, da wir die Textpositionen nicht mehr benötigen.

Es wird $\mathsf{SA}[0\dots m)$ von rechts nach links gescannt. Wenn ein Wert kleiner 0 ist, so erreichen wir ein Intervall, in dem die RMS-Suffixe im vorherigen Schritt nicht eindeutig sortiert wurden. Wir weisen allen von ihnen den größtmöglichen Rang $m-i$ zu. $i$ entspricht dabei der Anzahl der größeren RMS-Suffixe. Zusätzlich müssen die Referenzen bitweise negiert werden (da sie nun \glqq vergleichbar\grqq{} sind). Falls der Wert jedoch $\geq 0$ ist und es nicht nach einem negierten Wert gescannt wurde (d.~h. nicht der Anfang eines unsortierten Intervalls ist), so weisen wir den korrekten Rang $m-i$ zu. Wann immer ein komplett sortiertes Intervall erkannt wird, so wird die Anfangsposition dieses Intervalls in $\mathsf{SA}[0\dots m)$ mit $-k$ markiert, wobei $k$ der Größe dieses Intervalls entspricht. Somit können alle sortierten Intervalle erkannt werden, da diese mit einem negativen Wert beginnen, dessen Absolutwert der Länge dieses Intervalls gleicht. Tabelle \ref{dss:table:isa} zeigt die schrittweise Berechnung für das initiale partielle ISA. Die letzten beiden Indizes ergeben ein sortiertes Intervall, weshalb an $\mathsf{PAb}[2]$ der Wert -2 eingetragen wird. Die relativen Indizes 0 und 2 erhalten denselben Rang, da ihre Teilstrings identisch sind.


\subsubsection{RMS-Suffixe sortieren}
Zu guter Letzt können wir die korrekten Ränge aller RMS-Suffixe bestimmen und diese in $\mathsf{ISAb}$ abspeichern. Die Ränge sind dabei zum Sortieren ausreichend, d.h. $\mathsf{PAb}$ und der Zugriff auf den originalen Text wird nicht mehr benötigt. Der verfolgte Ansatz ähnelt dabei dem des Prefix-Doubling: in jeder Iteration $k$ betrachten wir nicht die doppelte Präfixlänge, sondern die doppelte Anzahl der zu betrachtenden RMS-Teilstrings, welche eine beliebige Länge haben können. Mit $\mathsf{ISAd}[i]$ referenzieren wir den Rang des $i + 2^k$-ten RMS-Suffixes. Die Ränge der RMS-Suffixe müssen dabei aktualisiert werden, wenn die Anzahl der betrachteten RMS-Teilstrings verdoppelt wird. Da die Ränge der RMS-Suffixe in $\mathsf{ISA}$ in Textreihenfolge gegeben sind, kann der Rang des nächstgelegenen RMS-Teilstrings jederzeit für beliebige Teilstrings abgerufen werden.

% Please add the following required packages to your document preamble:
% \usepackage[table,xcdraw]{xcolor}
% If you use beamer only pass "xcolor=table" option, i.e. \documentclass[xcolor=table]{beamer}
\begin{table}
	\begin{tabular}{l|lllllllllllllll}
		$i$  & 0                         & 1                         & 2                                        & 3                                       & 4 & 5 & 6 & 7 & 8 & 9 & 10 & 11 & 12 & 13 & 14 \\ \hline
		$\mathsf{T}$  & c                         & a                         & a                                        & b                                       & a & c & c & a & a & b & a  & c  & a  & a  & \$ \\ \hline
		$\mathsf{SA}$ & -4                        & 0                         & -2                                       & 1                                       & 1 & 3 & 0 & 2 & 0 & 0 & 0  & 2  & 4  & 8  & 10 \\ \hline
		$\mathsf{SA}$ & \cellcolor[HTML]{32CB00}8 & \cellcolor[HTML]{32CB00}2 & \cellcolor[HTML]{32CB00}$\underline{10}$ & \cellcolor[HTML]{32CB00}$\underline{4}$ & 1 & 3 & 0 & 2 & 0 & 0 & 0  & 2  & 4  & 8  & 10 \\ \hline
	\end{tabular}
	\caption{Bestimmung der absoluten RMS-Suffixindizes in der korrekten Reihenfolge (in $\mathsf{SA}[0\dots 4)$) anhand der Werte in $\mathsf{ISAb}$. Unterstrichene Werte kennzeichnen die Negierung vor dem ersten Induzieren.}
	\label{dss:table:correct-indices}
\end{table}

\subsubsection{Repetition-Detection}
Bevor wir uns mit dem Induzieren der L- und S-Suffixe beschäftigen, folgt zuvor noch eine Anpassung mittels \textit{repetition detection}. Diese Anpassung beschreibt den Sortierschritt im vorigen Absatz.

Definition \ref{repetition} besagt, wie eine Wiederholung in einem Text definiert ist. Dies kann problematisch werden, falls $\mathsf{S}_i$ ein RMS-Suffix ist, denn dann wäre $\mathsf{S}_{kp}$ ebenfalls ein RMS-Suffix für $k \leq r$. Um dies aufzulösen, können wir das erste Symbol betrachten, welches nicht Teil der Wiederholung ist, d.h. $\mathsf{T}[i + l] \neq \mathsf{T}[i + rp + l]$. Falls $\mathsf{T}[i + rp + l] < \mathsf{T}[i + l]$, dann gilt für $1 < i \leq r$:$\mathsf{T}[i + (r-1)p + 1, i+rp] < \mathsf{T}[(i-1) + (r-1)p + 1, (i-1) + rp]$. Mit anderen Worten, die RMS-Suffixe dieser Wiederholung sind in absteigender Reihenfolge sortiert. Der analoge Fall gilt für $\mathsf{T}[i + rp + l] > \mathsf{T}[i + l]$, d.h. $\mathsf{T}[i + (r-1)p + 1, i + rp] > \mathsf{T}[(i-1) + (r-1)p + 1, (i-1) + rp]$, die RMS-Suffixe liegen in aufsteigender Reihenfolge vor.

Verwenden wir Quicksort mit den zuvor sortierten Rängen als Schlüssel, so können wir diese \textit{repetition detection} verwenden. Wir verwenden den Rang des Medians der aktuell betrachteten RMS-Suffixe als Pivotelement. Ist der aktuelle Rang des ersten RMS-Suffixes im aktuell betrachteten Teilintervall gleich zu dem Pivotelement, so liegt eine Wiederholung vor ($\mathsf{ISAb}[i] = \mathsf{ISAd}[i]$). Wir verwenden wieder für $\lfloor \lg (\text{last} - \text{first})\rfloor $ Male Quicksort, bevor wir Heapsort zum Sortieren verwenden. Nachdem alle RMS-Suffixe durch das Sortieren in lexikografischer Reihenfolge liegen, durchlaufen wir den Text $\mathsf{T}$ erneut von rechts nach links. Beobachten wir RMS-Suffix $\mathsf{S}_i$ an Position $j$, so speichern wir den Index ab in $\mathsf{SA}[\mathsf{ISAb}[i]] = j$. Da wir im darauf folgenden Durchlauf zunächst nur S-Suffixe, aber keine L-Suffixe induzieren möchten, speichern wir die bitweise Negation von $j$, falls $S_{j-1}$ ein L-Suffix ist. Die korrekten RMS-Indizes für unser Beispiel sind in Tabelle \ref{dss:table:correct-indices} dargestellt.

% Please add the following required packages to your document preamble:
% \usepackage[table,xcdraw]{xcolor}
% If you use beamer only pass "xcolor=table" option, i.e. \documentclass[xcolor=table]{beamer}
\begin{table}
	\begin{tabular}{l|lllllllllllllll}
		& \multicolumn{1}{l|}{\$} & \multicolumn{8}{l|}{a}                                                                                                                                                                  & \multicolumn{2}{l|}{b} & \multicolumn{4}{l|}{c} \\ \hline
		$i$  & 0                       & 1 & 2                & 3               & 4 & 5                         & 6                         & 7                                        & 8                                       & 9         & 10        & 11   & 12  & 13  & 14  \\ \hline
		$\mathsf{T}$  & c                       & a & a                & b               & a & c                         & c                         & a                                        & a                                       & b         & a         & c    & a   & a   & \$  \\ \hline
		$\mathsf{SA}$ & 8                       & 2 & $\underline{10}$ & $\underline{4}$ & 1 & \cellcolor[HTML]{32CB00}8 & \cellcolor[HTML]{32CB00}2 & \cellcolor[HTML]{32CB00}$\underline{10}$ & \cellcolor[HTML]{32CB00}$\underline{4}$ & 0         & 0         & 2    & 4   & 8   & 10  \\ \hline
	\end{tabular}
	\caption{Setzen der Indizes der sortierten RMS-Suffixe an die richtige Position in $\mathsf{SA}$.}
	\label{dss:table:final-sort}
\end{table}

Nun sind die Positionen aller RMS-Suffixe im Text in $\mathsf{SA}[0\dots m)$ in lexikografischer Reihenfolge abgespeichert. Diese müssen als nächstes an ihre richtige Position gebracht werden. Das Ergebnis für unser Beispiel sieht man in Tabelle \ref{dss:table:final-sort}. Währenddessen können $\mathsf{BUCK\-ET\_S}$ sowie $\mathsf{BUCK\-ET\_RMS}$ derart angepasst werden, dass alle S-Buckets die rechte Grenze des jeweiligen Buckets beinhalten, um beim Induzieren das S-Suffix direkt an die richtige Position zu setzen. Für RMS-Suffixe wird dabei nur der Bucket $\mathsf{b}_{c0,c0+1}$ aktualisiert, da diese die linke Grenze eines Buckets $\mathsf{b}_{c0}$ beinhalten, bis zu welcher S-Suffixe eingefügt werden können. Tabelle \ref{dss:table:last-buckets} gibt die Berechnung der Buckets für unseren Beispielstring an.

% Please add the following required packages to your document preamble:
% \usepackage[table,xcdraw]{xcolor}
% If you use beamer only pass "xcolor=table" option, i.e. \documentclass[xcolor=table]{beamer}
\begin{table}
	\begin{tabular}{l|l|l|l|l|l|l|l|}
		\cline{2-8}
		& \$ & a & b & c  & (a,a)                     & (a,b)                     & (a,c)                     \\ \hline
		\multicolumn{1}{|l|}{$\mathsf{Bucket\_L}$}   & 0  & 1 & 9 & 11 &                           &                           &                           \\ \hline
		\multicolumn{1}{|l|}{$\mathsf{Bucket\_S}$}   &    &   &   &    & \cellcolor[HTML]{32CB00}4 & \cellcolor[HTML]{32CB00}6 & \cellcolor[HTML]{32CB00}8 \\ \hline
		\multicolumn{1}{|l|}{$\mathsf{Bucket\_RMS}$} &    &   &   &    &                           & \cellcolor[HTML]{34CDF9}3 & 2                         \\ \hline
	\end{tabular}
	\caption{Berechnung der S- und RMS-Buckets für das Induzieren. Für die S-Buckets wurden die rechten Grenzen der jeweiligen Buckets bestimmt (grün markiert). Bei RMS-Buckets wird nur der Bucket $\mathsf{b}_{c0, c0+1}$ berechnet (blau markiert), um die linke Grenze für S-Suffixe zu markieren (läuft beim Induzieren der S-Buckets nur bis zu dieser Grenze für $\mathsf{b}_{c0}$)}
	\label{dss:table:last-buckets}
\end{table}
%Beispiel für Suffix-Typen?


\subsubsection{Induzieren von L- und S-Suffixen}



% Please add the following required packages to your document preamble:
% \usepackage[table,xcdraw]{xcolor}
% If you use beamer only pass "xcolor=table" option, i.e. \documentclass[xcolor=table]{beamer}
% Please add the following required packages to your document preamble:
% \usepackage{graphicx}
% Please add the following required packages to your document preamble:
% \usepackage{graphicx}
% \usepackage[table,xcdraw]{xcolor}
% If you use beamer only pass "xcolor=table" option, i.e. \documentclass[xcolor=table]{beamer}
\begin{table}
	\centering
	\resizebox{\textwidth}{!}{%
		\begin{tabular}{l|lllllllllllllll|l}
			i  & 0                          & 1  & 2                & 3                                       & 4                                       & 5                                       & 6                                       & 7                          & 8                         & 9 & 10 & 11 & 12 & 13 & 14 & Bucket\_S{[}a, a{]}       \\ \hline
			SA & 14                         & 2  & $\underline{10}$ & $\underline{4}$                         & 1                                       & 8                                       & 2                                       & $\underline{10}$           & $\underline{\textbf{4}}$  & 0 & 0  & 2  & 4  & 8  & 10 & 4                         \\ \hline
			SA & 14                         & 13 & $\underline{10}$ & $\underline{4}$                         & 1                                       & 8                                       & 2                                       & $\underline{\textbf{10}}$  &\cellcolor[HTML]{34CDF9}4 & 0 & 0  & 2  & 4  & 8  & 10 & 4                         \\ \hline
			SA & 14                         & 13 & $\underline{10}$ & $\underline{4}$                         & 1                                       & 8                                       & \textbf{2}                              & \cellcolor[HTML]{34CDF9}10 & 4                         & 0 & 0  & 2  & 4  & 8  & 10 & 4                         \\ \hline
			SA & 14                         & 13 & $\underline{10}$ & $\underline{4}$                         & \cellcolor[HTML]{32CB00}$\underline{1}$ & \textbf{8}                              & \cellcolor[HTML]{34CDF9}$\underline{2}$ & 10                         & 4                         & 0 & 0  & 2  & 4  & 8  & 10 & \cellcolor[HTML]{32CB00}3 \\ \hline
			SA & 14                         & 13 & $\underline{10}$ & \cellcolor[HTML]{32CB00}$\underline{7}$ & $\underline{\textbf{1}}$                & \cellcolor[HTML]{34CDF9}$\underline{8}$ & $\underline{2}$                         & 10                         & 4                         & 0 & 0  & 2  & 4  & 8  & 10 & \cellcolor[HTML]{32CB00}2 \\ \hline
			SA & 14                         & 13 & $\underline{10}$ & $\underline{\textbf{7}}$                & \cellcolor[HTML]{34CDF9}1               & $\underline{8}$                         & $\underline{2}$                         & 10                         & 4                         & 0 & 0  & 2  & 4  & 8  & 10 & 2                         \\ \hline
			SA & 14                         & 13 & $\underline{10}$ & \cellcolor[HTML]{34CDF9}7               & 1                                       & $\underline{8}$                         & $\underline{2}$                         & 10                         & 4                         & 0 & 0  & 2  & 4  & 8  & 10 & 2                         \\ \hline
			SA & \cellcolor[HTML]{32CB00}14 & 2  & $\underline{10}$ & 7                                       & 1                                       & $\underline{8}$                         & $\underline{2}$                         & 10                         & 4                         & 0 & 0  & 2  & 4  & 8  & 10 & 2 \\ \hline                       
		\end{tabular}%
	}
	\caption{Induzieren der S-Suffixe. Neu eingefügte Indizes sind grün hinterlegt, wohingegen blau hinterlegte Werte negiert wurden. Es wird immer der fett markierte Index zum Induzieren betrachtet.}
	\label{dss:table:induce-s}
\end{table}

% Please add the following required packages to your document preamble:
% \usepackage[table,xcdraw]{xcolor}
% If you use beamer only pass "xcolor=table" option, i.e. \documentclass[xcolor=table]{beamer}
\begin{table}
	\centering
	\resizebox{\textwidth}{!}{%
		\begin{tabular}{l|l|lllllllllllllll}
			Schritt & i  & 0           & 1                                   & 2                                   & 3          & 4          & 5                         & 6                         & 7           & 8          & 9                                       & 10                                      & 11                                       & 12                        & 13                        & 14                        \\ \hline
			0       & SA & \textbf{14} & 2                                   & $\underline{10}$                    & 7          & 1          & $\underline{8}$           & $\underline{2}$           & 10          & 4          & 0                                       & 0                                       & 2                                        & 4                         & 8                         & 10                        \\ \hline
			1       & SA & 14          & \cellcolor[HTML]{32CB00}\textbf{13} & $\underline{10}$                    & 7          & 1          & $\underline{8}$           & $\underline{2}$           & 10          & 4          & 0                                       & 0                                       & 2                                        & 4                         & 8                         & 10                        \\ \hline
			2       & SA & 14          & 13                                  & \cellcolor[HTML]{32CB00}\textbf{12} & 7          & 1          & $\underline{8}$           & $\underline{2}$           & 10          & 4          & 0                                       & 0                                       & 2                                        & 4                         & 8                         & 10                        \\ \hline
			3       & SA & 14          & 13                                  & 12                                  & \textbf{7} & 1          & $\underline{8}$           & $\underline{2}$           & 10          & 4          & 0                                       & 0                                       & \cellcolor[HTML]{32CB00}$\underline{11}$ & 4                         & 8                         & 10                        \\ \hline
			4       & SA & 14          & 13                                  & 12                                  & 7          & \textbf{1} & $\underline{8}$           & $\underline{2}$           & 10          & 4          & 0                                       & 0                                       & $\underline{11}$                         & \cellcolor[HTML]{32CB00}6 & 8                         & 10                        \\ \hline
			5       & SA & 14          & 13                                  & 12                                  & 7          & 1          & $\underline{\textbf{8}}$  & $\underline{2}$           & 10          & 4          & 0                                       & 0                                       & $\underline{11}$                         & 6                         & \cellcolor[HTML]{32CB00}0 & 10                        \\ \hline
			6       & SA & 14          & 13                                  & 12                                  & 7          & 1          & \cellcolor[HTML]{34CDF9}8 & $\underline{\textbf{2}}$  & 10          & 4          & 0                                       & 0                                       & $\underline{11}$                         & 6                         & 0                         & 10                        \\ \hline
			7       & SA & 14          & 13                                  & 12                                  & 7          & 1          & 8                         & \cellcolor[HTML]{34CDF9}2 & \textbf{10} & 4          & 0                                       & 0                                       & $\underline{11}$                         & 6                         & 0                         & 10                        \\ \hline
			8       & SA & 14          & 13                                  & 12                                  & 7          & 1          & 8                         & 2                         & 10          & \textbf{4} & \cellcolor[HTML]{32CB00}$\underline{9}$ & 0                                       & $\underline{11}$                         & 6                         & 0                         & 10                        \\ \hline
			9       & SA & 14          & 13                                  & 12                                  & 7          & 1          & 8                         & 2                         & 10          & 4          & $\underline{\textbf{9}}$                & \cellcolor[HTML]{32CB00}$\underline{3}$ & $\underline{11}$                         & 6                         & 0                         & 10                        \\ \hline
			10      & SA & 14          & 13                                  & 12                                  & 7          & 1          & 8                         & 2                         & 10          & 4          & \cellcolor[HTML]{34CDF9}9               & $\underline{\textbf{3}}$                & $\underline{11}$                         & 6                         & 0                         & 10                        \\ \hline
			11      & SA & 14          & 13                                  & 12                                  & 7          & 1          & 8                         & 2                         & 10          & 4          & 9                                       & \cellcolor[HTML]{34CDF9}3               & $\underline{\textbf{11}}$                & 6                         & 0                         & 10                        \\ \hline
			12      & SA & 14          & 13                                  & 12                                  & 7          & 1          & 8                         & 2                         & 10          & 4          & 9                                       & 3                                       & \cellcolor[HTML]{34CDF9}11               & \textbf{6}                & 0                         & 10                        \\ \hline
			13      & SA & 14          & 13                                  & 12                                  & 7          & 1          & 8                         & 2                         & 10          & 4          & 9                                       & 3                                       & 11                                       & 6                         & 0                         & \cellcolor[HTML]{32CB00}5 \\ \hline
		\end{tabular}%
	}
% Please add the following required packages to your document preamble:
% \usepackage{graphicx}
	\centering
	\begin{tabular}{l|l|l|l|l}
		Schritt & Bucket\_L{[}\${]} & Bucket\_L{[}a{]}          & Bucket\_L{[}b{]}           & Bucket\_L{[}c{]}           \\ \hline
		0       & 1                 & 1                         & 9                          & 11                         \\ \hline
		1       & 1                 & \cellcolor[HTML]{32CB00}2 & 9                          & 11                         \\ \hline
		2       & 1                 & \cellcolor[HTML]{32CB00}3 & 9                          & 11                         \\ \hline
		3       & 1                 & 3                         & 9                          & \cellcolor[HTML]{32CB00}12 \\ \hline
		4       & 1                 & 3                         & 9                          & \cellcolor[HTML]{32CB00}13 \\ \hline
		5       & 1                 & 3                         & 9                          & \cellcolor[HTML]{32CB00}14 \\ \hline
		6       & 1                 & 3                         & 9                          & 14                         \\ \hline
		7       & 1                 & 3                         & 9                          & 14                         \\ \hline
		8       & 1                 & 3                         & \cellcolor[HTML]{32CB00}10 & 14                         \\ \hline
		9       & 1                 & 3                         & \cellcolor[HTML]{32CB00}11 & 14                         \\ \hline
		10      & 1                 & 3                         & 11                         & 14                         \\ \hline
		11      & 1                 & 3                         & 11                         & 14                         \\ \hline
		12      & 1                 & 3                         & 11                         & 14                         \\ \hline
		13      & 1                 & 3                         & 11                         & \cellcolor[HTML]{32CB00}15 \\ \hline
	\end{tabular}%
	\caption{Induzieren der L-Suffixe. Grün markierte Indizes wurden neu eingefügt, fett hervorgehobene Indizes werden zum Induzieren verwendet und blau hinterlegte Werte wurden negiert.}
	\label{dss:table:induce-l}
\end{table}


Durch die Typen der Suffixe wissen wir, dass in beliebigen Buckets $b_{c0,c1}$ L-Suffixe lexikografisch kleiner als S-Suffixe sowie RMS-Suffixe kleiner als S-Suffixe sind. Wir wissen ebenfalls, dass bei lexikografischer Ordnung alle direkt aneinandergereihten S-Suffixe links von mindestens einem RMS-Suffix sind (welches zu einem L-Suffix wechselt), da $S_{n-1}$ ein L-Suffix ist. Ebenso sind (in lexikografischer Reihenfolge) alle L-Suffixe rechts von mindestens einem S-Suffix. Das Suffix-Array wird nun zwei Mal durchlaufen. Beim ersten Mal laufen wir von rechts nach links und induzieren alle S-Suffixe, beim zweiten Mal laufen wir von links nach rechts und induzieren die L-Suffixe.

Beim ersten Scan des Suffix-Arrays \textit{SA} speichern wir jedes Mal den Eintrag $i-1$ an die rechteste freie Position im Bucket $b_{c0,c1}$, wenn wir einen Eintrag $i > 0$ lesen. Falls $T[i-2] > T[i-1]$, so ist $S_{i-2}$ ein L-Suffix und wir negieren den Wert von $i-1$ bitweise, da $S_{i-2}$ in diesem Schritt nicht induziert wird. Bei diesem Durchlauf wird jeder Wert durch seinen bitweise negierten Wert überschrieben. Falls eine Stelle bereits bitweise negiert war, so wird sie im nächsten Scan betrachtet, da es induziert wurde und das dazugehörige Suffix ein L-Suffix ist (daher in diesem Durchlauf nicht relevant). Alle Suffixe, welche zur Induzierung verwendet wurden, haben ihre Position bitweise negiert, da sie ein S-Suffix induziert haben. Alle Anderen werden hingegen durch ihre Position repräsentiert und sind für den nächsten Durchlauf relevant. Alle induzierten Suffixe sind dabei lexikografisch kleiner als jene, von denen induziert wurde, da in diesem Durchlauf S-Suffixe betrachtet wurden und damit $c0 \leq c1$ für alle Buckets $b_{c0,c1}$ gelten muss. Ebenso können wir nur in $b_{c0,c1}$ mit $c1 \leq c0$ induzieren, da nur S-Suffixe betrachtet wurden. Für unser Beispiel können wir in Tabelle \ref{dss:table:induce-s} sehen, wie Schrittweise die Indizes für $b_{a}$ überprüft und induziert werden. Der letzte zu prüfende Index 3 ist in \texttt{BUCK\-ET\_RMS[a,b]} gespeichert (s. Tabelle \ref{dss:table:last-buckets}).

Vor dem zweiten Durchlauf von links nach rechts wird $n-1$ an den Anfang des $T[n-1]$-Buckets gesetzt. Falls $S_{n-2}$ ein L-Suffix ist, so speichern wir $n-1$, da wir $S_{n-2}$ induzieren wollen. Andernfalls soll der bitweise negierte Wert von $n-1$ abgespeichert werden. Nun kann der Durchlauf beginnen.
Liegt ein Eintrag $i < 0$ vor, so wurde dieser bereits an seine richtige Position gesetzt und bitweise negiert, damit die korrekte Position an dieser Stelle steht. Ist $i > 0$, so muss das Suffix $S_{i-1}$ an die linkeste freie Position im Bucket $b_{T[i-1]}$ induziert werden. Alle übrigen Suffixe werden in diesem Durchlauf induziert, sodass die linke Grenze für $b_{c0}$, welche in \textit{BUCK\-ET\_L$[c0]$} gespeichert wird, ausreichend ist. Wenn das induzierte Suffix $S_{i-1}$ ein S-Suffix induzieren würde, so wird stattdessen der bitweise negierte Wert induziert, da das Suffix beim scannen des Indizes übersprungen (nur negiert) wird. In Tabelle \ref{dss:table:induce-l} sind im oberen Teil die Schritte für die Indizes beim Induzieren dargestellt. Im unteren Teil werden die Änderungen der jeweiligen \texttt{BUCK\-ET\_L} nach Einfügen eines Suffixes aufgelistet.

\subsection{Implementierung}

In unserer Implementierung wurden einige Stellen vereinfacht umgesetzt als bei der Referenz und in den vorigen Abschnitten beschrieben. Die Typen wurden (bis auf den Linksdurchlauf beim Induzieren der L-Suffixe) anhand von externen Methoden bestimmt, welche den Typen des Nachfolgers übergeben bekommt, um den Typen bei gleichen Zeichen korrekt zu bestimmen. Beim Sortieren der Teilstrings wurde statt MK-QS innerhalb des Introsorts eine Vergleichsfunktion gewählt, welche alle Teilstrings enthält und diese (bei zwei gegebenen Suffix-Indizes) Zeichenweise vergleicht. 
Der größte Unterschied liegt beim Sortieren der RMS-Suffixe, nachdem die Teilstrings vorsortiert und das initiale partielle \textbf{ISA} bestimmt wurde. In der Referenz wurde die Repetition Detection innerhalb des Quick- bzw. Introsorts integriert. In unserer Implementierung hingegen ist noch eine einfachere, iterative Idee verbaut: Wir durchlaufen das \textbf{PAb}, d.~h. die Referenzindizes bzw. Indikatoren für sortierte Intervalle und suchen unsortierte Intervalle bei einem Durchlauf von links nach rechts. Bei sortierten Intervallen können über den eingetragenen negierten Wert das komplette Intervall überspringen. Bei unsortierten Intervallen müssen wir zum Einen prüfen, ob es sich um ein einzelnes Intervall handelt. Dies können wir daran erkennen, dass alle Ränge identisch sind. Bei einem unterschiedlichen Rang beginnt ein weiteres, unsortiertes Intervall. Ist ein komplettes unsortiertes Intervall bestimmt, kann anhand des Teilstring-Doublings über den Rang des verdoppelten Teilstrings, d.~h. über den Rang des zusätzlich hinten dran gehangenen Teilstrings, sortiert werden. Danach müssen in diesem Intervall alle Ränge neu berechnet werden, um zwischen neuen sortierten und unterschiedlichen unsortierten Intervallen weiterhin unterscheiden zu können. Dies wird solange wiederholt, bis in einem vollständigen Durchlauf der Referenzen kein unsortiertes Intervall vorhanden ist. Dann erst sind alle Ränge eindeutig bestimmt und die Suffixe können an die richtige Position ins Suffix-Array gesetzt werden.

%TODO: Zusammenfassung überarbeiten
\subsection{Zusammenfassung}
Wir haben einen der schnellsten Algorithmen in der Praxis zur Konstruktion von Suffix-Arrays kennen gelernt. Neben der Unterteilung in L-, S-, und RMS-Suffixe nutzt der Algorithmus auch die Verteilung dieser in den jeweiligen Buckets aus. In der ersten Phase haben wir zunächst die Größen der Buckets gezählt, die RMS-Suffixe sowie ihre relative Positionierung im Text ins Suffix-Array eingetragen und über die Präfixsummen die Grenzen der jeweiligen Buckets berechnet. In der zweiten Phase haben wir erst die RMS-Teilstrings der einzelnen Buckets $b_{c0,c1}$ mithilfe des Introspective Sorts sortiert, daraufhin das partielle inverse Suffix-Array bestimmt und mit den Rängen des ISA und dem Doppeln der RMS-Substrings die RMS-Suffixe mit einem ähnlichen Verfahren zum Introspective Sort korrekt sortiert. In der letzten Phase haben wir zuerst in einem Durchlauf durch die Ausnutzung der Ordnung innerhalb einzelner Buckets von rechts nach links alle S-Suffixe induziert, in einem zweiten Durchlauf von links nach rechts dann auch die l-Suffixe, was zum korrekt sortierten Suffix-Array führt. 



\iffalse
% Please add the following required packages to your document preamble:
% \usepackage[table,xcdraw]{xcolor}
% If you use beamer only pass "xcolor=table" option, i.e. \documentclass[xcolor=table]{beamer}
\begin{table}[]
	\begin{tabular}{l|lllllllllllllll}
		i  & 0 & 1 & 2                & 3               & 4 & 5                         & 6                         & 7                                        & 8                                       & 9 & 10 & 11 & 12 & 13 & 14 \\ \hline
		T  & c & a & a                & b               & a & c                         & c                         & a                                        & a                                       & b & a  & c  & a  & a  & \$ \\ \hline
		SA & 8 & 2 & $\widetilde{10}$ & $\widetilde{4}$ & 1 & 3                         & 0                         & 2                                        & 0                                       & 0 & 0  & 2  & 4  & 8  & 10 \\ \hline
		SA & 8 & 2 & $\widetilde{10}$ & $\widetilde{4}$ & 1 & \cellcolor[HTML]{32CB00}8 & \cellcolor[HTML]{32CB00}2 & \cellcolor[HTML]{32CB00}$\widetilde{10}$ & \cellcolor[HTML]{32CB00}$\widetilde{4}$ & 0 & 0  & 2  & 4  & 8  & 10
	\end{tabular}
	\caption{Setzen der RMS-Suffixindizes an die richtige Position in \textbf{SA}.}
\end{table}
\fi

