\section{SACA-K}
\label{section:saca_k}

\newcommand {\suf} {\text{suf}}
\newcommand {\level} {\varepsilon}
\newcommand {\symbWidth} {0.3cm}

 \currentauthor{Jonas Bode} 
Ähnlich wie der SAIS-Algorithmus beruht auch der SACA-K-Algorithmus auf der Technik des \textbf{Induced Sortings}. Die wichtige Frage -- und mitunter der größte Unterschied von SAIS und SACA-K -- ist, wie sich mit möglichst wenig Platzverbrauch eine Sortierung der LMS-Suffixe für $T$ herleiten lässt. Insgesamt kommt der SACA-K-Algorithmus auf eine nur von der Alphabetgröße $|\Sigma|$ abhängige Platzschranke für den zusätzlich benötigten Speicher bei weiterhin linearer Laufzeit. Im originalen Paper wird das Alphabet als $K$ bezeichnet, daher hat der SACA-K seinen Namen. In Abbildung $\ref{sacak}$ ist der SACA-K-Algorithmus in Pseudocode angegeben. Im Folgenden werden die einzelnen Komponenten des Algorithmus erklärt. \todo{In der figure rutscht die Referenz in eine neue Zeile} \\

\begin{figure}
\begin{minted}[escapeinside=@@,numbers=left]{python}
def saca_k(T, SA, A, @$\varepsilon$@):
# Input text T
# Allocated space SA for suffixarray
# Alphabet A of text T
# Recursion Depth @$\varepsilon$@

# Stage 1 - First Induced Sorting
 if @$\varepsilon$@ = 0
   Induced Sorting SA(T) using bucket-pointer-array bkt 
   bkt stores a bucket pointer for each @$\sigma \in$@ K
 else
   Induced Sorting SA(T) 
   using the first/last bucket entrys as bucket counters
		
# Stage 2 - Renaming the LMS Substrings
 Assign individual supersymbols to distinct LMS-Substrings
 This creates string T@$_1$@ with new alphabet K@$_1$@
	
# Stage 3 - Recursion
 if all supersymbols in T@$_1$@ are distinct (length(T@$_1$@) = size(A@$_1$@))
   No recursion needed, calculate SA(T@$_1$@) directly
 else
   saca_k(T@$_1$@, SA@$_1$@, A@$_1$@, @$\varepsilon$@+1)
   where SA@$_1$@ reuses the space of SA

# Now we know the correct order of the LMS-Substrings, given by SA(T@$_1$@)
# Stage 4 - Second Induced Sorting
 if @$\varepsilon$@ = 0
   Induced Sorting of SA(T) from SA(T@$_1$@) 
   with Bucket-Pointer-Array bkt
 else
   Induced Sorting of SA(T) from SA(T@$_1$@) 
   with bucket counters inside the bucket

\end{minted}
\caption{SACA-K Algorithmus} \cite{saca:7}
\label{sacak}
\end{figure}

Insgesamt ergibt sich für den SACA-K-Algorithmus ein Laufzeitaufwand von $\mathcal O(n)$, da die Aufrufe von Induced Sort lineare Zeit haben und jeder konstruierte String $T_1$ für die Rekursionsinstanzen höchstens die Hälfte der Länge von $T$ hat. Im Worst-Case ist die Laufzeit des SACA-K-Algorithmus also $\mathcal T(n) = \mathcal T(\lfloor n/2 \rfloor) + \mathcal O(n) = \mathcal O(n) + \mathcal O(\lfloor n/2 \rfloor) + \mathcal O(\lfloor n/4 \rfloor) + \ldots = \mathcal O(n)$. \\
Der Speicherbedarf des Algorithmus ist -- neben der Eingabe $T$ und der Ausgabe $SA(T)$ -- nur abhängig von der Größe des Alphabets, wegen der Erstellung des Arrays \textit{bkt}. In tieferen Rekursionsstufen wird das \textit{bkt}-Array nicht benötigt, da wir dort die Pointer in die Buckets in den Bucketgrenzen selbst speichern \todo{Leichter verständlich wäre denke ich: da wir dort die Pointer der Bucketgrenzen in die Bucktes speichern}. Die Ersparnis des Speicherplatzes im Vergleich zum SAIS kommt genau durch dieses Verhalten des Algorithmus. Für unsere Anwendungsfälle betrachten wir konstante Alphabete, daher können wir von konstantem Speicherplatzbedarf $\mathcal O(1)$ ausgehen. Um dies zu erzielen wird mehrmals der Speicher, welcher für SA(T) reserviert ist, in den Rekursionsinstanzen mit Zwischenberechnungen gefüllt und danach wieder überschrieben.  \\

Bei Terminierung des Algorithmus ist das Array $SA(T)$ in den Speicherplatz $SA$ geschrieben. Im Folgenden werden wir zuerst beleuchten, wie sich die Induced Sort-Technik auf den tieferen Rekursionsstufen vom SAIS unterscheidet. Danach werden das Benennungsverfahren für die Symbole von $T_1$ und einige Tricks \todo{und einige Techniken, welche den Platzverbrauch ...}, welche den Platzverbrauch des Algorithmus reduzieren, erklärt.

\subsection{Induced Sort im SACA-K}

Wir beschäftigen uns im Folgenden mit den Details der Induced Sort-Technik. In diesem Abschnitt erläutern wir die generelle Funktionsweise von Induced Sorting auf einem String $T$. Zuerst wiederholen wir die normale Vorgehensweise des Induced Sortings, so wie sie im SACA-K in der obersten Stufe und auch im SAIS angewendet wird. Darauf aufbauend zeigen wir danach, wie das Induced Sorting auf den tieferen Ebenen funktioniert. Wir gehen dabei davon aus, dass die geordneten LMS-Suffixe als ein (geordnetes)\todo{geordnetes würde ich gar nicht klammern} Index-Array $SA_1$ vorliegen, welches die Länge $n_1 < n$ hat und in den ersten $0$ bis $n_1-1$ Stellen von $SA$ gespeichert ist. Es ist also $SA[0,n_1-1] = SA_1$. \pagebreak

Wir verwenden die Abkürzungen RTL und LTR für \textit{Right-To-Left} und \textit{Left-To-Right}, als Iterationsvarianten, welche ein Array der Länge $n$ entweder mit der Iterationsfolge $n-1, n-2, \ldots, 0$ oder $0, 1, \ldots, n - 1$ durchgehen. 

\subsubsection{ Induced Sort für Rekursionstiefe $\level = 0$}

Anders als im SAIS-Algorithmus werden im SACA-K die L- und S-Typen der Suffixe immer wieder neu bestimmt und nicht abgespeichert. Es existiert also kein Array t \todo{t kursiv, so wie anderen Symbole?}, welches die Typen beinhaltet und initial berechnet werden muss. Um sich die Bucket-Anfänge und -Enden zu merken, allokiert das Programm Speicherplatz für das Array \textit{bkt} der Größe $|\Sigma|$. Da es maximal $|\Sigma|$-viele Buckets geben kann, wird \textit{bkt} so befüllt, dass für $i \in \{0, \ldots, K-1\}$ der Wert \textit{bkt}$[i]$ genau die End-Position des Buckets von Symbol $A[i]$ wiedergibt. Die SACA-K-Version von Induced Sort ergibt sich dann wie folgt:
\begin{itemize}
\item \textbf{1) -- Initialisierungsphase} \\
Initialisiere $SA[n_1, n-1]$, als \textit{null} bzw.\ leer.

\item \textbf{2) -- RTL Einordnungsphase (mit \textit{bkt})} \\
Füge in \textit{bkt}  alle \textbf{Endpositionen} der jeweiligen Buckets ein, wie oben beschrieben. Iteriere $SA_1$ RTL und füge jeden Index eines LMS-Suffixes, welches mit dem Symbol $A[k]$ beginnt, in die Stelle \textit{bkt}$[k]$ ein. Danach sei \textit{bkt}$[k] = $ \textit{bkt}$[k]-1$.

\item \textbf{3) -- LTR L-Phase (mit \textit{bkt})} \\
Füge in \textit{bkt}  alle \textbf{Startpositionen} der jeweiligen Buckets ein. Iteriere $SA$ LTR und für jedes nicht-leere $SA[i]$, mit $j = SA[i]-1$, sodass $T[j]$ ein L-Typ ist, schreibe $j$ an die Stelle \textit{bkt}$[k]$, falls $T[j] = A[k]$. Danach sei \textit{bkt}$[k] = $ \textit{bkt}$[k]+1$.

\item \textbf{4) -- RTL S-Phase (mit \textit{bkt})} \\
Füge in \textit{bkt}  alle \textbf{Endpositionen} der jeweiligen Buckets ein. Iteriere $SA$ RTL und für jedes nicht-leere $SA[i]$, mit $j = SA[i]-1$, sodass $T[j]$ ein S-Typ ist, schreibe $j$ an die Stelle \textit{bkt}$[k]$, falls $T[j] = A[k]$. Danach sei \textit{bkt}$[k] = $ \textit{bkt}$[k]-1$.
\end{itemize}

Ein Beispiel dieser Art des Induced Sortings lässt sich im SAIS-Kapitel \todo{Referenz zum SAIS einfügen} finden.

\subsubsection{ Induced Sort für Rekursionstiefe $\level > 0$}

Wir beschäftigen uns nun mit den Feinheiten von Induced Sort für Suffixe auf den tieferen Rekursionsebenen. Beschrieben wird hier nur die Sortierung der Suffixe für Rekursionstiefe $\level = 1$, da das Vorgehen für alle anderen Rekursionstiefen analog erfolgt. Der String, den wir betrachten ist $T_1$, welcher aus vollkommen neuen Symbolen besteht -- je ein Symbol pro verschiedenem Substring von den zu sortierenden Suffixen. Würden wir weiterhin ein Array \textit{bkt} zum Zählen verwenden, so würden wir im Worst-Case einen Speicheraufwand von $\mathcal O(n)$ riskieren, da es bis zu $\mathcal O(\frac{n}{2}) = \mathcal O(n)$ viele verschiedene LMS-Substrings von $T$ -- und damit auch eine von $n$ linear abhängige Alphabetgröße $K_1$ von $T_1$ -- geben kann. Der Trick \todo{Ich würde Technik statt Trick verwenden}, mit dem wir auch weiterhin einen konstanten Speicherverbrauch in linearer Zeit beibehalten können, ist eine spezielle Konstruktion des Strings $T_1$ (zu dem Konstruktionsmechanismus werden wir später kommen), sodass jeder L- bzw.\ S-Typ in $T_1$ auch ein Pointer zu dem start- bzw.\ dem end-Element seines Buckets ist.  \\
Wie schon vorher erwähnt wird für $SA_1$ kein zusätzlicher Speicherplatz benötigt, da wir einfach \todo{-einfach- würde ich hier weglassen} die bisherigen, nicht mehr gebrauchten Daten in $SA$ überschreiben können. Da $n_1  \leq \lfloor n/2 \rfloor$ gilt, kann es in $T_1$ nur noch halb so viele zu sortierende Suffixe geben wie in $T$. Dies ermöglicht uns, da die in $T_1$ betrachteten Indizes nur noch maximal halb so groß werden können wie in $T$, das letzte Bit jedes Eintrages in $SA$ für eine zusätzliche Information zu benutzen: Es ist 0, falls dieser Eintrag ein Suffix-Index ist und es ist 1, falls der Eintrag leer ist (hier mit \textit{null} dargestellt, kann als größte negative Zahl implementiert werden) oder als bucket counter gebraucht wird. Im Folgenden werden nur die beiden kritischen Phasen 3) und 4) (genannt L- und S-Phase) von Induced Sort und ihre Implementierung für tiefere Rekursionsebenen betrachtet.

\subsubsection{ L-Phase für Rekursionstiefe $\level > 0$}

Wie in der ursprünglichen Form des Algorithmus vorgesehen iterieren wir durch $SA_1$ LTR. Für jedes Paar $e = SA[i], j = e -1$, für das gilt
\[
	e \neq \text{\textit{null}} \wedge e > 0  \wedge \suf(T_1,j) \text{ ist L-Typ} 
\]
wird $j$ in seinen entsprechenden Bucket in $SA$ platziert. In diesem Fall ist $\suf(T_1,j)$ genau dann ein L-Typ, wenn $T_1[j] \geq T_1[j+1]$ gilt. Da das Zeichen $c = T_1[j]$ auf das start-Element seines eigenen Buckets zeigt, muss dieses in den Bucket der Stelle $SA[c]$ eingefügt werden. Bei diesem Vorgang können die folgenden drei verschiedenen Fälle auftreten:

\begin{itemize}
\item Wenn $SA[c] = $ \textit{null} ist, dann ist $j$ der erste in diesen Bucket einzufügende Index, da an dieser Stelle bisher weder ein Zähler in den Bucket noch ein Element des Buckets ist. Anstatt $j$ jedoch genau an dieser Stelle einzufügen, prüfen wir ob auch $SA[c+1]  = $ \textit{null} ist. Ist dies nicht der Fall, so beginnt an der Stelle $c+1$ bereits ein neuer Bucket und $j$ ist das einzige Element seines Buckets (da der Bucket nur Größe 1 haben kann). Dann schreiben wir $j$ and die Stelle $SA[c]$. Ist an der Stelle $c+1$ jedoch noch kein Element enthalten, so benutzen wir $SA[c]$ vorerst als einen Bucketzähler -- dies ist eine negative Zahl, welches die Zähler von den echten Elementen unterscheidet, welche angibt wie viele Elemente bereits im zugehörigen Bucket stehen. Außerdem setzen wir $SA[c] = -1$ und $SA[c+1] = j$. Im weiteren Verlauf des Algorithmus wird $SA[c]$ dann entweder weiter dekrementiert, wenn mehr Elemente in den Bucket hinzukommen. Sobald der Bucket mit allen seinen Elementen befüllt wurde, werden alle Elemente nach links geshiftet und überschreiben damit den Zähler.

\item Wenn $ 0 > SA[c] \neq$ \textit{null} ist, dann wird $SA[c]$ für einen Zähler benutzt und enthält die momentane (negative) Anzahl der Elemente im Bucket. Es sei $p = c + |SA[c]| + 1$, dann ist $p$ die Position an die der Index $j$ einsortiert wird, falls $SA[p] = $ \textit{null} ist und $SA[c] = SA[c] - 1$. Sollte der Wert in $SA[p]$ jedoch nicht \textit{null} sein, so haben wir das Ende unseres Buckets erreicht und würden mit dem Schreiben von $j$ in $SA[p]$ über die Grenzen unseres Buckets hinausgehen. In diesem Fall ist $j$ der letzte in den Bucket einzusortierende Index und wir führen für den gesamten Bucket einen links-shift aus, so dass $j$ in $SA[p-1]$ geschrieben werden kann und alle anderen Elemente einen Eintrag nach links wandern, bis der Zähler in $SA[c]$ von $SA[c+1]$ überschrieben wurde. Danach ist dieser Bucket vollständig.

\item Wenn $ 0 < SA[c] \neq$ \textit{null} ist, so wird $SA[c]$ weder als Indikator gebraucht noch ist es leer -- in diesem Fall benutzt der links anliegende Bucket also $SA[c]$ für seine eigene Auslagerung. Daraus lässt sich schließen, dass der links anliegende Bucket bereits voll ist und wir können (wie oben) für diesen Bucket einen links-shift ausführen, bis zu dem ersten nicht leeren, negativen Eintrag. Dann verfahren wir genauso wie im Fall $SA[c] = $ \textit{null} weiter für den Bucket von $SA[c]$.
\end{itemize}

Der Zeitaufwand ist hierbei linear, da das Einfügen jedes Index in konstanter Zeit passiert und jeder eingefügte Index maximal einmal geshiftet wird. Da maximal ein Index pro Iteration eingefügt wird, wird die Laufzeit von der Länge von $SA_1$ dominiert, welche $\mathcal O(n_1)$ ist.

\subsubsection{ S-Phase für Rekursionstiefe $\level > 0$}

Für diese Phase der Sortierung kommt uns die Eigenschaft von $T_1$ zuhilfe, dass für jeden S-Typ $\suf(T_1, i)$ der Eintrag $SA[i]$ das letzte Element des Buckets von $T_1[i]$ ist. Wie im Ursprungsalgorithmus vorgesehen iterieren wir RTL durch $SA_1$. Für jedes Element $SA[i] \neq $ \textit{null} und $SA[i] > 0$ mit $j = SA[i] -1$, sodass $\suf(T_1,j)$ ein S-Typ ist (dies ist der Fall, wenn $T_1[j] > T_1[j+1]$ gilt oder $T[j] = T[j+1]$ und $T[j] > i$ ist), wird $j$ in seinen entsprechenden Bucket in $SA$ ähnlich wie oben platziert und zwar für $c = T[j]$ wie folgt:

\begin{itemize}
\item Wenn $SA[c] = $ \textit{null} ist, dann ist $j$ der erste in diesen Bucket einzufügende Index. Wir prüfen ob $SA[c-1]$ leer ist. Falls ja, dann setzen wir $SA[c-1] = j$ und benutzen $SA[c]$ als einen Zähler indem wir initial $SA[c] = -1$ setzen. Falls nein, so umfasst der S-Teil des Buckets nur ein Element, nämlich $j$, und wir setzen $SA[c] = j$.
\item Wenn $ 0 > SA[c] \neq$ \textit{null} ist, dann wird $SA[c]$ für einen Zähler benutzt und enthält die momentane (negative) Anzahl der Elemente im Bucket. Falls für $p = c + SA[c] - 1$ der Wert von $SA[p]$ leer ist, so ist $SA[p] = j$ und wir setzen $SA[c] = SA[c] - 1$. Falls $SA[p]$ nicht leer ist, so ist $SA[p-1]$ das Ende des S-Teils dieses Buckets und wir führen einen rechts-shift aller Elemente im Bereich $SA[c, p]$ aus, bis $SA[c]$ durch $SA[c-1]$ überschrieben wurde und setzen dann $SA[p-1] = j$.
\item Wenn $ 0 < SA[c] \neq$ \textit{null} ist, so wird $SA[c]$ weder als Indikator gebraucht noch ist es leer -- in diesem Fall benutzt der rechts anliegende Bucket also $SA[c]$ für seine eigene Auslagerung. Der rechts anliegende Bucket ist also bereits voll und wir können für diesen rechten Bucket einen rechts-shift ausführen, bis zu dem ersten nicht leeren, negativen Eintrag (dem Zähler dieses Buckets). Dann verfahren wir für den Bucket von $SA[c]$ genauso wie im Fall $SA[c] = $ \textit{null} weiter.
\end{itemize}

Der Zeitaufwand ist $\mathcal O(n)$, analog zu dem der L-Phase.

\subsection{ Benennungsverfahren der LMS-Substrings}

In diesem Abschnitt werden wir erfahren, wie sich die neuen Symbole des Alphabets für die Strings $T_1$ ergeben, sodass die oben vorausgesetzte Eigenschaft gilt, dass jeder L- bzw.\ S-Typ in $T_1$ auch ein Pointer zu dem start- bzw.\ dem end-Element seines Buckets ist. Es werden die folgenden Definitionen gebraucht:

\begin{definition}\textbf{S-Rang und SE-Rang}\\
Für einen String $T$ der Länge $n$ und einen Index $i \in \{0, \ldots, n-1\}$ ist der S-Rang bzw.\ der SE-Rang von $T[i]$ die Anzahl von $T[j]$ für $j \neq i$ mit $T[j] < T[i]$ bzw.\ $T[j] \leq T[i]$. 
\end{definition}

Um den String $T_1$ aus den geordneten LMS-Substrings in $SA$ von $T$ zu erzeugen, wird jedem LMS-Substring wie folgt sein S- bzw. SE-Rang zugewiesen:

\begin{itemize}
\item Iteriere über $SA_1$ LTR um jeden LMS-Substring den start-Index seines Buckets zuzuweisen in welchen er sortiert wurde.
\item Iteriere über den entstandenen String RTL und ersetze jedes Symbol mit S-Typ durch den end-Index des Buckets auf den es zeigt.
\end{itemize}

Um den ersten Schritt ausführen zu können, muss der start-Index der jeweiligen Buckets korrekt identifiziert werden, indem je zwei benachbarte LMS-Substrings in $T$ miteinander verglichen werden. Um das Ende eines LMS-Substring feststellen zu können, wird der Substring durchlaufen bis zum ersten mal ein L-Typ mit einem darauf folgenden S-Typ erkannt wird. Damit ist das Benennungsverfahren mit einer Laufzeit von $\mathcal O(n)$ ausführbar.
