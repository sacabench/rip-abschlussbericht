\chapter{Einleitung}

Im \currentauthor{Christopher Poeplau und Marvin Böcker} Zuge der Projektgruppe 2018/2019 beschäftigt sich
\emph{SACABench} mit Suffix-Array-Konstruktions-Algorithmen. \emph{SACA} ist das Akronym für
\textbf{S}uffix \textbf{A}rray \textbf{C}onstruction \textbf{A}lgorithm und \emph{Bench} die Abkürzung für Benchmark,
also dem Messen von Laufzeit und Speicherplatz der Algorithmen.

Die Forschung an effizienten Konstruktionsalgorithmen für Suffix-Arrays hat in den letzten Jahren durch die
immer komplexer werdenden Anwendungen enorm an Bedeutung gewonnen. Gerade in der Bioinformatik,
in der man es im Bereich der Genomforschung mit Größenordnungen von Milliarden von Zeichen zu tun hat,
sind effiziente Algorithmen in Bezug auf Zeit und Speicherplatz notwendig~\cite[ch.~1]{saca:6}.

Ein Suffix-Array repräsentiert die Suffixe eines Strings $T$ in lexikographischer Reihenfolge.
Sei $T=suffix$. Für den String $T$ ergibt sich die folgende Suffix-Tabelle:
%
\begin{center}
  \begin{tabular}{ | l | c | r }
    \hline
        $i$ & $T[i, n)$ \\ \hline
        0 & suffix \\ \hline
        1 & uffix \\ \hline
        2 & ffix \\ \hline
        3 & fix \\ \hline
        4 & ix \\ \hline
        5 & x \\ \hline
        6 & \$ \\
    \hline
  \end{tabular}
\end{center}
%
Es ist sofort ersichtlich, dass $T[0, n) = T$ und $T[n-2, n)=\$$ gilt.
Sortiert man die Suffixe nun lexikografisch, ergibt sich das Suffix-Array:
%
\begin{center}
  \begin{tabular}{ | l | c | r }
    \hline
        $i$ & $T[i, n)$ \\ \hline
        6 & \$ \\ \hline
        3 & fix \\ \hline
        2 & ffix \\ \hline
        4 & ix \\ \hline
        0 & suffix \\ \hline
        1 & uffix \\ \hline
        5 & x \\
    \hline
  \end{tabular}
\end{center}
%
Somit ergibt sich als Suffix-Array $SA(T)$: $\{6,3,2,4,0,1,5\}$

Ziel ist die effiziente Konstruktion dieses Arrays.
Dabei gibt es Algorithmen, die den Fokus auf die Laufzeit setzen,
andere wiederum auf die Speicheroptimierung und wieder andere versuchen den besten Kompromiss aus beiden Welten zu finden.
Die Aufgabe der Projektgruppe ist das Schaffen einer umfangreichen Library der bekanntesten SACAs,
eingebettet in ein Framework, das es ermöglicht auf intuitiver Art und Weise Algorithmen auf beliebigen
Texten zu testen und die Performance miteinander zu vergleichen. Grundziel des Frameworks ist die Vereinheitlichung.
Viele der Algorithmen existieren in einzelnen Repositorys und die Algorithmen werden in den meisten Fällen 
nicht auf vergleichbarer Basis getestet und analysiert. Diese Algorithmen gilt es zunächst zu verstehen
und dann zu implementieren, sodass sie den Schnittstellen des Frameworks genügen.
Es soll also ein erweiterbares Gesamtkonstrukt geschaffen werden, das bestehende SACAs
sammelt und repräsentatives Vergleichen der Algorithmen ermöglicht.

\section{Notation und Definitionen}
Einige\currentauthor{Rosa Pink} grundlegende Notationen und Definitionen werden hier kurz vorangestellt.

\subsection{Alphabet und lexikographische Sortierung}
Das konstante, indizierte Alphabet, $\Sigma$, besteht aus Zeichen $\sigma_j, j = 1,2,...,|\Sigma|$, die in dem Input String vorkommen können. Außerdem beinhaltet es das Sonderzeichen Sentinel \$, das das Ende des Strings markiert. Die Zeichen lassen sich aufsteigend lexikographisch (nach \textit{lexikographischer Sortierung}) ordnen: $\sigma_1 < \sigma_2 < ... < \sigma_{|\Sigma|}$. Das \$-Zeichen ist dabei definiert als das kleinste Zeichen im Alphabet.
Übliche Grenzen für zulässige Alphabete zum Suffix-Sortieren liegen bei $|\Sigma| \leq 256$ oder, seltener, bei $|\Sigma| \leq 65536$. Jedes Zeichen kann als Integer kodiert werden, und benötigt (entsprechend der Alphabetgröße) ein oder zwei Byte Speicherplatz.

\subsection{Input-String und Suffix}
Der Input-String wird \inputtext genannt und hat die Länge $n$, wobei angenommen wird, dass $n < 2^{32}$, damit ein Integer von $0..n$ in 4 Byte passt. Das $i$-te Zeichen im Input-String ist \inputtext[i], das $i$-te Suffix  (kurz: Suffix $i$) ist \suffix{i} = $\inputtext[i,n)$ =  \mbox{\inputtext[i]\inputtext[i+1]$...$\inputtext[n-1]}. \inputtext[n] ist dann das Terminalsymbol \$ (sowie alle \inputtext[m] mit $m>n$) und ist formal nicht Teil des Eingabe-Strings. Der Eingabe-String beginnt bei \inputtext[0].

\subsection{Suffix-Array}
Das Suffix-Array, kurz \sa, bezeichnet ein Array, in dem in lexikographischer Reihenfolge die Suffix-Indizes (Positionen des Anfangsbuchstabens) gespeichert sind.
Das bedeutet, \sa[j] = $i$ genau dann, wenn $\mathsf{T}[i,n)$ das $j$-te Suffix von \inputtext in Lexorder ist.

