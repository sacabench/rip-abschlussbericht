\chapter*{Abstract}

Das \currentauthor{Christopher Poeplau und Marvin Böcker} Suffixarray (kurz: SA) ist eine simple Datenstruktur, welche zur Text-indizierung benutzt wird.
Die effiziente Berechnung des Suffixarray ist nicht trivial, weshalb sich im Laufe der Jahre viele Algorithmen entwickelt haben, welche das SA unterschiedlich schnell berechnen können.
Da einige der bekannten Algorithmen für die Suffixarray-Konstruktion bereits sehr alt sind, ist nicht für jeden von ihnen eine moderne Implementierung in C++ verfügbar.
Im Falle des Algorithmus' von Nong und Zhang, 2007~\cite{saca:10} oder des Algorithmus' von Goto, 2017~\cite{saca:12} ist sogar noch gar keine Referenzimplementierung vorhanden.
Doch auch wenn eine Implementierung vorhanden ist, kann es sein, dass diese nicht optimal ist.
Daher versuchen wir in unserer Projektgruppe für zwölf ausgewählte SA-Algorithmen moderne und effiziente Implementierungen zu programmieren, welche dann ohne Bias verglichen werden können.
Dadurch werden diese zwölf Algorithmen besser messbar, durch ein einheitliches Framework, einfach vergleichbar und dadurch insgesamt bewertbar.
