\section{Notation und Definitionen}
\todo{Macht die Definitionen lieber wirklich als Definitionen und nicht als subsections}
Einige\currentauthor{Rosa Pink und\\Marvin Löbel} grundlegende Notationen und Definitionen werden hier kurz vorangestellt.

\subsection{Intervalle}
Wir nutzen $[i, j] = \{i, \dots, j\}$ und $[i, j) = [i, j - 1]$ als Kurzschreibweise für Integer-Intervalle.

\subsection{Alphabet und lexikographische Sortierung}
Das konstante Alphabet ist definiert als $\Sigma = \{\sigma_1, \sigma_2, ..., \sigma_{|\Sigma|-1}\} \cup \{\$\}$.
Es besteht aus Symbolen bzw. Zeichen $\sigma_i$, die im Eingabe-String vorkommen dürfen, und dem Sentinel-Symbol \$, welches das Ende des Strings markiert. 
Die Zeichen sind wie folgt lexikographisch (nach \textit{lexikographischer Sortierung})
geordnet: $\$ < \sigma_1 < \sigma_2 < ... < \sigma_{|\Sigma|-1}$. 
Sie lassen sich durch Integer-Werte repräsentieren, indem wir \$ den Wert 0 zuweisen und $\sigma_i$ den Wert $i$.
$\Sigma^+$ bezeichnet weiter die Menge aller Strings über diesem Alphabet mit echt positiver Länge.

\subsection{Input-String und Suffix}
Der Input-String wird \inputtext genannt und hat die Länge $n$. Das $i$-te Zeichen im
Input-String ist \inputtext[i], das $i$-te Suffix  (kurz: Suffix $i$) ist 
\begin{align*}
\suffix{i} := \inputtext[i,n) = \inputtext[i]\inputtext[i+1]...\inputtext[n-1]
\end{align*} \inputtext[n]
ist das Terminalsymbol \$ (sowie alle \inputtext[m] mit $m>n$) und ist formal nicht Teil des Eingabe-Strings.
Der Eingabe-String beginnt bei \inputtext[0].

\subsection{Suffix-Array}
Das Suffix-Array, kurz \sa, bezeichnet ein Array, in dem in lexikographischer Reihenfolge
die Suffix-Indizes (Positionen des Anfangsbuchstabens) gespeichert sind.
Das bedeutet, $\sa[j] = i$ genau dann, wenn $\mathsf{T}[i,n)$ das $j$-te Suffix von \inputtext in lexikografischer Ordnung ist.

\subsection{Buckets}
\label{def:bucket}
Alle Suffixe, die mit demselben Zeichen $c_0 \in \Sigma$ beginnen, formen ein zusammenhängendes Intervall im Suffix-Array.
Dieses Intervall wird $c_0$-Bucket genannt und mit $\mathsf b_{c_0}$ bezeichnet.
Der $(c_0,c_1)$-Bucket $\mathsf b_{c_0, c_1}$ bezeichnet das Intervall, dessen Suffixe mit denselben zwei Zeichen $c_0, c_1 \in \Sigma$ beginnen.
Ähnlich dazu bezeichnet der Bucket $\mathsf b_\omega$ alle Suffixe im \sa, die alle mit dem String $\omega \in \Sigma^+$ anfangen.

\subsection{Textwiederholung}
\label{def:repetition}
Eine Wiederholung in $\mathsf{T}$ ist ein Teilstring $\mathsf{T}[i, i + rp]$ mit $ r \geq 2, p \geq 0$ und $i, i + rp \in [0, n)$, sodass $\mathsf{T}[i, i+p) = \mathsf{T}[i + p, i + 2p) = \dots = \mathsf{T}[i + (r-1)p, i + rp)$.

\subsection{Praxisrelevante Grenzen}

Wir gehen im Folgenden davon aus, dass für unsere Eingabe $|\Sigma| = 256$ gilt, da sich so jedes Zeichen durch ein Byte repräsentieren lässt. Dies erlaubt auch das direkte Verarbeiten von Texten in Standardkodierungen wie ASCII und UTF-8~\cite{grundlagen:utf8}, die durch Byte-Arrays ohne 0-Bytes repräsentiert werden.

Auch gehen wir davon aus, das die Länge $n$ von Datenstrukturen, insbesondere der Eingabe und des Suffix-Arrays, durch die maximale Größe nativer Integer-Datentypen in Consumer-Computersystemen begrenzt ist. Es gilt somit in der Regel $n \leq 2^{32}$ oder $n \leq 2^{64}$, womit sich Indizes durch 4- bzw. 8-Byte Integer repräsentieren lassen. Wir betrachten außerdem $n = 2^{40}$ als 5-Byte Kompromiss zwischen den Beiden.

\subsection{Codebeispiele}

Wir geben alle algorithmischen Codebeispiele in an Python angelehnten Pseudocode an. \todo{Erweitern oder Weg}