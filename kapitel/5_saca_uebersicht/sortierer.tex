\subsection{Sortierer}
Ein \currentauthor{Marvin Böcker} prinzipiell einfacher Ansatz, um \sa zu konstruieren,
ist selbstverständlich das Sortieren von Suffixen mittels einem allgemeinen Sortieralgorithmus,
beispielsweise mit Quicksort oder Radixsort.

Für die Verwendung eines Sortieralgorithmus, um \sa direkt zu konstruieren,
muss eine passende Vergleichsfunktion benutzt werden:
Nehmen wir an, die Menge die sortiert werden soll, ist bereits ein Array von disjunkten Suffix-Indizes
\footnote{Dies kann zu Beginn sichergestellt werden, indem das
Array mit den Zahlen von 0 bis $n-1$ (beziehungsweise $n$, wenn das Sentinel mitsortiert werden soll) gefüllt wird.}.
In der Vergleichsfunktion vergleichen wir dann die Suffixe aus dem Text, die am entsprechenden Index starten.
In Abbildung 1 findet sich Pseudocode für eine simple Vergleichsfunktion,
welche die gesamten Suffixe naiv miteinander vergleicht.

\begin{figure}[!ht]
\begin{minted}[mathescape=true,escapeinside=||]{python}
def compare(i, j):
    if |$\inputtext[i]$| == |$\inputtext[j]$|:
        return compare(i + 1, j + 1)
    return |$\inputtext[i]$| < |\inputtext[j]|
\end{minted}
\caption{Pseudocode einer naiven Suffixvergleichsfunktion}
\end{figure}
%
Es ist klar, dass mit einem $\mathcal O(n \log n)$-Sortieralgorithmus wie beispielsweise Merge- oder Introsort
der gesamte SACA eine Worst-Case-Komplexität von\\ $\mathcal O(n^2 \log n)$ hat, da jeder Suffixvergleich
eine Worst-Case-Laufzeit von $\mathcal O(n)$ hat.
Zum Vergleich: Es gibt einige Linearzeitalgorithmen für die \sa-Konstruktion, beispielsweise DC3~\cite{saca:9} oder SAIS~\cite{saca:6}.

Die alleinige Sortierung von ganzen Suffixen ist offensichtlich nicht effizient.
Viele unserer implementierten Algorithmen verwenden allerdings zum Teil allgemeine Sortierverfahren,
um verschiedene Mengen zu sortieren,
beispielsweise Suffix-Indizes, eine Menge von Zeichen, ganze Suffixe oder beliebige Strings.
Beispielsweise verwendet Multikey-Quicksort den Quicksort-Algorithmus, um eine Menge von Zeichen zu sortieren.
Deep-Shallow verwendet Introsort, um die Suffixe nach ihrer Länge zu sortieren
und DC3 benutzt Radixsort, um Triplets zu sortieren.
In DivSufSort werden Stringsortierer benutzt, um bestimmte Substrings zu sortieren.
Daher ist es für unser Framework wichtig, auch diese allgemeinen Sortierverfahren angepasst
auf unseren Use-Case effizient zu implementieren.
In Abschnitt \ref{section:sortieralgorithmen} findet sich eine Übersicht über die von uns
implementierten beziehungsweise extern eingebundenen Sortieralgorithmen.
