\section{Notation und Definitionen}
Einige\currentauthor{Rosa Pink} grundlegende Notationen und Definitionen werden hier kurz vorangestellt.

\subsection{Alphabet und lexikographische Sortierung}
Das konstante, indizierte Alphabet, $\Sigma$, besteht aus Zeichen $\sigma_j, j = 1,2,...,|\Sigma|$,
die in dem Input String vorkommen können. Außerdem beinhaltet es das Sonderzeichen Sentinel \$, das das Ende des Strings markiert.
Die Zeichen lassen sich aufsteigend lexikographisch (nach \textit{lexikographischer Sortierung})
ordnen: $\sigma_1 < \sigma_2 < ... < \sigma_{|\Sigma|}$. Das \$-Zeichen ist dabei definiert als das kleinste Zeichen im Alphabet.
Übliche Grenzen für zulässige Alphabete zum Suffix-Sortieren liegen bei $|\Sigma| \leq 256$ oder, seltener, bei $|\Sigma| \leq 65536$.
Jedes Zeichen kann als Integer kodiert werden, und benötigt (entsprechend der Alphabetgröße) ein oder zwei Byte Speicherplatz.

\subsection{Input-String und Suffix}
Der Input-String wird \inputtext genannt und hat die Länge $n$, wobei angenommen wird,
dass $n < 2^{32}$, damit ein Integer von $0..n$ in 4 Byte passt. Das $i$-te Zeichen im
Input-String ist \inputtext[i], das $i$-te Suffix  (kurz: Suffix $i$) ist
\suffix{i} = $\inputtext[i,n)$ = \mbox{\inputtext[i]\inputtext[i+1]$...$\inputtext[n-1]}. \inputtext[n]
ist dann das Terminalsymbol \$ (sowie alle \inputtext[m] mit $m>n$) und ist formal nicht Teil des Eingabe-Strings.
Der Eingabe-String beginnt bei \inputtext[0].

\subsection{Suffix-Array}
Das Suffix-Array, kurz \sa, bezeichnet ein Array, in dem in lexikographischer Reihenfolge
die Suffix-Indizes (Positionen des Anfangsbuchstabens) gespeichert sind.
Das bedeutet, \sa[j] = $i$ genau dann, wenn $\mathsf{T}[i,n)$ das $j$-te Suffix von \inputtext in Lexorder ist.

%\section{Notationen und Definitionen}
%\label{definitions}
%Sei \currentauthor{Christopher Poeplau} $T$ ein Text mit Zeichenanzahl $\vert T \vert = n$
%\begin{itemize}
%    \item $\Sigma (T)$ := Alphabet von $T$
%    \item Ein Substring von $T[i, j) = T[i]...T[j-1]$
%    \item $T$ wird durch das lexikografisch kleinste Zeichen, dem $\$$ terminiert (engl. sentinel)
%\end{itemize} 
%\bigskip
%Suffix $S_i=T[i,n)$ als das in $i$ beginnende Suffix in $T$. 
%\begin{itemize}
%    \item Das Suffix ist ein S(hort)-Type-Suffix $\iff S_i<S_{i+1}$
%    \item Das Suffix ist ein L(arge)-Type-Suffix $\iff S_i>S_{i+1}$
%    \item $Type('\$') := S$
%    \item Ein Zeichen $T[i]$ ist S-Type $\iff Type(S_i) = S$
%    \item Ein Zeichen $T[i]$ ist L-Type $\iff Type(S_i) = L$
%\end{itemize}
%
%\subsection{Leftmost S-Type}
%Ein Zeichen $T[i]$ des Strings $T$ ist genau dann LMS, wenn $Type(T[i-1])=L$, also das Vorgängerzeichen vom Typ $L$ ist. Gleichzeitig ist ein Suffix $T[i, n)$ ein LMS-Suffix, wenn $T[i]$ LMS ist.\\
%Ein Substring $T[i, j]$, $i\neq j$, ist LMS, wenn $Type(T[i])=S$ und $Type(T[j])=S$. Zudem darf in diesem Substring kein weiteres LMS-Zeichen vorhanden sein. Insbesondere ist $T[j]$ dadurch exklusiv und gehört nicht zu dem Substring. \\
%Für die Gleichheit zweier LMS-Substrings gilt das folgende: \\
%Seien $T_1$ und $T_2$ LMS-Substrings.
%\begin{center}
%    $S_1=S_2 \iff \vert S_1 \vert = \vert S_2 \vert $ $\wedge$ $S_1$ besitzt dieselben Zeichen wie $S_2$ in derselben Reihenfolge
%\end{center}