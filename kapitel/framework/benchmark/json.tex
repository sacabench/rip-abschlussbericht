\subsection{JSON}

Wird die Option \termfont{-b} oder \termfont{-{}-benchmark} für den Befehl \texttt{sacabench construct} und \texttt{sacabench batch} des \sacabench-Frameworks angefügt, wird eine \texttt{JSON}-Datei erzeugt. \texttt{JSON} - Abkürzung für \texttt{JavaScript Object Notation} - ist ein Datenformat, das im Rahmen des \sacabench-Frameworks als Speicherformat aller benötigten Bench\-mark-Werte verwendet wird. In dieser Datei werden unter anderem der Name, die Anzahl der benötigten zuzüglichen Sentinels, der maximale Speicherverbrauch und die Laufzeit des Algorithmus, die Größe des zu untersuchenden Ausgangstextes, die Größe des Datentyps für das Ausgabe-Suffix-Array und die Titel jeder einzelnen Phase des Algorithmus gespeichert. Diese Datei wird anschließend mit einem \texttt{R}-Skript ausgewertet, indem die Option \termfont{-z} oder \termfont{-{}-plot} an die jeweiligen Befehle in der Kommandozeile angehangen werden.

