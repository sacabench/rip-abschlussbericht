\subsection{Automatische PDF-Generierung mit SqlPlotTools}
\label{framework:bechmark:sqlplottools}

\currentauthor{David Piper und Florian Grieskamp}
Wird \texttt{sacabench construct} zur Analyse eines einzelnen Algorithmus und dessen Phasen oder \texttt{sacabench batch} zum Vergleich mehrerer Algorithmen untereinander mit zusätzlicher Option \texttt{--automation} ausgeführt, werden die erhaltenen Daten direkt in einer PDF-Datei aufgearbeitet und präsentiert.
Hierzu wird zunächst in der Datei \texttt{sacabench.cpp} eine Konfigurations-Datei mit Metainformationen über die durchgeführte Messung erstellt.
Diese beinhaltet Informationen zum Computersystem, auf dem die Analyse ausgeführt wird und zu dem Text, welcher als Input für die Algorithmen verwendet wurde.
Zum System werden die Anzahl der CPUs, die Anzahl an Threads pro Socket, das Model und das verwendete Betriebssystem erfasst.
Bezüglich des Textes enthält die Konfigurations-Datei name des Textes, Prefix und Anzahl der Wiederholungen.
Diese Datei wird im JSON-Format in dem Unterordner sqlplot des Ordners zbmessungen gespeichert, welcher ebenfalls ein Skript names automation.sh beinhaltet.

Im Anschluss an das Benchmarktool wird dieses Skript ausgeführt.
Hierdurch wird zunächst ein temporärer Ordner vom System des Nutzers erzeugt, in den der Ordner sqlplot kopiert wird.
Neben der zuvor erstellten Konfigurations-Datei beinhaltet dieser Ordner auch zwei Latex-Dateien, welche als Vorlagen für die zu erstellende PDF-Datei dienen, ein Python-Skript names \\
json\_to\_result\_converter.py und ein Makefile, welches spätere Abläufe koordiniert.
Zusätzlich wird die JSON-Datei, welche bei der Messung durch das Benchmarktool erstellt wurde, in den Ordner sqlplot im temporären Ordner kopiert.
Nun sind alle benötigten Vorbereitungen getroffen und das Makefile im Unterordner sqlplot im temporären Ordner wird durch das Skript ausgeführt.

Das Makefile führt das Python-Skript json\_to\_result\_converter.py aus, welches als erstes eine Result-Datei erstellt.
Dies ist eine Datei, welche von SqlPlotTools verarbeitet werden kann.
Sie enthält die Daten aus der Konfigurations-Datei und der JSON-Datei, welche die Ergebnisse der Messung beinhaltet.
Da abhängig davon, ob \texttt{construct} oder \texttt{batch} aufgerufen wurde, unterschiedliche Daten im PDF benötigt werden, werden zwei unterschiedliche Result-Dateien erstellt.
Die eine Result-Datei enthält Daten für die genauere Analyse der unterschiedlichen Phasen eines einzelnen Algorithmus nach \texttt{construct} und die andere beinhaltet Daten für den Vergleich mehrerer Algoirhtmen untereinander nach \texttt{batch}.
Zusätzlich generiert das Python-Skript die Latex-Dateien, welche mit den Daten befüllt werden. 
Hierzu werden die beiden Latex-Vorlagen im Unterordner templates genutzt.
Nachdem all diese Dateien generiert wurden, klont das Makefile das SqlPlotTools-Repository von GitHub in den temporären Ordner und baue dort das Projekt.
Jede durch das Python-Skript erstellte Latex-Datei wird durch einen Aufruf des SqlPlotTools mit Daten befüllt.
Dieser Aufruf wird ebenfalls von dem Makefile ausgelöst. 
Daraufhin können die Latex-Dateien gesetzt werden, wodurch die fertige PDF-Datei mit den neuen Messdaten entsteht. 

Jetzt übernimmt wieder das Skript automation.sh und kopiert die generierte PDF-Datei an ihren Zielort.
Zuletzt wird der temporäre Ordner wieder gelöscht.

\newcommand{\comment}[1]{}
\comment{
In sacabench.cpp:
1. Erstellung einer Config-Datei mit Metainformationen über die durchgeführte Messung.
2. Speichern der Config-Datei in dem Ordner zbmessung/sqlplot.
3. Starte das Script zbmessung/automation.sh.

In zbmessung/automation.sh:
1. Erstelle temporären Ordner
2. Kopiere Ordner zbmessung/sqlplot in temporären Ordner
3. Kopiere durch Benchmark erstellte JSON-Datei in temporären Ordner
4. Rufe make im temporären Ordner auf.
5. Kopiere generierte PDF-Dateien an Zielort.
6. Lösche temporären Ordner.

Im temporären Ordner durch das Makefile:
1. Das Python-Skript json\_to\_result\_converter.py erstellt Result-Dateien, die von SqlPlotTools verarbeitet werden können. 
Diese enthalten die Daten aus der durch sacabench.cpp erstellten Config-Datei und der durch das Benchmark erstellten und durch automation.sh kopierten JSON-Datei mit den Messergebnissen.
Dazu werden zwei verschiedene Result-Dateien erstellt, eine mit Daten für den Vergleich mehrerer Algoirhtmen untereinander nach batch, und eine mit Daten für die genauere Analyse der einzelnen Phasen eines einzelnen Algorithmus nach construct.
Dieses Python-Skript generiert neben den Result-Dateien auch die Latex-Dateien, welche durch SqlPlotTools mit den Daten aus der Result-Datei befüllt werden.
Im Unterordner templates befinden sich zwei Latex-Dateien, eine für das Ergebnis eines construct-Aufrufs und eine für das Ergebnis eines batch-Aufrufs.
Abhänig vom Befülle die im Ordner templates befindlichen Latex-Dateien (je nach Ausführung) mit den Ergebnissen der Abfragen aus SqlPlotTools.
2. Klone SqlPlotTools von GitHub und baue das Projekt.
3. Jede durch das Python-Skript erstellte Latex-Datei wird durch einen Aufruf des SqlPlotTools mit Daten befüllt.
Dieser Aufruf wird ebenfalls von dem Makefile ausgelöst. 
4. Nachdem nun die Latex-Dateien generiert und mit den Daten befüllt sind, werden sie gesetzt.
}