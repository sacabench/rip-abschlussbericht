\section{Testdaten}

\begin{table}
\caption{Testdaten}
\label{messung:tab:testdaten}
\resizebox{\textwidth}{!}{
\begin{tabular}{lL{6cm}}
\toprule
	Testdatei                             & Beschreibung \\
\midrule
    \texttt{pc\_dblp.xml}          & \multirow{5}{*}{Pizza\&Chili Korpus~\cite{testdaten:pizzachilli2007}. } \\
    \texttt{pc\_dna}               & \\
    \texttt{pc\_english}           & \\
    \texttt{pc\_proteins}          & \\
    \texttt{pc\_sources}           & \\
\midrule
    \texttt{pcr\_cere}             & \multirow{4}{6cm}{Pizza\&Chili Repetitive Korpus~\cite{testdaten:pizzachilli2007}.\\Echte Texte. } \\
    \texttt{pcr\_para}  & \\
    \texttt{pcr\_einstein.en.txt}            & \\
    \texttt{pcr\_kernel}           & \\
    \midrule
    \texttt{pcr\_fib41}             & \multirow{3}{6cm}{Pizza\&Chili Repetitive Corpus~\cite{testdaten:pizzachilli2007}.\\ Künstliche Texte. }\\
    \texttt{pcr\_rs.13}            & \\
    \texttt{pcr\_tm29}             & \\
\midrule
    \texttt{tagme\_wiki-disamb30}  & Wiki-Disamb30 aus dem TAGME Dateset~\cite{testdaten:tagme}. \\
\midrule
    \texttt{wiki\_all\_vital.txt}  & Klartextauszug der \textit{Most-Vital} Wikipedia Artikel~\cite{testdaten:wiki}.\\
\midrule
    \texttt{cc\_commoncrawl.ascii} & Zufälliger Klartextauszug von ASCII Seiten aus Commoncrawl~\cite{testdaten:commoncrawl}. \\
\bottomrule
\end{tabular}
}
\end{table}

Die Testdaten in \cref{messung:tab:testdaten} werden durch das \texttt{make datasets} Target des Buildsystems bereitgestellt, und haben alle eine Größe von mindestens 200MiB.

% TAGME Datasets is a collection of datasets that contain short text fragments drawn from Wikipedia snapshot of Novembre 6, 2009. Fragments are composed by about 30 words, and they contains about 20 non-stopword on average. We gathered fragments of 2 types: Wiki-Disamb30, a list of 2M fragments each containing one ambiguous anchor.

