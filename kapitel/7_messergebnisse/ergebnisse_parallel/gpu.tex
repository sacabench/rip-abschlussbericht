\subsection{GPU-Implementierungen}
% IMPORT-DATA result_gpu result_gpu.txt

\begin{figure}[ht]
    \textbf{Configuration} \hfill Prefix size: 200\,MiB \\ Model name: Intel\textsuperscript{\textregistered} Xeon\textsuperscript{\textregistered} CPU E5-2640 v4 @ 2.40GHz
    \centering\small
    \begin{tikzpicture}[trim axis left]
        \begin{axis}[batchTimePlot,
            cycle list name={exotic},
            legend to name=runtime-gpu-0,
            width=\textwidth,
            height=50mm,
            xtick style = {draw=none},
            title={},
            bar width=12pt,
            ylabel={SA construction time [s]},
            legend style={
                /tikz/every even column/.append style={column sep=0.5cm,black},
                /tikz/every even column/.append style={black},
            },
            legend columns=2,
            symbolic x coords={ pc\_dblp.xml.200MB, pc\_dna.200MB, pc\_english.200MB, pc\_sources.200MB, pc\_proteins.200MB },
            every node near coord/.append style={color=black, rotate=90, anchor=west},
            ]

            %% MULTIPLOT(algo) SELECT algo, REPLACE(input, "_", "\_") AS x, time/1000 AS y
            %% FROM ( 
            %% SELECT algo, input, MEDIAN(memFinal) AS memFinal, MEDIAN(memOff) AS memOff, AVG(memPeak) AS memPeak, prefix, rep_id, MEDIAN(time) AS time FROM result_gpu GROUP BY algo, input, prefix, rep_id
            %% ) WHERE (input="pc_dblp.xml.200MB" OR input="pc_dna.200MB")
            %% AND rep_id=1 GROUP BY MULTIPLOT,x ORDER BY y
            \addplot coordinates { (pc\_dna.200MB,9.30329) (pc\_dblp.xml.200MB,11.5691) };
            \addlegendentry{algo=Osipov\_gpu};
            \addplot coordinates { (pc\_dblp.xml.200MB,18.7458) (pc\_dna.200MB,29.7007) };
            \addlegendentry{algo=DivSufSort\_par\_ref};
            \addplot coordinates { (pc\_dna.200MB,40.8382) (pc\_dblp.xml.200MB,54.6578) };
            \addlegendentry{algo=Osipov\_parallel\_wp};
        \end{axis}
    \end{tikzpicture}
    \medskip
    \ref{runtime-gpu-0}
    \caption{CPU- und GPU-Version des Osipov-Algorithmus im Vergleich mit DivSufSort auf pc\_dblp.xml.200MB und pc\_dna.200MB}
\label{gpu_benchmark:1}
\end{figure}

\begin{figure}[ht]
    \textbf{Configuration} \hfill Prefix size: 200\,MiB \\ Model name: Intel\textsuperscript{\textregistered} Xeon\textsuperscript{\textregistered} CPU E5-2640 v4 @ 2.40GHz
    \centering\small
    \begin{tikzpicture}[trim axis left]
        \begin{axis}[batchTimePlot,
            cycle list name={exotic},
            legend to name=runtime-gpu-1,
            width=\textwidth,
            height=50mm,
            xtick style = {draw=none},
            title={},
            bar width=12pt,
            ylabel={SA construction time [s]},
            legend style={
                /tikz/every even column/.append style={column sep=0.5cm,black},
                /tikz/every even column/.append style={black},
            },
            legend columns=2,
            symbolic x coords={ pc\_dblp.xml.200MB, pc\_dna.200MB, pc\_english.200MB, pc\_sources.200MB, pc\_proteins.200MB },
            every node near coord/.append style={color=black, rotate=90, anchor=west},
            ]

            %% MULTIPLOT(algo) SELECT algo, REPLACE(input, "_", "\_") AS x, time/1000 AS y
            %% FROM ( 
            %% SELECT algo, input, MEDIAN(memFinal) AS memFinal, MEDIAN(memOff) AS memOff, AVG(memPeak) AS memPeak, prefix, rep_id, MEDIAN(time) AS time FROM result_gpu GROUP BY algo, input, prefix, rep_id
            %% ) WHERE (input="pc_english.200MB" OR input="pc_proteins.200MB")
            %% GROUP BY MULTIPLOT,x ORDER BY y
            \addplot coordinates { (pc\_proteins.200MB,12.4394) (pc\_english.200MB,17.1103) };
            \addlegendentry{algo=Osipov\_gpu};
            \addplot coordinates { (pc\_proteins.200MB,23.5957) (pc\_english.200MB,25.0026) };
            \addlegendentry{algo=DivSufSort\_par\_ref};
            \addplot coordinates { (pc\_proteins.200MB,55.1025) (pc\_english.200MB,75.315) };
            \addlegendentry{algo=Osipov\_parallel\_wp};
        \end{axis}
    \end{tikzpicture}
    \medskip
    \ref{runtime-gpu-1}
    \caption{CPU- und GPU-Version des Osipov-Algorithmus im Vergleich mit DivSufSort auf pc\_proteins.200MB und pc\_english.200MB}
\label{gpu_benchmark:2}
\end{figure}
%TODO Eventuell noch pc_sources
\FloatBarrier


Die \currentauthor{Oliver Magiera und\\ Hermann Foot} Abbildungen \ref{gpu_benchmark:1} und \ref{gpu_benchmark:2} zeigen die Laufzeitvergleiche zwischen der parallelen Version des DivSufSort von Mori und den von uns implementierten Varianten des Prefix-Doublers nach Osipov, sowohl die parallele Variante für die CPU als auch für die GPU. Ersterer stellt dabei den schnellsten von uns gemessenen Algorithmus auf der CPU dar. 

Die beiden CPU-Algorithmen wurden jeweils, wie zuvor, auf dem Intel\textsuperscript{\textregistered} Xeon\textsuperscript{\textregistered} Chipsatz mit 20 Kernen ausführt, wohingegen beim GPU-Algorithmus als Koprozessor eine NVIDIA\textsuperscript{\textregistered} K40 Grafikkarte mit 2880 CUDA Kernen zum Einsatz kommt. Dieser stehen 12GB GDDR5 zur Verfügung. Als Programmierplattform nutzen wir dabei CUDA 10. Als Eingabetexte nutzen wir auch hier eine Auswahl von Texten aus dem Pizza$\&$Chilli Korpus, alle sind dabei jeweils 200MB groß.

Zu erkennen ist, dass auch hier der DivSufSort auf allen vier Eingabetexten den anderen CPU-Implementierung laufzeittechnisch zum Teil deutlich überlegen ist. Anders sieht es hingegen im Vergleich zur GPU-Implementierung aus: auf allen vier Eingabetexten erreicht diese bessere Laufzeiten als beide CPU-Algorithmen. Durchschnittlich braucht diese in unseren Benchmarks 12.6 Sekunden, der DivSufSort 24.26 Sekunden und das CPU-Pendant des Prefix-Doublers sogar 56.48 Sekunden. 
Zudem lässt sich beobachten, dass sowohl der GPU-Osipov mit 3,28 Sekunden, als auch der parallele DivSufSort mit 4,5 Sekunden, geringe Standardabweichungen aufweisen und dementsprechend nicht unverhältnismäßig verschieden auf unterschiedliche Eingabetexte reagieren. Anderes gilt dabei für die CPU-Variante des Osipov-Algorithmus, der für die vier Texte gleicher Größe eine Standardabweichung von 14.19 Sekunden und damit vergleichsweise große Schwankungen in der Laufzeit aufweist.
%Ist doch der selbe Algorithmus, warum also so unterschiedliche Kennzahlen? -> Rauslassen?

Aus diesen Grafiken allerdings nicht ersichtlich ist der hohe Speichererbrauch des Osipov-Algorithmus. Wie bereits in der Algorithmenbeschreibung in Kapitel \ref{algorithm:gpuprefix} erwähnt, wird neben einem zusätzlichen Hilfsarray $\mathsf{aux}$ darüber hinaus noch eine Liste aus Tripeln benötigt, die je nach Eingabe bis zu $n$ Elemente groß sein kann. Damit summiert sich der Speicherverbrauch der Osipov-Implementierungen zu $28n$ ($20n$ für unsere Implementierung, ca. $8n$ für den Radixsort) und ist damit deutlich höher als der des DivSufSort, welcher Zusatzspeicher abhängig von der Alphabetgröße $|\Sigma|$ benötigt.

Die Werte zeigen, dass über die GPU beschleunigte Algorithmen einen Geschwindigkeitsvorteil aufweisen können. Bei unserem GPU-SACA handelt es sich um eine unoptimierte Version des Osipov-Algorithmus, welcher optimierte Versionen der Präfixsumme und des Radixsorts aus dem externen CUB-Framework verwendet. Diese unoptimierte Variante konnte bereits den schnellsten CPU-SACA unseres Frameworks schlagen. Dies deutet darauf hin, dass die Verwendung der GPU zur Berechnung des Suffix-Arrays lohnenswert sein kann und daher näher untersucht werden sollte. Wie bereits in Abschnitt \ref{algorithm:gpuprefix} zum Osipov-Algorithmus erwähnt wurde, gibt es einige wenige SACAs, welche sogar noch schnellere Ergebnisse erzielen können. Leider ist es uns nicht gelungen, eine (lauffähige) Version dieser Algorithmen zu erhalten, weshalb ein noch größerer Augenmerk auf die Entwicklung von GPU-SACAs gelegt werden sollte. Ein besonders wichtiges Ziel könnte hierbei die Reduktion des Speicherbedarfs darstellen, da zum aktuellen Stand der Grafikspeicher (einer einzelnen Grafikkarte) deutlich geringer ist als der Arbeitsspeicher eines Testsystems und beispielsweise unsere Implementierung des Osipov-SACAs $28n$ Zusatzspeicher benötigt. Zusätzlich wäre die Nutzung mehrerer Grafikkarten für die Berechnung eines Suffix-Arrays ebenfalls eine Option. Dies stellt durch häufige Datenabhängigkeiten, wie durch das inverse Suffix-Array, jedoch eine besondere Herausforderung dar.