\section{Messverfahren}

Sei $input$ eine Eingabedatei, $algorithm$ ein Algorithmus, $k$ ein zu benutzender Prefix der Eingabe, $t$ eine Anzahl der zu benutzenden Threads, $r$ eine Anzahl von Messwiederholungen und $b$ eine Anzahl von Bits für den \texttt{sa\_index} Typ. Wir führen Messungen mit 4 verschiedenen Konfigurationen durch, wobei immer $r=1$\todo{Mehr Messungen durchführen!} gilt. Es gilt außerdem $b = 32$ für Eingabegrößen $< 2$GiB und $b = 64$ sonst:

\noindent \resizebox{\textwidth}{!}{
\begin{tabular}{L{2.2cm}llrc}
\toprule
Konfiguration & $input$ & $algorithm$ & $k$ & $t$\\
\midrule
Small-Sequential               & Small & Alle Sequentiellen & 200MiB     & N.A.                           \\
Large-Sequential-Input-Scaling & Large & Alle Sequentiellen & $t*200$MiB & $\in [1, 2, 4, 8, 12, 16, 20)$ \\
Large-Parallel-Weak-Scaling    & Large & Alle Parallelen    & $t*200$MiB & $\in [1, 2, 4, 8, 12, 16, 20)$ \\
Large-Parallel-Strong-Scaling  & Large & Alle Parallelen    & $200$MiB   & $\in [1, 2, 4, 8, 12, 16, 20)$ \\
\bottomrule
\end{tabular}
}

\bigskip
Für jede Konfiguration wird \texttt{taskset -c 0,1,...,$t$ sacabench batch $input$} mit den Optionen \texttt{-{}-whitelist $algorithm$}, \texttt{-{}-check}, \texttt{-{}-prefix $k$}, \texttt{-{}-repetitions $r$}, \texttt{-b $b$} und \texttt{-{}-benchmark $outfile$} aufgerufen. Hierbei ist $outfile$ jeweils ein eindeutiger Dateipfad.

Dieses Verfahren wird in der Framework Implementierung durch den Python Skript \texttt{zbmessung/lido3\_launcher.py} automatisiert.
