\subsection{Skalierbarkeit}

In den folgenden Kapiteln untersuchen wir Gruppen sequentieller Implementierungen in Bezug auf ihre Skalierbarkeit. Dabei wird die Präfixgröße kontinuierlich bis maximal 4GB Eingabegröße erhöht und dabei die Laufzeit und der Speicherverbrauch in Abhängigkeit davon gemessen.

\subsubsection{BPR}

\textbf{Configuration} \hfill Model name: none
% IMPORT-DATA stats2 ../results.txt

\begin{figure}[ht]
    \centering
    \begin{tikzpicture}
        \begin{axis}[
                name=axis1,
                cycle list name={exoticlines},
                width=0.5\textwidth,
                height=65mm,
                title={wiki.txt},
                xlabel={input size [200\,MB]},
                ylabel={SA construction time [s]},
                legend columns=2,
                legend to name=legend0,
                legend style={
                    /tikz/every even column/.append style={column sep=0.5cm,black},
                    /tikz/every even column/.append style={black},
                },
            ]

            %% MULTIPLOT(algo) SELECT thread_count AS x, time/1000 AS y, MULTIPLOT
            %% FROM (
            %% SELECT algo, input, MEDIAN(memFinal) AS memFinal, MEDIAN(memOff) AS memOff, AVG(memPeak) AS memPeak, prefix, rep, thread_count, MEDIAN(time) AS time, sacheck FROM stats2 GROUP BY algo, input, prefix, rep, thread_count
            %% ) WHERE input="wiki.txt" AND sacheck="ok" AND thread_count<9
            %% AND (algo="Naiv" OR algo="BPR" OR algo="BPR_ref")
            %% GROUP BY MULTIPLOT,x ORDER BY MULTIPLOT,x
            \addplot coordinates { (2,68.2405) (4,146.638) (8,315.01) };
            \addlegendentry{algo=BPR};
            \addplot coordinates { (2,89.0595) (4,164.762) (8,402.452) };
            \addlegendentry{algo=BPR\_ref};
            \addplot coordinates { (2,359.928) (4,778.531) (8,2125.23) };
            \addlegendentry{algo=Naiv};

        \end{axis}
        \begin{axis}[
                cycle list name={exoticlines},
                at={(axis1.outer north east)},
                anchor=outer north west,
                name=axis2,
                width=0.5\textwidth,
                height=65mm,
                title={wiki.txt},
                xlabel={input size [200\,MB]},
                ylabel={Extra Memory [MB]},
            ]

            %% MULTIPLOT(algo) SELECT thread_count AS x, memPeak/1000000 AS y, MULTIPLOT
            %% FROM (
            %% SELECT algo, input, MEDIAN(memFinal) AS memFinal, MEDIAN(memOff) AS memOff, AVG(memPeak) AS memPeak, prefix, rep, thread_count, MEDIAN(time) AS time, sacheck FROM stats2 GROUP BY algo, input, prefix, rep, thread_count
            %% ) WHERE input="wiki.txt" AND sacheck="ok" AND thread_count<9
            %% AND (algo="Naiv" OR algo="BPR" OR algo="BPR_ref")
            %% GROUP BY MULTIPLOT,x ORDER BY MULTIPLOT,x
            \addplot coordinates { (2,1750.76) (4,3429.53) (8,6784.97) };
            \addlegendentry{algo=BPR};
            \addplot coordinates { (2,7202.31) (4,14333.7) (8,28594.3) };
            \addlegendentry{algo=BPR\_ref};
            \addplot coordinates { (2,0.0) (4,0.0) (8,0.0) };
            \addlegendentry{algo=Naiv};

            \legend{}
        \end{axis}
    \end{tikzpicture}

    \medskip
    \ref{legend0}
\end{figure}

\begin{figure}[ht]
    \centering
    \begin{tikzpicture}
        \begin{axis}[
                name=axis1,
                cycle list name={exoticlines},
                width=0.5\textwidth,
                height=65mm,
                title={dna.txt},
                xlabel={input size [200\,MB]},
                ylabel={SA construction time [s]},
                legend columns=2,
                legend to name=legend1,
                legend style={
                    /tikz/every even column/.append style={column sep=0.5cm,black},
                    /tikz/every even column/.append style={black},
                },
            ]

            %% MULTIPLOT(algo) SELECT thread_count AS x, time/1000 AS y, MULTIPLOT
            %% FROM (
            %% SELECT algo, input, MEDIAN(memFinal) AS memFinal, MEDIAN(memOff) AS memOff, AVG(memPeak) AS memPeak, prefix, rep, thread_count, MEDIAN(time) AS time, sacheck FROM stats2 GROUP BY algo, input, prefix, rep, thread_count
            %% ) WHERE input="dna.txt" AND sacheck="ok" AND thread_count<9
            %% AND (algo="Naiv" OR algo="BPR" OR algo="BPR_ref")
            %% GROUP BY MULTIPLOT,x ORDER BY MULTIPLOT,x
            \addplot coordinates { (2,129.694) (4,265.874) (8,623.018) };
            \addlegendentry{algo=BPR};
            \addplot coordinates { (2,101.446) (4,209.787) (8,446.944) };
            \addlegendentry{algo=BPR\_ref};
            \addplot coordinates { (2,413.268) (4,908.502) (8,1904.01) };
            \addlegendentry{algo=Naiv};

        \end{axis}
        \begin{axis}[
                cycle list name={exoticlines},
                at={(axis1.outer north east)},
                anchor=outer north west,
                name=axis2,
                width=0.5\textwidth,
                height=65mm,
                title={dna.txt},
                xlabel={input size [200\,MB]},
                ylabel={Extra Memory [MB]},
            ]

            %% MULTIPLOT(algo) SELECT thread_count AS x, memPeak/1000000 AS y, MULTIPLOT
            %% FROM (
            %% SELECT algo, input, MEDIAN(memFinal) AS memFinal, MEDIAN(memOff) AS memOff, AVG(memPeak) AS memPeak, prefix, rep, thread_count, MEDIAN(time) AS time, sacheck FROM stats2 GROUP BY algo, input, prefix, rep, thread_count
            %% ) WHERE input="dna.txt" AND sacheck="ok" AND thread_count<9
            %% AND (algo="Naiv" OR algo="BPR" OR algo="BPR_ref")
            %% GROUP BY MULTIPLOT,x ORDER BY MULTIPLOT,x
            \addplot coordinates { (2,3630.85) (4,5308.57) (8,16476.5) };
            \addlegendentry{algo=BPR};
            \addplot coordinates { (2,7130.45) (4,14260.8) (8,28521.4) };
            \addlegendentry{algo=BPR\_ref};
            \addplot coordinates { (2,0.0) (4,0.0) (8,0.0) };
            \addlegendentry{algo=Naiv};

            \legend{}
        \end{axis}
    \end{tikzpicture}

    \medskip
    \ref{legend1}
\end{figure}

\begin{figure}[ht]
    \centering
    \begin{tikzpicture}
        \begin{axis}[
                name=axis1,
                cycle list name={exoticlines},
                width=0.5\textwidth,
                height=65mm,
                title={commoncrawl.txt},
                xlabel={input size [200\,MB]},
                ylabel={SA construction time [s]},
                legend columns=2,
                legend to name=legend2,
                legend style={
                    /tikz/every even column/.append style={column sep=0.5cm,black},
                    /tikz/every even column/.append style={black},
                },
            ]

            %% MULTIPLOT(algo) SELECT thread_count AS x, time/1000 AS y, MULTIPLOT
            %% FROM (
            %% SELECT algo, input, MEDIAN(memFinal) AS memFinal, MEDIAN(memOff) AS memOff, AVG(memPeak) AS memPeak, prefix, rep, thread_count, MEDIAN(time) AS time, sacheck FROM stats2 GROUP BY algo, input, prefix, rep, thread_count
            %% ) WHERE input="commoncrawl.txt" AND sacheck="ok" AND thread_count<9
            %% AND (algo="Naiv" OR algo="BPR" OR algo="BPR_ref")
            %% GROUP BY MULTIPLOT,x ORDER BY MULTIPLOT,x
            \addplot coordinates { (2,60.0865) (4,129.425) (8,278.686) };
            \addlegendentry{algo=BPR};
            \addplot coordinates { (2,74.6625) (4,172.898) (8,338.178) };
            \addlegendentry{algo=BPR\_ref};
            \addplot coordinates { (2,666.038) (4,1590.11) (8,4111.56) };
            \addlegendentry{algo=Naiv};

        \end{axis}
        \begin{axis}[
                cycle list name={exoticlines},
                at={(axis1.outer north east)},
                anchor=outer north west,
                name=axis2,
                width=0.5\textwidth,
                height=65mm,
                title={commoncrawl.txt},
                xlabel={input size [200\,MB]},
                ylabel={Extra Memory [MB]},
            ]

            %% MULTIPLOT(algo) SELECT thread_count AS x, memPeak/1000000 AS y, MULTIPLOT
            %% FROM (
            %% SELECT algo, input, MEDIAN(memFinal) AS memFinal, MEDIAN(memOff) AS memOff, AVG(memPeak) AS memPeak, prefix, rep, thread_count, MEDIAN(time) AS time, sacheck FROM stats2 GROUP BY algo, input, prefix, rep, thread_count
            %% ) WHERE input="commoncrawl.txt" AND sacheck="ok" AND thread_count<9
            %% AND (algo="Naiv" OR algo="BPR" OR algo="BPR_ref")
            %% GROUP BY MULTIPLOT,x ORDER BY MULTIPLOT,x
            \addplot coordinates { (2,1789.7) (4,3470.23) (8,6825.68) };
            \addlegendentry{algo=BPR};
            \addplot coordinates { (2,7240.91) (4,14374) (8,28634.7) };
            \addlegendentry{algo=BPR\_ref};
            \addplot coordinates { (2,0.0) (4,0.0) (8,0.0) };
            \addlegendentry{algo=Naiv};

            \legend{}
        \end{axis}
    \end{tikzpicture}

    \medskip
    \ref{legend2}
\end{figure}
\FloatBarrier

\subsubsection{Deep-Shallow}

\textbf{Configuration} \hfill Model name: none
% IMPORT-DATA stats2 ../results.txt

\begin{figure}[!h]
    \centering
    \begin{tikzpicture}
        \begin{axis}[
                name=axis1,
                cycle list name={exoticlines},
                width=0.5\textwidth,
                height=65mm,
                title={wiki.txt},
                xlabel={input size [200\,MB]},
                ylabel={SA construction time [s]},
                legend columns=2,
                legend to name=legend-seq-ds-0,
                legend style={
                    /tikz/every even column/.append style={column sep=0.5cm,black},
                    /tikz/every even column/.append style={black},
                },
            ]

            %% MULTIPLOT(algo) SELECT thread_count AS x, time/1000 AS y, MULTIPLOT
            %% FROM (
            %% SELECT algo, input, MEDIAN(memFinal) AS memFinal, MEDIAN(memOff) AS memOff, AVG(memPeak) AS memPeak, prefix, rep, thread_count, MEDIAN(time) AS time, sacheck FROM stats2 GROUP BY algo, input, prefix, rep, thread_count
            %% ) WHERE input="wiki.txt" AND sacheck="ok" 
            %% AND (algo="Naiv" OR algo="Deep-Shallow" OR algo="Deep-Shallow_ref" OR algo="Deep-Shallow_bb")
            %% GROUP BY MULTIPLOT,x ORDER BY MULTIPLOT,x
            \addplot coordinates { (1,85.1097) (2,190.63) (4,425.599) (8,1050.53) (12,1859.4) (16,2472.93) (20,3139.03) };
            \addlegendentry{algo=Deep-Shallow};
            \addplot coordinates { (1,79.191) (2,172.315) (4,372.287) (8,1022.63) (12,1782.47) (16,2396) (20,2548.68) };
            \addlegendentry{algo=Deep-Shallow\_bb};
            \addplot coordinates { (1,34.3903) (2,76.1794) (4,166.842) };
            \addlegendentry{algo=Deep-Shallow\_ref};
            \addplot coordinates { (1,162.841) (2,361.602) (4,783.025) (8,1729.44) (12,3448.59) (16,3842.25) (20,6183.09) };
            \addlegendentry{algo=Naiv};

        \end{axis}
        \begin{axis}[
                cycle list name={exoticlines},
                at={(axis1.outer north east)},
                anchor=outer north west,
                name=axis2,
                width=0.5\textwidth,
                height=65mm,
                title={wiki.txt},
                xlabel={input size [200\,MB]},
                ylabel={Extra Memory [MB]},
            ]

            %% MULTIPLOT(algo) SELECT thread_count AS x, memPeak/1000000 AS y, MULTIPLOT
            %% FROM (
            %% SELECT algo, input, MEDIAN(memFinal) AS memFinal, MEDIAN(memOff) AS memOff, AVG(memPeak) AS memPeak, prefix, rep, thread_count, MEDIAN(time) AS time, sacheck FROM stats2 GROUP BY algo, input, prefix, rep, thread_count
            %% ) WHERE input="wiki.txt" AND sacheck="ok" 
            %% AND (algo="Naiv" OR algo="Deep-Shallow" OR algo="Deep-Shallow_ref" OR algo="Deep-Shallow_bb")
            %% GROUP BY MULTIPLOT,x ORDER BY MULTIPLOT,x
            \addplot coordinates { (1,3.423512) (2,5.711312) (4,10.2978) (8,19.449) (12,47.34365) (16,62.59566) (20,77.8628) };
            \addlegendentry{algo=Deep-Shallow};
            \addplot coordinates { (1,3.336359) (2,5.624159) (4,10.2098) (8,19.361) (12,47.16767) (16,62.41968) (20,77.6852) };
            \addlegendentry{algo=Deep-Shallow\_bb};
            \addplot coordinates { (1,2.540872) (2,5.057524) (4,10.091) };
            \addlegendentry{algo=Deep-Shallow\_ref};
            \addplot coordinates { (1,0.0) (2,0.0) (4,0.0) (8,0.0) (12,0.0) (16,0.0) (20,0.0) };
            \addlegendentry{algo=Naiv};

            \legend{}
        \end{axis}
    \end{tikzpicture}

    \medskip
    \ref{legend-seq-ds-0}
\end{figure}

\begin{figure}[!h]
    \centering
    \begin{tikzpicture}
        \begin{axis}[
                name=axis1,
                cycle list name={exoticlines},
                width=0.5\textwidth,
                height=65mm,
                title={dna.txt},
                xlabel={input size [200\,MB]},
                ylabel={SA construction time [s]},
                legend columns=2,
                legend to name=legend-seq-ds-1,
                legend style={
                    /tikz/every even column/.append style={column sep=0.5cm,black},
                    /tikz/every even column/.append style={black},
                },
            ]

            %% MULTIPLOT(algo) SELECT thread_count AS x, time/1000 AS y, MULTIPLOT
            %% FROM (
            %% SELECT algo, input, MEDIAN(memFinal) AS memFinal, MEDIAN(memOff) AS memOff, AVG(memPeak) AS memPeak, prefix, rep, thread_count, MEDIAN(time) AS time, sacheck FROM stats2 GROUP BY algo, input, prefix, rep, thread_count
            %% ) WHERE input="dna.txt" AND sacheck="ok" 
            %% AND (algo="Naiv" OR algo="Deep-Shallow" OR algo="Deep-Shallow_ref" OR algo="Deep-Shallow_bb")
            %% GROUP BY MULTIPLOT,x ORDER BY MULTIPLOT,x
            \addplot coordinates { (1,111.326) (2,252.814) (4,558.929) (8,1148.98) (12,1821.64) (16,2482.4) (20,3182.25) };
            \addlegendentry{algo=Deep-Shallow};
            \addplot coordinates { (1,90.8697) (2,194.55) (4,425.169) (8,930.997) (12,1764.17) (16,2427.68) (20,2598.36) };
            \addlegendentry{algo=Deep-Shallow\_bb};
            \addplot coordinates { (1,38.7309) (2,81.5342) (4,175.302) };
            \addlegendentry{algo=Deep-Shallow\_ref};
            \addplot coordinates { (1,183.948) (2,415.412) (4,910.737) (8,1910.63) (12,3645.28) (16,4061.77) (20,6575.73) };
            \addlegendentry{algo=Naiv};

        \end{axis}
        \begin{axis}[
                cycle list name={exoticlines},
                at={(axis1.outer north east)},
                anchor=outer north west,
                name=axis2,
                width=0.5\textwidth,
                height=65mm,
                title={dna.txt},
                xlabel={input size [200\,MB]},
                ylabel={Extra Memory [MB]},
            ]

            %% MULTIPLOT(algo) SELECT thread_count AS x, memPeak/1000000 AS y, MULTIPLOT
            %% FROM (
            %% SELECT algo, input, MEDIAN(memFinal) AS memFinal, MEDIAN(memOff) AS memOff, AVG(memPeak) AS memPeak, prefix, rep, thread_count, MEDIAN(time) AS time, sacheck FROM stats2 GROUP BY algo, input, prefix, rep, thread_count
            %% ) WHERE input="dna.txt" AND sacheck="ok" 
            %% AND (algo="Naiv" OR algo="Deep-Shallow" OR algo="Deep-Shallow_ref" OR algo="Deep-Shallow_bb")
            %% GROUP BY MULTIPLOT,x ORDER BY MULTIPLOT,x
            \addplot coordinates { (1,2.288456) (2,4.576256) (4,9.151862) (8,18.3031) (12,45.9107) (16,61.16271) (20,76.41473) };
            \addlegendentry{algo=Deep-Shallow};
            \addplot coordinates { (1,2.288411) (2,4.576211) (4,9.151817) (8,18.303) (12,45.9107) (16,61.1627) (20,76.4147) };
            \addlegendentry{algo=Deep-Shallow\_bb};
            \addplot coordinates { (1,2.516592) (2,5.033172) (4,10.0663) };
            \addlegendentry{algo=Deep-Shallow\_ref};
            \addplot coordinates { (1,0.0) (2,0.0) (4,0.0) (8,0.0) (12,0.0) (16,0.0) (20,0.0) };
            \addlegendentry{algo=Naiv};

            \legend{}
        \end{axis}
    \end{tikzpicture}

    \medskip
    \ref{legend-seq-ds-1}
\end{figure}

\begin{figure}[!h]
    \centering
    \begin{tikzpicture}
        \begin{axis}[
                name=axis1,
                cycle list name={exoticlines},
                width=0.5\textwidth,
                height=65mm,
                title={commoncrawl.txt},
                xlabel={input size [200\,MB]},
                ylabel={SA construction time [s]},
                legend columns=2,
                legend to name=legend-seq-ds-2,
                legend style={
                    /tikz/every even column/.append style={column sep=0.5cm,black},
                    /tikz/every even column/.append style={black},
                },
            ]

            %% MULTIPLOT(algo) SELECT thread_count AS x, time/1000 AS y, MULTIPLOT
            %% FROM (
            %% SELECT algo, input, MEDIAN(memFinal) AS memFinal, MEDIAN(memOff) AS memOff, AVG(memPeak) AS memPeak, prefix, rep, thread_count, MEDIAN(time) AS time, sacheck FROM stats2 GROUP BY algo, input, prefix, rep, thread_count
            %% ) WHERE input="commoncrawl.txt" AND sacheck="ok" 
            %% AND (algo="Naiv" OR algo="Deep-Shallow" OR algo="Deep-Shallow_ref" OR algo="Deep-Shallow_bb")
            %% GROUP BY MULTIPLOT,x ORDER BY MULTIPLOT,x
            \addplot coordinates { (1,373.777) (2,727.454) };
            \addlegendentry{algo=Deep-Shallow};
            \addplot coordinates { (1,212.824) };
            \addlegendentry{algo=Deep-Shallow\_bb};
            \addplot coordinates { (1,38.4159) (2,85.7689) (4,193.81) };
            \addlegendentry{algo=Deep-Shallow\_ref};
            \addplot coordinates { (1,428.421) (2,669.456) (4,1602.12) (8,3723.05) };
            \addlegendentry{algo=Naiv};

        \end{axis}
        \begin{axis}[
                cycle list name={exoticlines},
                at={(axis1.outer north east)},
                anchor=outer north west,
                name=axis2,
                width=0.5\textwidth,
                height=65mm,
                title={commoncrawl.txt},
                xlabel={input size [200\,MB]},
                ylabel={Extra Memory [MB]},
            ]

            %% MULTIPLOT(algo) SELECT thread_count AS x, memPeak/1000000 AS y, MULTIPLOT
            %% FROM (
            %% SELECT algo, input, MEDIAN(memFinal) AS memFinal, MEDIAN(memOff) AS memOff, AVG(memPeak) AS memPeak, prefix, rep, thread_count, MEDIAN(time) AS time, sacheck FROM stats2 GROUP BY algo, input, prefix, rep, thread_count
            %% ) WHERE input="commoncrawl.txt" AND sacheck="ok" 
            %% AND (algo="Naiv" OR algo="Deep-Shallow" OR algo="Deep-Shallow_ref" OR algo="Deep-Shallow_bb")
            %% GROUP BY MULTIPLOT,x ORDER BY MULTIPLOT,x
            \addplot coordinates { (1,3.735902) (2,6.085712) };
            \addlegendentry{algo=Deep-Shallow};
            \addplot coordinates { (1,3.624746) };
            \addlegendentry{algo=Deep-Shallow\_bb};
            \addplot coordinates { (1,2.570416) (2,5.089404) (4,10.1495) };
            \addlegendentry{algo=Deep-Shallow\_ref};
            \addplot coordinates { (1,0.0) (2,0.0) (4,0.0) (8,0.0) };
            \addlegendentry{algo=Naiv};

            \legend{}
        \end{axis}
    \end{tikzpicture}

    \medskip
    \ref{legend-seq-ds-2}
\end{figure}
\FloatBarrier

\noindent
Auf den Abbildungen zu den beiden Texten \texttt{wiki} und \texttt{dna} lässt sich erkennen,
dass alle drei Implementierungen von Deep-Shallow die naive Sortiermethode in der Laufzeit schlagen.
Sie tun dies durch das Eintauschen von einigem Speicher (ca 1\% des Eingabetextes):
Der naive SACA verwendet hingegen keinen Extraspeicher.

In der dritten Abbildung zum Text \texttt{commoncrawl} zeigt sich die schlechte
Worst-Case-Laufzeit von Deep-Shallow.
Unsere Deep-Shallow-Implementierung ohne Big-Buckets (siehe \cref{ds:zweite}) schafft es hier zwar,
bis 400MB das SA zu berechnen, allerdings reicht die Zeit (2 Stunden) nicht aus, um das SA für 800MB Text zu berechnen.
Die Deep-Shallow-Big-Buckets-Implementierung schafft es nicht, das 200MB-SA zu berechnen.

In allen drei Plots schafft der Referenzalgorithmus es nicht, das SA für 1600MB Text zu berechnen.
Dies liegt daran, dass er innerhalb des Zeitlimits von zwei Stunden nicht terminiert ist.
Der Algorithmus verwendet zwar 32-Bit-Integer, und verwendet außerdem eins dieser Bits als Tag,
dies schränkt den adressierbaren Text jedoch nur auf $2^{32 - 1} \approx \SI{2.1}{GB}$ ein.
Daher ist zu vermuten, dass diese (relativ alte) Implementierung andere Annahmen trifft,
die die Textgröße weiter einschränken.

Vergleichend lässt sich sagen, dass unsere Implementierungen zwar den naiven SACA auf zwei Texten schlagen,
die optimiertere Referenzimplementierung allerdings in fast allen Fällen besser ist.
Lediglich der Speicherverbrauch unserer Implementierungen ist auf \texttt{dna} geringer.

\clearpage
\subsubsection{Doubling und Discarding}
\label{messeval:scaling:seq:doubling}

\textbf{Configuration} \hfill Model name: Intel\textsuperscript{\textregistered} Xeon\textsuperscript{\textregistered} CPU E5-2640 v4 @ 2.40GHz
% IMPORT-DATA stats2 ../results.txt

\begin{figure}[ht]
    \centering
    \begin{tikzpicture}
        \begin{axis}[
                name=axis1,
                cycle list name={exoticlines},
                width=0.5\textwidth,
                height=48mm,
                title={wiki.txt},
                xlabel={input size [GiB]},
                ylabel={SA construction time [min]},
                legend columns=2,
                legend to name=legend-seq-doubling-0,
                legend style={
                    /tikz/every even column/.append style={column sep=0.5cm,black},
                    /tikz/every even column/.append style={black},
                },
                ymax=40
            ]

            %% MULTIPLOT(algo) SELECT prefix/1024/1024/1024.0 AS x, time/1000/60 AS y, MULTIPLOT
            %% FROM (
            %% SELECT algo, input, MEDIAN(memFinal) AS memFinal, MEDIAN(memOff) AS memOff, AVG(memPeak) AS memPeak, prefix, rep, thread_count, MEDIAN(time) AS time, sacheck FROM stats2 GROUP BY algo, input, prefix, rep, thread_count
            %% ) WHERE thread_count < 9 AND input="wiki.txt" AND sacheck="ok" 
            %% AND (algo="Naiv" OR algo="Doubling" OR algo="Discarding2" OR algo="Discarding4")
            %% GROUP BY MULTIPLOT,x ORDER BY MULTIPLOT,x
            \addplot coordinates { (0.195312,2.4246) (0.390625,5.13328) (0.78125,10.9069) (1.5625,23.9962) };
            \addlegendentry{algo=Discarding2};
            \addplot coordinates { (0.195312,2.22196) (0.390625,4.77061) (0.78125,10.1198) (1.5625,22.3255) };
            \addlegendentry{algo=Discarding4};
            \addplot coordinates { (0.195312,11.0806) (0.390625,22.9862) (0.78125,47.518) (1.5625,97.2159) };
            \addlegendentry{algo=Doubling};
            \addplot coordinates { (0.195312,2.7048) (0.390625,6.0286) (0.78125,15.8137) (1.5625,28.8136) };
            \addlegendentry{algo=Naiv};

        \end{axis}
        \begin{axis}[
                cycle list name={exoticlines},
                at={(axis1.outer north east)},
                anchor=outer north west,
                name=axis2,
                width=0.5\textwidth,
                height=48mm,
                title={wiki.txt},
                xlabel={input size [GiB]},
                ylabel={Extra Memory [GiB]},
            ]

            %% MULTIPLOT(algo) SELECT prefix/1024/1024/1024.0 AS x, memPeak/1024/1024/1024 AS y, MULTIPLOT
            %% FROM (
            %% SELECT algo, input, MEDIAN(memFinal) AS memFinal, MEDIAN(memOff) AS memOff, AVG(memPeak) AS memPeak, prefix, rep, thread_count, MEDIAN(time) AS time, sacheck FROM stats2 GROUP BY algo, input, prefix, rep, thread_count
            %% ) WHERE thread_count < 9 AND input="wiki.txt" AND sacheck="ok" 
            %% AND (algo="Naiv" OR algo="Doubling" OR algo="Discarding2" OR algo="Discarding4")
            %% GROUP BY MULTIPLOT,x ORDER BY MULTIPLOT,x
            \addplot coordinates { (0.195312,2.34575) (0.390625,4.6895) (0.78125,9.377) (1.5625,18.752) };
            \addlegendentry{algo=Discarding2};
            \addplot coordinates { (0.195312,3.90825) (0.390625,7.8145) (0.78125,15.627) (1.5625,31.252) };
            \addlegendentry{algo=Discarding4};
            \addplot coordinates { (0.195312,2.34575) (0.390625,4.6895) (0.78125,9.377) (1.5625,18.752) };
            \addlegendentry{algo=Doubling};
            \addplot coordinates { (0.195312,0.0) (0.390625,0.0) (0.78125,0.0) (1.5625,0.0) };
            \addlegendentry{algo=Naiv};

            \legend{}
        \end{axis}
    \end{tikzpicture}

    \medskip
    \ref{legend-seq-doubling-0}
    \caption{Discarding2, Discarding4 und Doubling auf wiki.txt}
\end{figure}

\begin{figure}[ht]
    \centering
    \begin{tikzpicture}
        \begin{axis}[
                name=axis1,
                cycle list name={exoticlines},
                width=0.5\textwidth,
                height=48mm,
                title={dna.txt},
                xlabel={input size [GiB]},
                ylabel={SA construction time [min]},
                legend columns=2,
                legend to name=legend-seq-doubling-1,
                legend style={
                    /tikz/every even column/.append style={column sep=0.5cm,black},
                    /tikz/every even column/.append style={black},
                },
                ymax=40
            ]

            %% MULTIPLOT(algo) SELECT prefix/1024/1024/1024.0 AS x, time/1000/60 AS y, MULTIPLOT
            %% FROM (
            %% SELECT algo, input, MEDIAN(memFinal) AS memFinal, MEDIAN(memOff) AS memOff, AVG(memPeak) AS memPeak, prefix, rep, thread_count, MEDIAN(time) AS time, sacheck FROM stats2 GROUP BY algo, input, prefix, rep, thread_count
            %% ) WHERE thread_count < 9 AND input="dna.txt" AND sacheck="ok" 
            %% AND (algo="Naiv" OR algo="Doubling" OR algo="Discarding2" OR algo="Discarding4")
            %% GROUP BY MULTIPLOT,x ORDER BY MULTIPLOT,x
            \addplot coordinates { (0.195312,2.32164) (0.390625,4.98423) (0.78125,10.6304) (1.5625,21.3285) };
            \addlegendentry{algo=Discarding2};
            \addplot coordinates { (0.195312,2.3202) (0.390625,5.09165) (0.78125,10.9035) (1.5625,22.8657) };
            \addlegendentry{algo=Discarding4};
            \addplot coordinates { (0.195312,5.53637) (0.390625,11.3228) (0.78125,23.2906) (1.5625,47.8121) };
            \addlegendentry{algo=Doubling};
            \addplot coordinates { (0.195312,3.07022) (0.390625,6.92539) (0.78125,15.1993) (1.5625,31.8829) };
            \addlegendentry{algo=Naiv};

        \end{axis}
        \begin{axis}[
                cycle list name={exoticlines},
                at={(axis1.outer north east)},
                anchor=outer north west,
                name=axis2,
                width=0.5\textwidth,
                height=48mm,
                title={dna.txt},
                xlabel={input size [GiB]},
                ylabel={Extra Memory [GiB]},
            ]

            %% MULTIPLOT(algo) SELECT prefix/1024/1024/1024.0 AS x, memPeak/1024/1024/1024 AS y, MULTIPLOT
            %% FROM (
            %% SELECT algo, input, MEDIAN(memFinal) AS memFinal, MEDIAN(memOff) AS memOff, AVG(memPeak) AS memPeak, prefix, rep, thread_count, MEDIAN(time) AS time, sacheck FROM stats2 GROUP BY algo, input, prefix, rep, thread_count
            %% ) WHERE thread_count < 9 AND input="dna.txt" AND sacheck="ok" 
            %% AND (algo="Naiv" OR algo="Doubling" OR algo="Discarding2" OR algo="Discarding4")
            %% GROUP BY MULTIPLOT,x ORDER BY MULTIPLOT,x
            \addplot coordinates { (0.195312,2.34575) (0.390625,4.6895) (0.78125,9.377) (1.5625,18.752) };
            \addlegendentry{algo=Discarding2};
            \addplot coordinates { (0.195312,3.90825) (0.390625,7.8145) (0.78125,15.627) (1.5625,31.252) };
            \addlegendentry{algo=Discarding4};
            \addplot coordinates { (0.195312,2.34575) (0.390625,4.6895) (0.78125,9.377) (1.5625,18.752) };
            \addlegendentry{algo=Doubling};
            \addplot coordinates { (0.195312,0.0) (0.390625,0.0) (0.78125,0.0) (1.5625,0.0) };
            \addlegendentry{algo=Naiv};

            \legend{}
        \end{axis}
    \end{tikzpicture}

    \medskip
    \ref{legend-seq-doubling-1}
    \caption{Discarding2, Discarding4 und Doubling auf dna.txt}
\end{figure}

\begin{figure}[ht]
    \centering
    \begin{tikzpicture}
        \begin{axis}[
                name=axis1,
                cycle list name={exoticlines},
                width=0.5\textwidth,
                height=48mm,
                title={commoncrawl.txt},
                xlabel={input size [GiB]},
                ylabel={SA construction time [min]},
                legend columns=2,
                legend to name=legend-seq-doubling-2,
                legend style={
                    /tikz/every even column/.append style={column sep=0.5cm,black},
                    /tikz/every even column/.append style={black},
                },
            ]

            %% MULTIPLOT(algo) SELECT prefix/1024/1024/1024.0 AS x, time/1000/60 AS y, MULTIPLOT
            %% FROM (
            %% SELECT algo, input, MEDIAN(memFinal) AS memFinal, MEDIAN(memOff) AS memOff, AVG(memPeak) AS memPeak, prefix, rep, thread_count, MEDIAN(time) AS time, sacheck FROM stats2 GROUP BY algo, input, prefix, rep, thread_count
            %% ) WHERE thread_count < 9 AND input="commoncrawl.txt" AND sacheck="ok" 
            %% AND (algo="Naiv" OR algo="Doubling" OR algo="Discarding2" OR algo="Discarding4")
            %% GROUP BY MULTIPLOT,x ORDER BY MULTIPLOT,x
            \addplot coordinates { (0.195312,4.39624) (0.390625,9.27186) (0.78125,20.3309) (1.5625,43.9922) };
            \addlegendentry{algo=Discarding2};
            \addplot coordinates { (0.195312,3.39922) (0.390625,7.21615) (0.78125,15.8578) (1.5625,34.3377) };
            \addlegendentry{algo=Discarding4};
            \addplot coordinates { (0.195312,13.7193) (0.390625,30.3425) (0.78125,66.562) };
            \addlegendentry{algo=Doubling};
            \addplot coordinates { (0.195312,7.14747) (0.390625,11.1463) (0.78125,26.6099) (1.5625,62.0999) };
            \addlegendentry{algo=Naiv};

        \end{axis}
        \begin{axis}[
                cycle list name={exoticlines},
                at={(axis1.outer north east)},
                anchor=outer north west,
                name=axis2,
                width=0.5\textwidth,
                height=48mm,
                title={commoncrawl.txt},
                xlabel={input size [GiB]},
                ylabel={Extra Memory [GiB]},
            ]

            %% MULTIPLOT(algo) SELECT prefix/1024/1024/1024.0 AS x, memPeak/1024/1024/1024 AS y, MULTIPLOT
            %% FROM (
            %% SELECT algo, input, MEDIAN(memFinal) AS memFinal, MEDIAN(memOff) AS memOff, AVG(memPeak) AS memPeak, prefix, rep, thread_count, MEDIAN(time) AS time, sacheck FROM stats2 GROUP BY algo, input, prefix, rep, thread_count
            %% ) WHERE thread_count < 9 AND input="commoncrawl.txt" AND sacheck="ok" 
            %% AND (algo="Naiv" OR algo="Doubling" OR algo="Discarding2" OR algo="Discarding4")
            %% GROUP BY MULTIPLOT,x ORDER BY MULTIPLOT,x
            \addplot coordinates { (0.195312,2.34575) (0.390625,4.6895) (0.78125,9.377) (1.5625,18.752) };
            \addlegendentry{algo=Discarding2};
            \addplot coordinates { (0.195312,3.90825) (0.390625,7.8145) (0.78125,15.627) (1.5625,31.252) };
            \addlegendentry{algo=Discarding4};
            \addplot coordinates { (0.195312,2.34575) (0.390625,4.6895) (0.78125,9.377) };
            \addlegendentry{algo=Doubling};
            \addplot coordinates { (0.195312,0.0) (0.390625,0.0) (0.78125,0.0) (1.5625,0.0) };
            \addlegendentry{algo=Naiv};

            \legend{}
        \end{axis}
    \end{tikzpicture}

    \medskip
    \ref{legend-seq-doubling-2}
    \caption{Discarding2, Discarding4 und Doubling auf commoncrawl.txt}
\end{figure}
\FloatBarrier

Wir werten nun das Verhalten des \textit{Doubling}-Algorithmus, sowie zwei seiner Varianten mit \textit{Discarding} und $a$-Tupling für $a=2$ und $a=4$ (siehe \cref{chapter:saca:doubling}), bei skalierender Eingabegröße aus.

\paragraph{Die Speichermessung} zeigt, dass der Speicherverbrauch der Algorithmen rein von der Länge der Eingabe und der Größe von \texttt{sa\_index} abhängt. Dies ist aus den Diagrammen und Tabellen \ref{messung:tab:memory-small-seq-none} und \ref{messung:tab:memory-large-seq-weak} ersichtlich.

Dies stimmt mit ihren erwartete Verhalten basierend auf der Implementierung (\cref{chapter:saca:doubling:memory}) überein, gemäß der alle Varianten des Algorithmus nur auf einem Array von $|\inputtext|$ Elementen arbeiten. So liegt zum Beispiel der theoretische Speicherverbrauch des Discardings mit $a$=4 und der Arrayüberlagerungsoptimierung für eine 200 MiB Eingabe bei $200~\text{[MiB]} * (a+1) * \texttt{sizeof(sa\_index)} = 4000~\text{[MiB]} = 3.907~\text{[GiB]}$, was fast genau den gemessenen Betrag entspricht.

An \cref{messung:tab:sa-chk-large-seq-weak} erkennt man jedoch auch, das der Speicherverbrauch relativ hoch ist, da ab einer Eingabelänge von über 1600 MiB das Speicherlimit des Systems erreicht wird. Der Algorithmus ist somit nicht für Speicher-limitierte Systeme geeignet.

\paragraph{Die Laufzeitmessung} in \cref{messung:tab:time-small-seq-none} und  \cref{messung:tab:time-large-seq-weak} zeigt, dass das reine \textit{Doubling} einer der langsamsten Algorithmen ist. Dies ist bei seiner theoretischen Laufzeit von $\mathcal{O}(\text{sort}(n) \ceil*{\log \text{maxlcp}})$ (\cref{algo:doubling:sec:doubling}) plausibel, da Suffix-Indexe die eindeutig feststehen ggf. wiederholt neu bestimmt werden.

Der optimierte Algorithmus -- \textit{Discarding} mit $a$-Tupling, Pipelining, Arrayüberlagerung und Wordpacking -- behandelt stattdessen jeden Suffix-Index nur solange bis er eindeutig feststeht. Dies lässt sich aus den Diagrammen und Tabellen ablesen, bei denen der Algorithmus für beide $a$-Werte in der Laufzeit besser als der Naive Algorithmus skaliert.

In \cref{messung:tab:time-small-seq-none} sieht man jedoch auch, dass für kleine Eingaben der Unterschied zum Naiven Algorithmus teilweise nicht signifikant ist, der Laufzeitvorteil also nur für größere Eingaben zu trage kommt.

Laut der theoretischen Betrachtung sollte der Algorithmus mit $a=4$ schneller sein als mit $a=2$. Dies ist in der Messung für die meisten Eingaben der Fall, es gibt aber auch Ausnahmen bei denen der Unterschied gering oder sogar umgekehrt ausfällt. Am größten tritt dieser Effekt bei \texttt{pc\_dna} und \texttt{dna.txt} auf, was darauf schließen lässt das bei kleinen Alphabetgrößen $a=2$ effizienter zu sein scheint, bzw. die gemeinsamen Prefixe bei DNA kurz genug ausfallen das bei $a=4$ die Iterationsschritte zu grob-granular sind und eindeutige Suffix-Index zu lange beibehalten werden.

Im Vergleich zum Naiven Algorithmus fällt jedoch auch auf, das es keinen signifikanten Unterschied in der Laufzeit bei verschiedenen $a$ Werten gibt, weshalb man in Anbetracht des wesentlich größeren Speicherbedarfs bei $a=4$ für In-Memory Implementierungen vermutlich besser beim normalen verdoppeln, also $a=2$ bleiben sollte.


\subsubsection{mSufSort}

\textbf{Configuration} \hfill Model name: none
% IMPORT-DATA stats2 ../results.txt

\begin{figure}[ht]
    \centering
    \begin{tikzpicture}
        \begin{axis}[
                name=axis1,
                cycle list name={exoticlines},
                width=0.5\textwidth,
                height=65mm,
                title={wiki.txt},
                xlabel={input size [200\,MB]},
                ylabel={SA construction time [s]},
                legend columns=2,
                legend to name=legend0,
                legend style={
                    /tikz/every even column/.append style={column sep=0.5cm,black},
                    /tikz/every even column/.append style={black},
                },
            ]

            %% MULTIPLOT(algo) SELECT thread_count AS x, time/1000 AS y, MULTIPLOT
            %% FROM (
            %% SELECT algo, input, MEDIAN(memFinal) AS memFinal, MEDIAN(memOff) AS memOff, AVG(memPeak) AS memPeak, prefix, rep, thread_count, MEDIAN(time) AS time, sacheck FROM stats2 GROUP BY algo, input, prefix, rep, thread_count
            %% ) WHERE input="wiki.txt" AND sacheck="ok" 
            %% AND (algo="Naiv" OR algo="MSufSort_ref" OR algo="mSufSort" OR algo="mSufSort_scan" OR algo="mSufSortV2")
            %% GROUP BY MULTIPLOT,x ORDER BY MULTIPLOT,x
            \addplot coordinates { (1,47.0701) (2,100.6) (4,217.532) };
            \addlegendentry{algo=MSufSort\_ref};
            \addplot coordinates { (1,162.841) (2,361.602) (4,783.025) (8,1729.44) (12,3448.59) (16,3842.25) (20,6183.09) };
            \addlegendentry{algo=Naiv};
            \addplot coordinates { (1,116.37) (2,260.333) (4,573.609) (8,1315.46) (12,2614.41) (16,5210.64) (20,4964.64) };
            \addlegendentry{algo=mSufSort};
            \addplot coordinates { (1,117.286) (2,258.233) (4,578.421) (8,1325.42) (12,2624.13) (16,3647.39) (20,4986.78) };
            \addlegendentry{algo=mSufSortV2};
            \addplot coordinates { (1,92.7186) (2,196.271) (4,427.233) (8,971.369) (12,1759.33) (16,2433.8) (20,4373.97) };
            \addlegendentry{algo=mSufSort\_scan};

        \end{axis}
        \begin{axis}[
                cycle list name={exoticlines},
                at={(axis1.outer north east)},
                anchor=outer north west,
                name=axis2,
                width=0.5\textwidth,
                height=65mm,
                title={wiki.txt},
                xlabel={input size [200\,MB]},
                ylabel={Extra Memory [MB]},
            ]

            %% MULTIPLOT(algo) SELECT thread_count AS x, memPeak/1000000 AS y, MULTIPLOT
            %% FROM (
            %% SELECT algo, input, MEDIAN(memFinal) AS memFinal, MEDIAN(memOff) AS memOff, AVG(memPeak) AS memPeak, prefix, rep, thread_count, MEDIAN(time) AS time, sacheck FROM stats2 GROUP BY algo, input, prefix, rep, thread_count
            %% ) WHERE input="wiki.txt" AND sacheck="ok" 
            %% AND (algo="Naiv" OR algo="MSufSort_ref" OR algo="mSufSort" OR algo="mSufSort_scan" OR algo="mSufSortV2")
            %% GROUP BY MULTIPLOT,x ORDER BY MULTIPLOT,x
            \addplot coordinates { (1,849.4034) (2,1705.22) (4,3463.59) };
            \addlegendentry{algo=MSufSort\_ref};
            \addplot coordinates { (1,0.0) (2,0.0) (4,0.0) (8,0.0) (12,0.0) (16,0.0) (20,0.0) };
            \addlegendentry{algo=Naiv};
            \addplot coordinates { (1,233.586) (2,466.839) (4,934.499) (8,1883.2) (12,6801.19) (16,7627.81) (20,8444.25) };
            \addlegendentry{algo=mSufSort};
            \addplot coordinates { (1,233.587) (2,466.8409) (4,934.503) (8,1883.21) (12,6801.21) (16,7627.84) (20,8444.28) };
            \addlegendentry{algo=mSufSortV2};
            \addplot coordinates { (1,25.5203) (2,50.6866) (4,101.0216) (8,201.685) (12,806.023) (16,806.022) (20,806.032) };
            \addlegendentry{algo=mSufSort\_scan};

            \legend{}
        \end{axis}
    \end{tikzpicture}

    \medskip
    \ref{legend0}
\end{figure}

\begin{figure}[ht]
    \centering
    \begin{tikzpicture}
        \begin{axis}[
                name=axis1,
                cycle list name={exoticlines},
                width=0.5\textwidth,
                height=65mm,
                title={dna.txt},
                xlabel={input size [200\,MB]},
                ylabel={SA construction time [s]},
                legend columns=2,
                legend to name=legend1,
                legend style={
                    /tikz/every even column/.append style={column sep=0.5cm,black},
                    /tikz/every even column/.append style={black},
                },
            ]

            %% MULTIPLOT(algo) SELECT thread_count AS x, time/1000 AS y, MULTIPLOT
            %% FROM (
            %% SELECT algo, input, MEDIAN(memFinal) AS memFinal, MEDIAN(memOff) AS memOff, AVG(memPeak) AS memPeak, prefix, rep, thread_count, MEDIAN(time) AS time, sacheck FROM stats2 GROUP BY algo, input, prefix, rep, thread_count
            %% ) WHERE input="dna.txt" AND sacheck="ok" 
            %% AND (algo="Naiv" OR algo="MSufSort_ref" OR algo="mSufSort" OR algo="mSufSort_scan" OR algo="mSufSortV2")
            %% GROUP BY MULTIPLOT,x ORDER BY MULTIPLOT,x
            \addplot coordinates { (1,59.4812) (2,123.695) (4,261.569) };
            \addlegendentry{algo=MSufSort\_ref};
            \addplot coordinates { (1,183.948) (2,415.412) (4,910.737) (8,1910.63) (12,3645.28) (16,4061.77) (20,6575.73) };
            \addlegendentry{algo=Naiv};
            \addplot coordinates { (1,135.01) (2,293.153) (4,645.57) (8,1394) (12,3983.89) (16,5464.05) (20,6732.84) };
            \addlegendentry{algo=mSufSort};
            \addplot coordinates { (1,135.29) (2,295.803) (4,646.518) (8,1388.83) (12,2780.54) (16,3821.19) (20,6751.43) };
            \addlegendentry{algo=mSufSortV2};
            \addplot coordinates { (1,87.2189) (2,187.146) (4,404.273) (8,869.338) (12,1633.42) (16,2250.08) (20,4149.9) };
            \addlegendentry{algo=mSufSort\_scan};

        \end{axis}
        \begin{axis}[
                cycle list name={exoticlines},
                at={(axis1.outer north east)},
                anchor=outer north west,
                name=axis2,
                width=0.5\textwidth,
                height=65mm,
                title={dna.txt},
                xlabel={input size [200\,MB]},
                ylabel={Extra Memory [MB]},
            ]

            %% MULTIPLOT(algo) SELECT thread_count AS x, memPeak/1000000 AS y, MULTIPLOT
            %% FROM (
            %% SELECT algo, input, MEDIAN(memFinal) AS memFinal, MEDIAN(memOff) AS memOff, AVG(memPeak) AS memPeak, prefix, rep, thread_count, MEDIAN(time) AS time, sacheck FROM stats2 GROUP BY algo, input, prefix, rep, thread_count
            %% ) WHERE input="dna.txt" AND sacheck="ok" 
            %% AND (algo="Naiv" OR algo="MSufSort_ref" OR algo="mSufSort" OR algo="mSufSort_scan" OR algo="mSufSortV2")
            %% GROUP BY MULTIPLOT,x ORDER BY MULTIPLOT,x
            \addplot coordinates { (1,1054) (2,2107.9) (4,4215.7) };
            \addlegendentry{algo=MSufSort\_ref};
            \addplot coordinates { (1,0.0) (2,0.0) (4,0.0) (8,0.0) (12,0.0) (16,0.0) (20,0.0) };
            \addlegendentry{algo=Naiv};
            \addplot coordinates { (1,505.088) (2,939.525) (4,1879.05) (8,3758.1) (12,13584.9) (16,15263) (20,16878.1) };
            \addlegendentry{algo=mSufSort};
            \addplot coordinates { (1,505.088) (2,947.914) (4,1895.83) (8,3774.87) (12,13584.9) (16,15263) (20,16878.1) };
            \addlegendentry{algo=mSufSortV2};
            \addplot coordinates { (1,100.664) (2,402.654) (4,805.307) (8,1610.61) (12,3221.23) (16,3221.23) (20,6442.45) };
            \addlegendentry{algo=mSufSort\_scan};

            \legend{}
        \end{axis}
    \end{tikzpicture}

    \medskip
    \ref{legend1}
\end{figure}

\begin{figure}[ht]
    \centering
    \begin{tikzpicture}
        \begin{axis}[
                name=axis1,
                cycle list name={exoticlines},
                width=0.5\textwidth,
                height=65mm,
                title={commoncrawl.txt},
                xlabel={input size [200\,MB]},
                ylabel={SA construction time [s]},
                legend columns=2,
                legend to name=legend2,
                legend style={
                    /tikz/every even column/.append style={column sep=0.5cm,black},
                    /tikz/every even column/.append style={black},
                },
            ]

            %% MULTIPLOT(algo) SELECT thread_count AS x, time/1000 AS y, MULTIPLOT
            %% FROM (
            %% SELECT algo, input, MEDIAN(memFinal) AS memFinal, MEDIAN(memOff) AS memOff, AVG(memPeak) AS memPeak, prefix, rep, thread_count, MEDIAN(time) AS time, sacheck FROM stats2 GROUP BY algo, input, prefix, rep, thread_count
            %% ) WHERE input="commoncrawl.txt" AND sacheck="ok" 
            %% AND (algo="Naiv" OR algo="MSufSort_ref" OR algo="mSufSort" OR algo="mSufSort_scan" OR algo="mSufSortV2")
            %% GROUP BY MULTIPLOT,x ORDER BY MULTIPLOT,x
            \addplot coordinates { (1,44.15) (2,94.8815) (4,200.219) };
            \addlegendentry{algo=MSufSort\_ref};
            \addplot coordinates { (1,428.421) (2,669.456) (4,1602.12) (8,3723.05) };
            \addlegendentry{algo=Naiv};
            \addplot coordinates { (1,171.829) (2,374.05) (4,750.092) (8,1612.19) (12,3094.05) (16,4173.33) };
            \addlegendentry{algo=mSufSort};
            \addplot coordinates { (1,171.128) (2,370.818) (4,743.063) (8,1600.29) (12,3073.38) (16,4190.24) };
            \addlegendentry{algo=mSufSortV2};
            \addplot coordinates { (1,330.535) (2,685.191) (4,1206.6) (8,2327.13) (12,4102.11) };
            \addlegendentry{algo=mSufSort\_scan};

        \end{axis}
        \begin{axis}[
                cycle list name={exoticlines},
                at={(axis1.outer north east)},
                anchor=outer north west,
                name=axis2,
                width=0.5\textwidth,
                height=65mm,
                title={commoncrawl.txt},
                xlabel={input size [200\,MB]},
                ylabel={Extra Memory [MB]},
            ]

            %% MULTIPLOT(algo) SELECT thread_count AS x, memPeak/1000000 AS y, MULTIPLOT
            %% FROM (
            %% SELECT algo, input, MEDIAN(memFinal) AS memFinal, MEDIAN(memOff) AS memOff, AVG(memPeak) AS memPeak, prefix, rep, thread_count, MEDIAN(time) AS time, sacheck FROM stats2 GROUP BY algo, input, prefix, rep, thread_count
            %% ) WHERE input="commoncrawl.txt" AND sacheck="ok" 
            %% AND (algo="Naiv" OR algo="MSufSort_ref" OR algo="mSufSort" OR algo="mSufSort_scan" OR algo="mSufSortV2")
            %% GROUP BY MULTIPLOT,x ORDER BY MULTIPLOT,x
            \addplot coordinates { (1,850.518) (2,1689.38) (4,3387.6) };
            \addlegendentry{algo=MSufSort\_ref};
            \addplot coordinates { (1,0.0) (2,0.0) (4,0.0) (8,0.0) };
            \addlegendentry{algo=Naiv};
            \addplot coordinates { (1,216.915) (2,438.129) (4,876.56) (8,1751.19) (12,4176.83) (16,6999.77) };
            \addlegendentry{algo=mSufSort};
            \addplot coordinates { (1,216.919) (2,438.133) (4,876.564) (8,1751.2) (12,4176.87) (16,6999.84) };
            \addlegendentry{algo=mSufSortV2};
            \addplot coordinates { (1,25.61608) (2,50.801) (4,101.14) (8,201.804) (12,403.61) };
            \addlegendentry{algo=mSufSort\_scan};

            \legend{}
        \end{axis}
    \end{tikzpicture}

    \medskip
    \ref{legend2}
\end{figure}
\FloatBarrier

Es wurden alle drei sequentiellen Varianten des MSufSort parallelisiert.
In \todo{Fix these legends wtf} den Experimenten \ref{legend0}, \ref{legend1} und \ref{legend2} ist jeweils der naive Algorithmus am langsamsten, was bedeutet,
dass zumindest alle Varianten benutzbar sind. In \ref{legend0} und \ref{legend1} ist die Scan-Variante am schnellsten,
insbesondere bei dem DNA Eingabetext erkennt man auch eine deutlich flacher verlaufende Kurve,
vermutlich durch die lineare Laufzeit in einem der Sortierer.
Dieser (Scan-)Sortierer profitiert auch insbesondere von sehr kleinen Alphabeten,
da dann weniger Cache-Misses auftreten. Die beiden anderen Varianten unterscheiden sich fast gar nicht.
In \ref{legend2} liegt die Scan-Variante oberhalb versetzt zu den anderen Varianten,
skaliert aber zumindest nicht schlechter und liegt näher an den normalen MSufSort Varianten als am naiven Sortierer.
\clearpage
\subsubsection{qSufSort}

\textbf{Configuration} \hfill Model name: Intel\textsuperscript{\textregistered} Xeon\textsuperscript{\textregistered} CPU E5-2640 v4 @ 2.40GHz
% IMPORT-DATA stats2 ../results.txt

\begin{figure}[ht]
    \centering
    \begin{tikzpicture}
        \begin{axis}[
                name=axis1,
                cycle list name={exoticlines},
                width=0.5\textwidth,
                height=55mm,
                title={wiki.txt},
                xlabel={input size [200\,MB]},
                ylabel={SA construction time [min]},
                legend columns=2,
                legend to name=legend-seq-qsufsort-0,
                legend style={
                    /tikz/every even column/.append style={column sep=0.5cm,black},
                    /tikz/every even column/.append style={black},
                },
            ]

            %% MULTIPLOT(algo) SELECT thread_count AS x, time/1000/60 AS y, MULTIPLOT
            %% FROM (
            %% SELECT algo, input, MEDIAN(memFinal) AS memFinal, MEDIAN(memOff) AS memOff, AVG(memPeak) AS memPeak, prefix, rep, thread_count, MEDIAN(time) AS time, sacheck FROM stats2 GROUP BY algo, input, prefix, rep, thread_count
            %% ) WHERE input="wiki.txt" AND sacheck="ok" 
            %% AND (algo="Naiv" OR algo="qsufsort" OR algo="qsufsort_ref")
            %% GROUP BY MULTIPLOT,x ORDER BY MULTIPLOT,x
            \addplot coordinates { (1,2.7048) (2,6.0286) (4,15.8137) (8,28.8136) (12,59.9193) (16,65.215) (20,105.373) };
            \addlegendentry{algo=Naiv};
            \addplot coordinates { (1,2.40105) (2,5.22747) (4,11.5856) (8,27.6101) (12,55.4266) (16,84.8983) };
            \addlegendentry{algo=qsufsort};
            \addplot coordinates { (1,1.26217) (2,3.15719) (4,6.1854) (8,13.9982) };
            \addlegendentry{algo=qsufsort\_ref};

        \end{axis}
        \begin{axis}[
                cycle list name={exoticlines},
                at={(axis1.outer north east)},
                anchor=outer north west,
                name=axis2,
                width=0.5\textwidth,
                height=55mm,
                title={wiki.txt},
                xlabel={input size [200\,MB]},
                ylabel={Extra Memory [GiB]},
            ]

            %% MULTIPLOT(algo) SELECT thread_count AS x, memPeak/1024/1024/1024 AS y, MULTIPLOT
            %% FROM (
            %% SELECT algo, input, MEDIAN(memFinal) AS memFinal, MEDIAN(memOff) AS memOff, AVG(memPeak) AS memPeak, prefix, rep, thread_count, MEDIAN(time) AS time, sacheck FROM stats2 GROUP BY algo, input, prefix, rep, thread_count
            %% ) WHERE input="wiki.txt" AND sacheck="ok" 
            %% AND (algo="Naiv" OR algo="qsufsort" OR algo="qsufsort_ref")
            %% GROUP BY MULTIPLOT,x ORDER BY MULTIPLOT,x
            \addplot coordinates { (1,0.0) (2,0.0) (4,0.0) (8,0.0) (12,0.0) (16,0.0) (20,0.0) };
            \addlegendentry{algo=Naiv};
            \addplot coordinates { (1,0.783245) (2,1.5645) (4,3.127) (8,6.252) (12,18.752) (16,25.002) };
            \addlegendentry{algo=qsufsort};
            \addplot coordinates { (1,3.125) (2,6.25) (4,12.5) (8,25) };
            \addlegendentry{algo=qsufsort\_ref};

            \legend{}
        \end{axis}
    \end{tikzpicture}

    \medskip
    \ref{legend-seq-qsufsort-0}
    \caption{qSufSort auf wiki.txt}
\end{figure}

\begin{figure}[ht]
    \centering
    \begin{tikzpicture}
        \begin{axis}[
                name=axis1,
                cycle list name={exoticlines},
                width=0.5\textwidth,
                height=55mm,
                title={dna.txt},
                xlabel={input size [200\,MB]},
                ylabel={SA construction time [min]},
                legend columns=2,
                legend to name=legend-seq-qsufsort-1,
                legend style={
                    /tikz/every even column/.append style={column sep=0.5cm,black},
                    /tikz/every even column/.append style={black},
                },
            ]

            %% MULTIPLOT(algo) SELECT thread_count AS x, time/1000/60 AS y, MULTIPLOT
            %% FROM (
            %% SELECT algo, input, MEDIAN(memFinal) AS memFinal, MEDIAN(memOff) AS memOff, AVG(memPeak) AS memPeak, prefix, rep, thread_count, MEDIAN(time) AS time, sacheck FROM stats2 GROUP BY algo, input, prefix, rep, thread_count
            %% ) WHERE input="dna.txt" AND sacheck="ok" 
            %% AND (algo="Naiv" OR algo="qsufsort" OR algo="qsufsort_ref")
            %% GROUP BY MULTIPLOT,x ORDER BY MULTIPLOT,x
            \addplot coordinates { (1,3.07022) (2,6.92539) (4,15.1993) (8,31.8829) (12,50.4039) (16,86.2066) (20,90.5095) };
            \addlegendentry{algo=Naiv};
            \addplot coordinates { (1,1.80364) (2,4.12069) (4,9.03971) (8,18.6786) (12,37.2348) (16,45.4443) };
            \addlegendentry{algo=qsufsort};
            \addplot coordinates { (1,1.13116) (2,2.50599) (4,5.44533) (8,11.7436) };
            \addlegendentry{algo=qsufsort\_ref};

        \end{axis}
        \begin{axis}[
                cycle list name={exoticlines},
                at={(axis1.outer north east)},
                anchor=outer north west,
                name=axis2,
                width=0.5\textwidth,
                height=55mm,
                title={dna.txt},
                xlabel={input size [200\,MB]},
                ylabel={Extra Memory [GiB]},
            ]

            %% MULTIPLOT(algo) SELECT thread_count AS x, memPeak/1024/1024/1024 AS y, MULTIPLOT
            %% FROM (
            %% SELECT algo, input, MEDIAN(memFinal) AS memFinal, MEDIAN(memOff) AS memOff, AVG(memPeak) AS memPeak, prefix, rep, thread_count, MEDIAN(time) AS time, sacheck FROM stats2 GROUP BY algo, input, prefix, rep, thread_count
            %% ) WHERE input="dna.txt" AND sacheck="ok" 
            %% AND (algo="Naiv" OR algo="qsufsort" OR algo="qsufsort_ref")
            %% GROUP BY MULTIPLOT,x ORDER BY MULTIPLOT,x
            \addplot coordinates { (1,0.0) (2,0.0) (4,0.0) (8,0.0) (12,0.0) (16,0.0) (20,0.0) };
            \addlegendentry{algo=Naiv};
            \addplot coordinates { (1,0.783245) (2,1.5645) (4,3.127) (8,6.252) (12,18.752) (16,25.002) };
            \addlegendentry{algo=qsufsort};
            \addplot coordinates { (1,3.125) (2,6.25) (4,12.5) (8,25) };
            \addlegendentry{algo=qsufsort\_ref};

            \legend{}
        \end{axis}
    \end{tikzpicture}

    \medskip
    \ref{legend-seq-qsufsort-1}
    \caption{qSufSort auf dna.txt}
\end{figure}

\begin{figure}[ht]
    \centering
    \begin{tikzpicture}
        \begin{axis}[
                name=axis1,
                cycle list name={exoticlines},
                width=0.5\textwidth,
                height=55mm,
                title={commoncrawl.txt},
                xlabel={input size [200\,MB]},
                ylabel={SA construction time [min]},
                legend columns=2,
                legend to name=legend-seq-qsufsort-2,
                legend style={
                    /tikz/every even column/.append style={column sep=0.5cm,black},
                    /tikz/every even column/.append style={black},
                },
            ]

            %% MULTIPLOT(algo) SELECT thread_count AS x, time/1000/60 AS y, MULTIPLOT
            %% FROM (
            %% SELECT algo, input, MEDIAN(memFinal) AS memFinal, MEDIAN(memOff) AS memOff, AVG(memPeak) AS memPeak, prefix, rep, thread_count, MEDIAN(time) AS time, sacheck FROM stats2 GROUP BY algo, input, prefix, rep, thread_count
            %% ) WHERE input="commoncrawl.txt" AND sacheck="ok" 
            %% AND (algo="Naiv" OR algo="qsufsort" OR algo="qsufsort_ref")
            %% GROUP BY MULTIPLOT,x ORDER BY MULTIPLOT,x
            \addplot coordinates { (1,7.14747) (2,11.1463) (4,26.6099) (8,62.0999) };
            \addlegendentry{algo=Naiv};
            \addplot coordinates { (1,2.9285) (2,6.49794) (4,14.4873) (8,32.9371) (12,52.4071) };
            \addlegendentry{algo=qsufsort};
            \addplot coordinates { (1,1.44855) (2,3.1407) (4,6.92753) (8,15.4028) };
            \addlegendentry{algo=qsufsort\_ref};

        \end{axis}
        \begin{axis}[
                cycle list name={exoticlines},
                at={(axis1.outer north east)},
                anchor=outer north west,
                name=axis2,
                width=0.5\textwidth,
                height=55mm,
                title={commoncrawl.txt},
                xlabel={input size [200\,MB]},
                ylabel={Extra Memory [GiB]},
            ]

            %% MULTIPLOT(algo) SELECT thread_count AS x, memPeak/1024/1024/1024 AS y, MULTIPLOT
            %% FROM (
            %% SELECT algo, input, MEDIAN(memFinal) AS memFinal, MEDIAN(memOff) AS memOff, AVG(memPeak) AS memPeak, prefix, rep, thread_count, MEDIAN(time) AS time, sacheck FROM stats2 GROUP BY algo, input, prefix, rep, thread_count
            %% ) WHERE input="commoncrawl.txt" AND sacheck="ok" 
            %% AND (algo="Naiv" OR algo="qsufsort" OR algo="qsufsort_ref")
            %% GROUP BY MULTIPLOT,x ORDER BY MULTIPLOT,x
            \addplot coordinates { (1,0.0) (2,0.0) (4,0.0) (8,0.0) };
            \addlegendentry{algo=Naiv};
            \addplot coordinates { (1,0.783245) (2,1.5645) (4,3.127) (8,6.252) (12,18.752) };
            \addlegendentry{algo=qsufsort};
            \addplot coordinates { (1,3.125) (2,6.25) (4,12.5) (8,25) };
            \addlegendentry{algo=qsufsort\_ref};

            \legend{}
        \end{axis}
    \end{tikzpicture}

    \medskip
    \ref{legend-seq-qsufsort-2}
    \caption{qSufSort auf commoncrawl.txt}
\end{figure}
\FloatBarrier

\subsubsection{SAIS/SADS}

\textbf{Configuration} \hfill Model name: none
% IMPORT-DATA stats2 ../results.txt

\begin{figure}[ht]
    \centering
    \begin{tikzpicture}
        \begin{axis}[
                name=axis1,
                cycle list name={exoticlines},
                width=0.5\textwidth,
                height=65mm,
                title={wiki.txt},
                xlabel={input size [200\,MB]},
                ylabel={SA construction time [min]},
                legend columns=2,
                legend to name=legend-seq-sais-0,
                legend style={
                    /tikz/every even column/.append style={column sep=0.5cm,black},
                    /tikz/every even column/.append style={black},
                },
            ]

            %% MULTIPLOT(algo) SELECT thread_count AS x, time/1000/60 AS y, MULTIPLOT
            %% FROM (
            %% SELECT algo, input, MEDIAN(memFinal) AS memFinal, MEDIAN(memOff) AS memOff, AVG(memPeak) AS memPeak, prefix, rep, thread_count, MEDIAN(time) AS time, sacheck FROM stats2 GROUP BY algo, input, prefix, rep, thread_count
            %% ) WHERE input="wiki.txt" AND sacheck="ok" 
            %% AND (algo="Naiv" OR algo="SAIS" OR algo="SAIS_ref" OR algo="SADS" OR algo="SADS_ref" OR algo="SAIS-LITE_ref")
            %% GROUP BY MULTIPLOT,x ORDER BY MULTIPLOT,x
            \addplot coordinates { (1,2.71401) (2,6.02671) (4,13.0504) (8,28.824) (12,57.4765) (16,64.0376) (20,103.051) };
            \addlegendentry{algo=Naiv};
            \addplot coordinates { (1,2.61998) (2,5.66814) (4,11.9712) (8,24.3078) (12,39.6061) (16,60.6312) (20,94.8845) };
            \addlegendentry{algo=SADS};
            \addplot coordinates { (1,1.8516) (2,4.18147) (4,9.1634) (8,19.213) };
            \addlegendentry{algo=SADS\_ref};
            \addplot coordinates { (1,1.42253) (2,3.26143) (4,7.23316) (8,15.3443) (12,25.3336) (16,42.9405) (20,55.74) };
            \addlegendentry{algo=SAIS};
            \addplot coordinates { (1,0.629381) (2,1.33933) (4,2.80085) (8,5.76628) };
            \addlegendentry{algo=SAIS-LITE\_ref};
            \addplot coordinates { (1,1.42907) (2,3.27101) (4,9.25879) (8,18.3677) };
            \addlegendentry{algo=SAIS\_ref};

        \end{axis}
        \begin{axis}[
                cycle list name={exoticlines},
                at={(axis1.outer north east)},
                anchor=outer north west,
                name=axis2,
                width=0.5\textwidth,
                height=65mm,
                title={wiki.txt},
                xlabel={input size [200\,MB]},
                ylabel={Extra Memory [GiB]},
            ]

            %% MULTIPLOT(algo) SELECT thread_count AS x, memPeak/1024/1024/1024 AS y, MULTIPLOT
            %% FROM (
            %% SELECT algo, input, MEDIAN(memFinal) AS memFinal, MEDIAN(memOff) AS memOff, AVG(memPeak) AS memPeak, prefix, rep, thread_count, MEDIAN(time) AS time, sacheck FROM stats2 GROUP BY algo, input, prefix, rep, thread_count
            %% ) WHERE input="wiki.txt" AND sacheck="ok" 
            %% AND (algo="Naiv" OR algo="SAIS" OR algo="SAIS_ref" OR algo="SADS" OR algo="SADS_ref" OR algo="SAIS-LITE_ref")
            %% GROUP BY MULTIPLOT,x ORDER BY MULTIPLOT,x
            \addplot coordinates { (1,0.0) (2,0.0) (4,0.0) (8,0.0) (12,0.0) (16,0.0) (20,0.0) };
            \addlegendentry{algo=Naiv};
            \addplot coordinates { (1,0.564877) (2,1.099) (4,2.1401) (8,4.09191) (12,5.97584) (16,7.96063) (20,9.92052) };
            \addlegendentry{algo=SADS};
            \addplot coordinates { (1,0.0993436) (2,0.189524) (4,0.360081) (8,0.661858) };
            \addlegendentry{algo=SADS\_ref};
            \addplot coordinates { (1,0.118646) (2,0.226865) (4,0.433655) (8,0.804417) (12,1.87826) (16,2.49715) (20,3.09922) };
            \addlegendentry{algo=SAIS};
            \addplot coordinates { (1,0.781254) (2,1.5625) (4,3.125) (8,6.25) };
            \addlegendentry{algo=SAIS-LITE\_ref};
            \addplot coordinates { (1,0.0745086) (2,0.141785) (4,0.269095) (8,0.499429) };
            \addlegendentry{algo=SAIS\_ref};

            \legend{}
        \end{axis}
    \end{tikzpicture}

    \medskip
    \ref{legend-seq-sais-0}
\end{figure}

\begin{figure}[ht]
    \centering
    \begin{tikzpicture}
        \begin{axis}[
                name=axis1,
                cycle list name={exoticlines},
                width=0.5\textwidth,
                height=65mm,
                title={dna.txt},
                xlabel={input size [200\,MB]},
                ylabel={SA construction time [min]},
                legend columns=2,
                legend to name=legend-seq-sais-1,
                legend style={
                    /tikz/every even column/.append style={column sep=0.5cm,black},
                    /tikz/every even column/.append style={black},
                },
            ]

            %% MULTIPLOT(algo) SELECT thread_count AS x, time/1000/60 AS y, MULTIPLOT
            %% FROM (
            %% SELECT algo, input, MEDIAN(memFinal) AS memFinal, MEDIAN(memOff) AS memOff, AVG(memPeak) AS memPeak, prefix, rep, thread_count, MEDIAN(time) AS time, sacheck FROM stats2 GROUP BY algo, input, prefix, rep, thread_count
            %% ) WHERE input="dna.txt" AND sacheck="ok" 
            %% AND (algo="Naiv" OR algo="SAIS" OR algo="SAIS_ref" OR algo="SADS" OR algo="SADS_ref" OR algo="SAIS-LITE_ref")
            %% GROUP BY MULTIPLOT,x ORDER BY MULTIPLOT,x
            \addplot coordinates { (1,3.06579) (2,6.92353) (4,15.1789) (8,31.8439) (12,60.7547) (16,67.6962) (20,109.595) };
            \addlegendentry{algo=Naiv};
            \addplot coordinates { (1,2.66828) (2,5.41248) (4,15.9998) (8,23.1489) (12,37.848) (16,71.3739) (20,86.2139) };
            \addlegendentry{algo=SADS};
            \addplot coordinates { (1,1.94399) (2,4.12401) (4,8.98159) (8,19.3533) };
            \addlegendentry{algo=SADS\_ref};
            \addplot coordinates { (1,1.4321) (2,3.09755) (4,6.76803) (8,14.5458) (12,24.6139) (16,38.6081) (20,54.7418) };
            \addlegendentry{algo=SAIS};
            \addplot coordinates { (1,0.672712) (2,1.33055) (4,2.70553) (8,6.0267) };
            \addlegendentry{algo=SAIS-LITE\_ref};
            \addplot coordinates { (1,1.43087) (2,3.10891) (4,6.74562) (8,14.517) };
            \addlegendentry{algo=SAIS\_ref};

        \end{axis}
        \begin{axis}[
                cycle list name={exoticlines},
                at={(axis1.outer north east)},
                anchor=outer north west,
                name=axis2,
                width=0.5\textwidth,
                height=65mm,
                title={dna.txt},
                xlabel={input size [200\,MB]},
                ylabel={Extra Memory [GiB]},
            ]

            %% MULTIPLOT(algo) SELECT thread_count AS x, memPeak/1024/1024/1024 AS y, MULTIPLOT
            %% FROM (
            %% SELECT algo, input, MEDIAN(memFinal) AS memFinal, MEDIAN(memOff) AS memOff, AVG(memPeak) AS memPeak, prefix, rep, thread_count, MEDIAN(time) AS time, sacheck FROM stats2 GROUP BY algo, input, prefix, rep, thread_count
            %% ) WHERE input="dna.txt" AND sacheck="ok" 
            %% AND (algo="Naiv" OR algo="SAIS" OR algo="SAIS_ref" OR algo="SADS" OR algo="SADS_ref" OR algo="SAIS-LITE_ref")
            %% GROUP BY MULTIPLOT,x ORDER BY MULTIPLOT,x
            \addplot coordinates { (1,0.0) (2,0.0) (4,0.0) (8,0.0) (12,0.0) (16,0.0) (20,0.0) };
            \addlegendentry{algo=Naiv};
            \addplot coordinates { (1,0.547296) (2,1.01825) (4,1.97905) (8,3.96975) (12,5.99478) (16,7.99359) (20,10.0915) };
            \addlegendentry{algo=SADS};
            \addplot coordinates { (1,0.106607) (2,0.180249) (4,0.334526) (8,0.658049) };
            \addlegendentry{algo=SADS\_ref};
            \addplot coordinates { (1,0.101103) (2,0.185494) (4,0.356031) (8,0.716865) (12,1.74731) (16,2.34879) (20,2.97291) };
            \addlegendentry{algo=SAIS};
            \addplot coordinates { (1,0.781254) (2,1.5625) (4,3.125) (8,6.25) };
            \addlegendentry{algo=SAIS-LITE\_ref};
            \addplot coordinates { (1,0.0708514) (2,0.124751) (4,0.235499) (8,0.481298) };
            \addlegendentry{algo=SAIS\_ref};

            \legend{}
        \end{axis}
    \end{tikzpicture}

    \medskip
    \ref{legend-seq-sais-1}
\end{figure}

\begin{figure}[ht]
    \centering
    \begin{tikzpicture}
        \begin{axis}[
                name=axis1,
                cycle list name={exoticlines},
                width=0.5\textwidth,
                height=65mm,
                title={commoncrawl.txt},
                xlabel={input size [200\,MB]},
                ylabel={SA construction time [min]},
                legend columns=2,
                legend to name=legend-seq-sais-2,
                legend style={
                    /tikz/every even column/.append style={column sep=0.5cm,black},
                    /tikz/every even column/.append style={black},
                },
            ]

            %% MULTIPLOT(algo) SELECT thread_count AS x, time/1000/60 AS y, MULTIPLOT
            %% FROM (
            %% SELECT algo, input, MEDIAN(memFinal) AS memFinal, MEDIAN(memOff) AS memOff, AVG(memPeak) AS memPeak, prefix, rep, thread_count, MEDIAN(time) AS time, sacheck FROM stats2 GROUP BY algo, input, prefix, rep, thread_count
            %% ) WHERE input="commoncrawl.txt" AND sacheck="ok" 
            %% AND (algo="Naiv" OR algo="SAIS" OR algo="SAIS_ref" OR algo="SADS" OR algo="SADS_ref" OR algo="SAIS-LITE_ref")
            %% GROUP BY MULTIPLOT,x ORDER BY MULTIPLOT,x
            \addplot coordinates { (1,7.14036) (2,11.1576) (4,26.7019) (8,62.0508) };
            \addlegendentry{algo=Naiv};
            \addplot coordinates { (1,2.0736) (2,4.69942) (4,10.1292) (8,21.6334) (12,35.5195) (16,68.7152) (20,84.2733) };
            \addlegendentry{algo=SADS};
            \addplot coordinates { (1,1.45218) (2,3.41282) (4,7.58637) (8,16.5281) };
            \addlegendentry{algo=SADS\_ref};
            \addplot coordinates { (1,1.11341) (2,2.63141) (4,5.89157) (8,12.9557) (12,21.8599) (16,30.6487) (20,50.1443) };
            \addlegendentry{algo=SAIS};
            \addplot coordinates { (1,0.509132) (2,1.10527) (4,2.2822) (8,4.92302) };
            \addlegendentry{algo=SAIS-LITE\_ref};
            \addplot coordinates { (1,1.11334) (2,2.63203) (4,5.92267) (8,13.0398) };
            \addlegendentry{algo=SAIS\_ref};

        \end{axis}
        \begin{axis}[
                cycle list name={exoticlines},
                at={(axis1.outer north east)},
                anchor=outer north west,
                name=axis2,
                width=0.5\textwidth,
                height=65mm,
                title={commoncrawl.txt},
                xlabel={input size [200\,MB]},
                ylabel={Extra Memory [GiB]},
            ]

            %% MULTIPLOT(algo) SELECT thread_count AS x, memPeak/1024/1024/1024 AS y, MULTIPLOT
            %% FROM (
            %% SELECT algo, input, MEDIAN(memFinal) AS memFinal, MEDIAN(memOff) AS memOff, AVG(memPeak) AS memPeak, prefix, rep, thread_count, MEDIAN(time) AS time, sacheck FROM stats2 GROUP BY algo, input, prefix, rep, thread_count
            %% ) WHERE input="commoncrawl.txt" AND sacheck="ok" 
            %% AND (algo="Naiv" OR algo="SAIS" OR algo="SAIS_ref" OR algo="SADS" OR algo="SADS_ref" OR algo="SAIS-LITE_ref")
            %% GROUP BY MULTIPLOT,x ORDER BY MULTIPLOT,x
            \addplot coordinates { (1,0.0) (2,0.0) (4,0.0) (8,0.0) };
            \addlegendentry{algo=Naiv};
            \addplot coordinates { (1,0.552274) (2,1.06962) (4,2.04373) (8,3.91508) (12,5.72637) (16,7.50778) (20,9.28383) };
            \addlegendentry{algo=SADS};
            \addplot coordinates { (1,0.0930849) (2,0.180056) (4,0.338007) (8,0.634287) };
            \addlegendentry{algo=SADS\_ref};
            \addplot coordinates { (1,0.112834) (2,0.215545) (4,0.400654) (8,0.745269) (12,1.69918) (16,2.18314) (20,2.66373) };
            \addlegendentry{algo=SAIS};
            \addplot coordinates { (1,0.781254) (2,1.5625) (4,3.125) (8,6.25) };
            \addlegendentry{algo=SAIS-LITE\_ref};
            \addplot coordinates { (1,0.0728221) (2,0.140844) (4,0.26491) (8,0.498627) };
            \addlegendentry{algo=SAIS\_ref};

            \legend{}
        \end{axis}
    \end{tikzpicture}

    \medskip
    \ref{legend-seq-sais-2}
\end{figure}
\FloatBarrier

\clearpage
\subsubsection{Difference-Cover}

\textbf{Configuration} \hfill Model name: Intel\textsuperscript{\textregistered} Xeon\textsuperscript{\textregistered} CPU E5-2640 v4 @ 2.40GHz
% IMPORT-DATA stats2 ../results.txt

\begin{figure}[ht]
    \centering
    \begin{tikzpicture}
        \begin{axis}[
                name=axis1,
                cycle list name={exoticlines},
                width=0.5\textwidth,
                height=48mm,
                title={wiki.txt},
                xlabel={input size [GiB]},
                ylabel={SA construction time [min]},
                legend columns=2,
                legend to name=legend-seq-dc-0,
                legend style={
                    /tikz/every even column/.append style={column sep=0.5cm,black},
                    /tikz/every even column/.append style={black},
                },
            ]

            %% MULTIPLOT(algo) SELECT prefix/1024/1024/1024.0 AS x, time/1000/60 AS y, MULTIPLOT
            %% FROM (
            %% SELECT algo, input, MEDIAN(memFinal) AS memFinal, MEDIAN(memOff) AS memOff, AVG(memPeak) AS memPeak, prefix, rep, thread_count, MEDIAN(time) AS time, sacheck FROM stats2 GROUP BY algo, input, prefix, rep, thread_count
            %% ) WHERE input="wiki.txt" AND sacheck="ok" 
            %% AND (algo="Naiv" OR algo="DC3" OR algo="DC3-Lite" OR algo="DC3_ref" OR algo="DC7" OR algo="nzSufSort")
            %% GROUP BY MULTIPLOT,x ORDER BY MULTIPLOT,x
            \addplot coordinates { (0.195312,3.00654) (0.390625,5.74221) (0.78125,12.3272) (1.5625,28.7747) };
            \addlegendentry{algo=DC3};
            \addplot coordinates { (0.195312,24.2834) (0.390625,87.0219) };
            \addlegendentry{algo=DC3-Lite};
            \addplot coordinates { (0.195312,5.38767) (0.390625,11.1711) (0.78125,23.235) (1.5625,62.1154) };
            \addlegendentry{algo=DC3\_ref};
            \addplot coordinates { (0.195312,3.6085) (0.390625,7.42922) (0.78125,15.3815) (1.5625,32.3302) };
            \addlegendentry{algo=DC7};
            \addplot coordinates { (0.195312,2.7048) (0.390625,6.0286) (0.78125,15.8137) (1.5625,28.8136) (2.34375,59.9193) (3.125,65.215) (3.90625,105.373) };
            \addlegendentry{algo=Naiv};
            \addplot coordinates { (0.195312,15.6008) (0.390625,50.6913) };
            \addlegendentry{algo=nzSufSort};

        \end{axis}
        \begin{axis}[
                cycle list name={exoticlines},
                at={(axis1.outer north east)},
                anchor=outer north west,
                name=axis2,
                width=0.5\textwidth,
                height=48mm,
                title={wiki.txt},
                xlabel={input size [GiB]},
                ylabel={Extra Memory [GiB]},
            ]

            %% MULTIPLOT(algo) SELECT prefix/1024/1024/1024.0 AS x, memPeak/1024/1024/1024 AS y, MULTIPLOT
            %% FROM (
            %% SELECT algo, input, MEDIAN(memFinal) AS memFinal, MEDIAN(memOff) AS memOff, AVG(memPeak) AS memPeak, prefix, rep, thread_count, MEDIAN(time) AS time, sacheck FROM stats2 GROUP BY algo, input, prefix, rep, thread_count
            %% ) WHERE input="wiki.txt" AND sacheck="ok" 
            %% AND (algo="Naiv" OR algo="DC3" OR algo="DC3-Lite" OR algo="DC3_ref" OR algo="DC7" OR algo="nzSufSort")
            %% GROUP BY MULTIPLOT,x ORDER BY MULTIPLOT,x
            \addplot coordinates { (0.195312,3.10468) (0.390625,6.20936) (0.78125,12.4187) (1.5625,24.8374) };
            \addlegendentry{algo=DC3};
            \addplot coordinates { (0.195312,1.5625) (0.390625,3.125) };
            \addlegendentry{algo=DC3-Lite};
            \addplot coordinates { (0.195312,4.59605) (0.390625,9.1921) (0.78125,18.3842) (1.5625,36.7684) };
            \addlegendentry{algo=DC3\_ref};
            \addplot coordinates { (0.195312,1.9611) (0.390625,3.92219) (0.78125,7.84439) (1.5625,15.6888) };
            \addlegendentry{algo=DC7};
            \addplot coordinates { (0.195312,0.0) (0.390625,0.0) (0.78125,0.0) (1.5625,0.0) (2.34375,0.0) (3.125,0.0) (3.90625,0.0) };
            \addlegendentry{algo=Naiv};
            \addplot coordinates { (0.195312,0.000409931) (0.390625,0.000819862) };
            \addlegendentry{algo=nzSufSort};

            \legend{}
        \end{axis}
    \end{tikzpicture}

    \medskip
    \ref{legend-seq-dc-0}
    \caption{Algorithmen, die auf Difference Cover basieren, auf wiki.txt}
    \label{fig-seq-dc-0}
\end{figure}

\begin{figure}[ht]
    \centering
    \begin{tikzpicture}
        \begin{axis}[
                name=axis1,
                cycle list name={exoticlines},
                width=0.5\textwidth,
                height=48mm,
                title={dna.txt},
                xlabel={input size [GiB]},
                ylabel={SA construction time [min]},
                legend columns=2,
                legend to name=legend-seq-dc-1,
                legend style={
                    /tikz/every even column/.append style={column sep=0.5cm,black},
                    /tikz/every even column/.append style={black},
                },
            ]

            %% MULTIPLOT(algo) SELECT prefix/1024/1024/1024.0 AS x, time/1000/60 AS y, MULTIPLOT
            %% FROM (
            %% SELECT algo, input, MEDIAN(memFinal) AS memFinal, MEDIAN(memOff) AS memOff, AVG(memPeak) AS memPeak, prefix, rep, thread_count, MEDIAN(time) AS time, sacheck FROM stats2 GROUP BY algo, input, prefix, rep, thread_count
            %% ) WHERE input="dna.txt" AND sacheck="ok" 
            %% AND (algo="Naiv" OR algo="DC3" OR algo="DC3-Lite" OR algo="DC3_ref" OR algo="DC7" OR algo="nzSufSort")
            %% GROUP BY MULTIPLOT,x ORDER BY MULTIPLOT,x
            \addplot coordinates { (0.195312,2.71809) (0.390625,5.60346) (0.78125,11.5744) (1.5625,24.4515) };
            \addlegendentry{algo=DC3};
            \addplot coordinates { (0.195312,9.18303) (0.390625,41.7253) };
            \addlegendentry{algo=DC3-Lite};
            \addplot coordinates { (0.195312,4.11121) (0.390625,8.13608) (0.78125,16.6623) (1.5625,41.2953) };
            \addlegendentry{algo=DC3\_ref};
            \addplot coordinates { (0.195312,3.63665) (0.390625,7.5324) (0.78125,15.4058) (1.5625,31.8264) };
            \addlegendentry{algo=DC7};
            \addplot coordinates { (0.195312,3.07022) (0.390625,6.92539) (0.78125,15.1993) (1.5625,31.8829) (2.34375,50.4039) (3.125,86.2066) (3.90625,90.5095) };
            \addlegendentry{algo=Naiv};
            \addplot coordinates { (0.195312,8.55207) (0.390625,31.2622) (0.78125,101.737) };
            \addlegendentry{algo=nzSufSort};

        \end{axis}
        \begin{axis}[
                cycle list name={exoticlines},
                at={(axis1.outer north east)},
                anchor=outer north west,
                name=axis2,
                width=0.5\textwidth,
                height=48mm,
                title={dna.txt},
                xlabel={input size [GiB]},
                ylabel={Extra Memory [GiB]},
            ]

            %% MULTIPLOT(algo) SELECT prefix/1024/1024/1024.0 AS x, memPeak/1024/1024/1024 AS y, MULTIPLOT
            %% FROM (
            %% SELECT algo, input, MEDIAN(memFinal) AS memFinal, MEDIAN(memOff) AS memOff, AVG(memPeak) AS memPeak, prefix, rep, thread_count, MEDIAN(time) AS time, sacheck FROM stats2 GROUP BY algo, input, prefix, rep, thread_count
            %% ) WHERE input="dna.txt" AND sacheck="ok" 
            %% AND (algo="Naiv" OR algo="DC3" OR algo="DC3-Lite" OR algo="DC3_ref" OR algo="DC7" OR algo="nzSufSort")
            %% GROUP BY MULTIPLOT,x ORDER BY MULTIPLOT,x
            \addplot coordinates { (0.195312,3.05641) (0.390625,6.11283) (0.78125,12.2257) (1.5625,24.4513) };
            \addlegendentry{algo=DC3};
            \addplot coordinates { (0.195312,1.5625) (0.390625,3.125) };
            \addlegendentry{algo=DC3-Lite};
            \addplot coordinates { (0.195312,4.37885) (0.390625,8.7577) (0.78125,17.5154) (1.5625,35.0308) };
            \addlegendentry{algo=DC3\_ref};
            \addplot coordinates { (0.195312,1.9611) (0.390625,3.92219) (0.78125,7.84439) (1.5625,15.6888) };
            \addlegendentry{algo=DC7};
            \addplot coordinates { (0.195312,0.0) (0.390625,0.0) (0.78125,0.0) (1.5625,0.0) (2.34375,0.0) (3.125,0.0) (3.90625,0.0) };
            \addlegendentry{algo=Naiv};
            \addplot coordinates { (0.195312,5.00679e-06) (0.390625,5.00679e-06) (0.78125,5.0962e-06) };
            \addlegendentry{algo=nzSufSort};

            \legend{}
        \end{axis}
    \end{tikzpicture}

    \medskip
    \ref{legend-seq-dc-1}
    \caption{Algorithmen, die auf Difference Cover basieren, auf dna.txt}
    \label{fig-seq-dc-1}
\end{figure}

\begin{figure}[ht]
    \centering
    \begin{tikzpicture}
        \begin{axis}[
                name=axis1,
                cycle list name={exoticlines},
                width=0.5\textwidth,
                height=48mm,
                title={commoncrawl.txt},
                xlabel={input size [GiB]},
                ylabel={SA construction time [min]},
                legend columns=2,
                legend to name=legend-seq-dc-2,
                legend style={
                    /tikz/every even column/.append style={column sep=0.5cm,black},
                    /tikz/every even column/.append style={black},
                },
            ]

            %% MULTIPLOT(algo) SELECT prefix/1024/1024/1024.0 AS x, time/1000/60 AS y, MULTIPLOT
            %% FROM (
            %% SELECT algo, input, MEDIAN(memFinal) AS memFinal, MEDIAN(memOff) AS memOff, AVG(memPeak) AS memPeak, prefix, rep, thread_count, MEDIAN(time) AS time, sacheck FROM stats2 GROUP BY algo, input, prefix, rep, thread_count
            %% ) WHERE input="commoncrawl.txt" AND sacheck="ok" 
            %% AND (algo="Naiv" OR algo="DC3" OR algo="DC3-Lite" OR algo="DC3_ref" OR algo="DC7")
            %% GROUP BY MULTIPLOT,x ORDER BY MULTIPLOT,x
            \addplot coordinates { (0.195312,2.6809) (0.390625,5.64078) (0.78125,11.8365) (1.5625,25.7161) };
            \addlegendentry{algo=DC3};
            \addplot coordinates { (0.195312,19.5524) (0.390625,70.4523) };
            \addlegendentry{algo=DC3-Lite};
            \addplot coordinates { (0.195312,4.84048) (0.390625,10.6046) (0.78125,22.2288) (1.5625,60.4603) };
            \addlegendentry{algo=DC3\_ref};
            \addplot coordinates { (0.195312,3.46152) (0.390625,7.20483) (0.78125,15.7132) (1.5625,31.7633) };
            \addlegendentry{algo=DC7};
            \addplot coordinates { (0.195312,7.14747) (0.390625,11.1463) (0.78125,26.6099) (1.5625,62.0999) };
            \addlegendentry{algo=Naiv};

        \end{axis}
        \begin{axis}[
                cycle list name={exoticlines},
                at={(axis1.outer north east)},
                anchor=outer north west,
                name=axis2,
                width=0.5\textwidth,
                height=48mm,
                title={commoncrawl.txt},
                xlabel={input size [GiB]},
                ylabel={Extra Memory [GiB]},
            ]

            %% MULTIPLOT(algo) SELECT prefix/1024/1024/1024.0 AS x, memPeak/1024/1024/1024 AS y, MULTIPLOT
            %% FROM (
            %% SELECT algo, input, MEDIAN(memFinal) AS memFinal, MEDIAN(memOff) AS memOff, AVG(memPeak) AS memPeak, prefix, rep, thread_count, MEDIAN(time) AS time, sacheck FROM stats2 GROUP BY algo, input, prefix, rep, thread_count
            %% ) WHERE input="commoncrawl.txt" AND sacheck="ok" 
            %% AND (algo="Naiv" OR algo="DC3" OR algo="DC3-Lite" OR algo="DC3_ref" OR algo="DC7")
            %% GROUP BY MULTIPLOT,x ORDER BY MULTIPLOT,x
            \addplot coordinates { (0.195312,3.11898) (0.390625,6.23796) (0.78125,12.4759) (1.5625,24.9679) };
            \addlegendentry{algo=DC3};
            \addplot coordinates { (0.195312,1.5625) (0.390625,3.125) };
            \addlegendentry{algo=DC3-Lite};
            \addplot coordinates { (0.195312,4.6604) (0.390625,9.32081) (0.78125,18.6777) (1.5625,37.3555) };
            \addlegendentry{algo=DC3\_ref};
            \addplot coordinates { (0.195312,1.9611) (0.390625,3.92219) (0.78125,7.84439) (1.5625,15.6888) };
            \addlegendentry{algo=DC7};
            \addplot coordinates { (0.195312,0.0) (0.390625,0.0) (0.78125,0.0) (1.5625,0.0) };
            \addlegendentry{algo=Naiv};

            \legend{}
        \end{axis}
    \end{tikzpicture}

    \medskip
    \ref{legend-seq-dc-2}
    \caption{Algorithmen, die auf Difference Cover basieren, auf commoncrawl.txt}
    \label{fig-seq-dc-2}
\end{figure}
\FloatBarrier

In \currentauthor{Johannes Bahne} \cref{fig-seq-dc-0,fig-seq-dc-1,fig-seq-dc-2} sind die Implementierungen, die in dem \sacabench Framework umgesetzt worden sind, miteinander verglichen worden, die auf dem \emph{Difference Cover} basieren. Auffällig ist, dass die Implementierung des DC3 der Projektgruppe als einziger Algorithmus den naiven Algorithmus bezüglich der Laufzeit auf allen Testdaten schlägt. Die Referenzimplementierungen des DC3 und der DC7 sind jedoch nur auf dem \texttt{commoncrawl.txt} besser als der naive. Auf allen Eingabetexten belegen der nzSufSort und der DC3-Lite die letzten Plätze bezüglich der Laufzeit.

Dafür benötigen die Implementierungen des nzSufSort und DC3-Lite nur sehr geringen zusätzlichen Speicher zur Berechnung des Suffix-Arrays. Daher belegen die beiden Algorithmen nach dem naiven Algorithmus den zweiten und dritten Platz. Allerdings sind die Laufzeiten der beiden Implementierungen so schlecht, dass sie bereits vor $1$ GiB Eingabegröße das Zeitlimit des Messsystems überschreiten. Die Algorithmen DC3 und DC7 benötigen ab einer Größe von knapp $1,6$ GiB mehr Speicherplatz als auf dem Messsystem vorhanden, sodass sie ab dieser Eingabegröße ebenfalls abbrechen.

Auffällig ist ebenfalls, dass der Speicherverbrauch der Implementierung des DC3 der Projektgruppe geringer ist als der der Referenzimplementierung. Der DC7 benötigt sogar noch weniger Extra-Speicher. Dies liegt daran, dass die Rekursionstiefe des DC7 geringer ist, als die des DC3. Trotzdem ist der DC7 langsamer als der DC3, da in der dritten Phase mehr Vergleiche stattfinden um die Mengen zu mergen.
\clearpage
\subsubsection{DivSufSort}
\textbf{Configuration} \hfill Model name: Intel\textsuperscript{\textregistered} Xeon\textsuperscript{\textregistered} CPU E5-2640 v4 @ 2.40GHz
% IMPORT-DATA stats2 ../results.txt

\begin{figure}[ht]
    \centering
    \begin{tikzpicture}
        \begin{axis}[
                name=axis1,
                cycle list name={exoticlines},
                width=0.5\textwidth,
                height=48mm,
                title={wiki.txt},
                xlabel={input size [GiB]},
                ylabel={SA construction time [min]},
                legend columns=2,
                legend to name=legend-seq-divsufsort-0,
                legend style={
                    /tikz/every even column/.append style={column sep=0.5cm,black},
                    /tikz/every even column/.append style={black},
                },
            ]

            %% MULTIPLOT(algo) SELECT prefix/1024/1024/1024.0 AS x, time/1000/60 AS y, MULTIPLOT
            %% FROM (
            %% SELECT algo, input, MEDIAN(memFinal) AS memFinal, MEDIAN(memOff) AS memOff, AVG(memPeak) AS memPeak, prefix, rep, thread_count, MEDIAN(time) AS time, sacheck FROM stats2 GROUP BY algo, input, prefix, rep, thread_count
            %% ) WHERE input="wiki.txt" AND sacheck="ok" 
            %% AND (algo="Naiv" OR algo="DivSufSort" OR algo="DivSufSort_ref")
            %% GROUP BY MULTIPLOT,x ORDER BY MULTIPLOT,x
            \addplot coordinates { (0.195312,3.60188) (0.390625,7.23034) (0.78125,16.2487) (1.5625,38.0502) (2.34375,69.1509) (3.125,103.157) };
            \addlegendentry{algo=DivSufSort};
            \addplot coordinates { (0.195312,0.474866) (0.390625,0.89268) (0.78125,1.73052) (1.5625,3.45026) };
            \addlegendentry{algo=DivSufSort\_ref};
            \addplot coordinates { (0.195312,2.7048) (0.390625,6.0286) (0.78125,15.8137) (1.5625,28.8136) (2.34375,59.9193) (3.125,65.215) (3.90625,105.373) };
            \addlegendentry{algo=Naiv};

        \end{axis}
        \begin{axis}[
                cycle list name={exoticlines},
                at={(axis1.outer north east)},
                anchor=outer north west,
                name=axis2,
                width=0.5\textwidth,
                height=48mm,
                title={wiki.txt},
                xlabel={input size [GiB]},
                ylabel={Extra Memory [GiB]},
            ]

            %% MULTIPLOT(algo) SELECT prefix/1024/1024/1024.0 AS x, memPeak/1024/1024/1024 AS y, MULTIPLOT
            %% FROM (
            %% SELECT algo, input, MEDIAN(memFinal) AS memFinal, MEDIAN(memOff) AS memOff, AVG(memPeak) AS memPeak, prefix, rep, thread_count, MEDIAN(time) AS time, sacheck FROM stats2 GROUP BY algo, input, prefix, rep, thread_count
            %% ) WHERE input="wiki.txt" AND sacheck="ok" 
            %% AND (algo="Naiv" OR algo="DivSufSort" OR algo="DivSufSort_ref")
            %% GROUP BY MULTIPLOT,x ORDER BY MULTIPLOT,x
            \addplot coordinates { (0.195312,0.481461) (0.390625,0.962605) (0.78125,1.92422) (1.5625,3.81545) (2.34375,11.3931) (3.125,15.1968) };
            \addlegendentry{algo=DivSufSort};
            \addplot coordinates { (0.195312,0.000246726) (0.390625,0.000247501) (0.78125,0.000249021) (1.5625,0.000252061) };
            \addlegendentry{algo=DivSufSort\_ref};
            \addplot coordinates { (0.195312,0.0) (0.390625,0.0) (0.78125,0.0) (1.5625,0.0) (2.34375,0.0) (3.125,0.0) (3.90625,0.0) };
            \addlegendentry{algo=Naiv};

            \legend{}
        \end{axis}
    \end{tikzpicture}

    \medskip
    \ref{legend-seq-divsufsort-0}
\end{figure}

\begin{figure}[ht]
    \centering
    \begin{tikzpicture}
        \begin{axis}[
                name=axis1,
                cycle list name={exoticlines},
                width=0.5\textwidth,
                height=48mm,
                title={dna.txt},
                xlabel={input size [GiB]},
                ylabel={SA construction time [min]},
                legend columns=2,
                legend to name=legend-seq-divsufsort-1,
                legend style={
                    /tikz/every even column/.append style={column sep=0.5cm,black},
                    /tikz/every even column/.append style={black},
                },
            ]

            %% MULTIPLOT(algo) SELECT prefix/1024/1024/1024.0 AS x, time/1000/60 AS y, MULTIPLOT
            %% FROM (
            %% SELECT algo, input, MEDIAN(memFinal) AS memFinal, MEDIAN(memOff) AS memOff, AVG(memPeak) AS memPeak, prefix, rep, thread_count, MEDIAN(time) AS time, sacheck FROM stats2 GROUP BY algo, input, prefix, rep, thread_count
            %% ) WHERE input="dna.txt" AND sacheck="ok" 
            %% AND (algo="Naiv" OR algo="DivSufSort" OR algo="DivSufSort_ref")
            %% GROUP BY MULTIPLOT,x ORDER BY MULTIPLOT,x
            \addplot coordinates { (0.195312,4.14729) (0.390625,9.92467) (0.78125,21.8862) (1.5625,42.5506) (2.34375,71.4414) (3.125,111.881) };
            \addlegendentry{algo=DivSufSort};
            \addplot coordinates { (0.195312,0.5158) (0.390625,1.01201) (0.78125,2.05917) (1.5625,4.25621) };
            \addlegendentry{algo=DivSufSort\_ref};
            \addplot coordinates { (0.195312,3.07022) (0.390625,6.92539) (0.78125,15.1993) (1.5625,31.8829) (2.34375,50.4039) (3.125,86.2066) (3.90625,90.5095) };
            \addlegendentry{algo=Naiv};

        \end{axis}
        \begin{axis}[
                cycle list name={exoticlines},
                at={(axis1.outer north east)},
                anchor=outer north west,
                name=axis2,
                width=0.5\textwidth,
                height=48mm,
                title={dna.txt},
                xlabel={input size [GiB]},
                ylabel={Extra Memory [GiB]},
            ]

            %% MULTIPLOT(algo) SELECT prefix/1024/1024/1024.0 AS x, memPeak/1024/1024/1024 AS y, MULTIPLOT
            %% FROM (
            %% SELECT algo, input, MEDIAN(memFinal) AS memFinal, MEDIAN(memOff) AS memOff, AVG(memPeak) AS memPeak, prefix, rep, thread_count, MEDIAN(time) AS time, sacheck FROM stats2 GROUP BY algo, input, prefix, rep, thread_count
            %% ) WHERE input="dna.txt" AND sacheck="ok" 
            %% AND (algo="Naiv" OR algo="DivSufSort" OR algo="DivSufSort_ref")
            %% GROUP BY MULTIPLOT,x ORDER BY MULTIPLOT,x
            \addplot coordinates { (0.195312,0.430606) (0.390625,0.900643) (0.78125,1.8069) (1.5625,3.39691) (2.34375,10.1024) (3.125,13.3648) };
            \addlegendentry{algo=DivSufSort};
            \addplot coordinates { (0.195312,0.000246726) (0.390625,0.000247501) (0.78125,0.000249021) (1.5625,0.000252061) };
            \addlegendentry{algo=DivSufSort\_ref};
            \addplot coordinates { (0.195312,0.0) (0.390625,0.0) (0.78125,0.0) (1.5625,0.0) (2.34375,0.0) (3.125,0.0) (3.90625,0.0) };
            \addlegendentry{algo=Naiv};

            \legend{}
        \end{axis}
    \end{tikzpicture}

    \medskip
    \ref{legend-seq-divsufsort-1}
\end{figure}

\begin{figure}[ht]
    \centering
    \begin{tikzpicture}
        \begin{axis}[
                name=axis1,
                cycle list name={exoticlines},
                width=0.5\textwidth,
                height=48mm,
                title={commoncrawl.txt},
                xlabel={input size [GiB]},
                ylabel={SA construction time [min]},
                legend columns=2,
                legend to name=legend-seq-divsufsort-2,
                legend style={
                    /tikz/every even column/.append style={column sep=0.5cm,black},
                    /tikz/every even column/.append style={black},
                },
            ]

            %% MULTIPLOT(algo) SELECT prefix/1024/1024/1024.0 AS x, time/1000/60 AS y, MULTIPLOT
            %% FROM (
            %% SELECT algo, input, MEDIAN(memFinal) AS memFinal, MEDIAN(memOff) AS memOff, AVG(memPeak) AS memPeak, prefix, rep, thread_count, MEDIAN(time) AS time, sacheck FROM stats2 GROUP BY algo, input, prefix, rep, thread_count
            %% ) WHERE input="commoncrawl.txt" AND sacheck="ok" 
            %% AND (algo="Naiv" OR algo="DivSufSort" OR algo="DivSufSort_ref")
            %% GROUP BY MULTIPLOT,x ORDER BY MULTIPLOT,x
            \addplot coordinates { (0.195312,5.04956) (0.390625,10.7497) (0.78125,24.8351) (1.5625,56.6159) (2.34375,96.6195) };
            \addlegendentry{algo=DivSufSort};
            \addplot coordinates { (0.195312,0.438642) (0.390625,0.81735) (0.78125,1.5745) (1.5625,3.19154) };
            \addlegendentry{algo=DivSufSort\_ref};
            \addplot coordinates { (0.195312,7.14747) (0.390625,11.1463) (0.78125,26.6099) (1.5625,62.0999) };
            \addlegendentry{algo=Naiv};

        \end{axis}
        \begin{axis}[
                cycle list name={exoticlines},
                at={(axis1.outer north east)},
                anchor=outer north west,
                name=axis2,
                width=0.5\textwidth,
                height=48mm,
                title={commoncrawl.txt},
                xlabel={input size [GiB]},
                ylabel={Extra Memory [GiB]},
            ]

            %% MULTIPLOT(algo) SELECT prefix/1024/1024/1024.0 AS x, memPeak/1024/1024/1024 AS y, MULTIPLOT
            %% FROM (
            %% SELECT algo, input, MEDIAN(memFinal) AS memFinal, MEDIAN(memOff) AS memOff, AVG(memPeak) AS memPeak, prefix, rep, thread_count, MEDIAN(time) AS time, sacheck FROM stats2 GROUP BY algo, input, prefix, rep, thread_count
            %% ) WHERE input="commoncrawl.txt" AND sacheck="ok" 
            %% AND (algo="Naiv" OR algo="DivSufSort" OR algo="DivSufSort_ref")
            %% GROUP BY MULTIPLOT,x ORDER BY MULTIPLOT,x
            \addplot coordinates { (0.195312,0.536333) (0.390625,1.05382) (0.78125,2.10676) (1.5625,4.21944) (2.34375,12.6344) };
            \addlegendentry{algo=DivSufSort};
            \addplot coordinates { (0.195312,0.000246726) (0.390625,0.000247501) (0.78125,0.000249021) (1.5625,0.000252061) };
            \addlegendentry{algo=DivSufSort\_ref};
            \addplot coordinates { (0.195312,0.0) (0.390625,0.0) (0.78125,0.0) (1.5625,0.0) };
            \addlegendentry{algo=Naiv};

            \legend{}
        \end{axis}
    \end{tikzpicture}

    \medskip
    \ref{legend-seq-divsufsort-2}
\end{figure}
\FloatBarrier

\clearpage
\subsubsection{GSACA}

\textbf{Configuration} \hfill Model name: Intel\textsuperscript{\textregistered} Xeon\textsuperscript{\textregistered} CPU E5-2640 v4 @ 2.40GHz
% IMPORT-DATA stats2 ../results.txt

\begin{figure}[ht]
    \centering
    \begin{tikzpicture}
        \begin{axis}[
                name=axis1,
                cycle list name={exoticlines},
                width=0.5\textwidth,
                height=48mm,
                title={wiki.txt},
                xlabel={input size [200\,MB]},
                ylabel={SA construction time [min]},
                legend columns=2,
                legend to name=legend-seq-gsaca-0,
                legend style={
                    /tikz/every even column/.append style={column sep=0.5cm,black},
                    /tikz/every even column/.append style={black},
                },
            ]

            %% MULTIPLOT(algo) SELECT thread_count AS x, time/1000/60 AS y, MULTIPLOT
            %% FROM (
            %% SELECT algo, input, MEDIAN(memFinal) AS memFinal, MEDIAN(memOff) AS memOff, AVG(memPeak) AS memPeak, prefix, rep, thread_count, MEDIAN(time) AS time, sacheck FROM stats2 GROUP BY algo, input, prefix, rep, thread_count
            %% ) WHERE input="wiki.txt" AND sacheck="ok" 
            %% AND (algo="Naiv" OR algo="GSACA" OR algo="GSACA_ref" OR algo="GSACA_Opt")
            %% GROUP BY MULTIPLOT,x ORDER BY MULTIPLOT,x
            \addplot coordinates { (1,1.63248) (2,3.51828) (4,7.66738) (8,18.0422) };
            \addlegendentry{algo=GSACA};
            \addplot coordinates { (1,1.53868) (2,3.32921) (4,7.31669) (8,18.5135) };
            \addlegendentry{algo=GSACA\_ref};
            \addplot coordinates { (1,2.7048) (2,6.0286) (4,15.8137) (8,28.8136) (12,59.9193) (16,65.215) (20,105.373) };
            \addlegendentry{algo=Naiv};

        \end{axis}
        \begin{axis}[
                cycle list name={exoticlines},
                at={(axis1.outer north east)},
                anchor=outer north west,
                name=axis2,
                width=0.5\textwidth,
                height=48mm,
                title={wiki.txt},
                xlabel={input size [200\,MB]},
                ylabel={Extra Memory [GiB]},
            ]

            %% MULTIPLOT(algo) SELECT thread_count AS x, memPeak/1024/1024/1024 AS y, MULTIPLOT
            %% FROM (
            %% SELECT algo, input, MEDIAN(memFinal) AS memFinal, MEDIAN(memOff) AS memOff, AVG(memPeak) AS memPeak, prefix, rep, thread_count, MEDIAN(time) AS time, sacheck FROM stats2 GROUP BY algo, input, prefix, rep, thread_count
            %% ) WHERE input="wiki.txt" AND sacheck="ok" 
            %% AND (algo="Naiv" OR algo="GSACA" OR algo="GSACA_ref" OR algo="GSACA_Opt")
            %% GROUP BY MULTIPLOT,x ORDER BY MULTIPLOT,x
            \addplot coordinates { (1,3.125) (2,6.25) (4,12.5) (8,25) };
            \addlegendentry{algo=GSACA};
            \addplot coordinates { (1,2.34375) (2,4.6875) (4,9.375) (8,18.75) };
            \addlegendentry{algo=GSACA\_ref};
            \addplot coordinates { (1,0.0) (2,0.0) (4,0.0) (8,0.0) (12,0.0) (16,0.0) (20,0.0) };
            \addlegendentry{algo=Naiv};

            \legend{}
        \end{axis}
    \end{tikzpicture}

    \medskip
    \ref{legend-seq-gsaca-0}
\caption{GSACA und GSACA_ref auf wiki.txt}
\label{GSACA-seq-0}
\end{figure}

\begin{figure}[ht]
    \centering
    \begin{tikzpicture}
        \begin{axis}[
                name=axis1,
                cycle list name={exoticlines},
                width=0.5\textwidth,
                height=48mm,
                title={dna.txt},
                xlabel={input size [200\,MB]},
                ylabel={SA construction time [min]},
                legend columns=2,
                legend to name=legend-seq-gsaca-1,
                legend style={
                    /tikz/every even column/.append style={column sep=0.5cm,black},
                    /tikz/every even column/.append style={black},
                },
            ]

            %% MULTIPLOT(algo) SELECT thread_count AS x, time/1000/60 AS y, MULTIPLOT
            %% FROM (
            %% SELECT algo, input, MEDIAN(memFinal) AS memFinal, MEDIAN(memOff) AS memOff, AVG(memPeak) AS memPeak, prefix, rep, thread_count, MEDIAN(time) AS time, sacheck FROM stats2 GROUP BY algo, input, prefix, rep, thread_count
            %% ) WHERE input="dna.txt" AND sacheck="ok" 
            %% AND (algo="Naiv" OR algo="GSACA" OR algo="GSACA_ref" OR algo="GSACA_Opt")
            %% GROUP BY MULTIPLOT,x ORDER BY MULTIPLOT,x
            \addplot coordinates { (1,1.44488) (2,3.04899) (4,6.54744) (8,15.7501) };
            \addlegendentry{algo=GSACA};
            \addplot coordinates { (1,1.37708) (2,2.93697) (4,6.28076) (8,13.6695) };
            \addlegendentry{algo=GSACA\_ref};
            \addplot coordinates { (1,3.07022) (2,6.92539) (4,15.1993) (8,31.8829) (12,50.4039) (16,86.2066) (20,90.5095) };
            \addlegendentry{algo=Naiv};

        \end{axis}
        \begin{axis}[
                cycle list name={exoticlines},
                at={(axis1.outer north east)},
                anchor=outer north west,
                name=axis2,
                width=0.5\textwidth,
                height=48mm,
                title={dna.txt},
                xlabel={input size [200\,MB]},
                ylabel={Extra Memory [GiB]},
            ]

            %% MULTIPLOT(algo) SELECT thread_count AS x, memPeak/1024/1024/1024 AS y, MULTIPLOT
            %% FROM (
            %% SELECT algo, input, MEDIAN(memFinal) AS memFinal, MEDIAN(memOff) AS memOff, AVG(memPeak) AS memPeak, prefix, rep, thread_count, MEDIAN(time) AS time, sacheck FROM stats2 GROUP BY algo, input, prefix, rep, thread_count
            %% ) WHERE input="dna.txt" AND sacheck="ok" 
            %% AND (algo="Naiv" OR algo="GSACA" OR algo="GSACA_ref" OR algo="GSACA_Opt")
            %% GROUP BY MULTIPLOT,x ORDER BY MULTIPLOT,x
            \addplot coordinates { (1,3.125) (2,6.25) (4,12.5) (8,25) };
            \addlegendentry{algo=GSACA};
            \addplot coordinates { (1,2.34375) (2,4.6875) (4,9.375) (8,18.75) };
            \addlegendentry{algo=GSACA\_ref};
            \addplot coordinates { (1,0.0) (2,0.0) (4,0.0) (8,0.0) (12,0.0) (16,0.0) (20,0.0) };
            \addlegendentry{algo=Naiv};

            \legend{}
        \end{axis}
    \end{tikzpicture}

    \medskip
    \ref{legend-seq-gsaca-1}
\caption{GSACA und GSACA_ref auf dna.txt}
\label{GSACA-seq-1}
\end{figure}

\begin{figure}[ht]
    \centering
    \begin{tikzpicture}
        \begin{axis}[
                name=axis1,
                cycle list name={exoticlines},
                width=0.5\textwidth,
                height=48mm,
                title={commoncrawl.txt},
                xlabel={input size [200\,MB]},
                ylabel={SA construction time [min]},
                legend columns=2,
                legend to name=legend-seq-gsaca-2,
                legend style={
                    /tikz/every even column/.append style={column sep=0.5cm,black},
                    /tikz/every even column/.append style={black},
                },
            ]

            %% MULTIPLOT(algo) SELECT thread_count AS x, time/1000/60 AS y, MULTIPLOT
            %% FROM (
            %% SELECT algo, input, MEDIAN(memFinal) AS memFinal, MEDIAN(memOff) AS memOff, AVG(memPeak) AS memPeak, prefix, rep, thread_count, MEDIAN(time) AS time, sacheck FROM stats2 GROUP BY algo, input, prefix, rep, thread_count
            %% ) WHERE input="commoncrawl.txt" AND sacheck="ok" 
            %% AND (algo="Naiv" OR algo="GSACA" OR algo="GSACA_ref" OR algo="GSACA_Opt")
            %% GROUP BY MULTIPLOT,x ORDER BY MULTIPLOT,x
            \addplot coordinates { (1,1.34583) (2,2.9523) (4,6.47806) (8,14.6961) };
            \addlegendentry{algo=GSACA};
            \addplot coordinates { (1,1.25412) (2,2.79102) (4,6.09891) (8,15.3615) };
            \addlegendentry{algo=GSACA\_ref};
            \addplot coordinates { (1,7.14747) (2,11.1463) (4,26.6099) (8,62.0999) };
            \addlegendentry{algo=Naiv};

        \end{axis}
        \begin{axis}[
                cycle list name={exoticlines},
                at={(axis1.outer north east)},
                anchor=outer north west,
                name=axis2,
                width=0.5\textwidth,
                height=48mm,
                title={commoncrawl.txt},
                xlabel={input size [200\,MB]},
                ylabel={Extra Memory [GiB]},
            ]

            %% MULTIPLOT(algo) SELECT thread_count AS x, memPeak/1024/1024/1024 AS y, MULTIPLOT
            %% FROM (
            %% SELECT algo, input, MEDIAN(memFinal) AS memFinal, MEDIAN(memOff) AS memOff, AVG(memPeak) AS memPeak, prefix, rep, thread_count, MEDIAN(time) AS time, sacheck FROM stats2 GROUP BY algo, input, prefix, rep, thread_count
            %% ) WHERE input="commoncrawl.txt" AND sacheck="ok" 
            %% AND (algo="Naiv" OR algo="GSACA" OR algo="GSACA_ref" OR algo="GSACA_Opt")
            %% GROUP BY MULTIPLOT,x ORDER BY MULTIPLOT,x
            \addplot coordinates { (1,3.125) (2,6.25) (4,12.5) (8,25) };
            \addlegendentry{algo=GSACA};
            \addplot coordinates { (1,2.34375) (2,4.6875) (4,9.375) (8,18.75) };
            \addlegendentry{algo=GSACA\_ref};
            \addplot coordinates { (1,0.0) (2,0.0) (4,0.0) (8,0.0) };
            \addlegendentry{algo=Naiv};

            \legend{}
        \end{axis}
    \end{tikzpicture}

    \medskip
    \ref{legend-seq-gsaca-2}
\caption{GSACA und GSACA_ref auf commoncrawl.txt}
\label{GSACA-seq-2}
\end{figure}
\FloatBarrier

Die Diagramme stellen den Suffix-Array-Konstruktionsalgorithmus GSACA der Referenzimplementierung und dem naiven SACA gegebnüber.
Wie zu sehe ist, sind GSACA und die Referenzimplementierung bei allen drei Eingabetexten schneller als der naive Algorihtmus,
haben jedoch auch einen höheren Speicherbedarf.
Dieser ist bei GSACA wiederum größer als bei der Referenzimplementierung.
Die alternative Variante von GSACA, welche in Abschnitt \ref{gsaca:chapter7} beschrieben wurde, ist hingegen in keinem der Diagrammen aufgeführt.
Dies liegt an der verlängerten Laufzeit, welche durch den zusätzlichen Aufwand in der Berechnung der Werte von GSIZE entsteht.
Hierdurch schaffte es diese Variante nicht, in der vorgegebenen maximalen Zeit das Suffix-Array zu berechnen.
%\subsubsection{Osipov}

\textbf{Configuration} \hfill Model name: none
% IMPORT-DATA stats2 ../results.txt

\begin{figure}[ht]
    \centering
    \begin{tikzpicture}
        \begin{axis}[
                name=axis1,
                cycle list name={exoticlines},
                width=0.5\textwidth,
                height=65mm,
                title={wiki.txt},
                xlabel={input size [200\,MB]},
                ylabel={SA construction time [s]},
                legend columns=2,
                legend to name=legend-seq-osipov-0,
                legend style={
                    /tikz/every even column/.append style={column sep=0.5cm,black},
                    /tikz/every even column/.append style={black},
                },
            ]

            %% MULTIPLOT(algo) SELECT thread_count AS x, time/1000 AS y, MULTIPLOT
            %% FROM (
            %% SELECT algo, input, MEDIAN(memFinal) AS memFinal, MEDIAN(memOff) AS memOff, AVG(memPeak) AS memPeak, prefix, rep, thread_count, MEDIAN(time) AS time, sacheck FROM stats2 GROUP BY algo, input, prefix, rep, thread_count
            %% ) WHERE input="wiki.txt" AND sacheck="ok" 
            %% AND (algo="Naiv" OR algo="Osipov_sequential_wp" OR algo="Osipov_sequential")
            %% GROUP BY MULTIPLOT,x ORDER BY MULTIPLOT,x
            \addplot coordinates { (1,162.841) (2,361.602) (4,783.025) (8,1729.44) (12,3448.59) (16,3842.25) (20,6183.09) };
            \addlegendentry{algo=Naiv};
            \addplot coordinates { (1,173.123) (2,364.934) (4,770.468) (8,2098.53) };
            \addlegendentry{algo=Osipov\_sequential};
            \addplot coordinates { (1,150.597) (2,320.338) (4,679.733) (8,1604.35) };
            \addlegendentry{algo=Osipov\_sequential\_wp};

        \end{axis}
        \begin{axis}[
                cycle list name={exoticlines},
                at={(axis1.outer north east)},
                anchor=outer north west,
                name=axis2,
                width=0.5\textwidth,
                height=65mm,
                title={wiki.txt},
                xlabel={input size [200\,MB]},
                ylabel={Extra Memory [MB]},
            ]

            %% MULTIPLOT(algo) SELECT thread_count AS x, memPeak/1000000 AS y, MULTIPLOT
            %% FROM (
            %% SELECT algo, input, MEDIAN(memFinal) AS memFinal, MEDIAN(memOff) AS memOff, AVG(memPeak) AS memPeak, prefix, rep, thread_count, MEDIAN(time) AS time, sacheck FROM stats2 GROUP BY algo, input, prefix, rep, thread_count
            %% ) WHERE input="wiki.txt" AND sacheck="ok" 
            %% AND (algo="Naiv" OR algo="Osipov_sequential_wp" OR algo="Osipov_sequential")
            %% GROUP BY MULTIPLOT,x ORDER BY MULTIPLOT,x
            \addplot coordinates { (1,0.0) (2,0.0) (4,0.0) (8,0.0) (12,0.0) (16,0.0) (20,0.0) };
            \addlegendentry{algo=Naiv};
            \addplot coordinates { (1,5869.72) (2,11740.7) (4,23483.9) (8,46972.2) };
            \addlegendentry{algo=Osipov\_sequential};
            \addplot coordinates { (1,5869.72) (2,11740.7) (4,23483.9) (8,46972.2) };
            \addlegendentry{algo=Osipov\_sequential\_wp};

            \legend{}
        \end{axis}
    \end{tikzpicture}

    \medskip
    \ref{legend-seq-osipov-0}
\end{figure}

\begin{figure}[ht]
    \centering
    \begin{tikzpicture}
        \begin{axis}[
                name=axis1,
                cycle list name={exoticlines},
                width=0.5\textwidth,
                height=65mm,
                title={dna.txt},
                xlabel={input size [200\,MB]},
                ylabel={SA construction time [s]},
                legend columns=2,
                legend to name=legend-seq-osipov-1,
                legend style={
                    /tikz/every even column/.append style={column sep=0.5cm,black},
                    /tikz/every even column/.append style={black},
                },
            ]

            %% MULTIPLOT(algo) SELECT thread_count AS x, time/1000 AS y, MULTIPLOT
            %% FROM (
            %% SELECT algo, input, MEDIAN(memFinal) AS memFinal, MEDIAN(memOff) AS memOff, AVG(memPeak) AS memPeak, prefix, rep, thread_count, MEDIAN(time) AS time, sacheck FROM stats2 GROUP BY algo, input, prefix, rep, thread_count
            %% ) WHERE input="dna.txt" AND sacheck="ok" 
            %% AND (algo="Naiv" OR algo="Osipov_sequential_wp" OR algo="Osipov_sequential")
            %% GROUP BY MULTIPLOT,x ORDER BY MULTIPLOT,x
            \addplot coordinates { (1,183.948) (2,415.412) (4,910.737) (8,1910.63) (12,3645.28) (16,4061.77) (20,6575.73) };
            \addlegendentry{algo=Naiv};
            \addplot coordinates { (1,167.54) (2,356.139) (4,754.252) (8,1612.7) };
            \addlegendentry{algo=Osipov\_sequential};
            \addplot coordinates { (1,150.108) (2,320.971) (4,684.242) (8,1478.18) };
            \addlegendentry{algo=Osipov\_sequential\_wp};

        \end{axis}
        \begin{axis}[
                cycle list name={exoticlines},
                at={(axis1.outer north east)},
                anchor=outer north west,
                name=axis2,
                width=0.5\textwidth,
                height=65mm,
                title={dna.txt},
                xlabel={input size [200\,MB]},
                ylabel={Extra Memory [MB]},
            ]

            %% MULTIPLOT(algo) SELECT thread_count AS x, memPeak/1000000 AS y, MULTIPLOT
            %% FROM (
            %% SELECT algo, input, MEDIAN(memFinal) AS memFinal, MEDIAN(memOff) AS memOff, AVG(memPeak) AS memPeak, prefix, rep, thread_count, MEDIAN(time) AS time, sacheck FROM stats2 GROUP BY algo, input, prefix, rep, thread_count
            %% ) WHERE input="dna.txt" AND sacheck="ok" 
            %% AND (algo="Naiv" OR algo="Osipov_sequential_wp" OR algo="Osipov_sequential")
            %% GROUP BY MULTIPLOT,x ORDER BY MULTIPLOT,x
            \addplot coordinates { (1,0.0) (2,0.0) (4,0.0) (8,0.0) (12,0.0) (16,0.0) (20,0.0) };
            \addlegendentry{algo=Naiv};
            \addplot coordinates { (1,5872.03) (2,11744.1) (4,23488.1) (8,46976.2) };
            \addlegendentry{algo=Osipov\_sequential};
            \addplot coordinates { (1,5872.03) (2,11744.1) (4,23488.1) (8,46976.2) };
            \addlegendentry{algo=Osipov\_sequential\_wp};

            \legend{}
        \end{axis}
    \end{tikzpicture}

    \medskip
    \ref{legend-seq-osipov-1}
\end{figure}

\begin{figure}[ht]
    \centering
    \begin{tikzpicture}
        \begin{axis}[
                name=axis1,
                cycle list name={exoticlines},
                width=0.5\textwidth,
                height=65mm,
                title={commoncrawl.txt},
                xlabel={input size [200\,MB]},
                ylabel={SA construction time [s]},
                legend columns=2,
                legend to name=legend-seq-osipov-2,
                legend style={
                    /tikz/every even column/.append style={column sep=0.5cm,black},
                    /tikz/every even column/.append style={black},
                },
            ]

            %% MULTIPLOT(algo) SELECT thread_count AS x, time/1000 AS y, MULTIPLOT
            %% FROM (
            %% SELECT algo, input, MEDIAN(memFinal) AS memFinal, MEDIAN(memOff) AS memOff, AVG(memPeak) AS memPeak, prefix, rep, thread_count, MEDIAN(time) AS time, sacheck FROM stats2 GROUP BY algo, input, prefix, rep, thread_count
            %% ) WHERE input="commoncrawl.txt" AND sacheck="ok" 
            %% AND (algo="Naiv" OR algo="Osipov_sequential_wp" OR algo="Osipov_sequential")
            %% GROUP BY MULTIPLOT,x ORDER BY MULTIPLOT,x
            \addplot coordinates { (1,428.421) (2,669.456) (4,1602.12) (8,3723.05) };
            \addlegendentry{algo=Naiv};
            \addplot coordinates { (1,237.865) (2,522.81) (4,1084.11) (8,2467.35) };
            \addlegendentry{algo=Osipov\_sequential};
            \addplot coordinates { (1,215.233) (2,461.322) (4,994.943) (8,2566.84) };
            \addlegendentry{algo=Osipov\_sequential\_wp};

        \end{axis}
        \begin{axis}[
                cycle list name={exoticlines},
                at={(axis1.outer north east)},
                anchor=outer north west,
                name=axis2,
                width=0.5\textwidth,
                height=65mm,
                title={commoncrawl.txt},
                xlabel={input size [200\,MB]},
                ylabel={Extra Memory [MB]},
            ]

            %% MULTIPLOT(algo) SELECT thread_count AS x, memPeak/1000000 AS y, MULTIPLOT
            %% FROM (
            %% SELECT algo, input, MEDIAN(memFinal) AS memFinal, MEDIAN(memOff) AS memOff, AVG(memPeak) AS memPeak, prefix, rep, thread_count, MEDIAN(time) AS time, sacheck FROM stats2 GROUP BY algo, input, prefix, rep, thread_count
            %% ) WHERE input="commoncrawl.txt" AND sacheck="ok" 
            %% AND (algo="Naiv" OR algo="Osipov_sequential_wp" OR algo="Osipov_sequential")
            %% GROUP BY MULTIPLOT,x ORDER BY MULTIPLOT,x
            \addplot coordinates { (1,0.0) (2,0.0) (4,0.0) (8,0.0) };
            \addlegendentry{algo=Naiv};
            \addplot coordinates { (1,5870.94) (2,11741.7) (4,23485) (8,46972) };
            \addlegendentry{algo=Osipov\_sequential};
            \addplot coordinates { (1,5870.94) (2,11741.7) (4,23485) (8,46972) };
            \addlegendentry{algo=Osipov\_sequential\_wp};

            \legend{}
        \end{axis}
    \end{tikzpicture}

    \medskip
    \ref{legend-seq-osipov-2}
\end{figure}
\FloatBarrier

%\subsubsection{SACA-K}

\textbf{Configuration} \hfill Model name: none
% IMPORT-DATA stats2 ../results.txt

\begin{figure}[ht]
    \centering
    \begin{tikzpicture}
        \begin{axis}[
                name=axis1,
                cycle list name={exoticlines},
                width=0.5\textwidth,
                height=65mm,
                title={wiki.txt},
                xlabel={input size [200\,MB]},
                ylabel={SA construction time [s]},
                legend columns=2,
                legend to name=legend-seq-sacak-0,
                legend style={
                    /tikz/every even column/.append style={column sep=0.5cm,black},
                    /tikz/every even column/.append style={black},
                },
            ]

            %% MULTIPLOT(algo) SELECT thread_count AS x, time/1000 AS y, MULTIPLOT
            %% FROM (
            %% SELECT algo, input, MEDIAN(memFinal) AS memFinal, MEDIAN(memOff) AS memOff, AVG(memPeak) AS memPeak, prefix, rep, thread_count, MEDIAN(time) AS time, sacheck FROM stats2 GROUP BY algo, input, prefix, rep, thread_count
            %% ) WHERE input="wiki.txt" AND sacheck="ok" 
            %% AND (algo="Naiv" OR algo="SACA-K" OR algo="SACA-K_ref")
            %% GROUP BY MULTIPLOT,x ORDER BY MULTIPLOT,x
            \addplot coordinates { (1,162.841) (2,361.602) (4,783.025) (8,1729.44) (12,3448.59) (16,3842.25) (20,6183.09) };
            \addlegendentry{algo=Naiv};
            \addplot coordinates { (1,85.2816) (2,184.089) (4,387.585) (8,789.897) (12,1906.57) (16,1832.18) (20,2608.71) };
            \addlegendentry{algo=SACA-K};
            \addplot coordinates { (1,67.5895) (2,146.574) (4,309.243) (8,633.171) (12,1251.08) (16,1334.4) (20,2504.26) };
            \addlegendentry{algo=SACA-K\_ref};

        \end{axis}
        \begin{axis}[
                cycle list name={exoticlines},
                at={(axis1.outer north east)},
                anchor=outer north west,
                name=axis2,
                width=0.5\textwidth,
                height=65mm,
                title={wiki.txt},
                xlabel={input size [200\,MB]},
                ylabel={Extra Memory [MB]},
            ]

            %% MULTIPLOT(algo) SELECT thread_count AS x, memPeak/1000000 AS y, MULTIPLOT
            %% FROM (
            %% SELECT algo, input, MEDIAN(memFinal) AS memFinal, MEDIAN(memOff) AS memOff, AVG(memPeak) AS memPeak, prefix, rep, thread_count, MEDIAN(time) AS time, sacheck FROM stats2 GROUP BY algo, input, prefix, rep, thread_count
            %% ) WHERE input="wiki.txt" AND sacheck="ok" 
            %% AND (algo="Naiv" OR algo="SACA-K" OR algo="SACA-K_ref")
            %% GROUP BY MULTIPLOT,x ORDER BY MULTIPLOT,x
            \addplot coordinates { (1,0.0) (2,0.0) (4,0.0) (8,0.0) (12,0.0) (16,0.0) (20,0.0) };
            \addlegendentry{algo=Naiv};
            \addplot coordinates { (1,343.697) (2,625.246) (4,1126.38) (8,1925.11) (12,2610.55) (16,3418.62) (20,4180.56) };
            \addlegendentry{algo=SACA-K};
            \addplot coordinates { (1,0.000836) (2,0.000836) (4,0.00084) (8,0.00084) (12,0.00084) (16,0.00084) (20,0.000844) };
            \addlegendentry{algo=SACA-K\_ref};

            \legend{}
        \end{axis}
    \end{tikzpicture}

    \medskip
    \ref{legend-seq-sacak-0}
\end{figure}

\begin{figure}[ht]
    \centering
    \begin{tikzpicture}
        \begin{axis}[
                name=axis1,
                cycle list name={exoticlines},
                width=0.5\textwidth,
                height=65mm,
                title={dna.txt},
                xlabel={input size [200\,MB]},
                ylabel={SA construction time [s]},
                legend columns=2,
                legend to name=legend-seq-sacak-1,
                legend style={
                    /tikz/every even column/.append style={column sep=0.5cm,black},
                    /tikz/every even column/.append style={black},
                },
            ]

            %% MULTIPLOT(algo) SELECT thread_count AS x, time/1000 AS y, MULTIPLOT
            %% FROM (
            %% SELECT algo, input, MEDIAN(memFinal) AS memFinal, MEDIAN(memOff) AS memOff, AVG(memPeak) AS memPeak, prefix, rep, thread_count, MEDIAN(time) AS time, sacheck FROM stats2 GROUP BY algo, input, prefix, rep, thread_count
            %% ) WHERE input="dna.txt" AND sacheck="ok" 
            %% AND (algo="Naiv" OR algo="SACA-K" OR algo="SACA-K_ref")
            %% GROUP BY MULTIPLOT,x ORDER BY MULTIPLOT,x
            \addplot coordinates { (1,183.948) (2,415.412) (4,910.737) (8,1910.63) (12,3645.28) (16,4061.77) (20,6575.73) };
            \addlegendentry{algo=Naiv};
            \addplot coordinates { (1,81.6123) (2,168.851) (4,348.744) (8,1014.26) (12,1188.86) (16,1675.7) (20,2406.48) };
            \addlegendentry{algo=SACA-K};
            \addplot coordinates { (1,67.5694) (2,141.3) (4,294.461) (8,606.079) (12,1153.24) (16,1773.37) (20,2327.64) };
            \addlegendentry{algo=SACA-K\_ref};

        \end{axis}
        \begin{axis}[
                cycle list name={exoticlines},
                at={(axis1.outer north east)},
                anchor=outer north west,
                name=axis2,
                width=0.5\textwidth,
                height=65mm,
                title={dna.txt},
                xlabel={input size [200\,MB]},
                ylabel={Extra Memory [MB]},
            ]

            %% MULTIPLOT(algo) SELECT thread_count AS x, memPeak/1000000 AS y, MULTIPLOT
            %% FROM (
            %% SELECT algo, input, MEDIAN(memFinal) AS memFinal, MEDIAN(memOff) AS memOff, AVG(memPeak) AS memPeak, prefix, rep, thread_count, MEDIAN(time) AS time, sacheck FROM stats2 GROUP BY algo, input, prefix, rep, thread_count
            %% ) WHERE input="dna.txt" AND sacheck="ok" 
            %% AND (algo="Naiv" OR algo="SACA-K" OR algo="SACA-K_ref")
            %% GROUP BY MULTIPLOT,x ORDER BY MULTIPLOT,x
            \addplot coordinates { (1,0.0) (2,0.0) (4,0.0) (8,0.0) (12,0.0) (16,0.0) (20,0.0) };
            \addlegendentry{algo=Naiv};
            \addplot coordinates { (1,322.915) (2,493.271) (4,865.24) (8,1857.47) (12,2825.95) (16,3849.29) (20,4888.43) };
            \addlegendentry{algo=SACA-K};
            \addplot coordinates { (1,2.0e-05) (2,2.0e-05) (4,2.0e-05) (8,2.0e-05) (12,2.0e-05) (16,2.0e-05) (20,2.0e-05) };
            \addlegendentry{algo=SACA-K\_ref};

            \legend{}
        \end{axis}
    \end{tikzpicture}

    \medskip
    \ref{legend-seq-sacak-1}
\end{figure}

\begin{figure}[ht]
    \centering
    \begin{tikzpicture}
        \begin{axis}[
                name=axis1,
                cycle list name={exoticlines},
                width=0.5\textwidth,
                height=65mm,
                title={commoncrawl.txt},
                xlabel={input size [200\,MB]},
                ylabel={SA construction time [s]},
                legend columns=2,
                legend to name=legend-seq-sacak-2,
                legend style={
                    /tikz/every even column/.append style={column sep=0.5cm,black},
                    /tikz/every even column/.append style={black},
                },
            ]

            %% MULTIPLOT(algo) SELECT thread_count AS x, time/1000 AS y, MULTIPLOT
            %% FROM (
            %% SELECT algo, input, MEDIAN(memFinal) AS memFinal, MEDIAN(memOff) AS memOff, AVG(memPeak) AS memPeak, prefix, rep, thread_count, MEDIAN(time) AS time, sacheck FROM stats2 GROUP BY algo, input, prefix, rep, thread_count
            %% ) WHERE input="commoncrawl.txt" AND sacheck="ok" 
            %% AND (algo="Naiv" OR algo="SACA-K" OR algo="SACA-K_ref")
            %% GROUP BY MULTIPLOT,x ORDER BY MULTIPLOT,x
            \addplot coordinates { (1,428.421) (2,669.456) (4,1602.12) (8,3723.05) };
            \addlegendentry{algo=Naiv};
            \addplot coordinates { (1,69.7497) (2,155.464) (4,331.646) (8,1620.86) (12,1258.92) (16,1655.54) (20,2892.56) };
            \addlegendentry{algo=SACA-K};
            \addplot coordinates { (1,55.2005) (2,123.851) (4,266.413) (8,563.135) (12,1148.3) (16,1712.78) (20,2232.7) };
            \addlegendentry{algo=SACA-K\_ref};

        \end{axis}
        \begin{axis}[
                cycle list name={exoticlines},
                at={(axis1.outer north east)},
                anchor=outer north west,
                name=axis2,
                width=0.5\textwidth,
                height=65mm,
                title={commoncrawl.txt},
                xlabel={input size [200\,MB]},
                ylabel={Extra Memory [MB]},
            ]

            %% MULTIPLOT(algo) SELECT thread_count AS x, memPeak/1000000 AS y, MULTIPLOT
            %% FROM (
            %% SELECT algo, input, MEDIAN(memFinal) AS memFinal, MEDIAN(memOff) AS memOff, AVG(memPeak) AS memPeak, prefix, rep, thread_count, MEDIAN(time) AS time, sacheck FROM stats2 GROUP BY algo, input, prefix, rep, thread_count
            %% ) WHERE input="commoncrawl.txt" AND sacheck="ok" 
            %% AND (algo="Naiv" OR algo="SACA-K" OR algo="SACA-K_ref")
            %% GROUP BY MULTIPLOT,x ORDER BY MULTIPLOT,x
            \addplot coordinates { (1,0.0) (2,0.0) (4,0.0) (8,0.0) };
            \addlegendentry{algo=Naiv};
            \addplot coordinates { (1,319.277) (2,600.542) (4,1057.1) (8,1845.21) (12,2541.75) (16,3185.04) (20,3818.06) };
            \addlegendentry{algo=SACA-K};
            \addplot coordinates { (1,0.000944) (2,0.000964) (4,0.000972) (8,0.000972) (12,0.000972) (16,0.000972) (20,0.000972) };
            \addlegendentry{algo=SACA-K\_ref};

            \legend{}
        \end{axis}
    \end{tikzpicture}

    \medskip
    \ref{legend-seq-sacak-2}
\end{figure}
\FloatBarrier

