\clearpage
\subsubsection{GSACA}

\textbf{Configuration} \hfill Model name: Intel\textsuperscript{\textregistered} Xeon\textsuperscript{\textregistered} CPU E5-2640 v4 @ 2.40GHz
% IMPORT-DATA stats2 ../results.txt

\begin{figure}[ht]
    \centering
    \begin{tikzpicture}
        \begin{axis}[
                name=axis1,
                cycle list name={exoticlines},
                width=0.5\textwidth,
                height=48mm,
                title={wiki.txt},
                xlabel={input size [GiB]},
                ylabel={SA construction time [min]},
                legend columns=2,
                legend to name=legend-seq-gsaca-0,
                legend style={
                    /tikz/every even column/.append style={column sep=0.5cm,black},
                    /tikz/every even column/.append style={black},
                },
            ]

            %% MULTIPLOT(algo) SELECT prefix/1024/1024/1024.0 AS x, time/1000/60 AS y, MULTIPLOT
            %% FROM (
            %% SELECT algo, input, MEDIAN(memFinal) AS memFinal, MEDIAN(memOff) AS memOff, AVG(memPeak) AS memPeak, prefix, rep, thread_count, MEDIAN(time) AS time, sacheck FROM stats2 GROUP BY algo, input, prefix, rep, thread_count
            %% ) WHERE input="wiki.txt" AND sacheck="ok" 
            %% AND (algo="Naiv" OR algo="GSACA" OR algo="GSACA_ref" OR algo="GSACA_Opt")
            %% GROUP BY MULTIPLOT,x ORDER BY MULTIPLOT,x
            \addplot coordinates { (0.195312,1.63248) (0.390625,3.51828) (0.78125,7.66738) (1.5625,18.0422) };
            \addlegendentry{algo=GSACA};
            \addplot coordinates { (0.195312,1.53868) (0.390625,3.32921) (0.78125,7.31669) (1.5625,18.5135) };
            \addlegendentry{algo=GSACA\_ref};
            \addplot coordinates { (0.195312,2.7048) (0.390625,6.0286) (0.78125,15.8137) (1.5625,28.8136) (2.34375,59.9193) (3.125,65.215) (3.90625,105.373) };
            \addlegendentry{algo=Naiv};

        \end{axis}
        \begin{axis}[
                cycle list name={exoticlines},
                at={(axis1.outer north east)},
                anchor=outer north west,
                name=axis2,
                width=0.5\textwidth,
                height=48mm,
                title={wiki.txt},
                xlabel={input size [GiB]},
                ylabel={Extra Memory [GiB]},
            ]

            %% MULTIPLOT(algo) SELECT prefix/1024/1024/1024.0 AS x, memPeak/1024/1024/1024 AS y, MULTIPLOT
            %% FROM (
            %% SELECT algo, input, MEDIAN(memFinal) AS memFinal, MEDIAN(memOff) AS memOff, AVG(memPeak) AS memPeak, prefix, rep, thread_count, MEDIAN(time) AS time, sacheck FROM stats2 GROUP BY algo, input, prefix, rep, thread_count
            %% ) WHERE input="wiki.txt" AND sacheck="ok" 
            %% AND (algo="Naiv" OR algo="GSACA" OR algo="GSACA_ref" OR algo="GSACA_Opt")
            %% GROUP BY MULTIPLOT,x ORDER BY MULTIPLOT,x
            \addplot coordinates { (0.195312,3.125) (0.390625,6.25) (0.78125,12.5) (1.5625,25) };
            \addlegendentry{algo=GSACA};
            \addplot coordinates { (0.195312,2.34375) (0.390625,4.6875) (0.78125,9.375) (1.5625,18.75) };
            \addlegendentry{algo=GSACA\_ref};
            \addplot coordinates { (0.195312,0.0) (0.390625,0.0) (0.78125,0.0) (1.5625,0.0) (2.34375,0.0) (3.125,0.0) (3.90625,0.0) };
            \addlegendentry{algo=Naiv};

            \legend{}
        \end{axis}
    \end{tikzpicture}

    \medskip
    \ref{legend-seq-gsaca-0}
\caption{GSACA und GSACA_ref auf wiki.txt}
\label{GSACA-seq-0}
\end{figure}

\begin{figure}[ht]
    \centering
    \begin{tikzpicture}
        \begin{axis}[
                name=axis1,
                cycle list name={exoticlines},
                width=0.5\textwidth,
                height=48mm,
                title={dna.txt},
                xlabel={input size [GiB]},
                ylabel={SA construction time [min]},
                legend columns=2,
                legend to name=legend-seq-gsaca-1,
                legend style={
                    /tikz/every even column/.append style={column sep=0.5cm,black},
                    /tikz/every even column/.append style={black},
                },
            ]

            %% MULTIPLOT(algo) SELECT prefix/1024/1024/1024.0 AS x, time/1000/60 AS y, MULTIPLOT
            %% FROM (
            %% SELECT algo, input, MEDIAN(memFinal) AS memFinal, MEDIAN(memOff) AS memOff, AVG(memPeak) AS memPeak, prefix, rep, thread_count, MEDIAN(time) AS time, sacheck FROM stats2 GROUP BY algo, input, prefix, rep, thread_count
            %% ) WHERE input="dna.txt" AND sacheck="ok" 
            %% AND (algo="Naiv" OR algo="GSACA" OR algo="GSACA_ref" OR algo="GSACA_Opt")
            %% GROUP BY MULTIPLOT,x ORDER BY MULTIPLOT,x
            \addplot coordinates { (0.195312,1.44488) (0.390625,3.04899) (0.78125,6.54744) (1.5625,15.7501) };
            \addlegendentry{algo=GSACA};
            \addplot coordinates { (0.195312,1.37708) (0.390625,2.93697) (0.78125,6.28076) (1.5625,13.6695) };
            \addlegendentry{algo=GSACA\_ref};
            \addplot coordinates { (0.195312,3.07022) (0.390625,6.92539) (0.78125,15.1993) (1.5625,31.8829) (2.34375,50.4039) (3.125,86.2066) (3.90625,90.5095) };
            \addlegendentry{algo=Naiv};

        \end{axis}
        \begin{axis}[
                cycle list name={exoticlines},
                at={(axis1.outer north east)},
                anchor=outer north west,
                name=axis2,
                width=0.5\textwidth,
                height=48mm,
                title={dna.txt},
                xlabel={input size [GiB]},
                ylabel={Extra Memory [GiB]},
            ]

            %% MULTIPLOT(algo) SELECT prefix/1024/1024/1024.0 AS x, memPeak/1024/1024/1024 AS y, MULTIPLOT
            %% FROM (
            %% SELECT algo, input, MEDIAN(memFinal) AS memFinal, MEDIAN(memOff) AS memOff, AVG(memPeak) AS memPeak, prefix, rep, thread_count, MEDIAN(time) AS time, sacheck FROM stats2 GROUP BY algo, input, prefix, rep, thread_count
            %% ) WHERE input="dna.txt" AND sacheck="ok" 
            %% AND (algo="Naiv" OR algo="GSACA" OR algo="GSACA_ref" OR algo="GSACA_Opt")
            %% GROUP BY MULTIPLOT,x ORDER BY MULTIPLOT,x
            \addplot coordinates { (0.195312,3.125) (0.390625,6.25) (0.78125,12.5) (1.5625,25) };
            \addlegendentry{algo=GSACA};
            \addplot coordinates { (0.195312,2.34375) (0.390625,4.6875) (0.78125,9.375) (1.5625,18.75) };
            \addlegendentry{algo=GSACA\_ref};
            \addplot coordinates { (0.195312,0.0) (0.390625,0.0) (0.78125,0.0) (1.5625,0.0) (2.34375,0.0) (3.125,0.0) (3.90625,0.0) };
            \addlegendentry{algo=Naiv};

            \legend{}
        \end{axis}
    \end{tikzpicture}

    \medskip
    \ref{legend-seq-gsaca-1}
\caption{GSACA und GSACA_ref auf dna.txt}
\label{GSACA-seq-1}
\end{figure}

\begin{figure}[ht]
    \centering
    \begin{tikzpicture}
        \begin{axis}[
                name=axis1,
                cycle list name={exoticlines},
                width=0.5\textwidth,
                height=48mm,
                title={commoncrawl.txt},
                xlabel={input size [GiB]},
                ylabel={SA construction time [min]},
                legend columns=2,
                legend to name=legend-seq-gsaca-2,
                legend style={
                    /tikz/every even column/.append style={column sep=0.5cm,black},
                    /tikz/every even column/.append style={black},
                },
            ]

            %% MULTIPLOT(algo) SELECT prefix/1024/1024/1024.0 AS x, time/1000/60 AS y, MULTIPLOT
            %% FROM (
            %% SELECT algo, input, MEDIAN(memFinal) AS memFinal, MEDIAN(memOff) AS memOff, AVG(memPeak) AS memPeak, prefix, rep, thread_count, MEDIAN(time) AS time, sacheck FROM stats2 GROUP BY algo, input, prefix, rep, thread_count
            %% ) WHERE input="commoncrawl.txt" AND sacheck="ok" 
            %% AND (algo="Naiv" OR algo="GSACA" OR algo="GSACA_ref" OR algo="GSACA_Opt")
            %% GROUP BY MULTIPLOT,x ORDER BY MULTIPLOT,x
            \addplot coordinates { (0.195312,1.34583) (0.390625,2.9523) (0.78125,6.47806) (1.5625,14.6961) };
            \addlegendentry{algo=GSACA};
            \addplot coordinates { (0.195312,1.25412) (0.390625,2.79102) (0.78125,6.09891) (1.5625,15.3615) };
            \addlegendentry{algo=GSACA\_ref};
            \addplot coordinates { (0.195312,7.14747) (0.390625,11.1463) (0.78125,26.6099) (1.5625,62.0999) };
            \addlegendentry{algo=Naiv};

        \end{axis}
        \begin{axis}[
                cycle list name={exoticlines},
                at={(axis1.outer north east)},
                anchor=outer north west,
                name=axis2,
                width=0.5\textwidth,
                height=48mm,
                title={commoncrawl.txt},
                xlabel={input size [GiB]},
                ylabel={Extra Memory [GiB]},
            ]

            %% MULTIPLOT(algo) SELECT prefix/1024/1024/1024.0 AS x, memPeak/1024/1024/1024 AS y, MULTIPLOT
            %% FROM (
            %% SELECT algo, input, MEDIAN(memFinal) AS memFinal, MEDIAN(memOff) AS memOff, AVG(memPeak) AS memPeak, prefix, rep, thread_count, MEDIAN(time) AS time, sacheck FROM stats2 GROUP BY algo, input, prefix, rep, thread_count
            %% ) WHERE input="commoncrawl.txt" AND sacheck="ok" 
            %% AND (algo="Naiv" OR algo="GSACA" OR algo="GSACA_ref" OR algo="GSACA_Opt")
            %% GROUP BY MULTIPLOT,x ORDER BY MULTIPLOT,x
            \addplot coordinates { (0.195312,3.125) (0.390625,6.25) (0.78125,12.5) (1.5625,25) };
            \addlegendentry{algo=GSACA};
            \addplot coordinates { (0.195312,2.34375) (0.390625,4.6875) (0.78125,9.375) (1.5625,18.75) };
            \addlegendentry{algo=GSACA\_ref};
            \addplot coordinates { (0.195312,0.0) (0.390625,0.0) (0.78125,0.0) (1.5625,0.0) };
            \addlegendentry{algo=Naiv};

            \legend{}
        \end{axis}
    \end{tikzpicture}

    \medskip
    \ref{legend-seq-gsaca-2}
\caption{GSACA und GSACA_ref auf commoncrawl.txt}
\label{GSACA-seq-2}
\end{figure}
\FloatBarrier

Die Diagramme stellen den Suffix-Array-Konstruktionsalgorithmus GSACA der Referenzimplementierung und dem naiven SACA gegebnüber.
Wie zu sehe ist, sind GSACA und die Referenzimplementierung bei allen drei Eingabetexten schneller als der naive Algorihtmus,
haben jedoch auch einen höheren Speicherbedarf.
Dieser ist bei GSACA wiederum größer als bei der Referenzimplementierung.
Die alternative Variante von GSACA, welche in Abschnitt \ref{gsaca:chapter7} beschrieben wurde, ist hingegen in keinem der Diagrammen aufgeführt.
Dies liegt an der verlängerten Laufzeit, welche durch den zusätzlichen Aufwand in der Berechnung der Werte von GSIZE entsteht.
Hierdurch schaffte es diese Variante nicht, in der vorgegebenen maximalen Zeit das Suffix-Array zu berechnen.