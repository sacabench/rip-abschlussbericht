\subsection{\sa Korrektheit}

Wir überprüfen zunächst, ob alle Testdaten von allen Implementierungen korrekt verarbeitet werden konnten. Die Messergebnisse enthalten hierfür die von \texttt{-{}-check} erzeugte Informationen und können in \cref{anhang:messwerte:inputscale} (\cref{messung:tab:sa-chk-large-seq-weak}) eingesehen werden. Dabei wurde die Korrektheit aller sequentieller Algorithmen mit verschiedenen Testdaten der Größe 200 MiB festgehalten. Dabei fällt positiv auf, dass sowohl die Implementierungen der Projektgruppe, als auch die Referenzimplementierungen der originalen Paper bei den meisten Testdaten der Größe 200 MiB das Suffix-Array korrekt bestimmt wird. Lediglich bei den Texten \texttt{$pcr_{cere}$}, \texttt{$pcr_{kernel}$} und \texttt{$pcr_{para}$} überschreitet vereinzelt die Berechnungsdauer der Implementierungen das Zeitlimit des Messsystems. Das ist der Fall, weil diese Texte repetitiv sind und damit einige Algorithmen eine längere Berechnungszeit benötigen. Aber kein Algorithmus hat das Suffix-Array falsch berechnet oder ist gar abgestürzt.
