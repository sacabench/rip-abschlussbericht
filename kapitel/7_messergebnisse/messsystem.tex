\section{Messsystem}

\begin{table}
\caption{Messsystem}
\label{messung:tab:system}
\resizebox{\textwidth}{!}{
\begin{tabular}{ll}
\toprule
Betriebssystem       & GNU/Linux \\
Kernel               & Linux \\
Kernel Release       & 3.10.0-862.14.4.el7.x86\_64 \\
Kernel Version       & \#1 SMP Wed Sep 26 15:12:11 UTC 2018 \\
\midrule
Architecture:        & x86\_64 \\
CPU op-mode(s):      & 32-bit, 64-bit \\
Byte Order:          & Little Endian \\
CPU(s):              & 20 \\
On-line CPU(s) list: & 0-19 \\
Thread(s) per core:  & 1 \\
Core(s) per socket:  & 10 \\
Socket(s):           & 2 \\
NUMA node(s):        & 2 \\
Vendor ID:           & GenuineIntel \\
CPU family:          & 6 \\
Model:               & 79 \\
Model name:          & Intel(R) Xeon(R) CPU E5-2640 v4 @ 2.40GHz \\
Stepping:            & 1 \\
CPU MHz:             & 2599.951 \\
CPU max MHz:         & 3400.0000 \\
CPU min MHz:         & 1200.0000 \\
BogoMIPS:            & 4789.01 \\
Virtualization:      & VT-x \\
L1d cache:           & 32K \\
L1i cache:           & 32K \\
L2 cache:            & 256K \\
L3 cache:            & 25600K \\
NUMA node0 CPU(s):   & 0-9 \\
NUMA node1 CPU(s):   & 10-19 \\
\midrule
Speichergröße        & 64GiB \\
\bottomrule
\end{tabular}
}
\end{table}

Die Messung wurde auf dem in \cref{messung:tab:system} beschrieben System durchgeführt. Die Daten wurden durch die Linux Befehle \texttt{uname}, \texttt{lscpu} und \texttt{free} ermittelt.
