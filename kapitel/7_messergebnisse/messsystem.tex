\section{Messsystem}

\begin{table}
\caption{Messsystem}
\label{messung:tab:system}
\resizebox{\textwidth}{!}{
\begin{tabular}{ll}
\toprule
Betriebssystem       & GNU/Linux \\
Kernel               & Linux \\
Kernel Release       & 4.13.0-46-generic \\
Kernel Version       & \#51-Ubuntu SMP Tue Jun 12 12:36:29 UTC 2018 \\
\midrule
Architecture:        & x86\_64 \\
CPU op-mode(s):      & 32-bit, 64-bit \\
Byte Order:          & Little Endian \\
CPU(s):              & 4 \\
On-line CPU(s) list: & 0-3 \\
Thread(s) per core:  & 1 \\
Core(s) per socket:  & 4 \\
Socket(s):           & 1 \\
NUMA node(s):        & 1 \\
Vendor ID:           & GenuineIntel \\
CPU family:          & 6 \\
Model:               & 30 \\
Model name:          & Intel(R) Core(TM) i5 CPU         760  @ 2.80GHz \\
Stepping:            & 5 \\
CPU MHz:             & 3032.557 \\
CPU max MHz:         & 2801,0000 \\
CPU min MHz:         & 1200,0000 \\
BogoMIPS:            & 5617.49 \\
Virtualization:      & VT-x \\
L1d cache:           & 32K \\
L1i cache:           & 32K \\
L2 cache:            & 256K \\
L3 cache:            & 8192K \\
NUMA node0 CPU(s):   & 0-3 \\
\midrule
Speichergröße        & 8GiB \\
Speichermodul 0      & 4GiB DIMM DDR Synchronous 1333 MHz (0,8 ns) \\
Speichermodul 1      & DIMM [empty] \\
Speichermodul 2      & 4GiB DIMM DDR Synchronous 1333 MHz (0,8 ns) \\
Speichermodul 3      & DIMM [empty] \\
\bottomrule
\end{tabular}
}
\end{table}

Die Messung wurde auf dem in \cref{messung:tab:system} beschrieben System durchgeführt. Die Daten wurden durch die Linux Befehle \texttt{uname}, \texttt{lscpu} und \texttt{lshw -short} ermittelt.

