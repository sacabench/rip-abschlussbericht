\chapter{Extended Abstract}

% ###
% Beginning of the extended abstract. We will use english in this chapter.
\selectlanguage{english}
% ###

This \currentauthor{Marvin Böcker} is just a placeholder text until I know what to write here.

\section{Introduction}

The suffix array is a widely known text index, which can be used for various string operations, like full-text search~\cite{makinen} and construction of the Burrows-Wheeler Transform (BWT)~\cite{BWT}.
While the efficient construction is still an active research area,
divsufsort~\cite{saca:5,saca:5:repo} is the empirically fastest suffix array construction algorithm (SACA) since 2008\todo{wann?},
even though it has a theoretical complexity of $\mathcal O (n \log n)$,
while there are a few $\mathcal O(n)$ algorithms available.
In the last year we (re-)implemented eleven SACAs~\cite{saca:3,saca:11,saca:5,saca:9,saca:1,saca:8,saca:4,saca:7,saca:10,saca:6,saca:2} in modern C++ in order to beat divsufsort from its throne,
and although our attempts were not successful,
we created a sophisticated benchmark system for SACA algorithms, \emph{SACABench}, in the process.

In this paper we introduce SACABench and highlight its features,
as well as document some of the optimiziation strategies we used to implement the SACA algorithms. \blindtext

\section{Benchmark Tool}

SACABench is a CMake/C++14 project which contains many different SACA algorithms.
There are both sequential algorithms and parallel algorithms included and we also include the reference implementations for all of the algorithms, if there is one.
You are able to run a single, a subset of, or all of the algorithms, depending on your needs.
Time and memory consumption is automatically measured and can be output as JSON.
We also include tools to convert the JSON-format to graphs for a visual comparison of the algorithms.
Most of the graphs in this paper have been generated using SACABench directly\todo{Is that so?}.

\section{SACA Overview}

There are many SACAs that operate using different principles.
The most na\"ive SACA uses a general purpose string-sorter to sort the suffices of the input text.
Since a string comparison is $\mathcal O (n)$ in the worst-case, this would give $\mathcal O (n^2 \log n)$ runtime.
Even though this is a sub-optimal time bound,
there are several algorithms\footnote{These are Deep-Shallow, ...} in SACABench which don't improve on this,
but rather use methods to speed up the real-world performance.
These methods can be classified into the categories \emph{inducing} and \emph{doubling}:
%
\paragraph{Inducing} When you know the \emph{rank} of the suffix $i$, that is the position of $i$ in the SA, you also know the rank of the suffix $i-1$, if it is a L-Type.\todo{fixme. introduce L-types beforehand and fix this explaination}
\paragraph{Doubling} Imagine radix-sort, but with double the block size everytime. \todo{fix}

\bigskip

\blindtext

\section{Optimization Strategies}

\blindtext

\section{Conclusion}

\blindtext

\blindtext

% ###
% End of extended abstract, switch language to german again
\selectlanguage{ngerman}
% ###
